%
%   Subsection of Sect. 7.2 Main photometric equation
%

\subsection{Color Correction}% {\color{blue} Jean-Fran\c cois and Herv\'e}}
\label{se:color_correction}

The color correction $C_{s}$ is derived in
Sect.~\ref{ap:color_correction_HA}. Neglecting the effect of the
atmosphere on NIKA2 transmission, we compute the color correction
factor for target sources of spectral indices  $\alpha_{S}$ that are
different from Uranus using
\begin{equation}
  C_{s}(\nu_{0}, \alpha_s) = \frac{\int_{0}^{+\infty} (\nu/\nu_0)^{1.6} ~T({\nu}) d\nu}{ \int_{0}^{+\infty} (\nu
    /\nu_0)^{\alpha_S} ~ T({\nu}) d\nu}.
\end{equation}

Color correction factors for eight values of $\alpha_{S}$, and in particular
for $\alpha_{S}= 0.6$ which is the spectral index of MWC349, are
given in Table~\ref{tab:mod}. 

\begin{table*}[!h]
\caption{Color correction factor for a target source  $S \propto \nu ^{\alpha_S}$}
\label{tab:mod}
\centering 
\begin{tabular}{l| c c c c c c c c}
\hline\hline
\noalign{\smallskip}
Array  & \multicolumn{8}{c}{$\alpha_{S}$} \\
\hline
          &  -2 &  -1    &    0  & + 0.6 & +1  &  +2  & +3 & +4  \\
%            \noalign{\smallskip}
            \hline
%            \noalign{\smallskip}
          A1   & 0.876  &  0.916   &   0.951  & 0.969 &  0.981   &  1.005  &    1.024  &  1.037   \\
          A2   & 0.945  &  0.972   &   0.990  & 0.996 &  0.998   &  0.997  &    0.986  &  0.966      \\ 
          A3   & 0.907  &  0.940   &   0.967  & 0.980 &  0.987   &  1.001  &    1.009  &  1.011     \\
            \noalign{\smallskip}
            \hline
\multicolumn{8}{c}{Note : Uranus/Moreno model used for Uranus in this
  Table.}
\end{tabular}
\end{table*}





