%                                                                 aa.dem
% AA vers. 9.1, LaTeX class for Astronomy & Astrophysics
% demonstration file
%                                                       (c) EDP Sciences
%-----------------------------------------------------------------------
%
%\documentclass[referee]{aa} % for a referee version
%\documentclass[onecolumn]{aa} % for a paper on 1 column  
%\documentclass[longauth]{aa} % for the long lists of affiliations 
%\documentclass[letter]{aa} % for the letters 
%\documentclass[bibyear]{aa} % if the references are not structured 
%                              according to the author-year natbib style

%
\documentclass{aa}  

%
\usepackage{graphicx}
%%%%%%%%%%%%%%%%%%%%%%%%%%%%%%%%%%%%%%%%
\usepackage{txfonts}
%%%%%%%%%%%%%%%%%%%%%%%%%%%%%%%%%%%%%%%%
\usepackage{hyperref}
% To add links in your PDF file, use the package "hyperref"
% with options according to your LaTeX or PDFLaTeX drivers.
%
\begin{document} 


   \title{NIKA2 Calibration and Performance at the IRAM 30-m Telescope}

   \author{ Laurence Perotto
     \inst{1}
     \and 
     Nicolas Ponthieu
     \inst{2}
     \and Juan-F. Mac\'ias-P\'erez
     \inst{1}
     \and 
     F.-Xavier D\'esert
     \inst{2}
     \and 
     Jean-Fran\c cois Lestrade
     \inst{3}
     \and
     Herv\'e Aussel
     \inst{4}
     \and
     Fr\'ed\'eric Mayet
     \inst{1}
     \and
     Florian Ruppin
     \inst{1}
     \and
     NIKA2 CORE TEAM
   }

   \institute{Univ. Grenoble Alpes, CNRS, Grenoble INP, LPSC-IN2P3, 53, avenue des Martyrs, F-38000 Grenoble, France \\
     \email{laurence.perotto@lpsc.in2p3.fr}
     \and
     Univ. Grenoble Alpes, CNRS, IPAG, F-38000 Grenoble, France\\
     \and
     LERMA, Observatoire de Paris, PSL Research University, CNRS, Sorbonne Universités, UPMC Univ. Paris 06, F-75014, Paris, France\\
     \and
     AIM, CEA, CNRS, Universit\'e Paris-Saclay, Universit\'e Paris Diderot, Sorbonne Paris Cit\'e, F-91191 Gif-sur-Yvette, France
   }

   \date{Received April XX, 2019; Accepted XXXX XX, 2019}

% \abstract{}{}{}{}{} 
% 5 {} token are mandatory
   
   \abstract
   % context heading (optional)
   % {} leave it empty if necessary  
       {The New IRAM KID Array, NIKA2, is a dual-band millimetric camera of thousands of Kinetic
         Inductance Detectors (KID) installed at the IRAM 30-meter
         telescope in the Spanish Sierra Nevada. The instrument
         commissioning  was completed in September 2017, and NIKA2 is now
         open to the scientific community and will operate for the next
         decade. NIKA2 will be a key experiment for a vast variety of
         astrophysic and cosmology purposes, including planet
         formation, star formation in the Milky Way, gas in
         nearby galaxies, galaxy formation at high redshift and
         cosmology with galaxy cluster.}
       % aims heading (mandatory)
       {We present NIKA2 calibration and on-sky performance.}
       % methods heading (mandatory)
       {We used several
         thousands of observation scans that were acquired at the February
         2017 technical campaign, during which NIKA2 instrumental set-up was
         in the final configuration for the first time,
         and at the first science pools that took place in October 2017 and
         January 2018. This vast data set allows us to check the performance
         stability for one year and against a large range of observing
         elevations and atmospheric conditions. The calibration and
         performance assessment methods that we have developed are also
         thoroughly described, as well as the robustness tests that we have
         performed.}
       % results heading (mandatory)
       {We find an average fraction of usable detectors during a
         typical observing scan of 84 per cent in the 260~GHz bands and 90 per cent in the
         150~GHz, which well cover the 6.5 arcmin diameter field of
         view. The beam pattern has been measured and is characterized
         by a full width at half maximum (FWHM) of $11.1\pm0.1$ arcsec
         and $17.6 \pm 0.1$, and a beam efficiency of $55\pm 3$ and
         $77\pm 2$ per cent at 260 and 150~GHz respectively. The
         statistital relative calibration uncertainties are of about 6
       and 3 per cent for these two frequency bands, which validates
       the accuracy of the method we have deployed to correct for the
       atmosphere attenuation. The top-of-atmosphere noise equivalent
       flux density (NEFD) at 260 and 150~GHz are of $30 \pm 3$ and
       $9 \pm 1$ mJy.s$^{1/2}$/beam. This state-of-the-art performance
       confers NIKA2 with excellent mapping capabilities characterized
       by mapping speeds of about 100 and 1400
       arcmin$^2$/hours/mJy$^2$ at 260 and 150~GHz.}
       % conclusions heading (optional), leave it empty if necessary 
       {}

   \keywords{Submillimeter, Photometry, Method: observational, Cosmology: Cosmic Microwave Background}

   \maketitle

   
%
%-------------------------------------------------------------------

\section{Introduction}

   
\section{Conclusions}

   \begin{enumerate}
      \item The conditions for the stability of static, radiative
         layers in gas spheres, as described by Baker's
         standard one-zone model, can be expressed as stability
         equations of state. These stability equations of state depend
         only on the local thermodynamic state of the layer.
      \item If the constitutive relations -- equations of state and
         Rosseland mean opacities -- are specified, the stability
         equations of state can be evaluated without specifying
         properties of the layer.
      \item For solar composition gas the $\kappa$-mechanism is
         working in the regions of the ice and dust features
         in the opacities, the $\mathrm{H}_2$ dissociation and the
         combined H, first He ionization zone, as
         indicated by vibrational instability. These regions
         of instability are much larger in extent and degree of
         instability than the second He ionization zone
         that drives the Cephe{\"\i}d pulsations.
   \end{enumerate}

\begin{acknowledgements}
     
\end{acknowledgements}

% WARNING
%-------------------------------------------------------------------
% Please note that we have included the references to the file aa.dem in
% order to compile it, but we ask you to:
%
% - use BibTeX with the regular commands:
\bibliographystyle{aa} % style aa.bst
%\bibliography{Yourfile} % your references Yourfile.bib
%
% - join the .bib files when you upload your source files
%-------------------------------------------------------------------

\begin{appendix} %First appendix
\section{Background galaxy number counts and shear noise-levels}
\end{appendix}
%
%
\end{document}
