%%%%%%%%%%%%%%%%%%%%%%%%%%%%%%%%%%%%%%%%%
% Thin Sectioned Essay
% LaTeX Template
% Version 1.0 (3/8/11)
%
% This template has been downloaded from:
% http://www.LaTeXTemplates.com
%
% Original Author:
% Nicolas Diaz (nsdiaz@uc.cl) with extensive modifications by:
% Vel (vel@latextemplates.com)
%
% License:
% CC BY-NC-SA 3.0 (http://creativecommons.org/licenses/by-nc-sa/3.0/)
%
%%%%%%%%%%%%%%%%%%%%%%%%%%%%%%%%%%%%%%%%%

%----------------------------------------------------------------------------------------
%	PACKAGES AND OTHER DOCUMENT CONFIGURATIONS
%----------------------------------------------------------------------------------------

\documentclass[a4paper, 11pt]{report}
\usepackage[utf8]{inputenc}
\usepackage[T1]{fontenc}
\usepackage[french]{babel}
\usepackage[explicit]{titlesec}
%\textheight=21cm
\usepackage{hyperref}
\usepackage[protrusion=true,expansion=true]{microtype} % Better typography
\usepackage{graphicx} % Required for including pictures
\usepackage{wrapfig} % Allows in-line images
\usepackage{tablefootnote} % to display footnotes in tables
\usepackage{threeparttable}
\usepackage[toc,page]{appendix}
\usepackage{multirow}
\usepackage[usenames, dvipsnames]{color}
\usepackage{overpic} % add text to fig
\usepackage{mathpazo} % Use the Palatino font
\usepackage[T1]{fontenc} % Required for accented characters
\linespread{1.05} % Change line spacing here, Palatino benefits from a slight increase by default

% To have DRAFT displayed accross all pages
%\usepackage{draftwatermark}
%\SetWatermarkText{DRAFT}
%\SetWatermarkScale{1}
%\SetWatermarkLightness{0.95}

%\usepackage{background}
%\backgroundsetup{
%  position=current page.west,
%  angle=90,
%  nodeanchor=west,
%  vshift=-5mm,
%  opacity=1,
%  scale=4,
%  contents=Draft
%}

%30,144,255

%\definecolor{darkblue}{rgb}{0.12,0.47,0.87}
%\definecolor{darkblue}{rgb}{0.07,0.34,0.6}
\definecolor{bleu}{RGB}{30,144,255}
\definecolor{orange}{RGB}{255,200,0}%{255,77,0}
\definecolor{vert}{RGB}{60,180,0}

\titleformat{\chapter}[display] {\fontsize{24pt}{12pt}\selectfont \bfseries}{\textcolor{bleu} {#1}\\{\color{bleu}\hfil\rule{\dimexpr\textwidth}{1.2pt}}}{20pt}{\vspace{-2cm} \Huge}
%{\hfil\rule{\dimexpr\textwidth}{1.2pt}}}
\titleformat{name=\chapter,numberless}[display] {\fontsize{24pt}{12pt}\selectfont \bfseries}{\textcolor{bleu} {#1}}{20pt}{\Huge}


% SPECIAL COMMANDS  
\makeatletter
%\renewcommand\@biblabel[1]{\textbf{#1.}} % Change the square brackets for each bibliography item from '[1]' to '1.'
\renewcommand{\@listI}{\itemsep=0pt} % Reduce the space between items in the itemize and enumerate environments and the bibliography

% TITLE

\renewcommand{\maketitle}{ % Customize the title - do not edit title and author name here, see the TITLE block below
\begin{flushleft} % Right align
{\LARGE\@title} % Increase the font size of the title

\vspace{50pt} % Some vertical space between the title and author name

{\large\@author} % Author name
\\\@date % Date

\vspace{40pt} % Some vertical space between the author block and abstract
\end{flushleft}
}
\newcommand{\vu}[1]{#1}

%
% PAGE STYLE
%
\setlength{\textwidth}{16cm}
\setlength{\textheight}{23cm}
\oddsidemargin +0.5cm
\evensidemargin +0.5cm
\topmargin -0.3cm

%\setcounter{secnumdepth}{1}
\setcounter{tocdepth}{1}

\begin{document}

%----------------------------------------------------------------------------------------
%	TITLE
%----------------------------------------------------------------------------------------
\title{\parbox{16cm}{
    %\includegraphics[angle=-90, width=4cm]{logo-uga-vo-cmjn.pdf} \\ [3cm]   
    \textit{Pi\`eces compl\'ementaires au dossier LLSH} \\ [5cm]   
    \begin{center}\sf\bfseries\huge
      Candidature \`a l'habilitation \`a diriger des recherches \\ [-4mm]
      \rule{16cm}{1pt}\\ [2cm]   
    \end{center}
    \begin{center} 
      \sf \bfseries {\textcolor{bleu}{Laurence Perotto}} \\ [6cm]     
    \end{center}
  }
}
\author{\parbox{16cm}{
    %\includegraphics[width=4cm]{LPSC_logo.pdf}
    {\sf \Large Laboratoire de Physique Subatomique et Cosmologie} \\ [1cm]
    Univ. Grenoble Alpes, CNRS, Grenoble INP, LPSC-IN2P3, 53, avenue des Martyrs, F-38000 Grenoble, France
  }
}
\date{\today}

\maketitle % Print the title section
%
%
%
\tableofcontents

\pagenumbering{roman}
\pagenumbering{arabic}

%----------------------------------------------------------------------------------------
%
%   CV
%
% Un curriculum vitae faisant apparaitre la chronologie des activités
% dans les domaines de l'enseignement, de la recherche, et de
% l’administration / évaluation de la recherche
%----------------------------------------------------------------------------------------
\chapter{{\color[RGB]{30,144,255} Curriculum vitae}}% {\color{blue} Nico}}
%\label{se:intro}
\input{CV.tex}

%----------------------------------------------------------------------------------------
%
%   Lettre de motivation
%
% Une lettre de motivation (présentation du projet et du rôle de l'HDR dans ce projet)
%----------------------------------------------------------------------------------------
\chapter{{\color[RGB]{30,144,255} Lettre de motivation}}% {\color{blue} Nico}}
%\label{se:intro}
\input{lettre_motiv.tex}

% Si vous avez un.e garant.e, une lettre de soutien de sa part.

%----------------------------------------------------------------------------------------
%
%  Copie du diplome de Doctorat
%
%  La copie du diplome de Doctorat et de la qualification par le CNU, le cas échéant.
%----------------------------------------------------------------------------------------
\chapter{{\color[RGB]{30,144,255} Dipl\^omes}}
\begin{figure}[ht]
\begin{center}
\includegraphics[angle=90, clip=true, trim={0.8cm, 0, 0, 0.8cm},
        width=\textwidth]{Doctorat.pdf}
\caption{Duplicata du diplôme de doctorat}
\end{center}
\end{figure}


%----------------------------------------------------------------------------------------
%
%  Enseignement
%
% Un tableau des principaux enseignements dispensés indiquant la
% nature, le niveau, le volume et les modalités (enseignements
% magistraux, TD-cours, etc.). Pour les enseignements dispensés à
% l'étranger, indiquer le niveau d'équivalence dans le cursus
% français.
%----------------------------------------------------------------------------------------
\chapter{{\color[RGB]{30,144,255} Enseignements}}
\input{Table_enseignements.tex}

 
%----------------------------------------------------------------------------------------
%
%  Resumé travaux
%
% Un texte de synthèse des travaux académiques (2 pages maximum)
%----------------------------------------------------------------------------------------
\chapter{{\color[RGB]{30,144,255} Synthèse des travaux académiques}}% {\color{blue} Nico}}
%\label{se:intro}
\input{Synthese_travaux_.tex}

 
%----------------------------------------------------------------------------------------
%
%  Pricipales publications
%
% Une copie de cinq publications caractérisant au mieux l’activité de recherche du candidat ou de la candidate.
%----------------------------------------------------------------------------------------
\chapter{{\color[RGB]{30,144,255} Principales publications}}
%\label{se:intro}
\input{Principales_publi.tex}


 
%----------------------------------------------------------------------------------------
%
%  Liste de publications
%
% Une liste des travaux et publications classée selon les catégories suivantes:
% * les articles et notes publiés dans des revues à comité de lecture (préciser leur classement et la classification de référence, CNRS ou autre)
% * les articles et notes non publiés (dont working papers)
% * les ouvrages individuels
% * les contributions à des ouvrages collectifs
% * les contributions (orales ou affichées) à des colloques ou congrès ayant donné lieu à des actes (préciser si conférence sur invitation ou communication retenue après sélection).
% * les textes de vulgarisation ou de valorisation
% * les rapports de fin de contrat
%----------------------------------------------------------------------------------------
\chapter{{\color[RGB]{30,144,255} Liste des publications}}% {\color{blue} Nico}}
\label{se:liste_publi}
\input{liste_articles.tex}
\input{liste_preprints.tex}
\input{liste_livres.tex}
\input{liste_conferences.tex}
\input{liste_autres.tex}

 
%----------------------------------------------------------------------------------------
%
%  Liste des projets de recherche
%
% Une liste des projets de recherche financièrement soutenus par des programmes nationaux et internationaux.
%----------------------------------------------------------------------------------------
\chapter{{\color[RGB]{30,144,255} Liste des projets de recherche}}% {\color{blue} Nico}}
%\label{se:intro}
\input{Projets_liste2.tex}


%----------------------------------------------------------------------------------------
%
%  Autres
%
% Toute autre information jugee utile par le candidat ou la candidate.
%----------------------------------------------------------------------------------------
\chapter{{\color[RGB]{30,144,255} Complément d'informations}}% {\color{blue} Nico}}
%\label{se:intro}
\input{Complement.tex}



\end{document}
