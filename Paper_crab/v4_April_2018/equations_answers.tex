\documentclass[a4paper,10pt]{article}
\usepackage{epsfig}
\usepackage{latexsym}
\usepackage{graphicx}
\usepackage{amsfonts}
\usepackage{amsmath}
\usepackage{xcolor}

%-----------------------------------------
% Pour accepter les lettres accentuees de clavier azerty
% sans les \'e (utile) pour tapper directement en azerty 
% et ou faire passer aspell -c --lang=fr bidon.tex
%\usepackage[latin1]{inputenc}
%------------------------------------------


%\topmargin=-3cm
\topmargin=-1cm
\oddsidemargin=-1cm
\evensidemargin=-1cm
\textwidth=17cm
%\textheight=27cm
\textheight=25cm
\raggedbottom
\sloppy

\definecolor{Blue}{rgb}{0.,0.,1.}
\definecolor{LightSkyBlue}{rgb}{0.691,0.827,1.}
\definecolor{Red}{rgb}{1.,0.,0.}
\definecolor{Green}{rgb}{0.,1.,0.}
\definecolor{Purple}{rgb}{0.5, 0., 0.5}
\definecolor{Try}{rgb}{0.15,0.,1}
\definecolor{Black}{rgb}{0., 0., 0.}

%To get DRAFT accross all pages
%\usepackage{draftcopy}
%To replace ``DRAFT'' by ``ON GOING''
%\draftcopyName{ON GOING}{150}

\title{Derivation of uncertainties on the degree of polarization and the angle}
\author{Ritacco et al}

\begin{document}
\maketitle

\begin{equation}
 p    = \frac{\sqrt{Q^2 + U^2}}{I} \nonumber 
\end{equation}

so

\begin{eqnarray}
\frac{\partial p}{\partial Q} &=& \frac{Q}{I\sqrt{Q^2+U^2}}, \\
\frac{\partial p}{\partial U} &=& \frac{U}{I\sqrt{Q^2+U^2}}, \\
\frac{\partial p}{\partial I} &=& -\frac{\sqrt{Q^2+U^2}}{I^2},
\end{eqnarray}

thus

\begin{eqnarray}
\sigma_p^2 &=& \left(\frac{Q}{I\sqrt{Q^2+U^2}}\right)^2\sigma_Q^2 + 
\left(\frac{U}{I\sqrt{Q^2+U^2}}\right)^2\sigma_U^2 + 
\left(\frac{\sqrt{Q^2+U^2}}{I^2}\right)^2\sigma_I^2\\
&=&\frac{(Q^2\sigma_Q^2 + U^2\sigma_U^2)I^2 +
  (Q^2+U^2)^2\sigma_I^2}{I^4(Q^2+U^2)} \\
&=&\frac{(Q^2\sigma_Q^2 + U^2\sigma_U^2)I^2 +
  (p^2I^2)^2\sigma_I^2}{I^4(Q^2+U^2)} \\
&=&\frac{(Q^2\sigma_Q^2+U^2\sigma_U^2)+p^4I^2\sigma_I^2}{p^2I^4}
\end{eqnarray}

hence

\begin{equation}
\sigma_p = \frac{\sqrt{Q^2\sigma_Q^2+U^2\sigma_U^2+p^4I^2\sigma_I^2}}{pI^2}.
\end{equation}

As far as the angle is concerned:

\begin{eqnarray}
\psi & = & \frac{1}{2}\arctan{\frac{U}{Q}}\\
\frac{\partial \psi}{\partial Q} &=& \frac{1}{2}\frac{-U}{Q^2+U^2} \\
\frac{\partial \psi}{\partial U} &=& \frac{1}{2}\frac{Q}{Q^2+U^2}
\end{eqnarray}

\begin{eqnarray}
\sigma_\psi^2 &=& \frac{1}{4}\frac{U^2\sigma_Q^2+Q^2\sigma_U^2}{(Q^2+U^2)^2}\\
&=&\frac{1}{4}\frac{U^2\sigma_Q^2+Q^2\sigma_U^2}{p^4I^4}
\end{eqnarray}

so

\begin{equation}
\sigma_\psi = \frac{1}{2}\frac{\sqrt{U^2\sigma_Q^2+Q^2\sigma_U^2}}{(pI)^2}
\end{equation}
\end{document}
