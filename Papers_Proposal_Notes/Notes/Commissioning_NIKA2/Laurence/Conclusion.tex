%
%\chapter*{Conclusion et perspectives dans l'expérience \emph{Euclid}}
%\label{se:conclu}
%%
%\chapter*{Conclusion et perspectives dans l'expérience \emph{Euclid}}
%\label{se:conclu}
%%
%\chapter*{Conclusion et perspectives dans l'expérience \emph{Euclid}}
%\label{se:conclu}
%%
%\chapter*{Conclusion et perspectives dans l'expérience \emph{Euclid}}
%\label{se:conclu}
%\input{Conclusion.tex}
%
NIKA2 installée au télescope de 30 mètres de l'IRAM, est une expérience ouverte à la communauté scientifique depuis octobre 2017, et restera une instrument à demeure du télescope de 30-m pour la décennie à venir.

Nous avons présenté l'instrument en choisissant de montrer comment les choix instrumentaux se sont fondés sur l'expérience acquise avec l'instrument précurseur NIKA, et ont répondu à l'exigence d'exploiter au mieux les capacités offertes par le télescope de 30-m, en terme de champ de vue et de résolution angulaire. 

Entre l'installation de l'instrument au télescope de 30-m en octobre 2015 et la campagne de vérification scientifique d'avril 2017, s'est déroulée une intense phase de \emph{commissioning} dont les étapes clef ont été détaillées. 

Nous avons présenté les performances de NIKA2 telles qu'elles ont été évaluées à partir d'un vaste échantillon de données représentatif des conditions d'observation qui prévalent au télescope de 30-m. Une méthode de calibration de référence a été développée pour fournir des cartes de densité de flux, exploitable scientifiquement, à partir de la combinaison des données brutes issues de NIKA2 et du télescope. Cette méthode réalise la reconstruction du plan focal, la caractérisation des lobes, la correction de l'opacité atmosphérique, la calibration absolue et relative et enfin, la projection des signaux corrigés du bruit corrélé sur une carte. Plusieurs de ces étapes s'appuient sur des développements originaux nécessaires pour prendre en compte les spécificités de la technologie de détection utilisant les KIDs. \`A chaque étape, la robustesse de la méthode de référence aux effets systématiques a été validée par comparaison des résultats issus de jeux de données indépendants et de méthodes alternatives.

Les principales caractéristiques expérimentales de NIKA2 ont été résumées. NIKA2 est capable de cartographier un grand champ de vue de 6.5' de diamètre, simultanément dans deux bandes de fréquence centrées autour de 150 et 260\,GHz, avec une résolution angulaire meilleure que 18''. Une incertitude de calibration RMS, estimée à partir de la dispersion des flux de sources de calibration compactes, d'environ 3\% à 150\,GHz et 6\% à 260\,GHz, démontre une stabilité exceptionnelle de l'instrument et une grande précision des méthodes d'analyse déployées. Finalement, NIKA2 est une expérience à très haute sensibilité, en particulier dans le canal d'observation à 150\,GHz, pour lequel l'objectif de performance vers lequel tendre à terme est déjà atteint. La sensibilité de NIKA2 dans le canal à 260\,GHz, dont les tests en laboratoire promettait d'être plus spectaculaire encore, est limitée par l'effet d'absorption de la lumière par un élément optique (la lame dichroïque) pour atteindre un niveau simplement acceptable. Prises dans leur ensemble, ces performances confèrent à NIKA2 une vitesse de cartographie exceptionnelle. Elles en font l'un des meilleurs instruments millimétrique à haute résolution angulaire de sa génération. 

Des perspectives pour améliorer encore ces performances sont déjà envisagées et ont été discutées. Elles concernent d'une part l'instrument lui-même,et en particulier son dispositif optique, et d'autre part l'analyse des données, en particulier, la soustraction du bruit corrélé. 

NIKA2, forte d'une combinaison de performances unique, est d'ores et déjà en mesure de fournir des observations sans précédent et critiques pour progresser dans de nombreux domaines de l'astrophysique et la cosmologie. 

C'est actuellement le meilleur instrument millimétrique pour cartographier à haute résolution angulaire les amas de galaxies jusqu'à de hauts redshifts via l'effet Sunyaev-Zel'dovich. Avec un champ de vue qui englobe le lobe de \emph{Planck} et une résolution angulaire 17 fois meilleure, NIKA2 révèle le milieu intra-amas depuis le c\oe ur jusqu'en périphérie des amas. 

La cartographie à haute résolution angulaire via l'effet SZ dans un échantillon représentatif d'amas de galaxies mobilise un grand programme d'observation bénéficiant de 300 heures de temps garanti au télescope de 30-m de l'IRAM. L'échantillon se compose de 50 amas détectés par \emph{Planck} et ACT dans la gamme de redshift de 0.5 à 0.9 et couvrant un ordre de grandeur en masse. Ce grand programme, le \emph{Large Program} SZ ou LP-SZ, que je co-dirige, a pour finalité la cosmologie avec les amas de galaxies. L'objectif est d'améliorer les outils nécessaires pour dériver des contraintes cosmologiques avec les amas de galaxies. 

Le riche domaine de la cosmologie avec les amas de galaxie a été brièvement présenté. La fonction de masse et les propriétés statistiques de la distribution spatiale des amas de galaxies sont de puissantes sondes cosmologiques. 
\`A la fois sensibles à la croissance des structures et à l'expansion de l'univers, elles contraignent directement les paramètres du modèle $\Lambda$CDM minimal ($\sigma_8$ et $\Omega_{\rm{m}}$), ainsi que les extensions de ce modèle (\emph{e.g.} $\Sigma m_\nu$, $w$, $\gamma$). 
Les amas sont de façon inhérente, multi-sonde et sont actuellement bien détectés via leur luminosité X, leur richesse et leur effet de lentille sur les galaxies d'arrière-plan en optique et NIR et via l'effet tSZ. Chacune de ces sondes est affectées par des effets systématiques qui limitent aujourd'hui leur utilité pour la cosmologie. Un facteur limitant fondamental est la précision et la fiabilité avec laquelle peut être mesurée leur masse., en particulier pour les amas distants. Deux approches fondent les estimations de la masse : la reconstruction des effets de lentille gravitationnelle et la mesure de loi d'échelle reliant l'observable à la masse. La première méthode est affectée par des effets systématiques, tels la déviation à la sphéricité des amas (triaxialité) ou la contamination des sources. Les lois d'échelle disponibles actuellement sont également susceptibles d'être peu robustes, en particulier aux effets d'évolution avec le redshift, car calibrées à partir d'un faible nombre d'amas proches. 

La cartographie à haute résolution angulaire des amas via l'effet tSZ est une approche particulièrement puissante pour étendre les mesures de masse aux amas à haut redshift. Combinées à des cartes de la luminosité X, dont est dérivé un profil de densité de l'ICM, les cartes tSZ à résolution angulaire comparable fournissent une mesure du profil de pression de l'ICM, et sont ainsi utilisées pour reconstruire la masse hydrostatique. 

le grand programme SZ de NIKA2 est pensé pour réaliser ce projet. En combinaison avec les données de \emph{XMM-Newton}, nous mesurerons la loi d'échelle entre l'observable tSZ, $Y_{500}$, et la masse et testerons les effets d'évolution jusqu'à des redshifts de 0.9. Nous mesurerons également le profil de pression moyen via une observable directement sensible à la pression de l'ICM. Ce sont là, les deux outils nécessaires à l'exploitation de l'effet tSZ pour la cosmologie. 

Nous avons présenté les premiers résultats SZ de NIKA2. Notre analyse s'appuie sur les études pilotes réalisées avec le précurseur NIKA. Plusieurs résultats exceptionnels avaient été alors obtenus dans le cadre de ces études, telle la première carte de la température de l'ICM reconstruite sans données spectroscopiques en X et la première carte résolue de l'effet SZ cinétique. NIKA2 étant une version améliorée d'un ordre de grandeur du précurseur NIKA, ces résultats sont prometteurs quant aux réalisation à venir avec NIKA2. Le premier amas de galaxies observé par NIKA2 a fait l'objet d'une analyse combinée avec les données de \emph{XMM-Newton}, mettant au jour une première indication de l'impact de l'état thermodynamique des amas sur la dispersion de la relation masse-observable. Une autre force du LP-SZ est d'inclure des spécialistes des simulations numériques hydrodynamiques, qui nous permettra de valider nos méthodes, de tester la signifiance des éventuels biais mesurés et de comprendre leur origine physique. 

Nous avons détaillé le programme de NIKA2 en cosmologie et proposé plusieurs projets qui pourraient faire l'objet de futurs programmes sur demande de temps ouvert ou d'extensions du LP-SZ en temps garanti. L'objectif premier est la publication et la livraison à la communauté scientifique des données et résultats d'intérêt cosmologique. Cet objectif requiert une observation des amas garantissant un rapport signal sur bruit du paramètre de Compton qui soit homogène pour tout l'échantillon, l'analyse de chaque amas avec une méthode standard, préservant au mieux les plus grandes échelles angulaires accessibles à NIKA2, tout en soustrayant le bruit corrélé et les contaminants d'origines astrophysiques, et l'estimation des incertitudes statistiques et systématiques sur les produits finaux, qui sont la relation masse-observable et le profil de pression normalisé moyen, via des simulations numériques réalistes. Une première série de résultats à mi-parcours, fondés sur un demi-échantillon du LP-SZ pourrait être publiée dès 2021, avec une méthode d'analyse dérivée de la méthode actuelle. Ensuite, la publication des résultats finaux, utilisant l'échantillon complet et une méthode d'analyse améliorant la reconstruction des grandes échelles angulaires, pourrait constituer l'objectif d'une thèse commençant à l'automne 2021.

Le programme c\oe ur du LP-SZ repose sur la combinaison des données SZ de NIKA2 et X, en particulier de \emph{XMM-Newton}. En parallèle ou à plus long terme, des études multi-sondes complémentaires sont d'ores et déjà envisagées. Trois de ces études ont été décrites. Premièrement, un suivi optique de l'échantillon du LP-SZ par un observatoire équipé d'imageurs et de spectroscopes, telles les expériences installées au GTC à La Palma, permettrait d'explorer la relation entre richesse, masse dynamique et masse hydrostatique des amas et de progresser sur notre compréhension de l'évolution des galaxies dans le milieu intra-amas. Un tel programme aurait d'importantes répercussions pour la cosmologie avec les amas détectés par les futurs grands relevés optiques (VRO-LSST, \emph{Euclid}). Deuxièmement, une analyse multi-sondes incluant les données du LP-SZ et les données des relevés optiques existants ou à venir (HST, HSC, DES), permettrait de mesurer le biais hydrostatique et d'estimer la contribution des processus non-thermiques par comparaison de la masse hydrostatique et la masse reconstruite par effet de lentille, avec là encore, d'importantes implications pour l'utilisation des amas des futurs grands relevés optiques et X en cosmologie. Finalement, la complémentarité des deux grands observatoires de l'IRAM, le télescope de 30-m  et NOEMA, pourrait être exploitée pour fournir une vision globale des amas de galaxies : à la cartographie tSZ avec NIKA2 au télescope de 30-m, couvrant des échelles angulaire allant de la dizaine de seconde d'arc à la dizaine de minutes d'arc, nous adjoindrons une cartographie tSZ avec NOEMA, pour explorer le c\oe ur des amas à une résolution angulaire pouvant aller jusqu'à quelques dixièmes de secondes d'arc. Une telle étude, qui nous permettrait de quantifier l'impact des processus non-baryoniques au c\oe ur des amas, serait très précieuse, en particulier en vue des futurs relevés en X (\emph{e. g. } eROSITA). De manière générale, une analyse multi-sonde de l'échantillon d'amas de NIKA2 constitue une approche puissante pour étudier les effets systématiques qui affectent les différentes méthodes pour inférer la masse des amas de galaxies. 

Dans le contexte des futurs expériences incluant un fort programme en cosmologie avec les amas de galaxies, plusieurs extensions du LP-SZ sont prometteuses.
%
Dans le domaine du CMB, deux apports ont été identifiés. D'abord, un suivi d'un échantillon d'amas détectés par les prochaines expériences CMB au sol (\emph{Advanced} ACTpol, \emph{Simons Array}, \emph{Simons Observatory}, pour ne citer que celles accessibles à un suivi avec le télescope de 30-m de l'IRAM) permettrait d'étendre à plus basse masse la calibration de la relation masse-observable et du profil de pression moyen. Par ailleurs, NIKA2 pourrait obtenir la première reconstruction de l'effet de lentille sur le CMB par un amas individuel, ouvrant une nouvelle voie pour la mesure de la masse des amas étendant les méthodes cumulatives récemment utilisées. 
%
Dans le domaine X, NIKA2 offre l'opportunité d'effectuer un suivi en SZ des amas détectés par les futurs relevés (\emph{e. g. } eROSITA) avec une résolution angulaire comparable. Ainsi, des méthodes de mesure de la masse hydrostatique ou de la température de l'ICM sans données spectroscopiques pourraient être déployées, afin d'étendre la calibration de la masse à plus hauts redshifts.
%
Enfin, les apports de la cartographie SZ à haute-résolution angulaire sont également nombreux en optique/NIR. Par exemple, un suivi avec NIKA2 des amas détectés via l'effet de lentille gravitationnelle fort permettrait à la fois de mieux contrôler les effets systématiques inhérents à cette sonde, mais aussi d'améliorer notre connaissance de la population de galaxies à très hauts redshifts. Par ailleurs, un suivi des amas détectés via leur richesse dans les futurs relevés (VRO-LSST, \emph{Euclid}) serait utile à la calibration de la relation masse-observable en particulier à des redshifts pour lesquels une calibration en interne, fondée sur l'effet de lentille ou les dispersions de vitesses, devient peu fiable. \emph{Euclid} devrait détecter quelques $10^{4}$ amas à $z>1$, et ce jusqu'à $z\gtrsim 2$. Un suivi avec NIKA2 d'un échantillon d'amas de ce domaine de redshift peu exploré serait très souhaitable.        
   

La décennie qui s'ouvre s'annonce enthousiasmante pour la cosmologie avec les amas de galaxies. 
%
Elle verra émerger un foisonnement de données, avec la publication des données des relevés actuels et l'avènement des futurs grands relevés mis en service au début des années 2020, que ce soit en X (eROSITA), en optique (HSC, DES, KIDS, VRO-LSST, \emph{Euclid}) ou dans le domaine CMB via l'effet SZ (SPT-3G, \emph{Advanced} ACTPol, \emph{Simons Array}, \emph{Simons Observatory} puis CMB-S4).
%
Les avancées majeures concernent autant la précision des contraintes cosmologiques, liée à une augmentation de deux ordres de grandeurs du nombre d'amas détectés, que leur fiabilité, liée au contrôle accru des effets systématiques grâce aux comparaisons et corrélations croisées entre sondes des amas.
%
Dans ce paysage, NIKA2, via le LP-SZ d'abord puis d'éventuelles futures extensions, aura un apport critique, en particulier pour la calibration de la masse des amas à hauts redshifts, nécessaire pour exploiter à plein le potentiel scientifique de \emph{e. g. Euclid}.
%     
La cosmologie avec les amas de galaxies, s'appuyant sur des analyses multi-sondes fondées sur de vastes jeux de données, contribuera à forger la vison globale et cohérente de la formation des structures et de l'évolution de l'Univers, décisive pour contraindre les extensions au modèle $\Lambda$CDM minimal.  



             
     
       



%
NIKA2 installée au télescope de 30 mètres de l'IRAM, est une expérience ouverte à la communauté scientifique depuis octobre 2017, et restera une instrument à demeure du télescope de 30-m pour la décennie à venir.

Nous avons présenté l'instrument en choisissant de montrer comment les choix instrumentaux se sont fondés sur l'expérience acquise avec l'instrument précurseur NIKA, et ont répondu à l'exigence d'exploiter au mieux les capacités offertes par le télescope de 30-m, en terme de champ de vue et de résolution angulaire. 

Entre l'installation de l'instrument au télescope de 30-m en octobre 2015 et la campagne de vérification scientifique d'avril 2017, s'est déroulée une intense phase de \emph{commissioning} dont les étapes clef ont été détaillées. 

Nous avons présenté les performances de NIKA2 telles qu'elles ont été évaluées à partir d'un vaste échantillon de données représentatif des conditions d'observation qui prévalent au télescope de 30-m. Une méthode de calibration de référence a été développée pour fournir des cartes de densité de flux, exploitable scientifiquement, à partir de la combinaison des données brutes issues de NIKA2 et du télescope. Cette méthode réalise la reconstruction du plan focal, la caractérisation des lobes, la correction de l'opacité atmosphérique, la calibration absolue et relative et enfin, la projection des signaux corrigés du bruit corrélé sur une carte. Plusieurs de ces étapes s'appuient sur des développements originaux nécessaires pour prendre en compte les spécificités de la technologie de détection utilisant les KIDs. \`A chaque étape, la robustesse de la méthode de référence aux effets systématiques a été validée par comparaison des résultats issus de jeux de données indépendants et de méthodes alternatives.

Les principales caractéristiques expérimentales de NIKA2 ont été résumées. NIKA2 est capable de cartographier un grand champ de vue de 6.5' de diamètre, simultanément dans deux bandes de fréquence centrées autour de 150 et 260\,GHz, avec une résolution angulaire meilleure que 18''. Une incertitude de calibration RMS, estimée à partir de la dispersion des flux de sources de calibration compactes, d'environ 3\% à 150\,GHz et 6\% à 260\,GHz, démontre une stabilité exceptionnelle de l'instrument et une grande précision des méthodes d'analyse déployées. Finalement, NIKA2 est une expérience à très haute sensibilité, en particulier dans le canal d'observation à 150\,GHz, pour lequel l'objectif de performance vers lequel tendre à terme est déjà atteint. La sensibilité de NIKA2 dans le canal à 260\,GHz, dont les tests en laboratoire promettait d'être plus spectaculaire encore, est limitée par l'effet d'absorption de la lumière par un élément optique (la lame dichroïque) pour atteindre un niveau simplement acceptable. Prises dans leur ensemble, ces performances confèrent à NIKA2 une vitesse de cartographie exceptionnelle. Elles en font l'un des meilleurs instruments millimétrique à haute résolution angulaire de sa génération. 

Des perspectives pour améliorer encore ces performances sont déjà envisagées et ont été discutées. Elles concernent d'une part l'instrument lui-même,et en particulier son dispositif optique, et d'autre part l'analyse des données, en particulier, la soustraction du bruit corrélé. 

NIKA2, forte d'une combinaison de performances unique, est d'ores et déjà en mesure de fournir des observations sans précédent et critiques pour progresser dans de nombreux domaines de l'astrophysique et la cosmologie. 

C'est actuellement le meilleur instrument millimétrique pour cartographier à haute résolution angulaire les amas de galaxies jusqu'à de hauts redshifts via l'effet Sunyaev-Zel'dovich. Avec un champ de vue qui englobe le lobe de \emph{Planck} et une résolution angulaire 17 fois meilleure, NIKA2 révèle le milieu intra-amas depuis le c\oe ur jusqu'en périphérie des amas. 

La cartographie à haute résolution angulaire via l'effet SZ dans un échantillon représentatif d'amas de galaxies mobilise un grand programme d'observation bénéficiant de 300 heures de temps garanti au télescope de 30-m de l'IRAM. L'échantillon se compose de 50 amas détectés par \emph{Planck} et ACT dans la gamme de redshift de 0.5 à 0.9 et couvrant un ordre de grandeur en masse. Ce grand programme, le \emph{Large Program} SZ ou LP-SZ, que je co-dirige, a pour finalité la cosmologie avec les amas de galaxies. L'objectif est d'améliorer les outils nécessaires pour dériver des contraintes cosmologiques avec les amas de galaxies. 

Le riche domaine de la cosmologie avec les amas de galaxie a été brièvement présenté. La fonction de masse et les propriétés statistiques de la distribution spatiale des amas de galaxies sont de puissantes sondes cosmologiques. 
\`A la fois sensibles à la croissance des structures et à l'expansion de l'univers, elles contraignent directement les paramètres du modèle $\Lambda$CDM minimal ($\sigma_8$ et $\Omega_{\rm{m}}$), ainsi que les extensions de ce modèle (\emph{e.g.} $\Sigma m_\nu$, $w$, $\gamma$). 
Les amas sont de façon inhérente, multi-sonde et sont actuellement bien détectés via leur luminosité X, leur richesse et leur effet de lentille sur les galaxies d'arrière-plan en optique et NIR et via l'effet tSZ. Chacune de ces sondes est affectées par des effets systématiques qui limitent aujourd'hui leur utilité pour la cosmologie. Un facteur limitant fondamental est la précision et la fiabilité avec laquelle peut être mesurée leur masse., en particulier pour les amas distants. Deux approches fondent les estimations de la masse : la reconstruction des effets de lentille gravitationnelle et la mesure de loi d'échelle reliant l'observable à la masse. La première méthode est affectée par des effets systématiques, tels la déviation à la sphéricité des amas (triaxialité) ou la contamination des sources. Les lois d'échelle disponibles actuellement sont également susceptibles d'être peu robustes, en particulier aux effets d'évolution avec le redshift, car calibrées à partir d'un faible nombre d'amas proches. 

La cartographie à haute résolution angulaire des amas via l'effet tSZ est une approche particulièrement puissante pour étendre les mesures de masse aux amas à haut redshift. Combinées à des cartes de la luminosité X, dont est dérivé un profil de densité de l'ICM, les cartes tSZ à résolution angulaire comparable fournissent une mesure du profil de pression de l'ICM, et sont ainsi utilisées pour reconstruire la masse hydrostatique. 

le grand programme SZ de NIKA2 est pensé pour réaliser ce projet. En combinaison avec les données de \emph{XMM-Newton}, nous mesurerons la loi d'échelle entre l'observable tSZ, $Y_{500}$, et la masse et testerons les effets d'évolution jusqu'à des redshifts de 0.9. Nous mesurerons également le profil de pression moyen via une observable directement sensible à la pression de l'ICM. Ce sont là, les deux outils nécessaires à l'exploitation de l'effet tSZ pour la cosmologie. 

Nous avons présenté les premiers résultats SZ de NIKA2. Notre analyse s'appuie sur les études pilotes réalisées avec le précurseur NIKA. Plusieurs résultats exceptionnels avaient été alors obtenus dans le cadre de ces études, telle la première carte de la température de l'ICM reconstruite sans données spectroscopiques en X et la première carte résolue de l'effet SZ cinétique. NIKA2 étant une version améliorée d'un ordre de grandeur du précurseur NIKA, ces résultats sont prometteurs quant aux réalisation à venir avec NIKA2. Le premier amas de galaxies observé par NIKA2 a fait l'objet d'une analyse combinée avec les données de \emph{XMM-Newton}, mettant au jour une première indication de l'impact de l'état thermodynamique des amas sur la dispersion de la relation masse-observable. Une autre force du LP-SZ est d'inclure des spécialistes des simulations numériques hydrodynamiques, qui nous permettra de valider nos méthodes, de tester la signifiance des éventuels biais mesurés et de comprendre leur origine physique. 

Nous avons détaillé le programme de NIKA2 en cosmologie et proposé plusieurs projets qui pourraient faire l'objet de futurs programmes sur demande de temps ouvert ou d'extensions du LP-SZ en temps garanti. L'objectif premier est la publication et la livraison à la communauté scientifique des données et résultats d'intérêt cosmologique. Cet objectif requiert une observation des amas garantissant un rapport signal sur bruit du paramètre de Compton qui soit homogène pour tout l'échantillon, l'analyse de chaque amas avec une méthode standard, préservant au mieux les plus grandes échelles angulaires accessibles à NIKA2, tout en soustrayant le bruit corrélé et les contaminants d'origines astrophysiques, et l'estimation des incertitudes statistiques et systématiques sur les produits finaux, qui sont la relation masse-observable et le profil de pression normalisé moyen, via des simulations numériques réalistes. Une première série de résultats à mi-parcours, fondés sur un demi-échantillon du LP-SZ pourrait être publiée dès 2021, avec une méthode d'analyse dérivée de la méthode actuelle. Ensuite, la publication des résultats finaux, utilisant l'échantillon complet et une méthode d'analyse améliorant la reconstruction des grandes échelles angulaires, pourrait constituer l'objectif d'une thèse commençant à l'automne 2021.

Le programme c\oe ur du LP-SZ repose sur la combinaison des données SZ de NIKA2 et X, en particulier de \emph{XMM-Newton}. En parallèle ou à plus long terme, des études multi-sondes complémentaires sont d'ores et déjà envisagées. Trois de ces études ont été décrites. Premièrement, un suivi optique de l'échantillon du LP-SZ par un observatoire équipé d'imageurs et de spectroscopes, telles les expériences installées au GTC à La Palma, permettrait d'explorer la relation entre richesse, masse dynamique et masse hydrostatique des amas et de progresser sur notre compréhension de l'évolution des galaxies dans le milieu intra-amas. Un tel programme aurait d'importantes répercussions pour la cosmologie avec les amas détectés par les futurs grands relevés optiques (VRO-LSST, \emph{Euclid}). Deuxièmement, une analyse multi-sondes incluant les données du LP-SZ et les données des relevés optiques existants ou à venir (HST, HSC, DES), permettrait de mesurer le biais hydrostatique et d'estimer la contribution des processus non-thermiques par comparaison de la masse hydrostatique et la masse reconstruite par effet de lentille, avec là encore, d'importantes implications pour l'utilisation des amas des futurs grands relevés optiques et X en cosmologie. Finalement, la complémentarité des deux grands observatoires de l'IRAM, le télescope de 30-m  et NOEMA, pourrait être exploitée pour fournir une vision globale des amas de galaxies : à la cartographie tSZ avec NIKA2 au télescope de 30-m, couvrant des échelles angulaire allant de la dizaine de seconde d'arc à la dizaine de minutes d'arc, nous adjoindrons une cartographie tSZ avec NOEMA, pour explorer le c\oe ur des amas à une résolution angulaire pouvant aller jusqu'à quelques dixièmes de secondes d'arc. Une telle étude, qui nous permettrait de quantifier l'impact des processus non-baryoniques au c\oe ur des amas, serait très précieuse, en particulier en vue des futurs relevés en X (\emph{e. g. } eROSITA). De manière générale, une analyse multi-sonde de l'échantillon d'amas de NIKA2 constitue une approche puissante pour étudier les effets systématiques qui affectent les différentes méthodes pour inférer la masse des amas de galaxies. 

Dans le contexte des futurs expériences incluant un fort programme en cosmologie avec les amas de galaxies, plusieurs extensions du LP-SZ sont prometteuses.
%
Dans le domaine du CMB, deux apports ont été identifiés. D'abord, un suivi d'un échantillon d'amas détectés par les prochaines expériences CMB au sol (\emph{Advanced} ACTpol, \emph{Simons Array}, \emph{Simons Observatory}, pour ne citer que celles accessibles à un suivi avec le télescope de 30-m de l'IRAM) permettrait d'étendre à plus basse masse la calibration de la relation masse-observable et du profil de pression moyen. Par ailleurs, NIKA2 pourrait obtenir la première reconstruction de l'effet de lentille sur le CMB par un amas individuel, ouvrant une nouvelle voie pour la mesure de la masse des amas étendant les méthodes cumulatives récemment utilisées. 
%
Dans le domaine X, NIKA2 offre l'opportunité d'effectuer un suivi en SZ des amas détectés par les futurs relevés (\emph{e. g. } eROSITA) avec une résolution angulaire comparable. Ainsi, des méthodes de mesure de la masse hydrostatique ou de la température de l'ICM sans données spectroscopiques pourraient être déployées, afin d'étendre la calibration de la masse à plus hauts redshifts.
%
Enfin, les apports de la cartographie SZ à haute-résolution angulaire sont également nombreux en optique/NIR. Par exemple, un suivi avec NIKA2 des amas détectés via l'effet de lentille gravitationnelle fort permettrait à la fois de mieux contrôler les effets systématiques inhérents à cette sonde, mais aussi d'améliorer notre connaissance de la population de galaxies à très hauts redshifts. Par ailleurs, un suivi des amas détectés via leur richesse dans les futurs relevés (VRO-LSST, \emph{Euclid}) serait utile à la calibration de la relation masse-observable en particulier à des redshifts pour lesquels une calibration en interne, fondée sur l'effet de lentille ou les dispersions de vitesses, devient peu fiable. \emph{Euclid} devrait détecter quelques $10^{4}$ amas à $z>1$, et ce jusqu'à $z\gtrsim 2$. Un suivi avec NIKA2 d'un échantillon d'amas de ce domaine de redshift peu exploré serait très souhaitable.        
   

La décennie qui s'ouvre s'annonce enthousiasmante pour la cosmologie avec les amas de galaxies. 
%
Elle verra émerger un foisonnement de données, avec la publication des données des relevés actuels et l'avènement des futurs grands relevés mis en service au début des années 2020, que ce soit en X (eROSITA), en optique (HSC, DES, KIDS, VRO-LSST, \emph{Euclid}) ou dans le domaine CMB via l'effet SZ (SPT-3G, \emph{Advanced} ACTPol, \emph{Simons Array}, \emph{Simons Observatory} puis CMB-S4).
%
Les avancées majeures concernent autant la précision des contraintes cosmologiques, liée à une augmentation de deux ordres de grandeurs du nombre d'amas détectés, que leur fiabilité, liée au contrôle accru des effets systématiques grâce aux comparaisons et corrélations croisées entre sondes des amas.
%
Dans ce paysage, NIKA2, via le LP-SZ d'abord puis d'éventuelles futures extensions, aura un apport critique, en particulier pour la calibration de la masse des amas à hauts redshifts, nécessaire pour exploiter à plein le potentiel scientifique de \emph{e. g. Euclid}.
%     
La cosmologie avec les amas de galaxies, s'appuyant sur des analyses multi-sondes fondées sur de vastes jeux de données, contribuera à forger la vison globale et cohérente de la formation des structures et de l'évolution de l'Univers, décisive pour contraindre les extensions au modèle $\Lambda$CDM minimal.  



             
     
       



%
NIKA2 installée au télescope de 30 mètres de l'IRAM, est une expérience ouverte à la communauté scientifique depuis octobre 2017, et restera une instrument à demeure du télescope de 30-m pour la décennie à venir.

Nous avons présenté l'instrument en choisissant de montrer comment les choix instrumentaux se sont fondés sur l'expérience acquise avec l'instrument précurseur NIKA, et ont répondu à l'exigence d'exploiter au mieux les capacités offertes par le télescope de 30-m, en terme de champ de vue et de résolution angulaire. 

Entre l'installation de l'instrument au télescope de 30-m en octobre 2015 et la campagne de vérification scientifique d'avril 2017, s'est déroulée une intense phase de \emph{commissioning} dont les étapes clef ont été détaillées. 

Nous avons présenté les performances de NIKA2 telles qu'elles ont été évaluées à partir d'un vaste échantillon de données représentatif des conditions d'observation qui prévalent au télescope de 30-m. Une méthode de calibration de référence a été développée pour fournir des cartes de densité de flux, exploitable scientifiquement, à partir de la combinaison des données brutes issues de NIKA2 et du télescope. Cette méthode réalise la reconstruction du plan focal, la caractérisation des lobes, la correction de l'opacité atmosphérique, la calibration absolue et relative et enfin, la projection des signaux corrigés du bruit corrélé sur une carte. Plusieurs de ces étapes s'appuient sur des développements originaux nécessaires pour prendre en compte les spécificités de la technologie de détection utilisant les KIDs. \`A chaque étape, la robustesse de la méthode de référence aux effets systématiques a été validée par comparaison des résultats issus de jeux de données indépendants et de méthodes alternatives.

Les principales caractéristiques expérimentales de NIKA2 ont été résumées. NIKA2 est capable de cartographier un grand champ de vue de 6.5' de diamètre, simultanément dans deux bandes de fréquence centrées autour de 150 et 260\,GHz, avec une résolution angulaire meilleure que 18''. Une incertitude de calibration RMS, estimée à partir de la dispersion des flux de sources de calibration compactes, d'environ 3\% à 150\,GHz et 6\% à 260\,GHz, démontre une stabilité exceptionnelle de l'instrument et une grande précision des méthodes d'analyse déployées. Finalement, NIKA2 est une expérience à très haute sensibilité, en particulier dans le canal d'observation à 150\,GHz, pour lequel l'objectif de performance vers lequel tendre à terme est déjà atteint. La sensibilité de NIKA2 dans le canal à 260\,GHz, dont les tests en laboratoire promettait d'être plus spectaculaire encore, est limitée par l'effet d'absorption de la lumière par un élément optique (la lame dichroïque) pour atteindre un niveau simplement acceptable. Prises dans leur ensemble, ces performances confèrent à NIKA2 une vitesse de cartographie exceptionnelle. Elles en font l'un des meilleurs instruments millimétrique à haute résolution angulaire de sa génération. 

Des perspectives pour améliorer encore ces performances sont déjà envisagées et ont été discutées. Elles concernent d'une part l'instrument lui-même,et en particulier son dispositif optique, et d'autre part l'analyse des données, en particulier, la soustraction du bruit corrélé. 

NIKA2, forte d'une combinaison de performances unique, est d'ores et déjà en mesure de fournir des observations sans précédent et critiques pour progresser dans de nombreux domaines de l'astrophysique et la cosmologie. 

C'est actuellement le meilleur instrument millimétrique pour cartographier à haute résolution angulaire les amas de galaxies jusqu'à de hauts redshifts via l'effet Sunyaev-Zel'dovich. Avec un champ de vue qui englobe le lobe de \emph{Planck} et une résolution angulaire 17 fois meilleure, NIKA2 révèle le milieu intra-amas depuis le c\oe ur jusqu'en périphérie des amas. 

La cartographie à haute résolution angulaire via l'effet SZ dans un échantillon représentatif d'amas de galaxies mobilise un grand programme d'observation bénéficiant de 300 heures de temps garanti au télescope de 30-m de l'IRAM. L'échantillon se compose de 50 amas détectés par \emph{Planck} et ACT dans la gamme de redshift de 0.5 à 0.9 et couvrant un ordre de grandeur en masse. Ce grand programme, le \emph{Large Program} SZ ou LP-SZ, que je co-dirige, a pour finalité la cosmologie avec les amas de galaxies. L'objectif est d'améliorer les outils nécessaires pour dériver des contraintes cosmologiques avec les amas de galaxies. 

Le riche domaine de la cosmologie avec les amas de galaxie a été brièvement présenté. La fonction de masse et les propriétés statistiques de la distribution spatiale des amas de galaxies sont de puissantes sondes cosmologiques. 
\`A la fois sensibles à la croissance des structures et à l'expansion de l'univers, elles contraignent directement les paramètres du modèle $\Lambda$CDM minimal ($\sigma_8$ et $\Omega_{\rm{m}}$), ainsi que les extensions de ce modèle (\emph{e.g.} $\Sigma m_\nu$, $w$, $\gamma$). 
Les amas sont de façon inhérente, multi-sonde et sont actuellement bien détectés via leur luminosité X, leur richesse et leur effet de lentille sur les galaxies d'arrière-plan en optique et NIR et via l'effet tSZ. Chacune de ces sondes est affectées par des effets systématiques qui limitent aujourd'hui leur utilité pour la cosmologie. Un facteur limitant fondamental est la précision et la fiabilité avec laquelle peut être mesurée leur masse., en particulier pour les amas distants. Deux approches fondent les estimations de la masse : la reconstruction des effets de lentille gravitationnelle et la mesure de loi d'échelle reliant l'observable à la masse. La première méthode est affectée par des effets systématiques, tels la déviation à la sphéricité des amas (triaxialité) ou la contamination des sources. Les lois d'échelle disponibles actuellement sont également susceptibles d'être peu robustes, en particulier aux effets d'évolution avec le redshift, car calibrées à partir d'un faible nombre d'amas proches. 

La cartographie à haute résolution angulaire des amas via l'effet tSZ est une approche particulièrement puissante pour étendre les mesures de masse aux amas à haut redshift. Combinées à des cartes de la luminosité X, dont est dérivé un profil de densité de l'ICM, les cartes tSZ à résolution angulaire comparable fournissent une mesure du profil de pression de l'ICM, et sont ainsi utilisées pour reconstruire la masse hydrostatique. 

le grand programme SZ de NIKA2 est pensé pour réaliser ce projet. En combinaison avec les données de \emph{XMM-Newton}, nous mesurerons la loi d'échelle entre l'observable tSZ, $Y_{500}$, et la masse et testerons les effets d'évolution jusqu'à des redshifts de 0.9. Nous mesurerons également le profil de pression moyen via une observable directement sensible à la pression de l'ICM. Ce sont là, les deux outils nécessaires à l'exploitation de l'effet tSZ pour la cosmologie. 

Nous avons présenté les premiers résultats SZ de NIKA2. Notre analyse s'appuie sur les études pilotes réalisées avec le précurseur NIKA. Plusieurs résultats exceptionnels avaient été alors obtenus dans le cadre de ces études, telle la première carte de la température de l'ICM reconstruite sans données spectroscopiques en X et la première carte résolue de l'effet SZ cinétique. NIKA2 étant une version améliorée d'un ordre de grandeur du précurseur NIKA, ces résultats sont prometteurs quant aux réalisation à venir avec NIKA2. Le premier amas de galaxies observé par NIKA2 a fait l'objet d'une analyse combinée avec les données de \emph{XMM-Newton}, mettant au jour une première indication de l'impact de l'état thermodynamique des amas sur la dispersion de la relation masse-observable. Une autre force du LP-SZ est d'inclure des spécialistes des simulations numériques hydrodynamiques, qui nous permettra de valider nos méthodes, de tester la signifiance des éventuels biais mesurés et de comprendre leur origine physique. 

Nous avons détaillé le programme de NIKA2 en cosmologie et proposé plusieurs projets qui pourraient faire l'objet de futurs programmes sur demande de temps ouvert ou d'extensions du LP-SZ en temps garanti. L'objectif premier est la publication et la livraison à la communauté scientifique des données et résultats d'intérêt cosmologique. Cet objectif requiert une observation des amas garantissant un rapport signal sur bruit du paramètre de Compton qui soit homogène pour tout l'échantillon, l'analyse de chaque amas avec une méthode standard, préservant au mieux les plus grandes échelles angulaires accessibles à NIKA2, tout en soustrayant le bruit corrélé et les contaminants d'origines astrophysiques, et l'estimation des incertitudes statistiques et systématiques sur les produits finaux, qui sont la relation masse-observable et le profil de pression normalisé moyen, via des simulations numériques réalistes. Une première série de résultats à mi-parcours, fondés sur un demi-échantillon du LP-SZ pourrait être publiée dès 2021, avec une méthode d'analyse dérivée de la méthode actuelle. Ensuite, la publication des résultats finaux, utilisant l'échantillon complet et une méthode d'analyse améliorant la reconstruction des grandes échelles angulaires, pourrait constituer l'objectif d'une thèse commençant à l'automne 2021.

Le programme c\oe ur du LP-SZ repose sur la combinaison des données SZ de NIKA2 et X, en particulier de \emph{XMM-Newton}. En parallèle ou à plus long terme, des études multi-sondes complémentaires sont d'ores et déjà envisagées. Trois de ces études ont été décrites. Premièrement, un suivi optique de l'échantillon du LP-SZ par un observatoire équipé d'imageurs et de spectroscopes, telles les expériences installées au GTC à La Palma, permettrait d'explorer la relation entre richesse, masse dynamique et masse hydrostatique des amas et de progresser sur notre compréhension de l'évolution des galaxies dans le milieu intra-amas. Un tel programme aurait d'importantes répercussions pour la cosmologie avec les amas détectés par les futurs grands relevés optiques (VRO-LSST, \emph{Euclid}). Deuxièmement, une analyse multi-sondes incluant les données du LP-SZ et les données des relevés optiques existants ou à venir (HST, HSC, DES), permettrait de mesurer le biais hydrostatique et d'estimer la contribution des processus non-thermiques par comparaison de la masse hydrostatique et la masse reconstruite par effet de lentille, avec là encore, d'importantes implications pour l'utilisation des amas des futurs grands relevés optiques et X en cosmologie. Finalement, la complémentarité des deux grands observatoires de l'IRAM, le télescope de 30-m  et NOEMA, pourrait être exploitée pour fournir une vision globale des amas de galaxies : à la cartographie tSZ avec NIKA2 au télescope de 30-m, couvrant des échelles angulaire allant de la dizaine de seconde d'arc à la dizaine de minutes d'arc, nous adjoindrons une cartographie tSZ avec NOEMA, pour explorer le c\oe ur des amas à une résolution angulaire pouvant aller jusqu'à quelques dixièmes de secondes d'arc. Une telle étude, qui nous permettrait de quantifier l'impact des processus non-baryoniques au c\oe ur des amas, serait très précieuse, en particulier en vue des futurs relevés en X (\emph{e. g. } eROSITA). De manière générale, une analyse multi-sonde de l'échantillon d'amas de NIKA2 constitue une approche puissante pour étudier les effets systématiques qui affectent les différentes méthodes pour inférer la masse des amas de galaxies. 

Dans le contexte des futurs expériences incluant un fort programme en cosmologie avec les amas de galaxies, plusieurs extensions du LP-SZ sont prometteuses.
%
Dans le domaine du CMB, deux apports ont été identifiés. D'abord, un suivi d'un échantillon d'amas détectés par les prochaines expériences CMB au sol (\emph{Advanced} ACTpol, \emph{Simons Array}, \emph{Simons Observatory}, pour ne citer que celles accessibles à un suivi avec le télescope de 30-m de l'IRAM) permettrait d'étendre à plus basse masse la calibration de la relation masse-observable et du profil de pression moyen. Par ailleurs, NIKA2 pourrait obtenir la première reconstruction de l'effet de lentille sur le CMB par un amas individuel, ouvrant une nouvelle voie pour la mesure de la masse des amas étendant les méthodes cumulatives récemment utilisées. 
%
Dans le domaine X, NIKA2 offre l'opportunité d'effectuer un suivi en SZ des amas détectés par les futurs relevés (\emph{e. g. } eROSITA) avec une résolution angulaire comparable. Ainsi, des méthodes de mesure de la masse hydrostatique ou de la température de l'ICM sans données spectroscopiques pourraient être déployées, afin d'étendre la calibration de la masse à plus hauts redshifts.
%
Enfin, les apports de la cartographie SZ à haute-résolution angulaire sont également nombreux en optique/NIR. Par exemple, un suivi avec NIKA2 des amas détectés via l'effet de lentille gravitationnelle fort permettrait à la fois de mieux contrôler les effets systématiques inhérents à cette sonde, mais aussi d'améliorer notre connaissance de la population de galaxies à très hauts redshifts. Par ailleurs, un suivi des amas détectés via leur richesse dans les futurs relevés (VRO-LSST, \emph{Euclid}) serait utile à la calibration de la relation masse-observable en particulier à des redshifts pour lesquels une calibration en interne, fondée sur l'effet de lentille ou les dispersions de vitesses, devient peu fiable. \emph{Euclid} devrait détecter quelques $10^{4}$ amas à $z>1$, et ce jusqu'à $z\gtrsim 2$. Un suivi avec NIKA2 d'un échantillon d'amas de ce domaine de redshift peu exploré serait très souhaitable.        
   

La décennie qui s'ouvre s'annonce enthousiasmante pour la cosmologie avec les amas de galaxies. 
%
Elle verra émerger un foisonnement de données, avec la publication des données des relevés actuels et l'avènement des futurs grands relevés mis en service au début des années 2020, que ce soit en X (eROSITA), en optique (HSC, DES, KIDS, VRO-LSST, \emph{Euclid}) ou dans le domaine CMB via l'effet SZ (SPT-3G, \emph{Advanced} ACTPol, \emph{Simons Array}, \emph{Simons Observatory} puis CMB-S4).
%
Les avancées majeures concernent autant la précision des contraintes cosmologiques, liée à une augmentation de deux ordres de grandeurs du nombre d'amas détectés, que leur fiabilité, liée au contrôle accru des effets systématiques grâce aux comparaisons et corrélations croisées entre sondes des amas.
%
Dans ce paysage, NIKA2, via le LP-SZ d'abord puis d'éventuelles futures extensions, aura un apport critique, en particulier pour la calibration de la masse des amas à hauts redshifts, nécessaire pour exploiter à plein le potentiel scientifique de \emph{e. g. Euclid}.
%     
La cosmologie avec les amas de galaxies, s'appuyant sur des analyses multi-sondes fondées sur de vastes jeux de données, contribuera à forger la vison globale et cohérente de la formation des structures et de l'évolution de l'Univers, décisive pour contraindre les extensions au modèle $\Lambda$CDM minimal.  



             
     
       



%
NIKA2 installée au télescope de 30 mètres de l'IRAM, est une expérience ouverte à la communauté scientifique depuis octobre 2017, et restera une instrument à demeure du télescope de 30-m pour la décennie à venir.

Nous avons présenté l'instrument en choisissant de montrer comment les choix instrumentaux se sont fondés sur l'expérience acquise avec l'instrument précurseur NIKA, et ont répondu à l'exigence d'exploiter au mieux les capacités offertes par le télescope de 30-m, en terme de champ de vue et de résolution angulaire. 

Entre l'installation de l'instrument au télescope de 30-m en octobre 2015 et la campagne de vérification scientifique d'avril 2017, s'est déroulée une intense phase de \emph{commissioning} dont les étapes clef ont été détaillées. 

Nous avons présenté les performances de NIKA2 telles qu'elles ont été évaluées à partir d'un vaste échantillon de données représentatif des conditions d'observation qui prévalent au télescope de 30-m. Une méthode de calibration de référence a été développée pour fournir des cartes de densité de flux, exploitable scientifiquement, à partir de la combinaison des données brutes issues de NIKA2 et du télescope. Cette méthode réalise la reconstruction du plan focal, la caractérisation des lobes, la correction de l'opacité atmosphérique, la calibration absolue et relative et enfin, la projection des signaux corrigés du bruit corrélé sur une carte. Plusieurs de ces étapes s'appuient sur des développements originaux nécessaires pour prendre en compte les spécificités de la technologie de détection utilisant les KIDs. \`A chaque étape, la robustesse de la méthode de référence aux effets systématiques a été validée par comparaison des résultats issus de jeux de données indépendants et de méthodes alternatives.

Les principales caractéristiques expérimentales de NIKA2 ont été résumées. NIKA2 est capable de cartographier un grand champ de vue de 6.5' de diamètre, simultanément dans deux bandes de fréquence centrées autour de 150 et 260\,GHz, avec une résolution angulaire meilleure que 18''. Une incertitude de calibration RMS, estimée à partir de la dispersion des flux de sources de calibration compactes, d'environ 3\% à 150\,GHz et 6\% à 260\,GHz, démontre une stabilité exceptionnelle de l'instrument et une grande précision des méthodes d'analyse déployées. Finalement, NIKA2 est une expérience à très haute sensibilité, en particulier dans le canal d'observation à 150\,GHz, pour lequel l'objectif de performance vers lequel tendre à terme est déjà atteint. La sensibilité de NIKA2 dans le canal à 260\,GHz, dont les tests en laboratoire promettait d'être plus spectaculaire encore, est limitée par l'effet d'absorption de la lumière par un élément optique (la lame dichroïque) pour atteindre un niveau simplement acceptable. Prises dans leur ensemble, ces performances confèrent à NIKA2 une vitesse de cartographie exceptionnelle. Elles en font l'un des meilleurs instruments millimétrique à haute résolution angulaire de sa génération. 

Des perspectives pour améliorer encore ces performances sont déjà envisagées et ont été discutées. Elles concernent d'une part l'instrument lui-même,et en particulier son dispositif optique, et d'autre part l'analyse des données, en particulier, la soustraction du bruit corrélé. 

NIKA2, forte d'une combinaison de performances unique, est d'ores et déjà en mesure de fournir des observations sans précédent et critiques pour progresser dans de nombreux domaines de l'astrophysique et la cosmologie. 

C'est actuellement le meilleur instrument millimétrique pour cartographier à haute résolution angulaire les amas de galaxies jusqu'à de hauts redshifts via l'effet Sunyaev-Zel'dovich. Avec un champ de vue qui englobe le lobe de \emph{Planck} et une résolution angulaire 17 fois meilleure, NIKA2 révèle le milieu intra-amas depuis le c\oe ur jusqu'en périphérie des amas. 

La cartographie à haute résolution angulaire via l'effet SZ dans un échantillon représentatif d'amas de galaxies mobilise un grand programme d'observation bénéficiant de 300 heures de temps garanti au télescope de 30-m de l'IRAM. L'échantillon se compose de 50 amas détectés par \emph{Planck} et ACT dans la gamme de redshift de 0.5 à 0.9 et couvrant un ordre de grandeur en masse. Ce grand programme, le \emph{Large Program} SZ ou LP-SZ, que je co-dirige, a pour finalité la cosmologie avec les amas de galaxies. L'objectif est d'améliorer les outils nécessaires pour dériver des contraintes cosmologiques avec les amas de galaxies. 

Le riche domaine de la cosmologie avec les amas de galaxie a été brièvement présenté. La fonction de masse et les propriétés statistiques de la distribution spatiale des amas de galaxies sont de puissantes sondes cosmologiques. 
\`A la fois sensibles à la croissance des structures et à l'expansion de l'univers, elles contraignent directement les paramètres du modèle $\Lambda$CDM minimal ($\sigma_8$ et $\Omega_{\rm{m}}$), ainsi que les extensions de ce modèle (\emph{e.g.} $\Sigma m_\nu$, $w$, $\gamma$). 
Les amas sont de façon inhérente, multi-sonde et sont actuellement bien détectés via leur luminosité X, leur richesse et leur effet de lentille sur les galaxies d'arrière-plan en optique et NIR et via l'effet tSZ. Chacune de ces sondes est affectées par des effets systématiques qui limitent aujourd'hui leur utilité pour la cosmologie. Un facteur limitant fondamental est la précision et la fiabilité avec laquelle peut être mesurée leur masse., en particulier pour les amas distants. Deux approches fondent les estimations de la masse : la reconstruction des effets de lentille gravitationnelle et la mesure de loi d'échelle reliant l'observable à la masse. La première méthode est affectée par des effets systématiques, tels la déviation à la sphéricité des amas (triaxialité) ou la contamination des sources. Les lois d'échelle disponibles actuellement sont également susceptibles d'être peu robustes, en particulier aux effets d'évolution avec le redshift, car calibrées à partir d'un faible nombre d'amas proches. 

La cartographie à haute résolution angulaire des amas via l'effet tSZ est une approche particulièrement puissante pour étendre les mesures de masse aux amas à haut redshift. Combinées à des cartes de la luminosité X, dont est dérivé un profil de densité de l'ICM, les cartes tSZ à résolution angulaire comparable fournissent une mesure du profil de pression de l'ICM, et sont ainsi utilisées pour reconstruire la masse hydrostatique. 

le grand programme SZ de NIKA2 est pensé pour réaliser ce projet. En combinaison avec les données de \emph{XMM-Newton}, nous mesurerons la loi d'échelle entre l'observable tSZ, $Y_{500}$, et la masse et testerons les effets d'évolution jusqu'à des redshifts de 0.9. Nous mesurerons également le profil de pression moyen via une observable directement sensible à la pression de l'ICM. Ce sont là, les deux outils nécessaires à l'exploitation de l'effet tSZ pour la cosmologie. 

Nous avons présenté les premiers résultats SZ de NIKA2. Notre analyse s'appuie sur les études pilotes réalisées avec le précurseur NIKA. Plusieurs résultats exceptionnels avaient été alors obtenus dans le cadre de ces études, telle la première carte de la température de l'ICM reconstruite sans données spectroscopiques en X et la première carte résolue de l'effet SZ cinétique. NIKA2 étant une version améliorée d'un ordre de grandeur du précurseur NIKA, ces résultats sont prometteurs quant aux réalisation à venir avec NIKA2. Le premier amas de galaxies observé par NIKA2 a fait l'objet d'une analyse combinée avec les données de \emph{XMM-Newton}, mettant au jour une première indication de l'impact de l'état thermodynamique des amas sur la dispersion de la relation masse-observable. Une autre force du LP-SZ est d'inclure des spécialistes des simulations numériques hydrodynamiques, qui nous permettra de valider nos méthodes, de tester la signifiance des éventuels biais mesurés et de comprendre leur origine physique. 

Nous avons détaillé le programme de NIKA2 en cosmologie et proposé plusieurs projets qui pourraient faire l'objet de futurs programmes sur demande de temps ouvert ou d'extensions du LP-SZ en temps garanti. L'objectif premier est la publication et la livraison à la communauté scientifique des données et résultats d'intérêt cosmologique. Cet objectif requiert une observation des amas garantissant un rapport signal sur bruit du paramètre de Compton qui soit homogène pour tout l'échantillon, l'analyse de chaque amas avec une méthode standard, préservant au mieux les plus grandes échelles angulaires accessibles à NIKA2, tout en soustrayant le bruit corrélé et les contaminants d'origines astrophysiques, et l'estimation des incertitudes statistiques et systématiques sur les produits finaux, qui sont la relation masse-observable et le profil de pression normalisé moyen, via des simulations numériques réalistes. Une première série de résultats à mi-parcours, fondés sur un demi-échantillon du LP-SZ pourrait être publiée dès 2021, avec une méthode d'analyse dérivée de la méthode actuelle. Ensuite, la publication des résultats finaux, utilisant l'échantillon complet et une méthode d'analyse améliorant la reconstruction des grandes échelles angulaires, pourrait constituer l'objectif d'une thèse commençant à l'automne 2021.

Le programme c\oe ur du LP-SZ repose sur la combinaison des données SZ de NIKA2 et X, en particulier de \emph{XMM-Newton}. En parallèle ou à plus long terme, des études multi-sondes complémentaires sont d'ores et déjà envisagées. Trois de ces études ont été décrites. Premièrement, un suivi optique de l'échantillon du LP-SZ par un observatoire équipé d'imageurs et de spectroscopes, telles les expériences installées au GTC à La Palma, permettrait d'explorer la relation entre richesse, masse dynamique et masse hydrostatique des amas et de progresser sur notre compréhension de l'évolution des galaxies dans le milieu intra-amas. Un tel programme aurait d'importantes répercussions pour la cosmologie avec les amas détectés par les futurs grands relevés optiques (VRO-LSST, \emph{Euclid}). Deuxièmement, une analyse multi-sondes incluant les données du LP-SZ et les données des relevés optiques existants ou à venir (HST, HSC, DES), permettrait de mesurer le biais hydrostatique et d'estimer la contribution des processus non-thermiques par comparaison de la masse hydrostatique et la masse reconstruite par effet de lentille, avec là encore, d'importantes implications pour l'utilisation des amas des futurs grands relevés optiques et X en cosmologie. Finalement, la complémentarité des deux grands observatoires de l'IRAM, le télescope de 30-m  et NOEMA, pourrait être exploitée pour fournir une vision globale des amas de galaxies : à la cartographie tSZ avec NIKA2 au télescope de 30-m, couvrant des échelles angulaire allant de la dizaine de seconde d'arc à la dizaine de minutes d'arc, nous adjoindrons une cartographie tSZ avec NOEMA, pour explorer le c\oe ur des amas à une résolution angulaire pouvant aller jusqu'à quelques dixièmes de secondes d'arc. Une telle étude, qui nous permettrait de quantifier l'impact des processus non-baryoniques au c\oe ur des amas, serait très précieuse, en particulier en vue des futurs relevés en X (\emph{e. g. } eROSITA). De manière générale, une analyse multi-sonde de l'échantillon d'amas de NIKA2 constitue une approche puissante pour étudier les effets systématiques qui affectent les différentes méthodes pour inférer la masse des amas de galaxies. 

Dans le contexte des futurs expériences incluant un fort programme en cosmologie avec les amas de galaxies, plusieurs extensions du LP-SZ sont prometteuses.
%
Dans le domaine du CMB, deux apports ont été identifiés. D'abord, un suivi d'un échantillon d'amas détectés par les prochaines expériences CMB au sol (\emph{Advanced} ACTpol, \emph{Simons Array}, \emph{Simons Observatory}, pour ne citer que celles accessibles à un suivi avec le télescope de 30-m de l'IRAM) permettrait d'étendre à plus basse masse la calibration de la relation masse-observable et du profil de pression moyen. Par ailleurs, NIKA2 pourrait obtenir la première reconstruction de l'effet de lentille sur le CMB par un amas individuel, ouvrant une nouvelle voie pour la mesure de la masse des amas étendant les méthodes cumulatives récemment utilisées. 
%
Dans le domaine X, NIKA2 offre l'opportunité d'effectuer un suivi en SZ des amas détectés par les futurs relevés (\emph{e. g. } eROSITA) avec une résolution angulaire comparable. Ainsi, des méthodes de mesure de la masse hydrostatique ou de la température de l'ICM sans données spectroscopiques pourraient être déployées, afin d'étendre la calibration de la masse à plus hauts redshifts.
%
Enfin, les apports de la cartographie SZ à haute-résolution angulaire sont également nombreux en optique/NIR. Par exemple, un suivi avec NIKA2 des amas détectés via l'effet de lentille gravitationnelle fort permettrait à la fois de mieux contrôler les effets systématiques inhérents à cette sonde, mais aussi d'améliorer notre connaissance de la population de galaxies à très hauts redshifts. Par ailleurs, un suivi des amas détectés via leur richesse dans les futurs relevés (VRO-LSST, \emph{Euclid}) serait utile à la calibration de la relation masse-observable en particulier à des redshifts pour lesquels une calibration en interne, fondée sur l'effet de lentille ou les dispersions de vitesses, devient peu fiable. \emph{Euclid} devrait détecter quelques $10^{4}$ amas à $z>1$, et ce jusqu'à $z\gtrsim 2$. Un suivi avec NIKA2 d'un échantillon d'amas de ce domaine de redshift peu exploré serait très souhaitable.        
   

La décennie qui s'ouvre s'annonce enthousiasmante pour la cosmologie avec les amas de galaxies. 
%
Elle verra émerger un foisonnement de données, avec la publication des données des relevés actuels et l'avènement des futurs grands relevés mis en service au début des années 2020, que ce soit en X (eROSITA), en optique (HSC, DES, KIDS, VRO-LSST, \emph{Euclid}) ou dans le domaine CMB via l'effet SZ (SPT-3G, \emph{Advanced} ACTPol, \emph{Simons Array}, \emph{Simons Observatory} puis CMB-S4).
%
Les avancées majeures concernent autant la précision des contraintes cosmologiques, liée à une augmentation de deux ordres de grandeurs du nombre d'amas détectés, que leur fiabilité, liée au contrôle accru des effets systématiques grâce aux comparaisons et corrélations croisées entre sondes des amas.
%
Dans ce paysage, NIKA2, via le LP-SZ d'abord puis d'éventuelles futures extensions, aura un apport critique, en particulier pour la calibration de la masse des amas à hauts redshifts, nécessaire pour exploiter à plein le potentiel scientifique de \emph{e. g. Euclid}.
%     
La cosmologie avec les amas de galaxies, s'appuyant sur des analyses multi-sondes fondées sur de vastes jeux de données, contribuera à forger la vison globale et cohérente de la formation des structures et de l'évolution de l'Univers, décisive pour contraindre les extensions au modèle $\Lambda$CDM minimal.  



             
     
       


