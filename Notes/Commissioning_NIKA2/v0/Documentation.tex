%
%     STATUS OF THE TECHNICAL DOCUMENTATION
%____________________________________________________________________

This paragraph reviews the status of the technical documentation that should be delived to the collaboration by the NIKA2 consotium and the IRAM, as defined in the MOU.


\subsection{Consortium-lead documentation}

\begin{itemize}

\item Plan of the cryostat: 3D model (e.g. STEP, Solid Works, etc.) and plans as built (e.g. PDF)
  
\item List of hardware components (per module)

\item Optics filtering components. [Consortium].

\item Cryogenics system characteristics with basic and standard operating procedures (see section 6.3). 

\item Cryogenics monitoring and diagnostic tools plus procedure to contact a cryogenist from the consortium who is mandated to help IRAM in case of problem with the cryostat. 
  
\item  Electronics cards characteristics, implantation plans, and operating procedures. 

\item Programmable electronics.
  
\item Network needs (architecture, data rates, speed, memory, synchronization accuracy, internet access, storage, backup, archiving, etc). 

\item Software for instrument control and data acquisition (Camadia), aimed at general users. 

\end{itemize}


\subsection{IRAM-lead documentation}

\begin{itemize}

\item Optics imaging system characteristics and calculations

\item Observers interface: PaKo functions specific to NIKA2, plus useful scripts

\item Automated on line data processing tools. [IRAM lead, consortium input].

\item Off line data processing software. [IRAM lead, consortium input].

\item Cook book (for external users: including a short description of NIKA2 setup, check list and procedure to use the instrument at the telescope). Can be provided at the end of instrument commissioning. [IRAM, consortium input].

\end{itemize}
