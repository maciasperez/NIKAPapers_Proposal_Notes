%-------------------------------------------------------------------
% This is a template for IRAM Proposals (originally created by R.Lucas)
% Deadline 12 Sep 2013
% Please use always the most recent update of proposal.sty
% (R. Neri / C.Thum / J.M.Winters (11 Jul 2013)
% 
%-------------------------------------------------------------------
% ATTENTION:
%
% The following lines follow the LaTeX2e syntax.
%
\documentclass[11pt,a4]{article}
\usepackage{proposal}
\usepackage{graphicx}
%-------------------------------------------------------------------
\begin{document}
%-------------------------------------------------------------------
% Instrument (%sign  comment in or out)
%-------------------------------------------------------------------
\picoveleta
%\blank
%-------------------------------------------------------------------
% Title (replace content)
%-------------------------------------------------------------------
\title{Confirming $z \ge 2$ cluster candidates observed by Planck and Herschel \\ (NIKA guaranteed time proposal)}
%-------------------------------------------------------------------
% Proposal Type (%sign  comment in or out)
%-------------------------------------------------------------------
\exgalcont
%\exgalco
%\exgalother
%\solarcont
%\solarlines
%\solarother
%\galcont
%\gallines
%\galcloud
%\galcse
%\galyse
%\galchem
%\galother
%-------------------------------------------------------------------
% Abstract (replace content)
%-------------------------------------------------------------------
\abstract{Catching massive galaxy clusters during their major
  formation period at high redshift remains a challenge. The Planck
  all-sky survey has the unique capability of systematically finding
  the rarest, brightest high-redshift sub-millimeter sources across
  all of the extragalactic sky. We have collected a catalog of
  hundreds of Planck high-z candidates, and have now Herschel/SPIRE
  photometry for 230 of them. Our analysis shows significant
  overdensities of strongly clustered red-excess sources consistent
  with groups or clusters of dusty starbursts at high-z. Here we
  propose to follow-up 23 of these candidates with NIKA GT as part of
  a coordinated FIR/submm/mm follow-up including also GISMO and
  SCUBA2, with the goal of characterizing the SED and dust properties,
  and ultimately, to obtain photometric redshifts. This synergy
  between space and ground based submm/mm leading facilities will shed
  new lights on these enigmatic episodes of structure formation.}
%-------------------------------------------------------------------
% Proposal history : fill in or comment out
%
% If this proposal is a continuation or a resubmission of a previous
% IRAM proposal, please describe the history of the proposal using
% the \proposalhistory{} command at the end of this file. 
%
%-------------------------------------------------------------------
%\resubmit{000-86}
%\continue{000-86}
%-------------------------------------------------------------------
% Amount of telescope time 
%-------------------------------------------------------------------
% Total telescope time requested for this  semester
\hours{27}			
% telescope time, hours, requested 
% NOTE: telescope time includes all overheads, like in the Time Estimator
%-------------------------------------------------------------------
% Receivers (remove the '%' sign when using the receiver) 
% argument: telescope time, hours, requested for this receiver 
%-------------------------------------------------------------------
%\emir{1.6}
%\hera{2.7}
%\gismo{3.8}
\nika{27}
%-------------------------------------------------------------------
% Sidereal time intervals: {start hour}{end hour}{number of intervals} 
%-------------------------------------------------------------------
%\lsta{11}{11}{10}
%\lstb{23}{23}{10}
%----------------------------------------------------------------
% Special requirements 
%----------------------------------------------------------------
%\LargeProgram
\pooledobserving
%\serviceobserving
%\remoteobserving
%\polarimeter
%----------------------------------------------------------------
% Scheduling constraints
%----------------------------------------------------------------
\constraints{None}
%----------------------------------------------------------------
% Principal Investigator (name, institut, address, telephone, email)
%----------------------------------------------------------------
\pifirstname{Herve Dole \&}
\pilastname{J.F. Mac\'{\i}as-P\'erez}
\piinstitute{Institut d'Astrophysique Spatiale}
\pistreetandnumber{bat 121, Univ. Paris Sud}
\pizipandcity{91405 Orsay}
\picountry{France}
\piphone{(+33) 1 69 85 85 72}
\pifax{(+33) 1 69 85 86 75}
\piemail{herve.dole@ias.u-psud.fr \& macias@lpsc.in2p3.fr}
%----------------------------------------------------------------
% A brief remark in case the proposed observations are part 
% of a Diploma thesis, PhD thesis, Marie-Curie fellowship, ... 
% specify here who is the fellow, who the supervisor, ... 
%----------------------------------------------------------------
%\pinote{}      
%----------------------------------------------------------------
% Coauthors 
% macro: \coauthor{First_Name Last_Name}{affiliation}{country} 
% use one \coauthor{ }{ }{ } macro for each coauthor. 
% You may add as many \coauthor{ }{ }{ } macros as needed ...
%----------------------------------------------------------------
%\coauthor{Muhammad ibn Musa al-Khwarizmi}{House of Wisdom}{Bagdad}
%\coauthor{Gustav Robert Kirchhoff}{Ruprecht-Karls-Universit\"at Heidelberg}{Germany}
%\coauthor{Charles Messier}{Observatoire de l'H\^{o}tel de Cluny, Paris}{France}
%\coauthor{Jules Janssen}{Observatoire de Paris-Meudon}{France}
\coauthor{NIKA Collaboration -- see http://ipag.osug.fr/nika2/Collaborators.html}{}{}

\coauthor{A. Adam}{LPSC}{France}
\coauthor{A. Adane}{IRAM}{France}
\coauthor{P. Ade}{Cardiff University}{UK}
\coauthor{P. Andr\'e}{CEA Saclay}{France}
\coauthor{A. Beelen}{IAS}{France}
\coauthor{B. Belier}{IEF}{France}
\coauthor{A. Beno\^it}{Institut N\'eel}{France}
\coauthor{A. Bideaud}{Cardiff University}{UK}
\coauthor{N. Billot}{IRAM}{Spain}
\coauthor{O. Bourrion}{LPSC}{France}
\coauthor{M. Calvo}{Institut N\'eel}{France}
\coauthor{A. Catalano}{LPSC}{France}
\coauthor{G. Coiffard}{IRAM}{France}
\coauthor{B. Comis}{LPSC}{France}
\coauthor{A. D'Addabbo}{Institut N\'eel}{France}
\coauthor{F.-X. D\'esert}{IPAG}{France}
\coauthor{S. Doyle}{Cardiff University}{UK}
\coauthor{J. Goupy}{Institut N\'eel}{France}
\coauthor{C. Kramer}{IRAM}{Spain}
\coauthor{S. Leclercq}{IRAM}{France}
\coauthor{J. Martino}{IAS}{France}
\coauthor{P. Mauskopf}{Cardiff University}{UK}
\coauthor{F. Mayet}{LPSC}{France}
\coauthor{A. Monfardini}{Institut N\'eel}{France}
\coauthor{F. Pajot}{IAS}{France}
\coauthor{E. Pascale}{Cardiff University}{UK}
\coauthor{L. Perotto}{LPSC}{France}
\coauthor{E. Pointecouteau}{IRAP}{France}
\coauthor{N. Ponthieu}{IPAG}{France}
\coauthor{V. Rev\'eret}{CEA Saclay}{France}
\coauthor{L. Rodriguez}{CEA Saclay}{France}
\coauthor{G. Savini}{UCL}{UK}
\coauthor{K. Schuster}{IRAM}{France}
\coauthor{A. Sievers}{IRAM}{Spain}
\coauthor{C. Tucker}{Cardiff University}{UK}
\coauthor{R. Zylka}{IRAM}{France}
\coauthor{N. Nesvadba}{IAS}{France}
\coauthor{S. K\"onig}{IRAM}{France}
%\coauthor{A. Beelen}{IAS} {France}
\coauthor{R.-R. Chary}{Caltech}{USA}
\coauthor{I. Flores-Cacho}{IRAP}{France}
\coauthor{B. L. Frye}{UofA}{USA}
\coauthor{M. Giard}{IRAP}{France}
\coauthor{D. Guery (PhD)}{IAS} {France}
\coauthor{R. Kneissl}{ALMA} {Chile}
\coauthor{G. Lagache}{IAS} {France}
\coauthor{E. Le Floc'h}{CEA}{France}
\coauthor{T. McKenzie (PhD)}{UBC}{Canada}
\coauthor{L. Montier}{IRAP} {France}
%\coauthor{A. Omont}{IAP}{France}
%\coauthor{E. Pointecouteau}{IRAM}{France}
\coauthor{J.-L. Puget}{IAS}{France}
\coauthor{D. Scott}{UBC}{Canada}
%%\coauthor{N. Welikala}{APC}{France}
\coauthor{L. Yan}{IPAC}{USA}
% ...
%----------------------------------------------------------------
\observer{J.F.  Mac\'{\i}as-P\'erez}                 
%----------------------------------------------------------------
% Source list
% One line for each source should be given:
%
% Name   & RA      &    DEC    &   V(LSR) or z(radio) \\
% The epoch is given as a separate parameter.
% ADDITIONAL souces which do not fit here shall be entered in
% the macro "\extendedsourcelist" listed at the end of this file
%----------------------------------------------------------------
\epoch{J2000.0}  % J2000 epoch coordinates preferred
\sourcelist{
23 sources &  & & z$\sim$2-4
}

%----------------------------------------------------------------
% The following command will actually produce the front page. 
%----------------------------------------------------------------
\frontpage
%
%================================================================
%-NEW--30M--NEW--30M--NEW--30M--NEW--30m--NEW--30M--NEW--30M--NEW
%
%-30M--NEW--30M--NEW--30M--NEW--30M--NEW--30M--NEW--30M--NEW--30M
%----------------------------------------------------------------
%
%--- EMIR -------- EMIR -------- EMIR
%
%enter up to 2 \EMIR parameters macros for each setup 
%each macro MUST have 9 parameters (pairs {} of curly brackets) :
%setup - band - species - transition - GHz - Ta* - rms - width -backend 
%
\EMIRparameters
 {1}     % setup number number
 {E0}    % EMIR band: E0 (3mm), E1 (2mm), E2 (1.3mm), E3 (0.9mm)
 {HCN}   % species
 {1-0}   % transition
 {88.6}  % sky frequency, GHz
 {100}   % expected antenna temperature, mK
 {10}    % rms noise requested, mK
 {1.0}   % velocity resolution, km/s
 {V}     % backend(s) V:VESPA,W:WILMA,4: 4MHz filterbank, FTS50: FTS @ 50 kHz resolution, FTS200: FTS @ 200 kHz resolution 
\EMIRparameters{1}{E2}{HCN}{3-2}{265.9}{800}{100}{1.0}{V}
%
\EMIRparameters{2}{E0}{H$^{13}$CN}{1-0}{86.3}{50}{5}{0.5}{V}
\EMIRparameters{2}{E2}{H$^{13}$CN}{3-2}{259.0}{200}{10}{0.5}{V}
%
%EMIR observing parameters and telescope time requested, in hours.
%enter one macro for each EMIR setup
%            if maps   
%....setup..dRA...dDec.obsMode..switchMode...pwv...hours..remark
\EMIRmap
 {1}   % setup number					     
 {2.0} % delta L, map size in L  (empty when not mapping)     
 {2.0} % delta B, map size in B  (empty when not mapping)     
 {OTF} % mapping mode: OTF, Raster, or none		     
 {FSw} % switching mode: PSw, FSw, WSw, or none     	     
 {7}   % maximum acceptable pwv, mm: 1, 2, 4, 7, or 10.
 {5.1} % telescope time requested for this setup, hours 	     
 {---} % remark                                               
\EMIRmap{2}{}{}{none}{WSw}{2}{4.3}{second priority}
%
%
%--- HERA -------- HERA -------- HERA
%
%enter two \HERAparameters macros for each frequency setup 
%each macro MUST have 9 parameters:
% setup - HERA pol - species - transition - GHz - Ta* - rms - width - Backend
%
\HERAparameters
 {1}     % setup number
 {1}     % HERA 1 (H polarization array)
 {CO}    % species
 {2-1}   % transition
 {230.5} % sky frequency, GHz
 {50}    % expected antenna temperature, mK
 {20}    % rms noise requested, mK
 {10.}   % velocity resolution, km/s
 {W}     % backend(s) V:VESPA,W:WILMA,4: 4MHz filterbank, FTS50: FTS @ 50 kHz resolution, FTS200: FTS @ 200 kHz resolution 
\HERAparameters{1}{2}{CO} {2-1}{230.5}{50}{20}{10.}{W}
%
\HERAparameters{2}{1}{$^{13}$CO}{2-1}{220.4}{50}{10.}{10.}{W}
\HERAparameters{2}{2}{$^{13}$CO}{2-1}{220.4}{50}{10.}{10.}{W}
%
%HERA observing parameters and time request, hours.
%enter one macro for each HERA setup
%....setup..dRA...dDec.obsMode..switchMode...pwv...hours..remark
\HERAmap
 {1}    % setup number					     
 {}     % delta L, map size in L  (empty when not mapping)      
 {}     % delta B, map size in B  (empty when not mapping)     
 {none} % mapping mode: OTF, Raster, or none		     }     
 {FSw}  % switching mode: PSw, FSw, WSw, or none     	           
 {4}    % maximum acceptable pwv, mm: 1, 2, 4, 7, or 10.
 {6.9}  % telescope time requested for this setup, hours 	       
 {---}  % remark                                               
\HERAmap{2} {1.1}{1.1} {OTF}     {PSw}  {4}    {2.6}  {second priority}
%
%
%--- GISMO -------- GISMO -------- GISMO
%
% WARNING: MAMBO-type ONOFF observations are not possible anymore.
%          NEVER comment out the \BOLOmap macro.
%          It is needed for the calculation of the total GISMO time 
%          requested and for other input, like pwv.
%
% compact source observations with GISMO 
% Observations of compact (smaller than about 1 arcmin diameter) 
% sources are made through small maps.
% you may enter more than one \BOLOcompact macro (5 parameters !) 
\BOLOcompact
 {4.5}  % expected source flux density, mJy
 {0.8}  % rms noise requested, mJy
 {2}    % number of sources
 {4}    % maximum acceptable pwv, mm: 1, 2, 4, 7, or 10.
 {2.6}  % telescope time requested, hours
\BOLOcompact {3.0} {1.5} {15} {4} {6.2}
%
% larger maps with GISMO
% INFORMATION on the permitted/reasonable range of parameters is available
% on the Granada wiki pages
% enter one \BOLOmap macro (8 parameters !) for each different map
\BOLOmap
 {25}         % expected source peak brightness, mJy/beam
 {1.0}        % rms noise requested, mJy
 {3.0}        % map size in X (azimuth), arcmin
 {2.0}        % map size in Y (elevation), armin
 {1}          % number of maps
 {4}          % maximum acceptable pwv, mm: 1, 2, 4, 7, or 10.
 {3.8}        % telescope time requested, hours
 {Egg Nebula} % remark
\BOLOmap{55}{5.0}{4.0}{3.0}{2}{7}{0.8}{---}
%
%
%
%--- NIKA -------- NIKA -------- NIKA
%
% WARNING: MAMBO-type ONOFF observations are not possible.
%          Observations of compact sources (size <=40") are made in
%          Lissajous mode where the minimum map size of 1.0 x 1.0 arcmin 
%          should be used for both bands. Larger sources are mapped in 
%          Lissajous or on-the-fly modes.
% INFORMATION on the permitted/reasonable range of map sizes is available
%          on the Granada wiki pages.
% TWO macros, \NIKAparams with 7 parameters and \NIKAtimes (6 parameters), 
%          need to be specified for each NIKA setup. 
%
\NIKAparams
 {1}    % No. of setup
 {3}  % expected source flux density at 1.3mm, mJy
 {0.75}  % rms noise requested at 1.3mm, mJy
 {1}  % expected source flux density at 2mm, mJy
 {0.25}  % rms noise requested at 2mm, mJy
 {2.0}  % map size in X (azimuth), arcmin
 {2.0}  % map size in Y (elevation), armin
%\NIKAparams {2} {5.5} {1.8} {3.5} {0.5} {7.0} {5.0}
%
\NIKAtimes
 {1}          % No. of setup
 {1}          % priority band: enter 1 or 2 for the 1.3mm or 2mm band
              %                enter 0 if both band have equal priority  
 {4}          % maximum acceptable pwv, mm: 1, 2, 4, 7, or 10.
 {23}         % number of sources
 {27}        % telescope time requested, hours
 {} % remark
%\NIKAtimes {2} {2} {4} {33} {19.0} {---}
%
%
%----------------------------------------------------------------
% The following command will produce the Technical Sheet
%----------------------------------------------------------------
\techsheetPV
%----------------------------------------------------------------
%
\maketitle
%----------------------------------------------------------------
% Scientific justification (do not neglect this part ...)
% max. 2 pages of text (4 pages for large programs) 
% plus 2 pages of Figs., Tables and Refs.
%----------------------------------------------------------------

%\proposalhistory{Give a short history whenever the proposal is 
%                 a continuation or a resubmission.} 


\noindent Some of the greatest challenges of our hierarchical paradigms
of galaxy evolution and structure formation are at the high end of the
mass function, which traces the most extreme density fluctuations of
the primordial dark-matter distribution. Galaxy clusters are natural
probes of structure formation as well as galaxy evolution: with masses
of up to $10^{15}$ M$_{\odot}$, they are at the nexus of galaxy
evolution and cosmology, and they are the sites where most of the
massive galaxies in the nearby Universe are found (e.g., Renzini
2006). Archeological constraints like the remarkably tight red
sequence of passively evolving early-type galaxies in clusters out to
$z\sim$0.8 (e.g., Jaffe et al. 2011) and beyond (e.g., Kodama et
al. 2007) clearly show that massive cluster galaxies must have formed
most of their stars over short timescales in vigorous bursts of
dust-enshrouded, intense star formation at redshift $z\ge$2, giving
rise to bright FIR dust emission which is redshifted into the submm
range. 

In the {\it Herschel} era catching massive high-$z$ {\it galaxies}
during their major growth period may no longer be a challenge, however
identifying the most massive, most intensely star-forming {\it galaxy
  clusters} remains remarkably difficult. With short star-formation
time scales of few $10^8$ yrs and the intrinsic scarcity of very
massive objets at high-$z$, such objects are very rare. For example,
Mo \& White (2002) predict only 1 cluster with $10^{15}$ M$_{\odot}$
per $10^8$ Mpc$^3$ at $z\sim$1. Collecting a significant sample of the
most massive, most rapidly evolving high-$z$ density peaks continues
to be a challenge even for the wide {\it Herschel} surveys. This is a
``hot topic'' where any new selection can provide valuable
discoveries.

\vspace{2mm}

\noindent {\bf \emph{Planck} is the first sub-mm all-sky survey} with
the depth and spatial resolution necessary to probe the brightest end
of the high-$z$ FIR/sub-mm luminosity function down to $\sim$100~mJy
{\it and over the full 1/3 of the sky which is not dominated by
  Galactic foregrounds}. Sky coverage is essential to robustly
identify the most outstanding sources along the exponentially
declining tail of the luminosity function. We identify few hundreds of
potential high-redshift objects, about 1 per 30~deg$^{2}$ using
color-color criteria (Fig.~3). These targets are too rare to fall into
the large Herschel surveys in significant numbers. The only high-$z$
Planck source with published Herschel observations, a gravitational
lens at $z=$3.2 (Fu et al. 2012), was taken from the Planck Early
Release Compact Source Catalog (Planck Collab 2011, VII) which has a
different selection not optimized for high-$z$ galaxies like our
sample. From the sample, we have now 200 sources observed with
Herschel/SPIRE photometry (OT1, OT2, and exceptional must-do
allocation).

Most of our sources with existing Herschel observations are
significant overdensities of red sources that peak at 350$\mu$m or
500$\mu$m, consistent with galaxy groups and clusters at
$z_{phot,SPIRE}\ge$2 (Fig. 2 \& 3). Their joint emission boosts their
flux in the large (5'; 2.5~Mpc at $z=$2) beam of Planck. Our sample is
the only uniformly selected all-sky sample of high-$z$ submm candidates
currently available. Since there are no similar, brighter high-$z$ submm
regions at cluster scales on the observable FIR-submm sky, our
targets have an immense potential of being the most rapidly growing,
most actively star-forming overdensities of massive galaxies in the
early Universe, and the likely progenitors of the most massive galaxy
clusters today seen during their most rapid phase of star formation.

\noindent {\bf What are the intrinsic properties of these objects?}
We have embarked on a broad, on-going spectroscopic and photometric
multi-wavelength follow-up with sub-mm and optical/NIR facilities
across the world to find this out. This includes in particular SCUBA-2
(16 sources reduced, 17 to be observed, 39 proposed), Spitzer
DDT photometry at 3.6~$\mu$m and 4.5~$\mu$m for 25 targets (DDT+GO9
ongoing), as well as CFHT and VLT observations (11 targets observed):
we may have at least 1 cluster in J-K (analysis ongoing). For 1 source
we obtained 2mm PdBI interferometry through DDT W--1 (see below and
Fig. 2). {{\bf SPIRE alone already reveals a significant excess of
    strongly clustered, red sources, and likewise, other bands reveal
    unambiguous concentrations of sources.} We are currently working
  on cross-identifying our sources in different wavebands for
  individual sources where multi-wavelength data are already available
  (Figs. 1 \& 2). The SPIRE photometry of our sources is generally
  consistent with a broad redshift range of $z=$2$-$4, albeit with
  large uncertainties owing to the small wavelength coverage of the
  three SPIRE bands which only probe the very peak of the dust SED.}
The imperative next step to further characterize our sources is to
comprehensively sample the Rayleigh-Jeans tail of the dust SED at
several wavelengths. This is the primary motivation of our proposal.


\vspace{0.4cm}

\noindent {\Large Immediate Objectives}\\
We propose a deep 1.2 and 2.1~mm continuum imaging of 23 red, high-z
Planck and Herschel cluster candidates. Our primary goal is to
identify the mm counterparts of the Herschel sources. (Sub-)mm
photometry in addition to SPIRE FIR photometry is required to
characterize the dust SED. Note that our proposed sources are expected
to have flux densities larger than Ivison et
al. (2013). Multiwavelength data exist in some sources (see table
1). Our first GISMO run in April 2013 was successful (see figs 1, 4,5
) and proves the feasibility of the project.

{\bf {$\bullet$} Measure the dust SED} We will probe the
Rayleigh-Jeans tail of the dust heated by star formation over a much
larger wavelength range than our SPIRE data alone, which only cover
250$\mu$m, 350$\mu$m and 500$\mu$m, and longwards 850$\mu$m for which
we have some SCUBA2 data. {\it The goals is to combine SCUBA2, NIKA
  and SPIRE photometry} whenever possible.  This is ideal to probe the
dust peak of high-$z$ galaxies, and hence to confirm the high-redshift
nature of our sources. However, the proximity in wavelength boosts the
uncertainties of SPIRE-only photometric redshift estimates owing to
the unavoidable measurement uncertainties (Fig. 2). With NIKA %at 2~mm
we expect to significantly tighten our photometric redshift
constraints (modulo the unavoidable degeneracy with T$_d$). Since the
30-m beam is smaller than the SPIRE beam, we will have no uncertainties
from missing flux due to potential extended dust emission (Ivison et
al. 2012 -- and see our fig. 4 a-d).

{\bf {$\bullet$} Synergy with shorter wavelength photometry} We
already have extensive CFHT and Spitzer (DDT and GO9) photometry for
some sources (fig. 1), and are currently extending this sample. This
will allow us to fully characterize the stellar and star-formation
content of our targets and to directly compare with the specific star
formation rates in the general high-z galaxy population (are they
star-forming or on the main sequence, Elbaz et al. 2011, or is star
formation perhaps suppressed relative to the MS, which would perhaps
indicate accelerated galaxy ages as seen in nearby clusters?). For
galaxies with a 'simple' Herschel/SCUBA-2/NIKA dust SED consistent
with a single source this can already be done with our 30-m data
alone, in more complex cases we will follow-up using the PdBI to
disentangle individual sources.

{\bf {$\bullet$} Towards a sample for PdBI}.  This sample observed at
1/2mm will serve as a criterion for further observations at the PdBI,
based on the flux density measured, and on the refined SED and
photo-$z$. The goals will be to spatially resolve the dust emission
(and maybe CO lines) and correlate it with the NIR data (WISE or
Spitzer when available) to perform a co-analysis of stellar mass and
star formation rate where multiple sources contribute to the
SPIRE/NIKA fluxes with 10-15 arcsec beams.


\noindent Depending on the outcome in 2014, our observations may
serve as a pilot for a future 30-meter Large Program, to observe the
full SPIRE sample that we collected.  This would be complementary
to the detailed, on-going multi-wavelength follow-up of smaller
samples such as proposed here, and allow us to characterize the
ensemble properties of our sample in a statistical way (e.g.,
photometric redshift distribution once the range of dust temperatures
is known). This would be mandatory to fully exploit the immense and
unique potential that our all-sky survey has for cosmology (i.e.,
early collapse of massive dark-matter halos).

\vspace{0.4cm}

\noindent {\Large Technical Justification}\\
\noindent {\bf Sample size:} With almost 500 Planck high-$z$
candidates and 230 SPIRE targets, why pick these 23 sources? Our
overall sample is too large to follow up each and every source in all
wavebands, however to be able to characterize our sample overall, we
must reach a minimal statistical robustness, in particular to
characterize dust temperatures and FIR/mm redshifts. For this goal, we
need to reach a statistical uncertainty of $\sqrt{N}/N<20$\% or
better, i.e. more than 25 sources. Given our success in identifying
the Planck sources with Herschel, 'contamination' of our sample with
very bright gravitational lenses, etc., and taken into account the
available time slot, we require 23 sources to have a decent
statistical value: an uncertainty of 21\%. Table 1 highlights the
source characteristics.

\noindent {\bf Time estimate:} 
Using SED templates redshifted at $z=2$ , for example Arp220, we expect
flux ratio of 20 between 350 and 1.2 mm and 70 between 350 and 2.1
mm. To consider the SED follow-up of sources already detected with
SPIRE with a minimum flux of 60 mJy at 350~$\mu$m we need to detect
at 4$\sigma$ a 1.2 mm flux of about 3~mJy which corresponds to 0.75~mJy
rms.  Fig. 5 shows that Arp220 SED is a reasonable, although far from
perfect,  approximation.

\noindent Assuming point sources distributed within a 2 arcmin diameter,
Lissajous scans of 2 arcmin size are optimized to ensure homogeneous
sensitivity within 1 arcmin of radius of the center of the source.
Assuming winter weather conditions we expect a sensitivity of 40 mJy
s$^{1/2}$.  We therefore need to observe for 40 minutes each source.
We choose to observe 23 sources for a total of 18 hours of observation
on source and thus request 27 hours in total considering a third of
the time is spent for calibration (focus, pointing and photometry).
We note that the data reduction pipeline for Lissajous scans has
already been used successfully in the last NIKA run.

We thus request {\bf 27 hours in total for 23 sources (40min on source
  + overheads).}

%SEDs (NGC520 and Arp220) to each of the components to derive
%approximate photometric redshifts and flux density estimates at
%2mm. We obtain an average $z_{phot}$ of 3.3, and a predicted average
%flux density of 3.5 mJy, but our recent IRAM PdBI DDT at 2mm (weak
%detection at 0.42~mJy) suggests a wide diversity of dust SEDs in our
%sample, and that we need to reach the 1mJy level at 5$\sigma$,
%i.e. 0.2 mJy rms. Comparison with the Bethermin et al. (2011) model
%shows that this is well above the confusion limit. Assuming a dust spectrum
%at 1.2 mm we need to obtain 1.7 mJy rms per source. Assuming for NIKA 40 mJy s$^{1/2}$ = 0.66 mJy h$^{1/2}$
%at 1.2 mm the previous limit corresponds to about  0.71 hours per source accounting also for 
%on source time only (roughly a third of the time) and for the calibration runs (we apply a factor of 1.5).

%WHAT WE SHOULD SAY !!!
%Follow-up SED of sources detected with SPIRES with minimum of 60 mJy at 350 microns


%\newpage
\begin{figure}
\includegraphics[width=1.0\textwidth]{./figure_30m.ps}
\end{figure}

{\bf  TABLE 1 - Source follow-up status}


{\scriptsize
\bigskip
\begin{tabular}{lllcc}
\hline \\
\hline \\ 
Source Name & Spitzer GO9 & Spitzer GO10 & GISMO \\
\hline \\ 
G237.1+54.0& yes & no & yes\\
G191.3+62.0& yes & no & yes\\
G198.7+67.9& no & yes & no\\
G143.6+69.4& no & yes & no\\
G124.1+68.8& no & yes & no\\
G303.6+81.3& no & yes & no\\
G322.1+62.3& no & yes & no\\
G325.0+63.2& no & yes & no\\
G104.0+64.2& no & yes & no\\
G071.1+73.6& no & yes & no\\
G026.6+74.4& no & yes & no\\
G112.4+45.8& no & yes & no\\
G102.1+53.6& no & yes & no\\
G006.1+61.8& yes & no & yes (pending)\\
G058.5+64.6& no & yes & no\\
G57.3+63.0& no & yes & no\\
G56.7+62.6& yes & no & yes (pending)\\
G083.8+52.5& yes & no & no\\
G078.9+48.2& yes & no & yes (pending)\\
G107.6+36.9& no & yes & no\\
G063.7+47.7& yes & no & no\\
G059.1+37.4& no & yes & no\\
G052.2+28.1& no & yes & no\\
\hline \\
\end{tabular}
}

Targets have also been selected in SPIRE images to be concentrated,
i.e. allowing an efficient folow-up w/ NIKA within a 2 arcmin
diameter.



%----------------------------------------------------------------
% Extended Source List (sources which do not fit on cover page)
%  -  place anywhere after the cover page  
%  -  uncomment the following 5 lines if needed
%  -  do NOT modify the format
%----------------------------------------------------------------
\newpage
 \extendedsourcelist{
G237.1+54.0&10h39m46s&+08d52m48s& $z_p>2$ \\
G191.3+62.0&10h44m50s&+33d49m58s& $z_p>2$ \\
G198.7+67.9&11h11m30.00s&+30d17m59.6s& $z_p>2$ \\
G143.6+69.4&12h10m12.96s&+46d04m20.6s& $z_p>2$ \\
G124.1+68.8&12h48m59.76s&+48d19m57.4s& $z_p>2$ \\
G303.6+81.3&12h51m50.76s&+18d26m39.0s&  $z_p>2$ \\
G322.1+62.3&13h26m34.56s&+00d43m54.8s&  $z_p>2$ \\
G325.0+63.2&13h30m27.84s&+01d59m32.2s&  $z_p>2$ \\
G104.0+64.2&13h43m19.20s&+50d57m54.0s&  $z_p>2$ \\
G071.1+73.6&13h55m34.90s&+36d22m22.9s&   $z_p>2$ \\
G026.6+74.4&13h59m35.64s&+24d23m12.8s&   $z_p>2$ \\
G112.4+45.8&14h17m17.29s&+69d31m15.5s&   $z_p>2$ \\
G102.1+53.6&14h29m25.57s&+59d22m21.2s&   $z_p>2$ \\
G006.1+61.8&14h33m39.48s&+12d12m54.62s&   $z_p>2$ \\
G058.5+64.6&14h44m27.12s&+35d13m53.0s&   $z_p>2$ \\
G57.3+63.0&14h52m38.65s&+34d57m51.6s&   $z_p>2$ \\
G56.7+62.6&14h54m38.21s&+34d45m12.3s&   $z_p>2$ \\
G083.8+52.5&15h23m55.20s&+51d26m11.8s& $z_p>2$ \\
G078.9+48.2&15h56m01s&+50d03m48s& $z_p>2$ \\
G107.6+36.9&16h07m38.23s&+73d47m09.2s& $z_p>2$ \\
G063.7+47.7&16h07m53.70s&+40d02m40.6s& $z_p>2$ \\
G059.1+37.4&16h58m48.52s&+36d06m03.2s& $z_p>2$ \\
G052.2+28.1&17h35m22.95s&+28d17m03.9s&$z_p>2$ \\
}



%----------------------------------------------------------------



%----------------------------------------------------------------
% Extended Source List (sources which do not fit on cover page)
%  -  place anywhere after the cover page  
%  -  uncomment the following 5 lines if needed
%  -  do NOT modify the format
%----------------------------------------------------------------
% \extendedsourcelist{
% L1448       & 03:25:38.9   & +30:44:05       & +5.0  \\
% M33         & 01:33:50.9   & +30:39:35.8     & -170  \\
% ...         &  ...         &    ....         &  ...  \\
% }
%----------------------------------------------------------------

\end{document}
