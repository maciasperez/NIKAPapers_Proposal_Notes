
\begin{figure}[ht]
\begin{center}
  \includegraphics[width=0.65\textwidth]{Figures/opacity_tau1_tau2_ratio_N2R9_N2R10.png}
  \includegraphics[width=0.65\textwidth]{Figures/opacity_tau1_tau225_ratio_N2R9_N2R10.png}
  \includegraphics[width=0.65\textwidth]{Figures/opacity_tau2_tau225_ratio_N2R9_N2R10.png}
\caption[Opacity ratio]{Ratios between the 150 GHz and the 260 GHz NIKA2 zenith opacity
estimates and between the NIKA2 $\tau$ and the IRAM taumeter
values. The expectation values derived for NIKA2 bands
using the ATM model described in \ref{Pardo2002} are shown for
comparison (red and orange curves).}
  \label{fig:opacity_ratios}
\end{center}
\end{figure}

\begin{figure}[ht]
\begin{center}
  \includegraphics[width=0.8\textwidth]{Figures/opacity_tau1_tau2_byrun_ratio_N2R9_N2R10.png}
  \caption[Opacity ratio per campaigns]{Ratio between the 150 GHz and the 260 GHz NIKA2 zenith opacity
estimates. The expectation values derived for NIKA2 bands
using the ATM model described in \ref{Pardo2002} are shown for
comparison (red and orange curves). The observed NIKA2 opacity ratio
has a smooth, consistent behaviour over the overall probed opacity range,
and very few outlier estimates are seen although no scan selection has
been performed (out from discarding the dark tests). Also remarkable
is the consistency between estimates obtained during two campaigns
held two months apart in different weather conditions (good to average
during N2R9 and poor and often hightly unstable conditions during
N2R10). Some sub-structures are seen in the opacity ratio, which are
under investigations. They can have several origins (telescope cabin
temperature variation, variation of the $0_2$ fraction, atmospheric
temperature variation, internal temperature variations, etc).  
  }
  \label{fig:opacity_ratio_perrun}
\end{center}
\end{figure}


\begin{figure}[ht]
\begin{center}
  \includegraphics[width=0.8\textwidth]{Figures/opacity_tau1_tau2_ratio_perday_N2R9_N2R10.png}
  \includegraphics[width=0.8\textwidth]{Figures/opacity_tau1_tau2_ratio_perday_zoom_N2R9_N2R10.png}
  \caption[Opacity ratio per days]{Ratio between the 150 GHz and the 260 GHz NIKA2 zenith
    opacity estimates. The 4 outlier estimates on February, 24 (in
    cyan) correspond to a test using the external
    calibrator. Different regimes are seen on the 25th and 26th of
    February, while the weather conditions were too unstable to allow
    the astronomer team to focus.
  }
  \label{fig:opacity_ratio_perday}
\end{center}
\end{figure}


\begin{figure}[ht]
\begin{center}
  \includegraphics[width=0.8\textwidth]{Figures/opacity_tau1_tau2_ratio_bperror10pc_N2R9_N2R10.png}
  \includegraphics[width=0.8\textwidth]{Figures/opacity_tau1_tau2_ratio_o2fraction_N2R9_N2R10.png}
\caption[Robustness tests on opacity estimates]{Uncertainty of NIKA2 $\tau$ values. Upper panel: The impact
  of the NIKA2 transmission measurement uncertainties is illustrated
  using a very pessimistic relative uncertainty of $10\%$ (instead of
  the more realistic $1\%$ errors). Lower panel: The impact of the
uncertainty on the atmospheric absorption around $118\, \rm{GHz}$, due
to the lack of precise knowledge of the fraction of oxygene in the
atmosphere. The nominal absorption predicted by the ATM model is
modified by a factor from 0.5 to 2 in the $117-120\, \rm{GHz}$
frequency band, where the $0_2$ contributions largely dominates the
water vapor ones. }
  \label{fig:opacity_errors}
\end{center}
\end{figure}

\begin{figure}[ht]
\begin{center}
\includegraphics[width=0.9\textwidth]{Figures/opacity_tau1_tau2_emissionratio_N2R9_N2R10.png}
\caption[Emission opacity ratio]{Ratio of the atmospheric emission in NIKA2 bands defined as
  in Eq.~\ref{eq:opacity_emission_ratio}, compared with the ATM-model
  predicted ratio calculated as in Eq.~\ref{eq:opacity_emission_ratio_model}}
  \label{fig:opacity_emission}
\end{center}
\end{figure}

The ratios between the 150~GHz and the 260~GHz NIKA2 zenith opacity
estimates, quoted $\tau_{2mm}$ and $\tau_{1mm}$ , and
between the NIKA2 $\tau$ and the IRAM taumeter values are presented in
Fig.~\ref{opacity_ratios}, along with the expectation values derived for NIKA2 bands
using the ATM model described in \ref{Pardo2002}. Namely, these
predicted values $\tau^{th}$ are calculated from the ATM-model
atmospheric zenith opacity $\tau^{ATM}$ using:  
\begin{equation}
  \tau^{th}_{A_i} = - \ln{\frac{\int e^{-\tau^{ATM}(\nu)}
      T_{A_i}(\nu) d\nu}{ \int T_{A_i}(\nu) d\nu}},
\end{equation}

where the NIKA2 bandpasses $T_{A_i}$ for arrays $A_i$, $i=1, 2, 3$, are the Martin-Pupplet reference transmissions
corrected by a Rayleigh-Jeans term  $T'_{A_i}(\nu) /
\left( \frac{\nu}{\nu_0}\right)^2$. 

In Fig.~\ref{fig:opacity_emission}, we
show the ratio of the atmospheric emission in NIKA2 bands defined as:
\begin{equation}
  R_{\rm{atm}} = \frac{1-e^{-\tau_{2mm}}}{1-e^{-\tau_{1mm}}}.
    \label{eq:opacity_emission_ratio}
\end{equation}

It is compared with the ATM-model predicted ratio
\begin{equation}
  R_{\rm{atm}}^{th} = \frac{\int (1 - e^{-\tau^{\rm{ATM}}}) T_{A_2}(\nu) d\nu }{\int T_{A_2}(\nu) d\nu} / \frac{\int (1 -
      e^{-\tau^{\rm{ATM}}}) T_{A_{1}}(\nu) d\nu }{\int T_{A_1}(\nu)
        d\nu} .
      \label{eq:opacity_emission_ratio_model}
\end{equation}

In Fig.~\ref{fig:opacity_errors}, we investigate different effects that can impact the precision with
which the zenith opacities are determined: the upper panel shows the
expected dispersion in the NIKA2 $\tau$ values coming from the transmission
measurement uncertainties: to higlight this effect, we consider a very
pessimistic relative uncertainty of $10\%$ (whereas $1\%$ would have
been a more realistic value), and the lower panel shows the impact of the
uncertainty on the fraction of oxygene in the atmosphere, which mainly 
translates in an uncertainty on the atmospheric absorption around
$118\, \rm{GHz}$: the nominal absorption predicted by the ATM model is
modified by a factor from 0.5 to 2 in the $117-120\, \rm{GHz}$
frequency band, where the $0_2$ contributions largely dominates the
water vapor ones. 



We have compared $C_0$ values, the resonance frequency at zero atmosphere,
between different runs. It appears to vary in a systematic manner. For example
we have compared N2R6 and N2R7. The change of frequencies when converted to
temperature (with $c_1$) is of about $25$ and $86$~K at 1 and $2$~mm. This
cannot be a real change of the background. Translated back by a median value
of $c_1$ ($=2500$ and $1500$~Hz/K at 1 and 2 mm), we obtain a 62.5 and 128 kHz
median downward shift of all resonant frequencies between N2R6 (October 2016)
and N2R7 (December 2016). The likely explanation is that of a slight ageing of
the KIDs. A single monolayer of oxyde could be enough to produce the downward
shift.
