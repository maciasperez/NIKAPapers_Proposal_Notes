\section{Aperture Photometry Calibration {\color{blue} Jean-Fran\c cois}}

%  LP
%  --> petite introduction temporaire : peut etre modifiee/effacee
To be discussed in this section: the aperture photometry method, the conversion factor between Gaussian Beam photometry and aperture photometry and as example, some stability tests on Planets using aperture photometry

The aperture photometry method that was used to derive the first calibration results as reported in~\cite{Adam18} is detailed in Sect.~\ref{ap:cal_JFL}. 


%  JFL
%  --> extracted from results_Ap.tex

Aperture photometry integrates the intensities of all pixels within radius $r_{max}=180''$. In Table \ref{tab:ratio}, we provide the ratio between the 
the flux density derived by aperture photometry and a simple Gaussian fit to the source (Uranus or Neptune) with fixed fwhms $12.5''$ and $18.0''$).
This ratio is larger than unity and indiques that aperture photometry recovers more flux density than the gaussian fit. Caveat : a full analysis
of this inconsistency would require to take into account the fact that KIDs (kidpars) have actually been calibrated by fitting a gaussian to Uranus 
or Neptune.    


\begin{table*}[!h]
\caption{ratio aperture photometry / gaussian fit flux densities   }
\label{tab:ratio}
\centering
\begin{tabular}{l| c | c c c }
\hline\hline
\noalign{\smallskip}
run     & Nber of scans  &  A1    &    A2   &  A3    \\
\hline
r9    & 27  &  1.08$\pm$ 0.03    &  1.04$\pm$ 0.02 & 1.09 $\pm$ 0.03     \\
r12   & 20  &  1.12$\pm$ 0.02    &  1.05$\pm$ 0.02 & 1.13 $\pm$ 0.02     \\
r14   & 28  &  1.09$\pm$ 0.03    &  1.03$\pm$ 0.02 & 1.09 $\pm$ 0.03     \\
\noalign{\smallskip}
\hline
\end{tabular}
\end{table*}
