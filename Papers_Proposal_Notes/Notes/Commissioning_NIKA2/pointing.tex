\section{Pointing accuracy}
\label{se:pointing}
% + RTA pointing estimate method
% + pointing model
% + pointing error (scan-to-scan scattering)

\begin{figure}
\begin{center}
\includegraphics[clip, angle=0, scale = 0.30]{Figures/plot_20170418s192.png}
\caption{Summary plots of the reduction of pointing scan. There is one combined
  map per array to check the overall quality of the scan, and a set of azimuth
  and elevation profiles for one reference detector per array. The 2mm reference
  detector, highlighed in red, is the the pointing reference detector of
  NIKA2. The location of the peak in azimuth and elevation, as observed by the
  reference detector gives the pointing offset of the current scan.}
\label{fig:ptg}
\end{center}
\end{figure}

Based on general operating experience at the 30m telescope, we use the so-called
``pointing'' or ``cross'' scans to monitor the pointing during observations. The
telescope executes a back and forth scan in azimuth and a back and forth scan in
elevation, centered on the observed source. Looking at the timeline profiles of
the reference detector, we fit gaussian profiles and derive the current pointing
offsets of the system in azimuth and elevation. These offsets can then be passed
to PAKO to recenter the next scan (Fig.~\ref{fig:ptg}).

Such scans and their analyses are also used to improve the pointing model
of NIKA2. A pointing session consists in observing about 30 sources on a wide
range of elevations while monitoring the pointing offsets that are measured for
each observation. These offsets and then passed to the IRAM staff who then find
the pointing model parameters that minimize and symetrize the scattering of
these offsets (cf.~Fig.~\ref{fig:ptg_scattering}). Based on these observations,
we see that the pointing scattering of NIKA2 is overall XXX rms.
