
%\documentclass{article}
\documentclass[12pt]{article}
\usepackage[utf8]{inputenc}
\usepackage[english]{babel}

\title{Referee report: NIKA polarization observations of the Crab nebula}
%\author{Alessia Ritacco}
%\date{March 2018}

\begin{document}
\maketitle

\noindent Dear Referee,\\

Thank you very much for your attentive reading and very detailed report. 
To take into account your suggestions we have cross checked all our results
and went deeper in the analysis.
As a consequence, the paper has been entirely revised. Major changes have been made in the introduction and abstract to clarify the scope of the paper and account for
recommendations. Furthermore, we have revised significantly the sections including our results. In particular, we have established a collaboration with 
Planck team (notice that some of the members of our team are also members of Planck) to understand the behaviour of the Crab SED shown by the Planck HFI data. This has lead to a reanalysis of the Planck HFI data using the Planck 2018 data release (notice that this release is expected to have a better handling of polarisation systematics with respect to the 2015 one), which will be soon made public by the Planck team.  The new estimates of the Planck HFI Crab fluxes in intensity and polarisation have been included in this paper and are used to characterize
the Crab nebula SED in temperature and polarisation.

In the following we answer (in boldface) to your questions and comments. 
We have also included in the text your comments to ease the reading.
As indicated before, most of the paper has been rewritten, only section 2 had tiny modifications. Thus, changes in the text has not been boldfaced to avoid confusion. \\  \\
Best regards,\\
A. Ritacco, J.F. Macías-Pérez, N. Ponthieu on behalf of the authors.


\begin{enumerate}
    \item 
{
The Crab Nebula is indeed an important polarized calibration source
for CMB instruments, and efforts to measure its properties and assess
consistency between different instruments across a broad frequency
range is very valuable to the community.
}

{\bf We have made an effort to present our results on the Crab polarisation properties and SED so that they are as valuable as possible for the community.}

{\it For a paper that aims to contribute to the delicate science of
polarization calibration, I would have expected more attention to be
paid to the accounting of systematic errors, and to astrometric
accuracy. In addition, the section on SEDs should have been
straight-forward but instead is marred by confusing discussion of the
models and apparent mistakes in the fitting. }

{\bf A more detailed description of systematics on the NIKA data have been
performed. The SED fitting has been clarified and simplified in according
to the new Planck HFI data.}


In what follows, I discuss the most serious technical issues. I have
additional concerns with the interpretation of the results, but it
does not seem worth discussing these until the numerical reductions
have been revisited.


1. The handling of noise bias, and more generally the discussion of
errors and uncertainties, is not sufficiently sophisticated to meet the
stated goals of this paper (namely that the measurements might be used
for calibration of other instruments). There are two places where
this is most apparent:

In section 3, an effort is made to make statements about the spatially
integrated flux, polarized flux, and polarization angle of the nebula.
This information is potentially valuable for calibration of CMB
instruments, with larger beams. Ultimately the authors settle on a
method where pixels are excluded from the average if the polarized
flux in the pixel is below 3 times the noise level. This is
problematic for several reasons:

(a) Because the polarized intensity is distributed differently than
the unpolarized intensity, cutting on polarized brightness will bias
the measurement towards parts of the nebula with a pixels with a
higher polarization degree (p). This will inflate the "average" p
relative to what an experiment with a larger beam would be able to
observe.

(b) The reason given for the exclusion is to avoid "noise bias". The
noise bias is described earlier in the paper and is indeed important
when converting low S/N measurements of I,Q,U into estimates of p and
polarization angle (psi). But to compute integrated p and psi values
correctly, one should first obtain integrated values of I, Q, and U;
these will have quite high S/N, compared to individual pixel S/N, and
noise bias will not be a concern.

(c) This exclusion is hardly a minor technical point, since it changes
the inferred polarization angle by 3.5 degrees and the polarization
fraction by more than 1 percentage point. For the polarization angle,
at least, this change is much larger than the total uncertainty.

Although the authors perform this computation both with and without
the exclusion of S/N \textless 3 points, the values they carry forward, and
use in summary sentences and plots, are the biased ones (using S/N \textgreater 3
only). One suspects that the S/N \textgreater 3 results were ultimately chosen
because they agree better with the results from other experiments
(WMAP and Planck). The authors mention the need for an "additional
uncertainty at a level of 1\% on p and 4 deg on psi", but this
uncertainty is not mentioned again; it is not included in tables,
plots, or summary sentences.


Secondly, the authors often quote numbers with very small statistical
errors. For example, for one of the averages in section 3, the mean
polarization angle has an error of 0.0004 degrees. This suggests
either a numerical mistake, or that the map noise has been
treated as white when in fact it has significant correlated structure
on angular scales larger than the pixel. In the latter case, a
profitable approach is to check for consistency between different
subsets of the data (e.g., look at two co-adds of 12 maps instead of
the one co-add of 24 maps).

\textbf{As you indicated we were considering only statistical errors that underestimate the overall uncertainties. Following your suggestions we use jack-knife maps (null maps) to estimate the uncertainties in addition to the statistical ones. Moreover we also consider possible systematic errors induced by the
intensity to polarisation leakage correction. The calibration error is also considered in the estimation of the flux densities. As a consequence, this section has been completely revised. Furthermore, we have used aperture photometry methods to compute Stokes I,Q and U flux densities over the full extension of the Crab nebula so that we can properly compare with low resolution experiments. This new analysis has led to significantly larger uncertainties. As a consequence we have modified our conclusions with respect to the accuracy and implications of the NIKA results. }

\item  Basic errors in astrometry and coordinate handling.

The position of the Crab Pulsar is given, to sufficient precision, in
both the Introduction and the Conclusion. But the position of the
pulsar is not plotted correctly in any of the images that purport to
do so. Rather, the position appears to have been truncated to 0.01
degrees prior to plotting.

\textbf{The position of the Crab Pulsar was indeed plotted, by mistake, in a wrong position on the maps. The issue occurred due to a mistake in our plotting tools when zooming in. This issue has been corrected and all the values in the tables have been recalculated.} 

The position angle (PA) conversion between galactic and equatorial
coordinates is not done correctly. In tables 1 and 2 we find that the
Equatorial and Galactic PA differ by values ranging between 220 to 245
degrees, even though all of these points lie in a very small region of
the sky.
{\bf
For the conversion of the polarisation vectors between Equatorial and Galactic coordinates we have checked and modified if necessary all presented results. We have actually implemented different conversion tools and found consistent results between them. Furthermore, we have cross checked our results by recomputing the conversion for the data of other experiments like for example WMAP.}

\item Models are not explained clearly, and fits contain obvious errors.

(a) The intensity model described in equations (6) and (7) is not
completely described. It appears to be discontinuous, through 100
GHz, and the best-fit values are only given for 2 of the 4 model
parameters. Fig. 6 shows a very reasonable broken power-law fit to
the data, but the description in the text is inadequate and confusing.

(b) In 4.2, models are fit to the polarized intensity from WMAP and
Planck. It is alleged that only data below 100 GHz are used; this
raises the question of whether or not the Planck measurement at 100
GHz is included. The best-fit parameters for Planck, stated in
equation (9) and plotted in Fig. 7, do not appear to be the best fit
to these data. Indeed, if I take the Planck measurements and fit a
simple power-law, I find a very different best-fit line: A = 91.8 and
beta = -0.38.

These errors and/or omissions make it difficult to take any of the
model fitting seriously; they undermine the conclusions of the paper.

\textbf{As indicated before we have reanalysed the Planck HFI data that are now consistent with a single power law synchrotron model. As a consequence this section has been completely rewritten. Assuming a single power law model both in temperature and polarisation we have computed the best fits to the full available data set. The spectral index of the power law in polarisation is found to be consistent with the one for total intensity.}

\end{enumerate}

\end{document}
