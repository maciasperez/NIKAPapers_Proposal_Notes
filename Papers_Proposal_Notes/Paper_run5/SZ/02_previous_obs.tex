The object RX~J1347.5-1145 is among the clusters that have been intensively observed at several wavelengths and the most widely studied using tSZ at sub-arcmin resolution. It is a massive intermediate redshift galaxy cluster at $z = 0.4516$ undergoing a merging event.

This cluster is the most luminous \mbox{X-ray} cluster of galaxies known to date \citep[{\it e.g.}][]{allen_2002}. It was discovered in the {\it ROSAT} \mbox{X-ray} all-sky survey \citep{voges1999} and has been the object of many studies in \mbox{X-ray} \citep{schindler_1995, schindler_1997, allen_2002, gitti_2004, gitti_2005, gitti_2007, gitti_2007_bis, ota_2008}, optical \citep{cohen_kneib_2002, verdugo_2012}, infrared \citep{zemcov2007}, tSZ \citep{pointecouteau_1999, komatsu1999, pointecouteau_2001, komatsu_2001, kitayama_2004, mason_2010, korngut_2011, zemcov2012, plagge_2012}, and multiwavelength analysis \citep{bradac_2008, miranda_2008, johnson_2012}. From {\it ROSAT} \mbox{X-ray} observations, this cluster was thought to be a dynamically old relaxed cool-core cluster with an extremely strong cooling flow, due to its very spherical morphology and peaked \mbox{X-ray} profile \citep[{\it ROSAT};][]{schindler_1995, schindler_1997}. However, high angular resolution tSZ observations have proved RX~J1347.5-1145 to be an ongoing merger due to the measurement of an extension toward the southeast (SE) with respect to the \mbox{X-ray} center \citep{pointecouteau_1999, komatsu_2001, kitayama_2004}. This illustrates how tSZ and \mbox{X-ray} (and other wavelengths) observations are complementary. More recent \mbox{X-ray} \citep[{\it Chandra};][]{allen_2002} and lensing \citep{miranda_2008} observations are consistent with this interpretation and show a clear detection of the SE extension.

High resolution tSZ maps of RX~J1347.5-1145, such as the 90~GHz 8~arcsec (smoothed to 10~arcsec) resolution map of {\it MUSTANG} \citep{mason_2010}, have confirmed the presence of a strong SE extension. It is interpreted as being due to a hot gas that is heated by the merging of a subcluster crossing the main, originally relaxed, system from the south to the northeast (NE), which is perpendicular to the line-of-sight. The SE extension coincides with a radio mini-halo~\citep{gitti_2007_bis}, which indicates the presence of non-thermal electrons, that underlies a non-thermal contribution to the total pressure. Optical observations have also confirmed this scenario with the detection of a massive elliptical galaxy, which is located 20~arcsec on the east side of the \mbox{X-ray} center, while the central elliptical galaxy of the main cluster remains at the \mbox{X-ray} peak location \citep{cohen_kneib_2002}. 

The temperature profile of RX~J1347.5-1145 varies from $\sim$~6~keV in its core to $\sim$~20~keV at 80~arcsec and decreases to $\sim$~9~keV on the outer part of the cluster (120--300~arcsec form the core). The maximum temperature is located at the SE extension, reaching $k_B T_e \sim~25$~keV \citep{ota_2008}. The Compton $y$ parameter has been measured to be $y_{\mathrm{max}} \simeq~10^{-3}$ \citep{pointecouteau_1999}.

The object RX~J1347.5-1145 hosts a well-known radio source within 3~arcsec of the \mbox{X-ray} center in the central elliptical galaxy. Due to this contamination, the location of the tSZ maximum is still debated. Current single dish observations are consistent with the tSZ emission of the SE extension being stronger than that at the cluster \mbox{X-ray} center. However, taking advantage of the intrinsic point source removal power of interferometric data, \cite{plagge_2012} claim that it is only a secondary maximum. The point source has to be taken into account in the tSZ analysis. According to~\cite{pointecouteau_2001}, the source follows the spectrum $F_{\nu} = \left(77.8 \pm 1.7\right) \nu_{\mathrm{GHz}}^{-0.58 \pm 0.01}$ mJy. For {\it NIKA}, this corresponds to $4.4 \pm 0.3$ and $3.2 \pm 0.2$ mJy at 140 and 240~GHz, respectively.

Finally, in addition to the central radio source, \cite{zemcov2007} have reported the presence of two infrared galaxies. The first one (Z1 hereafter) is located at about 60~arcsec from the \mbox{X-ray} center in the southwest direction with a flux of  15.1 mJy (as measured with a signal to noise of 5.1) at 850 $\mu$m and 125 $\pm$ 34  mJy at 450 $\mu$m. The second source (Z2 hereafter) is located closer to the \mbox{X-ray} center at about 20~arcsec on the northeast side. However, it is only detected at 850 $\mu$m with a flux of 11.4 mJy (measured with a signal to noise of 4.7). The best-fit value at 450 $\mu$m is 10  $\pm$ 32 mJy. The contamination of these sources in the {\it NIKA} bands is estimated and accounted for in our analysis, as discussed in Sect.~\ref{sec:2band_decor}.
