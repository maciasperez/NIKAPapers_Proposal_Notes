\documentclass[iop,numberedappendix,apj]{emulateapj}
%\documentclass[iop,numberedappendix,apj]{aastex}
%\documentclass{article}
%\usepackage{emulateapj}
%\pdfoutput=1
\usepackage{color}
\usepackage{amssymb}
\usepackage{natbib}
\usepackage{graphicx}
\usepackage{epsfig}
\usepackage{url}
%\usepackage{lscape}
\usepackage{afterpage}

\usepackage[tbtags]{amsmath}
\usepackage{hyperref,xcolor}
\hypersetup{colorlinks,linkcolor={blue!50!black},citecolor={blue!50!black},urlcolor={blue!80!black}}
\setlength{\tabcolsep}{0.04in} 
%\usepackage{epsfig}
%\usepackage{fullpage}
%\usepackage{hyperref}
%%%%%%%%%%%%%%%%%%%%%%%%%%%%%%%%%%%%%%%%%%%%%%%%%%%%%%%%%%%%%%%%%%%%%%%%%%%%%%%
%%% NOTES ON COMPILING / PRINTING THIS DOCUMENT
%%% do the latex <filename>
%%% dvips -O 0cm,2.0cm <filename.dvi>
%%% dvips -O 0cm,2.0cm clash_pressures.dvi

\newcommand {\apgt} {\ {\raise-.5ex\hbox{$\buildrel>\over\sim$}}\ }
\newcommand {\aplt} {\ {\raise-.5ex\hbox{$\buildrel<\over\sim$}}\ }
\newcommand{\dmod}{\overrightarrow{d}_{mod}}
\newcommand{\dvec}{\overrightarrow{d}}
\newcommand{\avec}{\overrightarrow{a}}
\newcommand{\asec}{$^{\prime \prime}$}
\newcommand{\asecs}{$^{\prime \prime}\ $}
\newcommand{\amin}{$^{\prime}$}
\newcommand{\amins}{$^{\prime}\ $}
\newcommand{\sigT}{\mbox{$\sigma_{\mbox{\tiny T}}$}}
\newcommand{\Tcmb}{\mbox{$T_{\mbox{\tiny CMB}}$}}
\newcommand{\kB}{\mbox{$k_{\mbox{\tiny B}}$}}
\newcommand{\kBT}{\mbox{$k_{\mbox{\tiny B}}T_{\mbox{\tiny e}}$}}
\newcommand{\nH}{\mbox{$n_{\mbox{\tiny H}}$}}
\newcommand{\NH}{\mbox{$N_{\mbox{\tiny H}}$}}
\newcommand{\LameH}{\mbox{$\Lambda_{e \mbox{\tiny H}}$}}
\newcommand{\Lamee}{\mbox{$\Lambda_{ee}$}}
\newcommand{\rhogas}{\mbox{$\rho_{\mbox{\scriptsize gas}}$}}
\newcommand{\rhotot}{\mbox{$\rho_{\mbox{\scriptsize tot}}$}}
\newcommand{\Mgas}{\mbox{$M_{\mbox{\scriptsize gas}}$}}
\newcommand{\Mtot}{\mbox{$M_{\mbox{\scriptsize tot}}$}}
\newcommand{\Mvir}{\mbox{$M_{\mbox{\scriptsize vir}}$}}
\newcommand{\Yint}{\mbox{$Y_{\mbox{\scriptsize int}}$}}
\newcommand{\Ycyl}{\mbox{$Y_{\mbox{\scriptsize cyl}}$}}
\newcommand{\Ysph}{\mbox{$Y_{\mbox{\scriptsize sph}}$}}
\newcommand{\fgas}{\mbox{$f_{\mbox{\scriptsize gas}}$}}
\newcommand{\LCDM}{\mbox{$\Lambda$CDM}}
\newcommand{\Pe}{\mbox{$P_{\mbox{\scriptsize e}}$}}
\newcommand{\msun}{$M_{\odot}$}
\newcommand{\etal}{{\it et al.}}
\newcommand{\mJy}{\,{\rm mJy} }
\newcommand{\um}{\,\mu {\rm m} }
\newcommand{\mJySr}{\,{\rm MJy/Sr} }
\newcommand{\mJyBm}{\,{\rm mJy/Bm} }
\newcommand{\mK}{\,{\rm mK} }
\newcommand{\K}{\,{\rm K} }
\newcommand{\uJy}{\,{\rm \mu Jy} }
\newcommand{\uK}{\,{\rm \mu K} }
\newcommand{\kHz}{\, {\rm kHz} }
\newcommand{\eg}{{\it e.g.}}
\newcommand{\ie}{{\it i.e.}}
\newcommand{\etc}{{\it etc.}}
\newcommand{\aips}{{\tt AIPS++}}
\newcommand{\nusp}{\nu_{sp}}
\newcommand{\ghz}{{\, \rm GHz}}
\newcommand{\db}{{\, \rm dB}}
\newcommand{\degsqr}{\, {\rm deg^2}}
\newcommand{\Tew}{\mbox{$T_{\mathrm{ew}}$}}
\newcommand{\Tspec}{\mbox{$T_{\mathrm{spec}}$}}
\newcommand{\chandra}{{\it Chandra}}
\newcommand{\asca}{{ASCA}}
\newcommand{\wmap}{{WMAP}}
\newcommand{\rosat}{{ROSAT}}
\newcommand{\xmm}{{XMM-{\it Newton}}}
\newcommand{\planck}{{\it Planck}}
\newcommand{\hubble}{{\it Hubble}}
\newcommand{\rxj}{RX J1347.5-1145}
\newcommand{\clj}{CL J1226.9+3332}
\newcommand{\macsa}{MACS J0647.7+7015}
\newcommand{\macsb}{MACS J1206.2-0847}
\newcommand{\macsc}{MACS J0717.5+3745}
\newcommand{\macsd}{MACS J1423.8+2404}
\newcommand{\macse}{MACS J0329.6-0211}
\newcommand{\macsf}{MACS J0429-0253}
\newcommand{\macsg}{MACS J0744.9+3927}
\newcommand{\macsh}{MACS J1149+2223}
\newcommand{\macsi}{MACS J1115+0130}
\newcommand{\Tx}{\mbox{$T_{\mbox{\tiny X}}$}}
\newcommand{\te}{\mbox{$T_{\mbox{\tiny e}}$}}
\newcommand{\mec}{\mbox{$m_{\mbox{\tiny e}} c^2$}}
\newcommand{\dene}{\mbox{$n_{\mbox{\tiny e}}$}}
\newcommand{\denesq}{\mbox{$n^2_{\mbox{\tiny e}}$}}
\newcommand{\yx}{\mbox{$Y_{\mbox{\tiny X}}$}}
\newcommand{\ysze}{\mbox{$Y_{\mbox{\tiny tSZE}}$}}
\newcommand{\sx}{\mbox{$S_{\mbox{\tiny X}}$}}
\newcommand{\Itsz}{\mbox{$I_{\mbox{\tiny tSZE}}$}}
\newcommand{\Iksz}{\mbox{$I_{\mbox{\tiny kSZE}}$}}
\newcommand{\chisq}{\mbox{$\chi^{2}$}}
\newcommand{\chired}{\mbox{$\chi^{2}_{red}$}}


\slugcomment{}
\shortauthors{Romero \etal}
\shorttitle{Joint SZE Map Fitting with MUSTANG and Bolocam}
%altaffilmark{#}

\begin{document}

\title{Galaxy Cluster Pressure Profiles as Determined by Sunyaev Zel'dovich Effect 
  Observations with MUSTANG and Bolocam II: Joint Analysis of Fourteen Clusters}
\author{
%  Order TBD ,
  Charles E. Romero\altaffilmark{1,2},
  Brian S. Mason\altaffilmark{3},
  Jack Sayers\altaffilmark{4}
  Tony Mroczkowski\altaffilmark{5,6},
  Tracy E. Clarke\altaffilmark{6},
  Craig Sarazin\altaffilmark{7},
  Alexander H.\ Young\altaffilmark{8},
  Jonathan Sievers\altaffilmark{9},
  Simon R. Dicker\altaffilmark{10},
  Erik D.\ Reese\altaffilmark{11},
  Nicole Czakon \altaffilmark{4,12},
  Mark Devlin\altaffilmark{10},
  Phillip M.\ Korngut\altaffilmark{4},
  Sunil Golwala\altaffilmark{4}
} 
\date{\today}

%%%%%%%%%%%%%%%%%%%%%%%%%%%%%%%%%%%%%%%%%%%%%%%%%%%%%%%%%%%%%%%%%%%%%%%%%%%%%%%
\altaffiltext{1}{Institut de Radioastronomie Millim\'{e}trique
300 rue de la Piscine, Domaine Universitaire
38406 Saint Martin d'H\`{e}res, France} 
\altaffiltext{2}{Author contact: \email{romero@iram.fr}}
\altaffiltext{3}{National Radio Astronomy Observatory, 520 Edgemont Rd.,
Charlottesville, VA 22904, USA}
\altaffiltext{4}{Department of Physics, Math, and Astronomy,
  California Institute of Technology, Pasadena, CA 91125, USA}
\altaffiltext{5}{National Research Council Fellow} 
\altaffiltext{6}{U.S.\ Naval Research Laboratory,
  4555 Overlook Ave SW, Washington, DC 20375, USA}
\altaffiltext{7}{Department of Astronomy, University of Virginia,
  P.O. Box 400325, Charlottesville, VA 22904, USA}
\altaffiltext{8}{MIT Lincoln Laboratories}
\altaffiltext{9}{Astrophysics \& Cosmology Research Unit, University of KwaZulu-Natal,
  Private Bag X54001, Durban 4000, South Africa}
\altaffiltext{10}{Department of Physics and Astronomy, University of
  Pennsylvania, 209 South 33rd Street, Philadelphia, PA, 19104, USA}
\altaffiltext{11}{Department of Physics, Astronomy, and Engineering, 
  Moorpark College, 7075 Campus Rd., Moorpark, CA 93021, USA} 
\altaffiltext{12}{Academia Sinica, 128 Academia Road, Nankang, Taipei 115, Taiwan}
%\begin{document}

%%%%%%%%%%%%%%%%%%%%%%%%%%%%%%%%%%%%%%%%%%%%%%%%%%%%%%%%%%%%%%%%%%%%%%%%%%%%%%%

\begin{abstract}
We present pressure profiles of galaxy clusters determined from high resolution 
Sunyaev-Zel'dovich (SZ) effect observations of fourteen clusters, which span the
redshift range $ 0.25 < z < 0.89$. 
%We compare our pressure profiles to those from the X-ray data presented by the ACCEPT 
%collaboration, both under the assumption of spherical symmetry. 
The fitting procedure simultaneously fits spherical cluster models to MUSTANG and 
Bolocam data. In this analysis, we adopt the 
generalized NFW parameterization of pressure profiles to produce our models.
Our constraints on pressure profile parameters, in this study $\gamma$, $C_{500}$, and $P_0$,
are consistent with those in previous studies, but for individual clusters we find discrepancies 
with the X-ray derived pressure profiles from the ACCEPT database. 
We investigate potential sources of these discrepancies, especially in the
context of cluster geometry, electron temperature of the intracluster medium, and
substructure.
We find that the ensemble mean profile for all clusters in our sample is described by the parameters: 
$[\gamma,C_{500},P_0] = [0.3,1.3,7.94]$, for cool core clusters: $[\gamma,C_{500},P_0] = [0.6,0.9,3.55]$,
and for disturbed clusters: $[\gamma,C_{500},P_0] = [0.0,1.5,12.6]$.
 Three of the fourteen clusters have established substructure, while an additional
two clusters exhibit potential substructure.
\end{abstract}

\keywords{galaxy clusters: general --- galaxy clusters}

\maketitle

%%%%%%%%%%%%%%%%%%%%%%%%%%%%%%%%%%%%%%%%%%%%%%%%%%%%%%%%%%%%%%%%%%%%%%%%%%%%%%%
\section{Introduction}
\label{sec:intro}
%%%%%%%%%%%%%%%%%%%%%%%%%%%%%%%%%%%%%%%%%%%%%%%%%%%%%%%%%%%%%%%%%%%%%%%%%%%%%%%

\textcolor{red}{[PREFACE: I have added some comments throughout this draft in bracketed and red text. I think I've added
most comments in the relevant places. That said, I have yet to do a general revision of text in
Section~\ref{sec:ind_notes}. I have left some figure references in that section, even if the figures don't exist in
this file. The question is whether to include them.]}

Galaxy clusters are the largest gravitationally bound objects in the universe and thus serve as ideal cosmological probes 
and astrophysical laboratories. Within a galaxy cluster, the gas in the intracluster medium (ICM) constitutes 90\% of the
baryonic mass \citep{vikhlinin2006b} and is directly observable in the X-ray due to bremsstrahlung emission. 
At millimeter and sub-millimeter wavelengths, the ICM is observable via the Sunyaev-Zel'dovich effect (SZE) 
\citep{sunyaev1972}: the inverse Compton scattering of cosmic microwave background (CMB) photons off of
the hot ICM electrons. The thermal SZE is observed as an intensity decrement relative to the CMB at wavelengths longer 
than $\sim$1.4 mm (frequencies less than $\sim$220 GHz). The amplitude of the thermal SZE is proportional to the integrated
line-of-sight electron pressure, and is often parameterized as Compton y: $y = (\sigma_T / m_e c^2) \int P_e dl$, where
$\sigma_T$ is the Thomson cross section, $m_e$ is the electron mass, $c$ is the speed of light, and $P_e$ is the electron
pressure.
%At longer radio wavelengths, if relativistic electrons are present, parts of the ICM may emit synchrotron emission.

%\textcolor{red}{[I need to revise this paragraph.]}
Cosmological constraints are generally limited by the accuracy of mass calibration of galaxy clusters 
\citep[e.g.][]{hasselfield2013, reichardt2013}, which is often calculated via a scaling relation with respect to some 
integrated observable quantity. Scatter in the scaling relations will then depend on the regularity of
clusters and the adopted intergration radius of the clusters. 
Determining pressure profiles of galaxy clusters provides an assessment of 
the relative impact and frequency of various astrphysical processes in the ICM and can refine the choice of extent of 
galaxy clusters to reduce the scatter in scaling relations.

In the core of a galaxy cluster, some observed astrophysical processes include shocks and cold fronts 
\citep[e.g.][]{markevitch2007}, sloshing \citep[e.g.][]{fabian2006}, and X-ray cavities \citep{mcnamara2007}. 
It is also theorized that helium sedimentation should occur, most noticeably in low redshift, dynamically-relaxed 
clusters \citep{abramopoulos1981, gilfanov1984} 
and recently the expected helium enhancement via sedimentation has been numerically simulated \citep{peng2009}. 
This would result in an offset between X-ray and SZE derived pressure profiles.

At large radii ($R \gtrsim R_{500}$),\footnote{$R_{500}$
is the radius at which the enclosed average mass density is 500 times the critical density, 
$\rho_c(z)$, of the universe} equilibration timescales are longer, accretion is ongoing, 
and hydrostatic equilibrium (HSE) is a poor approximation. 
Several numerical simulations show that the fractional contribution
 from non-thermal pressure increases with radius \citep{shaw2010,battaglia2012,nelson2014}. 
For all three studies, non thermal pressure fractions between 15\% and 30\% are found at ($R \sim R_{500}$)
for redshifts $0 < z < 1$. However, the intermediate radii, between the core and outer regions of the 
galaxy cluster, offer a region where self-similar scalings derived from HSE can be used to describe simulations 
and observations \citep[e.g.][]{kravtsov2012}. Moreover, both simulations and observations find low
cluster-to-cluster scatter in pressure profiles within this intermediate radial range \citep[e.g.][]{borgani2004,
nagai2007,arnaud2010,bonamente2012,planck2013a,sayers2013}.

While many telescopes capable of making SZE observations are already operational or are being built, most have
angular resolutions of one arcminute or larger. The MUSTANG camera \citep{dicker2008}
on the 100 meter Robert C. Byrd Green Bank Telescope \citep[GBT, ][]{jewell2004} with its angular resolution of 9\asec 
(full-width, half-maximum FWHM) is one of only a few SZE instruments with sub-arcminute resolution.
However, MUSTANG's instantaneous field of view (FOV) limits its sensitivity to scales larger than $1$\amin. 
To probe a wider range of scales we complement our MUSTANG data with SZE data from Bolocam \citep{glenn1998}. 
Bolocam is a 144-element bolometer
array on the Caltech Submillimeter Observatory (CSO) with a beam FWHM of 58\asecs at 140 GHz and circular FOV with 8\amins 
diameter, which is well matched to the angular size of $R_{500}$ ($\sim 4$\amin) of the clusters in our sample. 

This paper is organized as follows. In Section~\ref{sec:obs} we describe the MUSTANG and Bolocam observations and reduction. 
In Section~\ref{sec:jointfitting} we review the method used to jointly fit pressure profiles to MUSTANG and Bolocam data. We
present results from the joint fits in Section~\ref{sec:pp_constraints} and compare our results to X-ray derived pressures 
in Section~\ref{sec:xray_comp}. 
Throughout this paper we assume a $\Lambda$CDM cosmology with $\Omega_m = 0.3$, $\Omega_{\lambda} = 0.7$, and $H_0 = 70$ 
km s$^{-1}$ Mpc$^{-1}$.
%consistent with the 9-year \emph{Wilkinson Microwave Anisotropy Probe} (WMAP) results reported in \cite{hinshaw2013}.

%%%%%%%%%%%%%%%%%%%%%%%%%%%%%%%%%%%%%%%%%%%%%%%%%%%%%%%%%%%%%%%%%%%%%%%%%%%%%%%
\section{Observations and Data Reduction}
\label{sec:obs}
%%%%%%%%%%%%%%%%%%%%%%%%%%%%%%%%%%%%%%%%%%%%%%%%%%%%%%%%%%%%%%%%%%%%%%%%%%%%%%%

\subsection{Sample}

Our cluster sample is based primarily on the Cluster Lensing And Supernova survey with Hubble (CLASH) sample, which is a
524-orbit multi-cycle treasury program. 
%One of its main goals is to ``measure the profiles and substructures of dark matter 
%in galaxy clusters with unprecedented precision and resolution'' \citep{postman2012}. 
The CLASH sample has 25 massive galaxy 
clusters, 20 of which are selected from X-ray data (from Chandra X-ray Observatory), and 5 based on exceptional lensing strength. 
These clusters have the following properties: $0.187 < z < 0.890$, $5.5 < k T_e$ (keV)$ < 15.5$, and $6.7 < L_{bol} / 10^{44} $ 
(erg s$^{-1}$) $<90.8$. Thus, these clusters are large enough that we should expect to detect them with MUSTANG with a reasonable 
amount of time on the sky (on average, $<$25 hours per cluster).

%While these clusters are not a complete sample, many already have SZ effect observations from the Sunyaev-Zel'dovich 
%Array (SZA), AMiBA, or Bolocam, making them well studied, and deserving of high resolution SZ effect measurements. 
%The wealth of observations on these clusters will allow us to constrain pressure and 
%mass profiles of clusters as well as the impact of substructure. Additionally, we will be able to assess discrepancies
%between X-ray derived properties, shown in Table~\ref{tbl:cluster_properties} 
%and compare to SZ derived properties. 

\begin{figure*}[!h]
  \centering
  \begin{tabular}{cccc}
   \epsfig{file=figures/MBO_Contours_a1835_xray_14_Feb_2016.eps,width=0.25\linewidth,clip=}   &
   \epsfig{file=figures/MBO_Contours_a611_xray_14_Feb_2016.eps,width=0.25\linewidth,clip=}    &
   \epsfig{file=figures/MBO_Contours_m1115_xray_14_Feb_2016.eps,width=0.25\linewidth,clip=}    &
   \epsfig{file=figures/MBO_Contours_m0429_xray_14_Feb_2016.eps,width=0.25\linewidth,clip=}    \\
   \epsfig{file=figures/MBO_Contours_m1206_xray_14_Feb_2016.eps,width=0.25\linewidth,clip=}    &
   \epsfig{file=figures/MBO_Contours_m0329_xray_14_Feb_2016.eps,width=0.25\linewidth,clip=}    &
   \epsfig{file=figures/MBO_Contours_rxj1347_xray_14_Feb_2016.eps,width=0.25\linewidth,clip=}    &
   \epsfig{file=figures/MBO_Contours_m1311_xray_14_Feb_2016.eps,width=0.25\linewidth,clip=}    \\
   \epsfig{file=figures/MBO_Contours_m1423_xray_14_Feb_2016.eps,width=0.25\linewidth,clip=}    &
   \epsfig{file=figures/MBO_Contours_m1149_xray_14_Feb_2016.eps,width=0.25\linewidth,clip=}    &
   \epsfig{file=figures/MBO_Contours_m0717_xray_14_Feb_2016.eps,width=0.25\linewidth,clip=}    &
   \epsfig{file=figures/MBO_Contours_m0647_xray_14_Feb_2016.eps,width=0.25\linewidth,clip=}    \\
   \epsfig{file=figures/MBO_Contours_m0744_xray_14_Feb_2016.eps,width=0.25\linewidth,clip=}    &
   \epsfig{file=figures/MBO_Contours_clj1226_xray_14_Feb_2016.eps,width=0.25\linewidth,clip=}    &
     &
  \end{tabular}
  \caption{MUSTANG maps of the clusters in our sample. Pale contours are MUSTANG contours;
    blue contours are Bolocam. Both start at $3\sigma$ decrement, with $1\sigma$ intervals for MUSTANG
    and $2\sigma$ intervals for Bolocam.
    Red contours are X-ray surface brightness contours at arbitrary levels. The red asterisk is the 
    ACCEPT centroid; the blue asterisk is the Bolocam centroid.}
  \label{fig:mustang_maps_sample}
\end{figure*}

\begin{deluxetable*}{lllllllllll}
\tabletypesize{\scriptsize}
\tablecolumns{10}
\tablewidth{0pt} 
\tablecaption{CLASH cluster properties \label{tbl:cluster_properties}}
\tablehead{ 
    \colhead{Cluster} & \colhead{$z$} & \colhead{$M_{500}$} & \colhead{$P_{500}$} & \colhead{$R_{500}$} & \colhead{$T_x^1$} 
              & \colhead{$T_x^2$} & \colhead{$T_{mg}$} & \colhead{Dynamical} & \colhead{$\Delta r_0$} \\
              \colhead{}  & \colhead{} & \colhead{($10^{14} M_{\odot}$)} & \colhead{(keV/cm$^{3}$)} & \colhead{(kpc)} & 
              \colhead{(keV)} & \colhead{(keV)} & \colhead{(keV)} & \colhead{state} & \colhead{(\asec)}
}
\startdata
    \textbf{Abell 1835}  & 0.253 & 12   & 0.00594   & 1490   & 9.0  & 10.0 & 8.17 & CC      & 6.8   \\
    \textbf{Abell 611}   & 0.288 & 7.4  & 0.00445   & 1240   & 6.8  & --   & 6.55 & --      & 18.7  \\
    \textbf{MACS1115}    & 0.355 & 8.6  & 0.00545   & 1280   & 9.2  & 9.14 & 6.47 & CC      & 34.8  \\
    \textbf{MACS0429}    & 0.399 & 5.8  & 0.00448   & 1100   & 8.3  & 8.55 & 4.32 & CC      & 18.7  \\
    \textbf{MACS1206}    & 0.439 & 19   & 0.01059   & 1610   & 10.7 & 11.4 & 8.81 & --     & 6.9   \\
    \textbf{MACS0329}    & 0.450 & 7.9  & 0.00596   & 1190   & 6.3  & 5.85 & 5.05 & CC \& D & 14.8  \\
    \textbf{RXJ1347}     & 0.451 & 22   & 0.01171   & 1670   & 10.8 & 13.6 & 7.99 & CC      & 9.6   \\
    \textbf{MACS1311}    & 0.494 & 3.9  & 0.00399   & 930    & 6.0  & 6.36 & 4.70 & CC      & 27.7  \\
    \textbf{MACS1423}    & 0.543 & 6.6  & 0.00612   & 1090   & 6.9  & 6.81 & 4.14 & CC      & 19.8  \\
    \textbf{MACS1149}    & 0.544 & 19   & 0.01228   & 1530   & 8.5  & 8.76 & 7.29 & D       & 6.0   \\
    \textbf{MACS0717}    & 0.546 & 25   & 0.01490   & 1690   & 11.8 & 10.6 & 7.70 & D       & 32.4  \\
    \textbf{MACS0647}    & 0.591 & 11   & 0.00923   & 1260   & 11.5 & 12.6 & 9.97 & --      & 6.9   \\
    \textbf{MACS0744}    & 0.698 & 13   & 0.01199   & 1260   & 8.1  & 8.90 & 7.34 & D       & 4.9   \\
    \textbf{CLJ1226}     & 0.888 & 7.8  & 0.01184   & 1000   & 12.0 & 11.7 & 8.39 & --      & 15.3  \\
    \hline
    Abell 383            & 0.187 & 4.7  & 0.00285   & 1110   & 5.4  & 5.47 & --   & CC      & --    \\
    Abell 209            & 0.206 & 13   & 0.00564   & 1530   & 8.2  & 8.69 & --   & --      & --    \\
    Abell 1423           & 0.213 & 8.7  & 0.00445   & 1350   & 5.8  & 6.61 & --   & --      & --    \\
    Abell 2261           & 0.224 & 14   & 0.00632   & 1590   & 6.1  & 8.09 & --   & CC      & --    \\
    RXJ2129              & 0.234 & 7.7  & 0.00423   & 1280   & 6.3  & 7.78 & --   & CC      & --    \\
    MS 2137              & 0.313 & 4.7  & 0.00342   & 1060   & 4.7  & --   & --   & CC      & --    \\
    RXC J2248            & 0.348 & 22   & 0.01014   & 1760   & 10.9 & 11.5 & --   & --      & --    \\
    MACS1931             & 0.352 & 9.9  & 0.00595   & 1340   & 7.5  & 7.92 & --   & CC      & --    \\
    MACS1532             & 0.362 & 9.5  & 0.00589   & 1310   & 6.8  & 6.47 & --   & CC      & --    \\
    MACS1720             & 0.387 & 6.3  & 0.00465   & 1140   & 7.9  & 6.50 & --   & CC      & --    \\
    MACS0416             & 0.397 & 9.1  & 0.00625   & 1270   & 8.2  & 8.14 & --   & --      & --    \\
    MACS2129             & 0.570 & 11   & 0.00903   & 1250   & 8.6  & 8.11 & --   & D       & --    
\enddata
\tablecomments{$z$, $M_{500}$, and $T_X^1$ are taken as tabulated in \citet{mantz2010}:  $T_X^1$ is calculated from a 
  single spectrum over $0.15 R_{500} < r < R_{500}$ for each cluster. $T_X^2$ is from \citet{morandi2015},
  and is calculated over $0.15 R_{500} < r < 0.75 R_{500}$.  $T_{mg}$ is a fitted gas mass weighted temperature,
  determined by fitting the ACCEPT \citep{cavagnolo2009} temperature profiles to the profile found in
  \citet{vikhlinin2006}. The bolded clusters are the 14 clusters in our sample.
  $\Delta r_0$ denotes the offset between the ACCEPT and Bolocam centroids. Dynamical states of clusters,
  cool core (CC) or disturbed (D), are determined in \citet{sayers2013}.}
\end{deluxetable*}

Of the 25 clusters in the CLASH sample, four are too far south to be observed with MUSTANG from Green Bank, WV.
Of the remaining 21, we were able to observe fourteen given the available good weather and their limited visibility
during the observational campaign from 2008 to 2014.
Abell 209 was observed, but was relatively noisy and showed no trace of any detection. Thus, our final sample
includes thirteen CLASH clusters and Abell 1835, which are shown in bold in Table~\ref{tbl:cluster_properties}. These clusters
were also observed with Bolocam, and have been analyzed in \citet{sayers2012, sayers2013,czakon2014}. The centroid differences
between the \emph{Archive of Chandra Cluster Entropy Profile Tables} \citep[ACCEPT][]{cavagnolo2009}) 
and Bolocam ($\Delta r_0$) are also listed as  in Table~\ref{tbl:cluster_properties}. The total integration times of
MUSTANG and Bolocam observations of our sample is listed in Table~\ref{tbl:cluster_obs}. 

\begin{deluxetable*}{lllllllllll}
\tabletypesize{\footnotesize}
\tablecolumns{10}
\tablewidth{0pt} 
\tablecaption{Bolocam and MUSTANG observational properties. \label{tbl:cluster_obs}}
\tablehead{ 
    \colhead{Cluster} & \colhead{$z$} & \colhead{R.A.} & \colhead{Decl.} & 
              \colhead{$T_{obs,B}$} & \colhead{Noise$_{B}$} & \colhead{A10$_{B}$} & 
              \colhead{$T_{obs,M}$} & \colhead{Noise$_{M}$} & \colhead{A10$_{M}$}    \\
            & \colhead{} & \colhead{(J2000)} & \colhead{(J2000)} &  
              \colhead{(hours)} & \colhead{$\mu K_{CMB}$-amin} & \colhead{($\sigma$)} &
              \colhead{(hours)} & \colhead{$\mu$Jy/bm}        & \colhead{($\sigma$)}
}
\startdata
    \textbf{Abell 1835}  & 0.253 & 14:01:01.9 & +02:52:40 & 14.0 & 16.2 & 28.9 & 8.6  & 53.4 & 10.0  \\
    \textbf{Abell 611}   & 0.288 & 08:00:56.8 & +36:03:26 & 18.7 & 25.0 & 13.9 & 12.0 & 46.2 & 1.73  \\
    \textbf{MACS1115}    & 0.355 & 11:15:51.9 & +01:29:55 & 15.7 & 22.8 & 16.3 & 10.0 & 56.4 & 8.66  \\
    \textbf{MACS0429}    & 0.399 & 04:29:36.0 & -02:53:06 & 17.0 & 24.1 & 13.2 & 11.6 & 47.2 & -0.02 \\
    \textbf{MACS1206}    & 0.439 & 12:06:12.3 & -08:48:06 & 11.3 & 24.9 & 28.7 & 13.3 & 42.5 & 8.89  \\
    \textbf{MACS0329}    & 0.450 & 03:29:41.5 & -02:11:46 & 10.3 & 22.5 & 17.4 & 13.1 & 39.9 & 8.63  \\
    \textbf{RXJ1347}     & 0.451 & 13:47:30.8 & -11:45:09 & 15.5 & 19.7 & 45.3 & 1.9  & 276. & 8.90  \\
    \textbf{MACS1311}    & 0.494 & 13:11:01.7 & -03:10:40 & 14.2 & 22.5 & 11.3 & 10.6 & 64.5 & 0.71  \\
    \textbf{MACS1423}    & 0.543 & 14:23:47.9 & +24:04:43 & 21.7 & 22.3 & 11.8 & 11.2 & 35.7 & 6.15  \\
    \textbf{MACS1149}    & 0.544 & 11:49:35.4 & +22:24:04 & 17.7 & 24.0 & 22.0 & 13.9 & 32.7 & -1.47 \\
    \textbf{MACS0717}    & 0.546 & 07:17:32.1 & +37:45:21 & 12.5 & 29.4 & 31.3 & 14.6 & 27.1 & 3.05  \\
    \textbf{MACS0647}    & 0.591 & 06:47:49.7 & +70:14:56 & 11.7 & 22.0 & 24.1 & 16.4 & 20.3 & 11.3  \\
    \textbf{MACS0744}    & 0.698 & 07:44:52.3 & +39:27:27 & 16.3 & 20.6 & 17.8 & 7.6  & 48.5 & 7.67  \\
    \textbf{CLJ1226}     & 0.888 & 12:26:57.9 & +33:32:49 & 11.8 & 22.9 & 13.7 & 4.9  & 85.6 & 9.43 
\enddata
\tablecomments{Subscripts $_{B}$ and $_{M}$ denote Bolocam and MUSTANG properties respectively. Noise$_{B}$
  and $T_{obs,B}$ are those reported in \citet{sayers2013}. Noise$_{M}$ is calculated on MUSTANG maps with 
  $10$\asecs smoothing, in the central arcminute. $T_{obs}$ are the integration times (on source) for the 
  given instruments. A10$_B$ and A10$_M$ values indicate the significance (in $\sigma$) of $P_0$ when we
  fit a spherical A10 \citep{arnaud2010} profile (see Section~\ref{sec:bulk_ICM}).}
\end{deluxetable*}

%The clusters were observed with
%MUSTANG over the projects AGBT08A\_056, AGBT09A\_052, AGBT09C\_059, AGBT10A\_056, AGBT10C\_017, AGBT10C\_026, AGBT10C\_042, 
%AGBT10C\_031, AGBT11A\_009, and AGBT11B\_001.


%%%%%%%%%%%%%%%%%%%%%%%%%%%%%%%%%%%%%%%%%%%%%%%%%%%%%%%%%%%%%%%%%%%%%%%%%%%%%%%
\subsection{MUSTANG Observations and Reduction}
\label{sec:musobs}
%%%%%%%%%%%%%%%%%%%%%%%%%%%%%%%%%%%%%%%%%%%%%%%%%%%%%%%%%%%%%%%%%%%%%%%%%%%%%%%

MUSTANG is a 64 pixel array of Transition Edge Sensor (TES) bolometers arranged in an $8 \times 8$ array
located at the Gregorian focus on the 100 m GBT. Operating at 90 GHz (81--99~GHz),
MUSTANG has an angular resolution of 9\asec and pixel spacing of 0.63$f \lambda$ resulting in a FOV
of 42\asec. More detailed information about the instrument can be found in \citet{dicker2008}.

Our observations and data reduction are described in detail in \citet{romero2015a}, and we briefly review them
here. Absolute flux calibrations are based on the planets Mars, Uranus, or Saturn, nebulae, or the star Betelgeuse 
($\alpha_{Ori}$). At least one of these flux calibrators was observed at least once per night, and we find our 
calibration is accurate to a 10\% RMS uncertainty. We also observe bright point sources every half hour
to track our pointing and beam shape. To observe the target galaxy clusters, we employ Lissajous daisy scans 
with a $3\arcmin$ radius and in many of the clusters we broadened our coverage with a hexagonal pattern of 
daisy centers (with $1\arcmin$ offsets). 
For most clusters, the coverage (weight) drops to 50\% of its peak value at a radius of $1.3'$.

Processing of MUSTANG data is performed using a custom IDL pipeline. Raw data is recorded as time ordered data (TOD)
from each of the 64 detectors. An outline of the data processing for each scan on a galaxy cluster is given below.
  
  (1) We define a pixel mask from the nearest preceding CAL scan; unresponsive detectors are masked out.
  The CAL scan provides us with unique gains to be applied to each of the responsive detectors.

  (2) A common mode template, calculated as the arithmetic mean of the TOD across detectors, polynomial, and sinusoid
  are fit to the data and subtracted.

  (3) After the common mode and polynomial subtraction each scan is subjected to spike (glitch), skewness, and Allan 
  variance tests and are flagged if they do not meet our criteria.

  (4) Individual detector weights are calculated as $1/ \sigma_i^2$, where $\sigma_i$ is the RMS of the non-flagged
  TOD for that detector. 

  (5) Maps are produced by gridding the TOD in 1\asec pixels in Right Ascension (R.A.) and Declination (Dec). A weight 
  map is produced in addition to the signal map.



\begin{figure}
  \begin{center}
  \includegraphics[width=0.5\textwidth]{figures/Transfer_Function_All_2_Dec_2015.eps}
  \end{center}
  \caption{Effective average transfer function of our data reduction over our sample. 
    The variations between cluster is minimal. \textcolor{red}{[I
        will quantify this...but it's quite small.]} For each cluster, attenuation is
    calculated based on simulated observations of 25 fake skies. The plotted one-dimensional
    transfer function is the weighted average of the transfer functions
    of individual clusters. The transfer functions of individual clusters are calculated as
    the ratio of the one dimensional power spectra of the power spectra of observed fake sky
    and input fake sky. We have labelled the relevant angles for the FOV and FWHM.}
  \label{fig:xfer_all}
\end{figure}


\subsection{Bolocam Observations and Reduction}
\label{sec:bolocamredox}

Bolocam is a 144-element camera that was a facility instrument on the Caltech Submillimeter Observatory (CSO) from
2003 until 2012. Its field of view is 8\amins in diameter, and at 140 GHz it has a resolution of 58\asec FWHM
(\citet{glenn1998,haig2004}). The clusters were observed with a Lissajous pattern that results in a tapered
coverage dropping to 50\% of the peak value at a radius of roughly 5\amin, and to 0 at a radius of 10\amin.
The Bolocam maps used in this analysis are $14\arcmin \times 14\arcmin$. The Bolocam data are the same as those 
used in \citet{czakon2014} and \citet{sayers2013}; the details of the reduction are given therein, along with \citet{sayers2011}. 
%Bolocam observed Abell 1835 for 14.0 hours resulting in a noise of 16.2 $\mu K_{CMB}$-arcminute, and observed 
%MACS 0647 for 11.4 hours resulting in a noise of 22.0 $\mu K_{CMB}$-arcminute.
%In addition to the data maps, for each cluster 1000 noise maps are also provided, which
%included relevant sources of instrumental, atmospheric, and astronomical noise. 
The reduction and calibration is similar to that used for MUSTANG, and Bolocam achieves a 
5\% calibration accuracy and 5\asecs pointing accuracy.

%%%%%%%%%%%%%%%%%%%%%%%%%%%%%%%%%%%%%%%%%%%%%%%%%%%%%%%%%%%%%%%%%%%%%%%%%%%%%%%
\section{Joint Map Fitting Technique}
\label{sec:jointfitting}
%%%%%%%%%%%%%%%%%%%%%%%%%%%%%%%%%%%%%%%%%%%%%%%%%%%%%%%%%%%%%%%%%%%%%%%%%%%%%%%

\subsection{Overview}
\label{sec:jf_overview}

The joint map fitting technique used in this paper is described in detail in \citet{romero2015a}. We review
it briefly here. The general approach follows that of a least squares fitting procedure, which assumes that
we can make a model map as a linear combination of model components. 

This linear combination can be written as:
\begin{equation}
  \vec{d}_m = \mathbf{A} \vec{a}_m,
\end{equation}
where $d_m$ is the total model, each column in $\mathbf{A}$ is a filtered model component, and $\vec{a}_m$ is 
an array of amplitudes of the components. There are, in effect, four types of components for which we consider 
fitting: a bulk component, point source(s), residual component(s),and a mean level. Of these, we produce a
sky model for the bulk component and point source to be filtered. The residual component is calculated directly
as a filtered component.

We wish to fit $\vec{d}_m$ to our data, $\vec{d}$, and allow for a calibration offset between Bolocam and
MUSTANG data. We therefore define our data vector as:
\begin{equation}
  \vec{d} = [ \vec{d}_{B}, k \vec{d}_{M}, k ] ,
\end{equation}
where $\vec{d}_{B}$ is the Bolocam data, $\vec{d}_{M}$ is the MUSTANG data, and $k$ is the calibration offset of
MUSTANG relative to Bolocam, with an uncertainty of 11.2\%.

We use the $\chi^2$ statistic as our goodness of fit:
\begin{equation}
  \chi^2 = (\overrightarrow{d} - \overrightarrow{d}_m)^T \mathbf{N}^{-1} (\overrightarrow{d} - \overrightarrow{d}_m),
\end{equation}
where $\mathbf{N}$ is the covariance matrix; however, because we wish to fit for $k$ in addition to the 
amplitude of model components, we no longer have
completely linearly independent variables, and thus we employ MPFIT \citep{markwardt2009} to solve for these
variables.

\subsection{Simulated Observations}
\label{sec:jf_filtering}

Simulated observations are performed by converting an input sky into input TOD and processing the simulated TOD
as the true TOD are processed (Section~\ref{sec:musobs}). The time requirements for simulated observations are  
reduced by storing the necessary telescope and detector information from the true TOD in an IDL structure. 
Due to the time requirements to cover the
necessary parameter space (Section~\ref{sec:param_space}), TOD were produced with a fraction of the scans (termed
short TOD) on a given cluster, where care was taken to ensure the same coverage (relative weight distribution). 
The filtering is observed to be the same between full TOD and short TOD (Figure~\ref{fig:long_vs_short_qv}). 

\begin{figure}[!h]
  \centering
  \includegraphics[width=0.5\textwidth]{figures/Long_vs_short_qv_bestfit_m0647_22_Jun_2015_v2.eps}
  \caption{A fitted model to MACS 0647. The scatter in the short TOD is $\lesssim 3$\% 
    of the peak Compton y. In absolute terms, this translates to roughly $2\times 10^{-5}$ in Compton y. 
    Typical pixel noise in maps is $15 \times 10^{-5}$.}
  \label{fig:long_vs_short_qv}
\end{figure}

\subsection{Components}
\label{sec:components}

\subsection{Bulk ICM}
\label{sec:bulk_ICM}

 As in \citet{romero2015a}, the bulk component is taken to be a 
spherically symmetric 3D electron pressure profile as parameterized by a generalized Navarro, Frenk,
and White profile \citep[hereafter, gNFW][]{navarro1997,nagai2007}:
\begin{equation}
  \Tilde{P} = \frac{P_0}{(C_{500} X)^{\gamma} [1 + (C_{500} X)^{\alpha}]^{(\beta - \gamma)/\alpha}}
\end{equation}
where $X = R / R_{500}$, and $C_{500}$ is the concentration parameter; one can also write ($C_{500} X$) as
($R / R_s$), where $R_s = R_{500}/C_{500}$. $\Tilde{P}$ is the electron pressure in units of the characteristic
pressure $P_{500}$. This pressure profile is integrated along the line of sight to produce 
a Compton $y$ profile, given as 
\begin{equation}
  y(r) = \frac{P_{500} \sigma_{T}}{m_e c^2} \int_{-\infty}^{\infty} \Tilde{P}(r,l) dl
\end{equation}
where $R^2 = r^2 + l^2$, $r$ is the projected radius, and $l$ is the distance from the center of the cluster
along the line of sight. Once integrated, $y(r)$ is gridded as $y(\theta)$ and  is realized as a map 
(pixels of 1\asecs and 20\asecs on a side). From here, we produce two model maps: one for Bolocam and one for MUSTANG. 
In each case, we convolve the Compton $y$ map by the appropriate beam shape. For Bolocam we use a Gaussian with FWHM
$= 58$\asec, and for MUSTANG we use the double Gaussian as determined in \citet{romero2015a}.

\subsubsection{Point Sources}
\label{sec:ptsrcs}

Point sources are treated in the same manner as in \citet{romero2015a}. All compact sources in our sample
are well modelled as a point source. We clearly detect point sources in Abell 1835, MACS 1115, MACS 0429, 
MACS 1206, RXJ1347, MACS 1423, and MACS 0717 in the MUSTANG maps. A point source is identified by NIKA 
\citep{adam2015} in CLJ1226, which is posited to be a submillimeter galaxy (SMG) behind the cluster. 
The point source in MACS 0717 is due to a foreground elliptical galaxy \citep{mroczkowski2012}.
All of the remaining point sources (six) are coincident (within 3\asecs of reported coordinnates) with the BCGs 
of their respective clusters \citep[][]{crawford1999,donahue2015}. Moreover, of these six BCGs, four of them
exhibit unambiguous UV excess, with the remaining two (MACS 1206 and Abell 1835) not being included because
of ambiguity for MACS 1206 with lensed background systems \citep{donahue2015}. Abell 1835 (not in the CLASH
sample) was observed by \citet{odea2010} and found to have a FUV flux corresponding to a star formation rate
of 11.7 $M_{\odot}$ per year, which fits within the SFR range (5 - 80 $M_{\odot}$ yr$^{-1}$) of the UV excess BCGs
found in \citep{donahue2015}. 
For the Bolocam image, the point sources in Abell 1835, MACS 0429, RXJ1347, and MACS 1423
has been subtracted based on an extrapolation of a power law fit to the 1.4 GHz NVSS \citep{condon1998}
and 30 GHz SZA \citep{mroczkowski2009} measurements \citep{sayers2012}. The flux densities for
the point sources fitted are shown in Table~\ref{tbl:sample_ptsrc}.

\begin{deluxetable}{c c c c c}
\tabletypesize{\footnotesize}
\tablecolumns{5}
\tablewidth{0pt} 
\tablecaption{Point source flux densities \label{tbl:sample_ptsrc}}
\tablehead{ 
    \colhead{Cluster} & \colhead{R.A. (J2000)} & \colhead{Dec (J2000)} & 
    \colhead{$S_{90}$ (mJy)}  & \colhead{$S_{140}$ (mJy)}
}
\startdata
Abell 1835  & 14:01:02.07  &  +2:52:47.52  & $1.37 \pm 0.08$ & $0.7 \pm 0.2$ \\
MACS 1115   & 11:15:51.82  &  +1:29:56.82  & $1.04 \pm 0.11$ & --            \\  
MACS 0429   & 04:29:35.97  &  -2:53:04.74  & $7.67 \pm 0.84$ & $6.0 \pm 1.8$ \\
MACS 1206   & 12:06:12.11  &  -8:48:00.85  & $0.75 \pm 0.08$ & --            \\  
RXJ1347     & 13:47:30.61  & -11:45:09.48  & $7.40 \pm 0.58$ & $4.0 \pm 1.2$ \\  
MACS 1423   & 14:23:47.71  & +24:04:43.66  & $1.36 \pm 0.13$ & $0.7 \pm 0.2$ \\  
MACS 0717   & 07:17:37.03  & +37:44:24.00  & $2.08 \pm 0.25$ & --            \\   
CLJ1226     & 12:27:00.01  & +33:32:42.00  & $0.36 \pm 0.11$ & --   
\enddata
  \tablecomments{$S_{90}$ is the best fit flux density to MUSTANG, and $S_{140}$ is the assumed flux density in
  the Bolocam maps (at 140 GHz). The location of the point source is reported from the fitted centroid to the
  MUSTANG data. The conversion from mJy to the equivalent uK$_{CMB}$ is given as: 
  $S_{140} (\text{mJy/bm}) \sim S_{140} / 30 (\mu\text{K}_{CMB})$.}
\end{deluxetable}

\subsubsection{Residual Components}

Residual components are selected primarily as $4 \sigma$ decrements within the central arcminute of MUSTANG maps,
which are not well fitted by a bulk model. We fit residual components for MACS 1206, RXJ 1347, and MACS 0744.
Although we do not fit for residual components in Abell 611 and MACS 1115, we report properties of potential
residual components for these two clusters. We do not fit the residual component for Abell 611 because the peak
significance is not $4\sigma$ and the bulk cluster model appears to account for much of the decrement. For MACS 1115, 
the residual component is outside the central arcminute and does not affect our fit.

To create the residual component, we first select the feature of interest based on the MUSTANG signal-to-noise (SNR) map
of the cluster \citep[see][]{romero2015a}. All pixels below $-3 \sigma$ pertaining to the feature are selected, and the 
shape is determined by fitting a two dimensional Gaussian. This Gaussian is then fit to the unsmoothed MUSTANG data map 
(in units of Compton y), where only its amplitude is allowed to vary.

\begin{deluxetable*}{c | c c c c c c c}
\tabletypesize{\footnotesize}
\tablecolumns{5}
\tablewidth{0pt} 
\tablecaption{Parameters of Residual Components from MUSTANG \label{table:resid_comps}}
\tablehead{
Cluster & RA      & Dec     & Modeled Peak y    & FWHM$_A$ & FWHM$_B$ & $\theta$ & Fitted Peak y \\
        & (J2000) & (J2000) & ($10^{-5}$) & (\asec) &  (\asec) & (deg.) & ($10^{-5}$)     
}
\startdata
Abell 611 &  8:00:56.20 & 36:03:00.08 &  8.4  &  20.7 &  35.3 &   160 & --        \\ 
MACS 1115 & 11:15:56.66 &  1:30:02.82 &  14   &  17.8 &  28.8 &   138 & --        \\ 
MACS 1206 & 12:06:12.91 & -8:47:33.48 &  7.6  &  23.5 &  23.5 &  -115 & $3.6 \pm 0.7$   \\ 
RXJ1347   & 13:47:31.06 &-11:45:18.38 &  42   &  12.2 &  30.1 &   -52 & $52  \pm 9$     \\ 
MACS 0744 &  7:44:52.22 & 39:27:28.71 &  11   &  17.0 &  23.5 &     1 & $9.0 \pm 2.8$   \\ 
\enddata
\tablecomments{Residual components modeled with a two dimensional Gaussian with associated. $\theta$ is 
  measured CCW (going east) from due north.}
\end{deluxetable*}

\subsubsection{Mean Level}
\label{sec:mean_level}

Similar to \citet{czakon2014}, we wish to account for a mean level (signal offset) in the MUSTANG maps.
We do not wish to fit for a mean level simultaneously as a bulk component given the degeneracies. Therefore,
to determine the mean level independent of the other components, we create a MUSTANG noise map
% from time-flipped TOD 
and calculate the mean within the inner arcminute for each cluster. This mean is then subtracted before 
the other components are fit. 

\subsection{Parameter Space}
\label{sec:param_space}

As in \citet{romero2015a}, we fix MUSTANG's centroid, but allow Bolocam's pointing to vary
by $\pm 10$\asecs in RA and Dec with a prior on Bolocam's radial pointing accuracy of $1\sigma = 5$\asec.
Our approach to find the absolute calibration offset between Bolocam and MUSTANG is also the same as in
\citet{romero2015a}. 

In \citet{romero2015a}, we performed a grid search over $\gamma$ and $C_{500}$, marginalizing over $P_0$,
where $\alpha$ and $\beta$ are fixed to values determined from previous studies. We find that for differing
$\alpha$ and $\beta$, the pressure profiles are in very good agreement with one another and that the 
differences in $\chi^2$ values between these fits is not significant. Thus, for fits to other clusters, 
we limit our $\alpha$ and $\beta$ in the gNFW parametrization to those values used in A10.

We search over $0 < \gamma < 1.3$ in steps of $\delta \gamma = 0.1$, and over
$0.1 < C_{500} < 3.3$ in steps of $\delta C_{500} = 0.1$. To create models in finer steps than $\delta \gamma$ 
and $\delta C_{500}$, we interpolate filtered model maps from nearest neighbors from the grid of original 
filtered models. 

\subsubsection{Centroid}

The default centroids used when gridding our bulk ICM component are the ACCEPT centroids. Given the offsets
between ACCEPT and Bolocam centroids (Table~\ref{tbl:cluster_properties}), we perform a second set of
fits where we grid the bulk ICM component using the Bolocam centroids.

%%%%%%%%%%%%%%%%%%%%%%%%%%%%%%%%%%%%%%%%%%%%%%%%%%%%%%%%%%%%%%%%%%%%%%%%%%%%%%%
\section{SZ Pressure Profile Constraints}
\label{sec:pp_constraints}
%%%%%%%%%%%%%%%%%%%%%%%%%%%%%%%%%%%%%%%%%%%%%%%%%%%%%%%%%%%%%%%%%%%%%%%%%%%%%%%

We have constrained the gNFW parameters $P_0$, $C_{500}$, and $\gamma$ for fourteen individual clusters;
these are presented in Table~\ref{tbl:pressure_profile_results}. 
%We find that six of the fourteen clusters are best fit by $\gamma = 0$. 
We find that six of our sample of fourteen have a best fit $\gamma = 0$, where we do not allow $\gamma <0$. 
We find that our range of $C_{500}$ is sufficient, and that it is generally found to be $0.5 < C_{500} < 2.0$. 

As it stands, finding slopes in the cores of galaxy clusters that are fit with $\gamma = 0$ is not unprecedented; 
\citet{arnaud2010} find six of their 31 analyzed clusters in the REXCESS sample 
have $\gamma=0$, where all gNFW parameters except $\beta$ were fitted for individual clusters. They found a 
similar range in $C_{500}$, and $0.3 < \alpha < 2.5$. The sample in \citet{arnaud2010} is a local ($z < 0.2$), 
flux limited sample, and for their analysis, they have excluded two clusters (a supercluster, Abell 901/902, 
and a bimodal cluster, RXC J2157.4-0747) from the full REXCESS sample of 33 clusters. \citet{sayers2013} 
determined pressure profile parameters over a sample of 45 clusters with the redshift range $0.15 < z < 0.89$, 
where most (60\%) lie between $0.35 < z < 0.59$. They fit profiles to the stack of deprojected pressure 
profiles, restricting $C_{500}$ to the A10 value, and fit for the other gNFW parameters. 

Despite the relative discrepancies in pressure profiles of individual galaxy clusters as derived from MUSTANG and
Bolocam to those derived by ACCEPT, the pressure profile found over all clusters 
has very similar parameters to \citet{arnaud2010}. This can also be seen in Figure~\ref{fig:pp_sets},
where the A10 pressure profile is consistently close to the profile from this work (R16). While all 14 clusters
in this work are in \citet{sayers2013}, they find a consistently higher universal pressure profile.
\citet{planck2013a} find higher pressure are large radii and lower pressure at small radii compared to our
and A10 pressure profiles. 

%\textcolor{red}{[I think I want to expand on this - perhaps postulating why we
%find these trends.]}

%%%%%%%%%%%%%%%%%%%%%%%%%%%%%%%%%%%%%%%%%%%%%%%%%%%%%%%%%%%%%%%%%%%%%%%%%%%%%%%%%%%%%%%%%%%%%%%%%%%%%%%%%%%
%%%                                                SOME FIGURES                                         %%%
%%%%%%%%%%%%%%%%%%%%%%%%%%%%%%%%%%%%%%%%%%%%%%%%%%%%%%%%%%%%%%%%%%%%%%%%%%%%%%%%%%%%%%%%%%%%%%%%%%%%%%%%%%%

\begin{figure}[!h]
  \centering
  \begin{tabular}{cc}
     \epsfig{file=figures/JF_Conf_Intervals_Disturbed_23_Jan_2016.eps,width=0.50\linewidth,clip=} &
     \epsfig{file=figures/JF_Conf_Intervals_Cool_core_23_Jan_2016.eps,width=0.50\linewidth,clip=} \\
     \epsfig{file=figures/JF_Conf_Intervals_Well_behaved_23_Jan_2016.eps,width=0.50\linewidth,clip=} &
     \epsfig{file=figures/JF_Conf_Intervals_All_23_Jan_2016.eps,width=0.50\linewidth,clip=} 
   \end{tabular}
  \caption{Confidence intervals over all disturbed clusters, cool-core clusters, and the entire sample.
           Cool core clusters include: Abell 1835, MACS 1115, MACS 0429, MACS 0329, RXJ 1347, 
           MACS 1311 and MACS 1423. Disturbed clusters include: MACS 0329, MACS 1149, MACS 0717, and
           MACS 0744. Well behaved clusters include: Abell 1835, MACS 1115, MACS 1206, RXJ 1347,
           MACS 0647, MACS 0744, and CLJ 1226. Well behaved clusters are identified above.}
  \label{fig:ensemble_cis}
\end{figure}

\begin{figure}
  \begin{center}
%  \includegraphics[width=0.5\textwidth]{figures/profile_sets_plot_v2.eps}
  \includegraphics[width=0.5\textwidth]{figures/profile_sets_plot_v2_23_Jan_2016.eps}
  \end{center}
  \caption{Pressure Profiles from this (R16) and other works. A09 indicates ACCEPT \citep{cavagnolo2009} 
    pressure profile for the 14 clusters in this sample. A09 falls below R16.}
  \label{fig:pp_sets}
\end{figure}

% Now I want to compare to A10, P12, S13? OK...this is done.
%%% What else to say?

While our stacked pressure profiles are in excellent agreement with the previously derived pressure profiles
in the region $0.1 R_{500} < r < R_{500}$, we see deviations at small and large radii. It is not too surprising
that we agree with A10 at large radii, as we have fixed $\alpha$ and $\beta$ to the A10 values. Despite our
fourteen clusters being included in the BOXSZ sample \citep{sayers2013}, we see that S13 shows higher pressure
at all radii. We note that S13 presents a higher pressure at small radii than found in this work, where the
addition of MUSTANG data contributes significantly to constraining the pressure at small radii (towards smaller
pressures). At larger radii, our restriction of $\alpha$ and $\beta$ again explain our reduced pressure relative
to S13.

Figure~\ref{fig:ppr_ensembles} shows the ratios of the pressure profiles derived from this work to those from 
other works, when clusters are characterized by dynamical type. We calculate these average ratios by weighting 
the ratios of individual clusters, where the specific fitted gNFW pressure profile for
each cluster is taken for ACCEPT, but for the other sets (A10, P12, and S13), we assume the gNFW profile 
found for all clusters in their sample. In fitting profiles to ACCEPT, we impose the same restriction on 
$\alpha$ and $\beta$ (fixing them to A10 values). In this manner, this imposition no longer biases our results
at large radii, and we consitently see larger pressures at large radii in the SZ as compared to X-ray (ACCEPT) 
data. \textcolor{red}{I also started thinking about \emph{XMM-Newton} vs. \emph{Chandra} data. However,
\citet{donahue2014} would indicate that \emph{XMM-Newton} data tends to be lower in $n_e$ and $T_e$, which is
\textbf{opposite} what we see in A09 vs. A10.}

\begin{figure}
  \begin{center}
%  \includegraphics[width=0.5\textwidth]{figures/PPRs_ensembles_All_24_Oct_2015_v2.eps}
%  \includegraphics[width=0.5\textwidth]{figures/PPRs_ensembles_All_11_Jan_2016.eps}
  \begin{tabular}{cc}
    \epsfig{file=figures/PPRs_ensembles_All_ACCEPT_scalerr_v2_24_Jan_2016.eps,width=0.50\linewidth,clip=}   &
    \epsfig{file=figures/PPRs_ensembles_All_A10_scalerr_v2_24_Jan_2016.eps,width=0.50\linewidth,clip=}  \\
    \epsfig{file=figures/PPRs_ensembles_All_P12_scalerr_v2_24_Jan_2016.eps,width=0.50\linewidth,clip=}   &
    \epsfig{file=figures/PPRs_ensembles_All_S13_scalerr_v2_24_Jan_2016.eps,width=0.50\linewidth,clip=}  
  \end{tabular}
  \end{center}
  \caption{Pressure ratios as compared to different sets. For ACCEPT, pressure ratios are calculated per
  cluster and weighted by the error in the ratio (per radial bin). For the others, the ratio is again 
  calculated per individual cluster, but the comparison pressure profile always assumes the gNFW profile
  for the entire dataset, respectively (i.e. A10, P12, or S13). These ratios are weighted in the same manner.}
  \label{fig:ppr_ensembles}
\end{figure}

%%%%%%%%%%%%%%%%%%%%%%%%%%%%%%%%%%%%%%%%%%%%%%%%%%%%%%%%%%%%%%%%%%%%%%%%%%%%%%%
\subsection{Potential Sources of Error}
\label{sec:pp_error}
%%%%%%%%%%%%%%%%%%%%%%%%%%%%%%%%%%%%%%%%%%%%%%%%%%%%%%%%%%%%%%%%%%%%%%%%%%%%%%%

Shallow slopes in the cores of clusters could be suggestive of a centroid offset either between MUSTANG and Bolocam or
between SZ and X-ray data. Given the MUSTANG and Bolocam pointing accuracies (2\asec and 5\asec, respectively),
it is unlikely that the centroid offsets between MUSTANG and Bolocam are driving the fits to shallow slopes. 
The difference between
SZ (Bolocam) centroids and ACCEPT (Table~\ref{tbl:cluster_properties}) are large relative to pointing accuracies
and thus potentially more important. However, when we adopt Bolocam's centroid 
%(\textcolor{red}{[Is this worth another figure? I'm OK without another figure for this.]}) 
we find negligible change to the SZ pressure profile as compared to adopting the ACCEPT centroid. 

Individually, the Bolocam and MUSTANG data sets yield consistent fits with each other, where changes in best fit 
parameters generally occur along the shallow gradient in confidence intervals (i.e. along the degeneracy
between  $C_{500}$, and $\gamma$).

Additionally, we consider the impact of the assumed fluxes of point sources in the Bolocam maps of Abell 1835, 
MACS 0429, RXJ 1347, and MACS 1423. The conversion for $S_{140}$ values from mJy to the equivalent $\mu K_{CMB-amin}$ 
is $\sim 30$, which puts the uncertainties of these point sources at 6, 52, 35, and 6 $\mu K_{CMB-amin}$ respectively. 
From Table~\ref{tbl:cluster_obs}, we see that the noise in the Bolocam maps of these clusters are 16.2, 24.1, 19.7, 
and 22.3 $\mu K_{CMB-amin}$ respectively. Thus, for MACS 0429 and RXJ 1347, we see that the potential impact of the
point sources assumed by Bolocam could be non-trivial.

%%%%%%%%%%%%%%%%%%%%%%%%%%%%%%%%%%%%%%%%%%%%%%%%%%%%%%%%%%%%%%%%%%%%%%%%%%%%%%%
\subsection{Potential Sources of Bias}
\label{sec:pp_bias}
%%%%%%%%%%%%%%%%%%%%%%%%%%%%%%%%%%%%%%%%%%%%%%%%%%%%%%%%%%%%%%%%%%%%%%%%%%%%%%%

\textcolor{red}{This subsection would depend on if we think my confidence intervals are in fact biased...}

%%%%%%%%%%%%%%%%%%%%%%%%%%%%%%%%%%%%%%%%%%%%%%%%%%%%%%%%%%%%%%%%%%%%%%%%%%%%%%%%%%%%%%%%%%%%%%%%%%%%%%%%%%%
%%%                  Table of just the ensembles (and A10 to compare against)                           %%%
%%%%%%%%%%%%%%%%%%%%%%%%%%%%%%%%%%%%%%%%%%%%%%%%%%%%%%%%%%%%%%%%%%%%%%%%%%%%%%%%%%%%%%%%%%%%%%%%%%%%%%%%%%%
%
% %\begin{deluxetable*}{l | l l l l l l l l l l l}
%\tabletypesize{\footnotesize}
%\tablecolumns{6}
%\tablewidth{0pt} 
%\tablecaption{Fitted Pressure Profiles for ensembles \label{tbl:pressure_profile_results}}
%\tablehead{
%Sample & $P_0$ & $C_{500}$ & $\alpha$ & $\beta$ & $\gamma$ \\ 
%      & (Mpc) & ($10^{-5}$ Mpc$^2$) & $10^{-3}$ keV cm$^{-3}$         
%}
%\startdata
%All          &  $7.94  \pm 0.10$ & $1.3_{-0.1}^{+0.1}$ & 1.05 & 5.49 & $0.3_{-0.1}^{+0.1}$ \\ 
%Cool Core    &  $3.55  \pm 0.06$ & $0.9_{-0.1}^{+0.1}$ & 1.05 & 5.49 & $0.6_{-0.1}^{+0.1}$ \\
%Disturbed    &  $12.56 \pm 0.29$ & $1.5_{-0.2}^{+0.1}$ & 1.05 & 5.49 & $0.0^{+0.1}$       \\ 
%Well behaved &  $5.34 \pm 0.08$ & $1.2_{-0.1}^{+0.1}$ & 1.05 & 5.49 & $0.5_{-0.1}^{+0.1}$  \\ 
%\hline
%All (A10)    &  $8.403 h_{70}^{-3/2}$ & 1.177 & 1.0510 & 5.4905 & 0.3081 \\
%Cool core (A10) &  $3.249 h_{70}^{-3/2}$ & 1.128 & 1.2223 & 5.4905 & 0.7736 \\
%Disturbed (A10) &  $3.202 h_{70}^{-3/2}$ & 1.083 & 1.4063 & 5.4905 & 0.3798 
%\enddata
%\tablecomments{We have assumed A10 values of $\alpha$ and $\beta$.
%    The findings from A10 are reproduced in the last three rows. The $h_{70}$ dependence is included for explicit replication
%    of A10 results; all $P_0$ values have this dependence (the assumed cosmologies are the same). Well behaved clusters are
%    identified in the next section.}
%\end{deluxetable*}
 % Not really intending to use this.
%
%%%%%%%%%%%%%%%%%%%%%%%%%%%%%%%%%%%%%%%%%%%%%%%%%%%%%%%%%%%%%%%%%%%%%%%%%%%%%%%%%%%%%%%%%%%%%%%%%%%%%%%%%%%
%%%    END OF THAT TABLE!    NOW LET'S PUT IN THE TABLE WE WANT INSTEAD                                 %%%
%%%%%%%%%%%%%%%%%%%%%%%%%%%%%%%%%%%%%%%%%%%%%%%%%%%%%%%%%%%%%%%%%%%%%%%%%%%%%%%%%%%%%%%%%%%%%%%%%%%%%%%%%%%

%%\begin{landscape}

\begin{deluxetable*}{l|lllllllllll}
\tabletypesize{\scriptsize}
\tablecolumns{12}
\tablewidth{\columnwidth} 
\tablecaption{Summary of Fitted Pressure Profiles \label{tbl:pressure_profile_results}}
\tablehead{
\colhead{Cluster} & \colhead{$R_{500}^a$} & \colhead{$Y_{sph}(R_{500})$} & \colhead{$P_{500}^a$} & 
        \colhead{$P_0$} & \colhead{$C_{500}$} & \colhead{$\alpha$} & \colhead{$\beta$} & \colhead{$\gamma$} & 
                  \colhead{$k$} & \colhead{$\tilde{\chi}^2$} &  \colhead{d.o.f.}        \\ 
      & \colhead{(Mpc)} & \colhead{($10^{-5}$ Mpc$^2$)} & \colhead{$10^{-3}$ keV cm$^{-3}$} & \colhead{}  & 
      \colhead{} & \colhead{}  & \colhead{}  & \colhead{}  &   \colhead{}    
}
\startdata
Abell 1835 & 1.49 & $22.50_{-4.49}^{+4.12}$ &  5.94 & $2.15 \pm 0.07$ & $0.77_{-0.17}^{+0.23}$ & 
1.05 & 5.49 & $0.78_{-0.13}^{+0.12}$ & 1.08 & 0.99 & 12880   \\ 
Abell 611  & 1.24 &  $8.14_{-2.21}^{+3.68}$ &  4.45 & $35.43 \pm 2.46$ & $2.00_{-0.30}^{+0.40}$ & 
1.05 & 5.49 & $0.00^{+0.15}$ & 0.96 & 1.02 & 12882   \\ 
MACS 1115  & 1.28 & $20.53_{-3.52}^{+3.84}$ &  5.45 & $0.67 \pm 0.04$ & $0.35_{-0.10}^{+0.15}$ & 
1.05 & 5.49 & $0.87_{-0.27}^{+0.18}$ & 1.11 & 1.04 & 12875   \\ 
MACS 0429  & 1.10 & $19.85_{-3.74}^{+4.00}$ &  4.48 & $11.01 \pm 0.77$ & $0.59_{-0.09}^{+0.11}$ & 
1.05 & 5.49 & $0.00^{+0.15}$ & 1.00 & 1.03 & 12875   \\ 
MACS 1206  & 1.61 & $43.24_{-8.27}^{+8.19}$ & 10.59 & $2.39 \pm 0.10$ & $0.74_{-0.14}^{+0.16}$ & 
1.05 & 5.49 & $0.51_{-0.16}^{+0.14}$ & 1.09 & 1.01 & 12874   \\ 
MACS 0329  & 1.19 & $12.91_{-2.37}^{+2.93}$ &  5.93 & $9.30 \pm 0.50$ & $1.18_{-0.28}^{+0.72}$ & 
1.05 & 5.49 & $0.41_{-0.41}^{+0.19}$ & 1.03 & 0.99 & 12876   \\ 
RXJ1347    & 1.67 & $37.69_{-5.11}^{+5.78}$ & 11.71 & $3.24 \pm 0.08$ & $1.18_{-0.48}^{+1.02}$ & 
1.05 & 5.49 & $0.80_{-0.70}^{+0.30}$ & 1.15 & 0.99 & 12880   \\ 
MACS 1311  & 0.93 & $10.16_{-1.73}^{+1.79}$ &  3.99 & $2.75 \pm 0.22$ & $0.35_{-0.05}^{+0.15}$ & 
1.05 & 5.49 & $0.41_{-0.41}^{+0.34}$ & 0.98 & 1.00 & 12881   \\ 
MACS 1423  & 1.09 &  $8.47_{-2.07}^{+2.53}$ &  6.12 & $22.39 \pm 1.71$ & $1.58_{-0.48}^{+0.22}$ & 
1.05 & 5.49 & $0.00^{+0.35}$ & 1.04 & 0.98 & 12876   \\ 
MACS 1149  & 1.53 & $42.77_{-5.67}^{+4.99}$ & 12.28 & $5.50 \pm 0.25$ & $0.83_{-0.03}^{+0.07}$ & 
1.05 & 5.49 & $0.00^{+0.05}$ & 0.87 & 1.00 & 13584   \\ 
MACS 0717  & 1.69 & $43.44_{-8.00}^{+9.28}$ & 14.90 & $21.28 \pm 0.68$ & $1.97_{-0.37}^{+0.53}$ & 
1.05 & 5.49 & $0.00^{+0.25}$ & 0.48 & 1.04 & 12876   \\ 
MACS 0647  & 1.26 & $26.22_{-4.72}^{+5.37}$ &  9.23 & $2.78 \pm 0.11$  & $0.70_{-0.20}^{+0.30}$ & 
1.05 & 5.49 & $0.60_{-0.20}^{+0.15}$ & 1.14 & 1.01 & 12876   \\ 
MACS 0744  & 1.26 & $12.59_{-2.29}^{+3.18}$ & 11.99 & $13.15 \pm 0.81$ & $1.71_{-0.21}^{+0.29}$ & 
1.05 & 5.49 & $0.00^{+0.15}$ & 0.90 & 1.02 & 12875   \\ 
CLJ1226    & 1.00 &  $9.03_{-1.60}^{+2.03}$ & 11.84 & $19.29 \pm 1.25$ & $1.90_{-0.50}^{+0.60}$ & 
1.05 & 5.49 & $0.29_{-0.29}^{+0.36}$ & 0.92 & 1.03 & 12875   \\ 
\hline
All          &  --    &  --    &  --    &  $7.94  \pm 0.10$ & $1.3_{-0.1}^{+0.1}$ & 1.05 & 5.49 & $0.3_{-0.1}^{+0.1}$ & -- & -- & -- \\ 
Cool Core    &  --    &  --    &  --    &  $3.55  \pm 0.06$ & $0.9_{-0.1}^{+0.1}$ & 1.05 & 5.49 & $0.6_{-0.1}^{+0.1}$ & -- & -- & -- \\
Disturbed    &  --    &  --    &  --    &  $12.56 \pm 0.29$ & $1.5_{-0.2}^{+0.1}$ & 1.05 & 5.49 & $0.0^{+0.1}$       & -- & -- & -- \\ 
Well behaved &  --    &  --    &  --    &  $5.34 \pm 0.08$ & $1.2_{-0.1}^{+0.1}$ & 1.05 & 5.49 & $0.5_{-0.1}^{+0.1}$  & -- & -- & -- \\ 
\hline
All (A10)    &  --    &  --    &  --    &  $8.403 h_{70}^{-3/2}$ & 1.177 & 1.0510 & 5.4905 & 0.3081 & -- & -- & -- \\
Cool core (A10) &  --    &  --    &  --    &  $3.249 h_{70}^{-3/2}$ & 1.128 & 1.2223 & 5.4905 & 0.7736 & -- & -- & -- \\
Disturbed (A10) &  --    &  --    &  --    &  $3.202 h_{70}^{-3/2}$ & 1.083 & 1.4063 & 5.4905 & 0.3798 & -- & -- & --
\enddata
\tablecomments{Results from our pressure profile analysis. $Y_{sph}$ is calculated using the tabulated value of $R_{500}$.
    $^a$Values of $R_{500}$ and $P_{500}$ are taken from \citet{sayers2013}. We have assumed A10 values of $\alpha$ and $\beta$.
    The findings from A10 are reproduced in the last three rows. The $h_{70}$ dependence is included for explicit replication
    of A10 results; all $P_0$ values have this dependence (the assumed cosmologies are the same).}
\end{deluxetable*}

%\end{landscape}
 % reference: {tbl:pressure_profile_results}
%%% Maybe \include does something I don't want it to?
\begin{deluxetable*}{l|llllllllllll}
\tabletypesize{\footnotesize}
\tablecolumns{13}
%\tablewidth{\columnwidth} 
\tablewidth{0pt} 
\tablecaption{Summary of Fitted Pressure Profiles \label{tbl:pressure_profile_results}}
\tablehead{
\colhead{Cluster} & \colhead{$R_{500}^a$} & \colhead{$Y_{cyl}(R_{500})$} & \colhead{$Y_{sph}(R_{500})$} & \colhead{$10^3 P_{500}^a$} & 
        \colhead{$P_0$} & \colhead{$C_{500}$} & \colhead{$\alpha$} & \colhead{$\beta$} & \colhead{$\gamma$} & 
                  \colhead{$k$} & \colhead{$\tilde{\chi}^2$} &  \colhead{d.o.f.}        \\ 
      & \colhead{(Mpc)} & \colhead{($10^{-5}$ Mpc$^2$)} & \colhead{($10^{-5}$ Mpc$^2$)} & \colhead{keV cm$^{-3}$} & \colhead{}  & 
      \colhead{} & \colhead{}  & \colhead{}  & \colhead{}  &   \colhead{}    
}
\startdata
Abell 1835 & 1.49 & $26.75_{-6.15}^{+6.05}$ & $21.81_{-4.49}^{+4.12}$ &  5.94 & $2.15 \pm 0.07$ & $0.77_{-0.17}^{+0.23}$ & 
1.05 & 5.49 & $0.78_{-0.13}^{+0.12}$ & 1.08 & 0.99 & 12880   \\ 
Abell 611  & 1.24 & $9.67_{-2.57}^{+4.85}$ & $8.73_{-2.21}^{+3.68}$ &  4.45 & $35.43 \pm 2.46$ & $2.00_{-0.30}^{+0.40}$ & 
1.05 & 5.49 & $0.00^{+0.15}$ & 0.96 & 1.02 & 12882   \\ 
MACS 1115  & 1.28 & $30.28_{-6.30}^{+7.32}$ & $20.10_{-3.52}^{+3.84}$ &  5.45 & $0.67 \pm 0.04$ & $0.35_{-0.10}^{+0.15}$ & 
1.05 & 5.49 & $0.87_{-0.27}^{+0.18}$ & 1.11 & 1.04 & 12875   \\ 
MACS 0429  & 1.10 & $30.41_{-6.88}^{+7.72}$ & $19.57_{-3.74}^{+4.00}$ &  4.48 & $11.01 \pm 0.77$ & $0.59_{-0.09}^{+0.11}$ & 
1.05 & 5.49 & $0.00^{+0.15}$ & 1.00 & 1.03 & 12875   \\ 
MACS 1206  & 1.61 & $61.52_{-12.63}^{+12.49}$ & $48.16_{-8.27}^{+8.19}$ & 10.59 & $2.39 \pm 0.10$ & $0.74_{-0.14}^{+0.16}$ & 
1.05 & 5.49 & $0.51_{-0.16}^{+0.14}$ & 1.09 & 1.01 & 12874   \\ 
MACS 0329  & 1.19 & $13.38_{-2.99}^{+3.83}$ & $11.86_{-2.37}^{+2.93}$ &  5.93 & $9.30 \pm 0.50$ & $1.18_{-0.28}^{+0.72}$ & 
1.05 & 5.49 & $0.41_{-0.41}^{+0.19}$ & 1.03 & 0.99 & 12876   \\ 
RXJ1347    & 1.67 & $42.47_{-6.81}^{+8.29}$ & $37.80_{-5.11}^{+5.78}$ & 11.71 & $3.24 \pm 0.08$ & $1.18_{-0.48}^{+1.02}$ & 
1.05 & 5.49 & $0.80_{-0.70}^{+0.30}$ & 1.15 & 0.99 & 12880   \\ 
MACS 1311  & 0.93 & $17.18_{-3.49}^{+3.80}$ & $10.08_{-1.73}^{+1.79}$ &  3.99 & $2.75 \pm 0.22$ & $0.35_{-0.05}^{+0.15}$ & 
1.05 & 5.49 & $0.41_{-0.41}^{+0.34}$ & 0.98 & 1.00 & 12881   \\ 
MACS 1423  & 1.09 & $10.35_{-2.73}^{+4.00}$ & $8.89_{-2.07}^{+2.53}$ &  6.12 & $22.39 \pm 1.71$ & $1.58_{-0.48}^{+0.22}$ & 
1.05 & 5.49 & $0.00^{+0.35}$ & 1.04 & 0.98 & 12876   \\ 
MACS 1149  & 1.53 & $56.87_{-9.00}^{+8.04}$ & $41.62_{-5.67}^{+4.99}$ & 12.28 & $5.50 \pm 0.25$ & $0.83_{-0.03}^{+0.07}$ & 
1.05 & 5.49 & $0.00^{+0.05}$ & 0.87 & 1.00 & 13584   \\ 
MACS 0717  & 1.69 & $64.50_{-10.18}^{+12.42}$ & $55.72_{-8.00}^{+9.28}$ & 14.90 & $21.28 \pm 0.68$ & $1.97_{-0.37}^{+0.53}$ & 
1.05 & 5.49 & $0.00^{+0.25}$ & 0.48 & 1.04 & 12876   \\ 
MACS 0647  & 1.26 & $34.06_{-7.76}^{+10.21}$ & $26.33_{-4.72}^{+5.37}$ &  9.23 & $2.78 \pm 0.11$  & $0.70_{-0.20}^{+0.30}$ & 
1.05 & 5.49 & $0.60_{-0.20}^{+0.15}$ & 1.14 & 1.01 & 12876   \\ 
MACS 0744  & 1.26 & $15.10_{-3.01}^{+4.50}$ & $13.20_{-2.29}^{+3.18}$ & 11.99 & $13.15 \pm 0.81$ & $1.71_{-0.21}^{+0.29}$ & 
1.05 & 5.49 & $0.00^{+0.15}$ & 0.90 & 1.02 & 12875   \\ 
CLJ1226    & 1.00 & $10.50_{-1.94}^{+2.65}$ & $9.46_{-1.60}^{+2.03}$ & 11.84 & $19.29 \pm 1.25$ & $1.90_{-0.50}^{+0.60}$ & 
1.05 & 5.49 & $0.29_{-0.29}^{+0.36}$ & 0.92 & 1.03 & 12875   \\ 
\hline
All          &  --    &  --    &  --    &  --    &  $7.94  \pm 0.10$ & $1.3_{-0.1}^{+0.1}$ & 1.05 & 5.49 & $0.3_{-0.1}^{+0.1}$ & -- & -- & -- \\ 
Cool Core    &  --    &  --    &  --    &  --    &  $3.55  \pm 0.06$ & $0.9_{-0.1}^{+0.1}$ & 1.05 & 5.49 & $0.6_{-0.1}^{+0.1}$ & -- & -- & -- \\
Disturbed    &  --    &  --    &  --    &  --    &  $12.56 \pm 0.29$ & $1.5_{-0.2}^{+0.1}$ & 1.05 & 5.49 & $0.0^{+0.1}$       & -- & -- & -- \\ 
Well behaved &  --    &  --    &  --    &  --    &  $5.34 \pm 0.08$ & $1.2_{-0.1}^{+0.1}$ & 1.05 & 5.49 & $0.5_{-0.1}^{+0.1}$  & -- & -- & -- \\ 
\hline
All (A10)    &  --    &  --    &  --    &  --    &  $8.403 h_{70}^{-3/2}$ & 1.18 & 1.05 & 5.49 & 0.31 & -- & -- & -- \\
Cool core (A10) &  --    &  --    &  --    &  --    &  $3.249 h_{70}^{-3/2}$ & 1.13 & 1.22 & 5.49 & 0.78 & -- & -- & -- \\
Disturbed (A10) &  --    &  --    &  --    &  --    &  $3.202 h_{70}^{-3/2}$ & 1.08 & 1.41 & 5.49 & 0.38 & -- & -- & --
\enddata
\tablecomments{Results from our pressure profile analysis. $Y_{sph}$ is calculated using the tabulated value of $R_{500}$.
    $^a$Values of $R_{500}$ and $P_{500}$ are taken from \citet{sayers2013}. We have assumed A10 values of $\alpha$ and $\beta$.
    The findings from A10 are reproduced in the last three rows. The $h_{70}$ dependence is included for explicit replication
    of A10 results; all $P_0$ values have this dependence (the assumed cosmologies are the same).}
\end{deluxetable*}

%%%%%%%%%%%%%%%%%%%%%%%%%%%%%%%%%%%%%%%%%%%%%%%%%%%%%%%%%%%%%%%%%%%%%%%%%%%%%%%%%%%%%%%%%%%%%%%%%%%%%%%%%%%
%%%                                             END OF THAT TABLE!                                      %%%
%%%%%%%%%%%%%%%%%%%%%%%%%%%%%%%%%%%%%%%%%%%%%%%%%%%%%%%%%%%%%%%%%%%%%%%%%%%%%%%%%%%%%%%%%%%%%%%%%%%%%%%%%%%

%%%%%%%%%%%%%%%%%%%%%%%%%%%%%%%%%%%%%%%%%%%%%%%%%%%%%%%%%%%%%%%%%%%%%%%%%%%%%%%%%%%%%%%%%%%%%%%%%%%%%%%%%%%
%%%                                                NEXT SECTION                                         %%%
%%%%%%%%%%%%%%%%%%%%%%%%%%%%%%%%%%%%%%%%%%%%%%%%%%%%%%%%%%%%%%%%%%%%%%%%%%%%%%%%%%%%%%%%%%%%%%%%%%%%%%%%%%%

\section{Integrated Compton Y Scaling Relations}

%\textcolor{red}{[I'm not sure where to put this, or if it's worth including as it is right now. I know the
%calculation of $Y_{cyl}$ is desirable - I will do that soon.]}

We have calculated $Y_{sph}$ using the tabulated value of $R_{500}$, where $Y_{sph}$ is given by:
\begin{equation}
  Y_{sph}(r) = \frac{\sigma_T}{m_e c^2} \int_0^r P_e(r') 4 \pi r'^2 dr' .
  \label{eqn:ysph}
\end{equation}
The error bars on $Y_{Sph}(R_{500})$ are found by calculating $Y_{Sph}(R_{500})$ to the fits over the 1000 noise realizations, and
taking the values encompassing the middle 68\%.
The $Y_{Sph} - M_{500}$ scaling relation calculated in \citet{arnaud2010} is given as:
\begin{equation}
%  h(z)^{-2/3} Y_{sph}(x R_{500}) = A_x \left{[}\frac{M_{500}}{3 \times 10^{14} h_{70}^{-1} M_{\odot}} \right{]}^{\alpha} ,
  h(z)^{-2/3} Y_{sph}(x R_{500}) = A_x \left[ \frac{M_{500}}{3 \times 10^{14} h_{70}^{-1} M_{\odot}} \right] ^{\alpha} ,
  \label{eqn:ysph_scaling}
\end{equation}
where $\alpha = 1.78$, $A_X = 2.925 \times 10^{-5} I(x) h_{70}^{-1}\text{Mpc}^2$, and $I(1) = 0.6145$.
We compare our values of $Y_{Sph}(R_{500})$ to that of A10 in Figure~\ref{fig:ysph_scaling}
and find six clusters which deviate by more than $2 \sigma$, in $Y_{Sph}$, from the scaling relation: 
MACS 1311, MACS 0429, RXJ 1347, MACS 1149, MACS 0717, and MACS 0744.

We obtain $M_{500}$ from \citet{mantz2010}, who calculate it as follows: (1) calculating $M_{gas,2500}$, the total gas mass enclosed in
$R_{2500}$ from deprojected gas mass (non-parametric) profiles, (2) calculating $R_{500}$ by a ratio of $R_{500} / R{2500} \sim 2.3$
assuming an NFW profile with concentration parameter $c=4$, and (3) calculating to $M_{500}$ as 
$\frac{4 \pi 500 \rho_{cr}(z) r_{500}^3}{3(1+B)}$, where $B = 0.03 \pm 0.06$ is a systematic fractional bias. 
\citet{mantz2010} note that the dominant source of systematic uncertainty associated with $M_{500}$ comes from the uncertainty 
in the assumed $f_{gas}(r_{2500}) = 0.1104$. 

\textcolor{red}{The bias of the (two) cool core clusters above the scaling relation is fundamentally caused by the shallower
slope at large radii for these two clusters. The only way to achieve a shallow slope at larger radii with $\alpha$ and $\beta$ 
fixed is then to drive $C_{500}$ to low values (and keep $\gamma$ relatively low too). We, in fact, see this. MACS 1311 and MACS
0429, the two ``offending'' clusters, have two of the three lowest $C_{500}$ values, where MACS 1115 has the lowest, but is also
fit with a relatively high $\gamma$ value. The question then becomes, what is driving the pressure high at larger radii? And
subsequently, is it just coincidence that these two clusters are cool core clusters? Or colloquially, \emph{what's up with that?}.
Finally, with only four disturbed clusters, one of which is also a cool core cluster, it's tempting to see the reverse trend -
that disturbed clusters are falling below the relation, and cool core clusters are falling above the scaling relation. I'm less
inclined to make a point of this, but perhaps it's worth mentioning.}

\textcolor{red}{I would suspect the source of pressure at large radii is largely assymetries seen at these larger radii. That is,
looking at the maps with X-ray and Bolocam contours, we see extended pressure, which is not strictly circular about the [either]
centroid. Again, on centroids, as discussed in Section~\ref{sec:pp_error}, there is generally not a significant difference in 
fitted parameters, and specifically for these two clusters, there is visually \textbf{no} difference when comparing fits using 
the ACCEPT centroid to the fits using the Bolocam centroid. Thus, I have to return to the notion that there \textbf{is} some
pressure that SZ observations are finding at large radii that the X-ray observations are not. This higher pressure is not seen
globally (it's not symmetrically present), but arises assymetrically and could be due to locally elevated temperatures at large
radii, to which X-rays are not sensitive.}

\textcolor{red}{I wonder if this tendency is not seen in X-ray data as much simply because exposure times for cool core clusters
may be restricted. That is, perhaps the cost to measure the pressure at these radii is too much for what proposers have expected
to find, especially when they can constrain the pressure profile in the core very well in a much shorter time, relative to what
would be required for these larger radii.}


%%%%%%%%%%%%%%%%%%%%%%%%%%%%%%%%%%%%%%%%%%%%%%%%%%%%%%%%%%%%%%%%%%%%%%%%%%%%%%%%%%%%%%%%%%%%%%%%%%%%%%%%%%%
%%%                                                SOME FIGURES                                         %%%
%%%%%%%%%%%%%%%%%%%%%%%%%%%%%%%%%%%%%%%%%%%%%%%%%%%%%%%%%%%%%%%%%%%%%%%%%%%%%%%%%%%%%%%%%%%%%%%%%%%%%%%%%%%

\begin{figure}[!ht]
  \begin{center}
%  \includegraphics[width=1.0\textwidth]{analysis_figures/ysph_scaling_relation_2sigma}
%  \includegraphics[width=0.5\textwidth]{figures/ysph-M_all_color_coded_unlabeled_connected.eps}
  \includegraphics[width=0.5\textwidth]{figures/ysph-M_all_color_coded_unlabeled_accept_14_Jan_2016_connected.eps}
  \end{center}
  \caption{$Y_{Sph,SZ}(R_{500})$ as calculated in this work (Table~\ref{tbl:pressure_profile_results}),
  and $M_{500}$ as calculated from \citet{mantz2010}. The scaling relation (dashed line) and triangles
  are from \citet{arnaud2010} and \citet{pratt2010}. The diamonds are $Y_{Sph,X}(R_{500})$ as calculated from the gNFW fits
  to the ACCEPT pressure profiles. MACS 1311 and MACS 0429 are the notable outliers above the scaling relation.}
% The labelled clusters are those whose $Y_{Sph}(R_{500})$ values lie
%  more than $2\sigma$ from the scaling relation from \citet{arnaud2010}, the dot-dashed line.
  \label{fig:ysph_scaling}
\end{figure}

%%%%%%%%%%%%%%%%%%%%%%%%%%%%%%%%%%%%%%%%%%%%%%%%%%%%%%%%%%%%%%%%%%%%%%%%%%%%%%%
\section{CLuster Properties by Combining SZ and X-ray Data}
\label{sec:xray_comp}
%%%%%%%%%%%%%%%%%%%%%%%%%%%%%%%%%%%%%%%%%%%%%%%%%%%%%%%%%%%%%%%%%%%%%%%%%%%%%%%

The primary observables of SZ and X-ray observations of galaxy clusters differ in their 
dependence upon the physical properties in the intracluster medium (ICM). This difference
has, in the past, been exploited to make calculations of the Hubble parameter, $H_0$, 
assuming spherical geometry of galaxy clusters. However, one can relax the spherical assumption
and use the differences in SZ and X-ray inferred quantities to calculate cluster elongation
along the line of sight, helium sedimentation, or (recalculate) the ICM electron temperature.
These are all degenerate (i.e. these can not all be independently constrained). We investigate
each constraint individually, and conclude that differences in the SZ and X-ray spherically derived
pressure profiles are due to some combination cluster elongation and ICM temperature distribution.

%%% Add some general discussion of SZ vs. ACCEPT?

\subsection{Ellipsoidal Geometry}
\label{sec:ellgeo}

The geometry of a cluster, along the line of sight, may be calculated by comparing SZ and
X-ray pressure profiles. We take $n_e$ as the true distribution, and let 
it be a function of ellipsoidal radius, $E^2 = \frac{x^2}{a^2} + \frac{y^2}{b^2} + \frac{z^2}{c^2}$, 
where we will define $z$ to be along the line of sight, $s = \frac{z}{c}$ to be the ``scaled'' line of sight,
and $\rho = \sqrt{\frac{x^2}{a^2} + \frac{y^2}{b^2}}$ to be the scaled in-the-plane of the sky radius.
We wish to compare $c$ to $a$ and $b$, where we assume $a=b=1$. Given that our observables
are integrals along the line of sight, we can write $O(\rho) = \int Q(\rho, z) dz$, where
 $O(\rho)$ is our observable as a function of position on the sky, and $Q(\rho, z)$ is our
integrable quantity (source function) with ellipsoidal symmetry. We can rewrite this as 
$O(\rho) = \int c \times Q(\rho, s) ds = \int 1 \times Q\prime(\rho, s) ds$, where $Q\prime(\rho, s)$
is our spherical source function assumed in our derivation of pressure profiles (i.e. 
$Q\prime(\rho, s) = Q\prime(r)$).

The observable, $O(\rho)$ is fixed, and thus $Q(r) \propto \frac{Q(E)}{c}$. Of interest to us are 
$n_e(E), n_{e,X}(r), \text{and} n_{e,SZ}(r)$, the elliptical electron distribution, the X-ray derived spherical
electron distribution, and the SZ derived spherical electron distribution. 
These are related as follows: $n_{e,X}^2(r) = \frac{n_e^2(E)}{c}$, and $n_{e,SZ}(r) = \frac{n_e(E)}{c}$.
Thus, the ratio of derived (spherical) radial electron densities has the following proportionality:
$\frac{n_{e,SZ}(r)}{n_{e,X}(r)} = c^{1/2}$. To be explicit, we wish to compare axis ratios, and as such
we define $\eta = \frac{a}{c} = \frac{b}{c}$. Noting that $\eta$ should be independent of $r$,
we see that it can be ascertained through the normalization of the electron density profiles,
or similarly, the electron pressure profiles: $\frac{P_{e,SZ}}{P_{e,X}} = \eta^{-1/2}$. 
We can sum this up as saying that a prolate cluster (along
the line of sight, $\eta < 1$) will yield greater pressure as derived from SZ than from X-ray,
whereas an oblate cluster (along the line of sight) will do the opposite.

To estimate the ellipticity of clusters, we fit the ACCEPT pressure profiles with
a gNFW pressure profile, with $\alpha$ and $\beta$ fixed at their A10 values: 1.05 and 5.49,
respectively. 
%These are tabulated in Table~\ref{tbl:accept_gnfw}. 
The resultant gNFW profile
is then integrated along the line of sight (LOS) to create a Compton $y$ map, and then filtered as 
discussed in \citet{romero2015a}. The amplitude of the ACCEPT model fitted to Bolocam data,
$P_{Bolo}$, is used to account for the geometry of the cluster.
The POS-to-LOS ratio given as $\eta = P_{Bolo}^{-2}$, and its associated uncertainty
is calculated as $\sigma_{\eta}^2 = 2 P_{Bolo}^{-2} ((\sigma_{P_{Bolo}}/P_{Bolo})^2 + 
(\sigma_{Bolo}/P_{Bolo})^2 + (\sigma_{ACCEPT}/P_{Bolo})^2)$, where $(\sigma_{Bolo}/P_{Bolo}) = 0.05$ 
and $(\sigma_{ACCEPT}/P_{Bolo}) = 0.10$ are the calibration uncertainties of Bolocam and ACCEPT 
respectively. Table~\ref{tbl:accept_gnfw} presents relevant fitted gNFW parameters used in calculating
the cluster geometry; some of the $\eta$ values we find to be implausible, but are tabulated nonetheless.

\begin{deluxetable*}{l | l l l | l l l | l l | l }
\tabletypesize{\footnotesize}
\tablecolumns{10}
\tablewidth{0pt} 
\tablecaption{ACCEPT gNFW Fitted Parameters and Comparison to SZ data \label{tbl:accept_gnfw}}
\tablehead{
    Cluster & $P_0$ & $C_{500}$ & $\gamma$ & $P_{SZ}$ & k & $P_{Bolo}$ & $\eta$ & $\sigma_{\eta}$ & $\Delta k$ ($\sigma$)
}
%\begin{table}
%  \centering
%  \begin{tabular}{l l l l l | l l l | l l | l }
%    Cluster & $P_0$ & $C_{500}$ & $\gamma$ & $P_{SZ}$ & k & $P_{Bolo}$ & $\eta$ & $\sigma_{\eta}$ & $\Delta k$ ($\sigma$) \\
%    \hline   
\startdata
    Abell 1835  & 12.8   & 1.3   & 0.43  & 0.77 & 1.21 & 0.75 & 1.77 & 0.41  & 1.08  \\
    Abell 611   & 3.24   & 1.2   & 0.42  & 3.08 & 0.94 & 3.15 & 0.10 & 0.03  & 0.17  \\
    MACS1115    & 18.0   & 1.7   & 0.32  & 0.93 & 1.17 & 0.89 & 1.28 & 0.32  & -0.50 \\
    MACS0429    & 3.76   & 1.2   & 0.73  & 1.79 & 0.57 & 2.06 & 0.24 & 0.06  & 3.58  \\
    MACS1206    & 2.82   & 1.0   & 0.56  & 1.36 & 0.97 & 1.36 & 0.54 & 0.13  & 1.00  \\
    MACS0329    & 3.49   & 1.1   & 0.66  & 1.90 & 0.89 & 1.94 & 0.26 & 0.07  & 1.17  \\
    RXJ1347     & 85.3   & 3.8   & 0.028 & 1.13 & 1.18 & 1.12 & 0.79 & 0.18  & -0.25 \\
    MACS1311    & 35.7   & 2.1   & 0.084 & 1.64 & 0.81 & 1.85 & 0.29 & 0.08  & 1.42  \\
    MACS1423    & 14.5   & 1.9   & 0.45  & 1.26 & 0.82 & 1.38 & 0.52 & 0.15  & 1.83  \\
    MACS1149    & 6.21   & 1.1   & 0.00  & 1.41 & 0.82 & 1.45 & 0.48 & 0.12  & 0.42  \\
    MACS0717    & 12.7   & 1.9   & 0.00  & 1.80 & 0.51 & 1.84 & 0.30 & 0.07  & 0.25  \\
    MACS0647    & 24.8   & 1.7   & 0.00  & 1.24 & 1.09 & 1.22 & 0.67 & 0.16  & -0.42 \\
    MACS0744    & 1.53   & 0.8   & 0.71  & 1.14 & 0.97 & 1.15 & 0.76 & 0.19  &  0.58 \\
    CLJ1226     & 21.5   & 1.8   & 0.12  & 1.21 & 1.00 & 1.22 & 0.68 & 0.18  & -0.67 
\enddata
\tablecomments{Tabulated gNFW fits to the ACCEPT pressure profiles. $P_{SZ}$ denotes the fitted
  amplitude (renormalization) of the ACCEPT model to the SZ data. $P_{Bolo}$ denotes the
  fitted amplitude of the ACCEPT model to just Bolocam data. the gNFW parameters $\alpha$ and
  $\beta$ are fixed at A10 values of 1.05 and 5.49. The column $\Delta k$ ($\sigma$) lists the 
  significances of a more spherical core, as compared to the outer regions. $\Delta k$ was calculated
  as the difference between the $k$ in this table (column 5), and the values listed in
  Table~\ref{tbl:pressure_profile_results}. Negative $\Delta k$ 
  significances indicate the core is measured to be more ellipsoidal than the outer regions.}
\end{deluxetable*}


The CLASH sample contains X-ray (20) and lensing (5) selected clusters and was not explicitly designed to
be orientation unbiased. It is, therefore, not too surprising that we find indications that many of the clusters 
in our sample are elongated along the line of sight ($\eta < 1$).
Abell 1835 is not in the CLASH sample, but is a notably well studied cool core cluster, i.e. its the subject 
of many studies on the basis of its cool core.

This investigation has made the assumption that the geometry of a given cluster is globally consistent.
That is, one ellipsoidal geometry applies to all regions of the cluster. However, this need not be the
case \citep{kravtsov2012}. The cluster should appear more spherical towards the center, where baryons have
condensed \citep[e.g.][and references therein]{kravtsov2012}. Also, the DM and baryonic distributions
need not align (one need only look at the Bullet cluster \citep{Markevitch2004} for a dramatic example).
This is not a particular concern to this analysis as we are comparing quantities based on the baryonic
distribution, but would be more of a concern when including lensing. 

One way to infer a difference in geometries between the inner and outer regions is to note the calibration
offset between Bolocam and MUSTANG in our fits from Table~\ref{tbl:accept_gnfw}. In almost all cases, we
find that $k$ tends to be inversely related to $P_{SZ}$, which may suggest that the central pressure distribution
is more spherical than the outer pressure distribution. A quick estimate of this significance of this signature
is found by comparing to the calibration offset values found in Table~\ref{tbl:pressure_profile_results}, and
finding those clusters that show a preference in $\Delta k$ towards a more spherical center. Recalling that $k$ 
has a prior on it of $12$\%, we can calculate significances shown in Table~\ref{tbl:accept_gnfw}.

%%%%%%%%%%%%%%%%%%%%%%%%%%%%%%%%%%%%%%%%%%%%%%%%%%%%%%%%%%%%%%

\subsection{Temperature profiles}
\label{sec:temp_profiles}

%%%%%%%%%%%%%%%%%%%%%%%%%%%%%%%%%%%%%%%%%%%%%%%%%%%%%%%%%%%%%%

If we assume a given geometry (known ellipticity), then instead of solving for the ellipticity, we can
derive a temperature profile, making use of the direct pressure constraints from SZ observations and the
electron density constraints from X-ray observations. That is, we calculate
\begin{equation}
  T_{e} = \frac{P_{SZ}}{n_{e,X}},
  \label{eqn:telec}
\end{equation}
where $P_{SZ}$ is the pressure derived from pressure profile fits to the SZ data (Section~\ref) and
$n_{e,X}$ is the deprojected electron density derived from X-ray data by the ACCEPT collaboration.
For each bin, we assign radial values as the arithmetic mean of its radial bounds.
%(i.e. $r = (r_i + r_{i+1})/2$), where $r_i$ are the bounds between bins.
Binned values of $P_{SZ}$ are then calculated from the fitted gNFW profile for each radial value 
for the corresponding bins used for $n_{e,X}$.

%may lead to more accurate than that found solely by X-rays, depending on
%the quality of both data sets. 
%X-rays do not constrain electron temperature as well as electron density because
%of the scaling in emission, and foremost because of the photon count requirement for X-ray spectra used to derive 
%a temperature from X-rays, which effectively provides lower resolution temperature data, especially at larger radii.
%Thus, combining the SZ derived pressure and X-ray derived electron density makes use of the strength of each data
%set.

We find two models (beyond the isothermal model) to describe temperature profiles in the literature: those in
\citet{vikhlinin2006}, denoted as (V06) and \citet{bulbul2010}, denoted as (B10). These are given as:
\begin{equation}
  T_{e,V06} = \frac{[(r / r_{cool})^{a_{cool}} + T_{min}/T_0]}{(r / r_{cool})^{a_{cool}} + 1} \times
             \frac{T_0}{(r/r_t)^a [1 + (r/r_t)^b]^{c/b}}
  \label{eqn:tvik06}
\end{equation}
where $r_{cool}$ is a fitted parameter, indicated the radial scale of the cool core, $r_t$ is a transitional radius,
which has been called a scaling radius in other profiles (e.g. NFW). $T_{min}$ is the minimum temperature observed
within the cool core. The remaining parameters, $a_{cool}$, $a$, $b$, $c$, and $T_0$ are all fit for. The first term
in this equation is denoted as the cool core taper:
\begin{equation}
  t_{cool} = \frac{[(r / r_{cool})^{a_{cool}} + T_{min}/T_0]}{(r / r_{cool})^{a_{cool}} + 1}
  \label{eqn:cc_taper}
\end{equation}
Thus, the V06 model should be seen as a gNFW temperature profile with a cool core taper, where Greek letters are used
for the gNFW profiles used in our pressure profile analysis (Section~\ref{sec:components}). The letter equivalents
are $a = \gamma$, $b = \alpha$, and $c = \beta - \gamma$. Thus, given that we have fixed $\alpha$ and $\beta$ from A10
values, we fix $b$ and $c$ in fitting for the V06 temperature model, where we calculate $c$ based on the best fit 
$\gamma$ value from our joint fits (Section~\ref{sec:components}). 
Thus, we fit for $T_0$, $r_{cool}$, and $a_{cool}$, and $a$.

The other temperature model \citep{bulbul2010} is given as:
\begin{equation}
  T_{e,B10} = T_0 \left[ \frac{1}{(\beta - 2)} \frac{(1+ r/r_s)^{\beta-2} - 1}{r/r_s (1 + r/r_s)^{\beta-2}}\right] t_{cool}
  \label{eqn:tbul10}
\end{equation}
where $T_0$ and parameters in $t_{cool}$ are the only parameters specific to the temperature profile (independent of
pressure or density profile). $\beta$ is a power law term in a generalized NFW profile proposed in \citet{bulbul2010}:
\begin{equation}
  \rho_{tot}(r) = \frac{\rho_i}{(r/r_s)(1+r/r_s)^{\beta}}
  \label{eqn:bulgnfw}
\end{equation}
This formulation of density allows for an analytical formulation of $P(r)$ under the assumption of hydrostatic 
equilibrium. With the assumption of a polytropic equation of state ($P = k \rho_g^{n+1}$, where $k$ is simply a
constant, and $n$ is the polytropic index), the temperature profile (Equation~\ref{eqn:tbul10}) can be derived.
It is worth noting that Equation~\ref{eqn:tbul10} does not diverge at $\beta = 2$ and can be calculated from
L'Hopital's rule. As with $r_t$ in the V06 model, we fix $r_s = R_{500}/C_{500}$ based on the fitted value of $C_{500}$.

We use MPFIT \citep{markwardt2009} to solve for the free parameters in the two temperature profiles. For
V06 model, we fit for  $r_{cool}$, $a_{cool}$, $T_0$, and $a$, while for the B10 model we fit for $r_{cool}$, $a_{cool}$, 
$T_0$, and $\beta$.

The results of our SZ and X-ray derived temperature profiles are shown in Figure~\ref{fig:tprofs_all}. 
We noted that the V06 model typically performs worse than B10 in our fits. This can be
attributed to fixing $b$ and $c$ (especially $c$) as these are fixed at power laws that incorporate the
behavior of $n_e$ with radius.

%\afterpage{
%\clearpage
\thispagestyle{empty}
\begin{figure}
  \centering
  \begin{tabular}{ccc}
   \epsfig{file=figures/temperature_profiles_and_models_a1835_18_Oct_2015.eps,width=0.33\linewidth,clip=}   &
   \epsfig{file=figures/temperature_profiles_and_models_a611_18_Oct_2015.eps,width=0.33\linewidth,clip=}   &
   \epsfig{file=figures/temperature_profiles_and_models_m1115_18_Oct_2015.eps,width=0.33\linewidth,clip=}   \\
   \epsfig{file=figures/temperature_profiles_and_models_m0429_18_Oct_2015.eps,width=0.33\linewidth,clip=}   &
   \epsfig{file=figures/temperature_profiles_and_models_m1206_18_Oct_2015.eps,width=0.33\linewidth,clip=}   &
   \epsfig{file=figures/temperature_profiles_and_models_m0329_18_Oct_2015.eps,width=0.33\linewidth,clip=}   \\
   \epsfig{file=figures/temperature_profiles_and_models_rxj1347_18_Oct_2015.eps,width=0.33\linewidth,clip=}   &
   \epsfig{file=figures/temperature_profiles_and_models_m1311_18_Oct_2015.eps,width=0.33\linewidth,clip=}   &
   \epsfig{file=figures/temperature_profiles_and_models_m1423_18_Oct_2015.eps,width=0.33\linewidth,clip=}   \\
%  \end{tabular}
%  \caption{Temperature Profiles. The green triangles are derived as $T_e = P_{e,SZ} / n_{e,X}$, and the shaded green
%    indicates $1\sigma$ uncertainties.The blue plus signs are from ACCEPT. The solid lines are our fitted
%    Vikhlinin temperature models and the dashed lines are our fitted Bulbul temperature models.}
%  \label{fig:tprofs_1}
%\end{figure}
%\begin{figure}
%  \centering
%  \begin{tabular}{ccc}
   \epsfig{file=figures/temperature_profiles_and_models_m1149_18_Oct_2015.eps,width=0.33\linewidth,clip=}   &
   \epsfig{file=figures/temperature_profiles_and_models_m0717_18_Oct_2015.eps,width=0.33\linewidth,clip=}   &
   \epsfig{file=figures/temperature_profiles_and_models_m0647_18_Oct_2015.eps,width=0.33\linewidth,clip=}   \\
   \epsfig{file=figures/temperature_profiles_and_models_m0744_18_Oct_2015.eps,width=0.33\linewidth,clip=}   &
   \epsfig{file=figures/temperature_profiles_and_models_clj1226_18_Oct_2015.eps,width=0.33\linewidth,clip=}  &
  \end{tabular}
  \caption{Temperature Profiles. The green triangles are derived as $T_e = P_{e,SZ} / n_{e,X}$, and the shaded green
    indicates $1\sigma$ uncertainties.The blue plus signs are X-ray spectroscopically derived temperatures from ACCEPT. 
    The solid lines are our fitted
    Vikhlinin temperature models and the dashed lines are our fitted Bulbul temperature models.}
  \label{fig:tprofs_all}
\end{figure}

%}
\begin{deluxetable}{l | l l l}
\tabletypesize{\footnotesize}
\tablecolumns{4}
\tablewidth{0pt} 
\tablecaption{Summary of Fitted Temperature Profiles \label{tbl:temperature_profile_results}}
\tablehead{
Cluster & $\chi_{V06}^2$ & $\chi_{B10}^2$ & DOF 
}
\startdata
    Abell 1835  & 12.2  & 6.45  & 25  \\
    Abell 611   & 11.2  & 3.52  & 25  \\
    MACS1115    & 23.1  & 12.1  & 28  \\
    MACS0429    & 4.96  & 4.10  & 12  \\
    MACS1206    & 21.6  & 2.62  & 30  \\
    MACS0329    & 13.7  & 10.2  & 28  \\
    RXJ1347     & 19.2  & 10.4  & 28  \\
    MACS1311    & 8.98  & 7.01  & 17  \\
    MACS1423    & 3.75  & 6.76  & 17  \\
    MACS1149    & 29.0  & 7.67  & 34  \\
    MACS0717    & 55.8  & 58.9  & 32  \\
    MACS0647    & 9.25  & 3.73  & 18  \\
    MACS0744    & 14.5  & 10.7  & 21  \\
    CLJ1226     & 21.8  & 18.9  & 15
\enddata
\tablecomments{The reported $\chi^2$ values show that the temperature error bars are generally quite large.}
\end{deluxetable}

Given the degeneracy between the geometry of the cluster and temperature, as calculated here, it is most probable
that the difference between SZ and X-ray pressure profiles is due to a combination of the ICM temperatures and
cluster geometries, including deviations from ellipsoidal symmetry of electron density and temperature. 
We find it implausible
that the temperature is in some cases twice that found by ACCEPT, especially those inferred values of 
$k_B T_e \gtrsim 15$ keV. However, even accounting for geometry as calculated in Section~\ref{sec:ellgeo},
we find that there are some clusters that still appear discrepant with ACCEPT, particularly those whose pressure
falls below ACCEPT's pressure towards the center, but above ACCEPT at larger radii.  Some cases, especially
%(Figure~\ref{fig:CI_all}) %% This figure is found in pressure_profiles.tex (in thesis).
MACS 0717 and MACS 1149 are likely due to their known merging status. Others, like MACS 1115, a cool core cluster
with no documented merger activity, with higher pressure at large radii in the SZ relative to X-rays could likely 
be explained by the geometry in the plane of the sky. MACS 1115
shows a northern elongation in the Bolocam map that is not present in the X-ray surface brightness maps
(Figure~\ref{fig:mustang_maps_sample}), 
and MACS 0429 shows a northwest-southeast elongation that is not evident
in the X-ray surface brightness maps. Thus, there is a geometric discrepancy between the SZ and X-ray, especially
at moderate to large radii. It is interesting to note that this discrepancy appears to bias the SZ inferred pressure 
high relative to the X-ray inferred pressure. This could be indicative of higher temperature (lower density) regions,
as the X-ray measurements would be less sensitive to such a region.

%%%%%%%%%%%%%%%%%%%%%%%%%%%%%%%%%%%%%%%%%%%%%%%%%%%%%%%%%%%%%%%%%%%%%%%%%%%%%%%
\section{Notes on Individual Clusters}
\label{sec:ind_notes}
%%%%%%%%%%%%%%%%%%%%%%%%%%%%%%%%%%%%%%%%%%%%%%%%%%%%%%%%%%%%%%%%%%%%%%%%%%%%%%%

%%%%%%%%%%%%%%%%%%%%%%%%%%%%%%%%%%%%%%%%%%%%%%%%%%%%%%%%%%%%%%

\subsection{Abell 1835 (z=0.25)}
\label{sec:results_a1835}

%%%%%%%%%%%%%%%%%%%%%%%%%%%%%%%%%%%%%%%%%%%%%%%%%%%%%%%%%%%%%%

Abell 1835 is a well studied massive cool core cluster. The cool core was noted to have substructure in the central
10\asecs by \citet{schmidt2001}, and identified as being due the central AGN by \citet{mcnamara2006}. Abell 1835 has also
been extensively studied via the SZ effect \citep{reese2002,benson2004,bonamente2006,sayers2011,mauskopf2012}. The models adopted
were either beta models or generalized beta models, and tend to suggest a shallow slope for the pressure interior
to 10\asec. Previous analysis of Abell 1835 with MUSTANG data \citep{korngut2011} detected the SZ effect decrement, but not
at high significance, which is consistent with a featureless, smooth, broad signal. Our updated MUSTANG reduction
of Abell 1835, shown in Figure \ref{fig:mustang_maps_sample}, has the same features as in \citet{korngut2011}.

%%%%%%%%%%%%%%%%%%%%%%%%%%%%%%%%%%%%%%%%%%%%%%%%%%%%%%%%%%%%%%

\subsection{Abell 611 (z=0.29)}
\label{sec:results_a611}

%%%%%%%%%%%%%%%%%%%%%%%%%%%%%%%%%%%%%%%%%%%%%%%%%%%%%%%%%%%%%%

Abell 611 is unique among our clusters for the severity of the discrepancy between our joint SZ fitted pressure
profile and that found in the X-rays. The MUSTANG map (Figure~\ref{fig:mustang_maps_sample}) shows an enhancement
south of the X-ray centroid, and the Bolocam map shows elongation towards the south-southwest. Weak lensing maps
are suggestive of a southwest-northeast elongation \citep{newman2009, zitrin2015}.
Using the density of galaxies, \citet{lemze2013} find a core and a halo which align with the elongation seen 
in the SZ (Figure~\ref{fig:a611_supp_figs}). We note that AMI
\citep{hurley-walker2012} also sees this elongation, while they also note that Abell 611 is the most relaxed
cluster in their sample and that the X-ray data presented from \citet{laroque2006} is very circular and uniform.
Despite being relaxed, Abell 611 is not listed as a cool core cluster (nor disturbed) \citep{sayers2013}.

In an analysis of the dark matter distribution, \citet{newman2009} find that the core (logarithmic) slope of the
cluster is shallower than an NFW model, with $\beta_{DM} = 0.3$, where the dark matter distribution has been characterized
by yet another generalization of the NFW profile:
\begin{equation}
  \rho(r) = \frac{\rho_0}{(r/r_s)^{\beta_{tot}}(1 + r/r_s)^{3-\beta_{tot}}}
\end{equation}
They find the distribution of dark matter within Abell 611 to be inconsistent with a single NFW model. 

%%%%%%%%%%%%%%%%%%%%%%%%%%%%%%%%%%%%%%%%%%%%%%%%%%%%%%%%%%%%%%

\subsection{MACS 1115 (z=0.36)}
\label{sec:results_m1115}

%%%%%%%%%%%%%%%%%%%%%%%%%%%%%%%%%%%%%%%%%%%%%%%%%%%%%%%%%%%%%%

MACS 1115 is listed as a cool core cluster \citep{sayers2013}. It is among seven CLASH clusters that show
unambiguous ultraviolet (UV) excesses attributed to unabsorbed star formation rates of 5-80 $M_{\odot} $yr$^{-1}$
\citep{donahue2015}. MUSTANG detects a point source in MACS 1115, which is coincident with its BCG. 
%The NVSS, at 1.4 GHz, finds the flux of the point source to be $16.2$ mJy.
MACS 1115 is fit by a fairly steep inner pressure profile slope to the SZ data (Figure~\ref{fig:CI_all}).
Adopting the Bolocam centroid, the inner pressure profile slope is notably reduced, yet the goodness of fit is
not significantly changed. In particular, the Bolocam image shows a north-south elongation (particularly to the
north of the centroids). In contrast, weak and strong lensing \citep{zitrin2015} show a more southeast-northwest
elongation.

%%%%%%%%%%%%%%%%%%%%%%%%%%%%%%%%%%%%%%%%%%%%%%%%%%%%%%%%%%%%%%

\subsection{MACS 0429 (z=0.40)}
\label{sec:results_m0429}

%%%%%%%%%%%%%%%%%%%%%%%%%%%%%%%%%%%%%%%%%%%%%%%%%%%%%%%%%%%%%%

MACS 0429 has been well studied in the X-ray \citep{schmidt2007,comerford2007,maughan2008,allen2008,mann2012}
MACS 0429 is identified as a cool core cluster \citep[cf.][]{mann2012,sayers2013}. The bright point source in 
the MUSTANG image is the cluster BCG, which is noted as having an excesses UV emission \citep{donahue2015}.
Of the point sources observed by MUSTANG, this has the shallowest spectral index between 90 GHz and 140 GHz
of $\alpha_{\nu} = 0.55$.
%At 90 GHz, we find the flux density as $7.67 \pm 0.84$ mJy. The point source subtracted from the Bolocam data
%is a $6.0 \pm 1.8$ mJy source at 140 GHz. At 1.4 GHz, NVSS finds the point source to have a flux density of
%$138.8 \pm 4.2$ mJy \citep{condon1998}. 

Despite MACS 0429's stature as a cool core cluster, its pressure profile
(Figure~\ref{fig:CI_all}) is surprisingly shallow in the core, and shows elevated pressure relative to
X-ray derived pressure at moderate radii. The offset between the Bolocam centroid \citep{sayers2013} and ACCEPT
\citep{cavagnolo2009} centroid is 100 kpc, which is notably larger than the X-ray-optical separations of the cluster
peaks and centroids reported in \citet{mann2012} of 12.8 and 19.5 kpc respectively.


%\afterpage{
%\clearpage
%\thispagestyle{empty}
%\begin{figure}
%  \centering
%  \includegraphics[width=0.85\textwidth]{analysis_figures/MACS0429_flux_figure_with_centroid_ptsub_mnsub_9_Jul_2015}
%  \includegraphics[width=0.85\textwidth]{analysis_figures/MBO_Contours_m0429_lens_22_Jan_2015.eps}
%  \caption{MACS 0429}
%  \label{fig:macs_0429params}
%\end{figure}
%\clearpage
%}

%%%%%%%%%%%%%%%%%%%%%%%%%%%%%%%%%%%%%%%%%%%%%%%%%%%%%%%%%%%%%%

\subsection{MACS 1206 (z=0.44)}
\label{sec:results_m1206}

%%%%%%%%%%%%%%%%%%%%%%%%%%%%%%%%%%%%%%%%%%%%%%%%%%%%%%%%%%%%%%

MACS 1206 has been observed extensively \citep[e.g.][]{ebeling2001,ebeling2009,gilmour2009,umetsu2012,
zitrin2012a,biviano2013,sayers2013}. It is not categorized as a cool core or a disturbed cluster
\citep{sayers2013}. Using weak lensing data from Subaru, \citet{umetsu2012} find that it the major-minor 
axis ratio of projected mass is $\gtrsim 1.7$ at $1\sigma$. They infer that this high ellipticity and 
alignment with the BCG, optical, X-ray, and LSS shapes are suggestive that the major axis is aligned 
close to the plane of the sky. In \citet{young2014}, substructure is identified that corresponds to an 
optically-identified subcluster, which may either be a merging subcluster, or a foreground cluster. 
In this analysis, the SZ signal observed by MUSTANG is well modelled by a residual component (coincident 
with the subcluster) and a spherical bulk ICM component. We note that the Bolocam contours of MACS 1206
do not exhibit much ellipticity. We do find that MACS 1206 has a major-minor axis 
ratio of $1.85 \pm 0.45$ (Section~\ref{sec:ellgeo}), where the major axis is along the line of sight.

%The point source was found to have a flux density of $0.77 \pm 0.06$ mJy with the best fit model in
%\citet{young2014}. In this analysis, we find it to have a flux density of $0.75 \pm 0.08$ mJy. A proposal 
%has been accepted for \emph{XMM-Newton} observations of this substructure (PI: Sarazin).

%\afterpage{
%\clearpage
%\thispagestyle{empty}
%\begin{figure}
%  \centering
%  \includegraphics[width=0.85\textwidth]{analysis_figures/cres/JF_Conf_Intervals_2params_both_default_speedy_9_Feb_2015_m1206.eps}
%  \includegraphics[width=0.85\textwidth]{analysis_figures/cres/PPP_arnaud_v3_log-log_30_Mar_2015_m1206.eps}
%  \caption{MACS 1206}
%  \label{fig:macs_1206params}
%\end{figure}
%\clearpage
%}

%%%%%%%%%%%%%%%%%%%%%%%%%%%%%%%%%%%%%%%%%%%%%%%%%%%%%%%%%%%%%%

\subsection{MACS 0329 (z=0.45)}
\label{sec:results_m0329}

%%%%%%%%%%%%%%%%%%%%%%%%%%%%%%%%%%%%%%%%%%%%%%%%%%%%%%%%%%%%%%

MACS 0329 has the distinction of being listed as both a cool core and disturbed cluster. Although it has
been classified as relaxed \citep{schmidt2007}, subtructure has been noted \citep{maughan2008}, and it earns
its cool core and disturbed classifications based on central weighting of X-ray luminsoity and comparing
centroid offsets between optical and X-ray data \citep{sayers2013}. The elongation of the weak lensing and
strong lensing are towards the northwest and southeast of the centroid.

MACS 0329 has two systems with multiple images: one at $z = 6.18$ and the other at $z = 2.17$. The Einstein
radii for these two systems are $r_E = 34$\asecs and $r_E = 28$\asec, respectively \citep{zitrin2012b}, which is
noted as being typical for relaxed, well-concentrated lensing clusters. 


%\afterpage{
%\clearpage
%\thispagestyle{empty}
%\begin{figure}
%  \centering
%  \includegraphics[width=0.85\textwidth]{analysis_figures/cres/JF_Conf_Intervals_2params_both_default_speedy_9_Feb_2015_m0329.eps}
%  \includegraphics[width=0.85\textwidth]{analysis_figures/cres/PPP_arnaud_v3_log-log_23_Feb_2015_m0329.eps}
%  \caption{MACS 0329}
%  \label{fig:macs_0329params}
%\end{figure}
%\clearpage
%}

%%%%%%%%%%%%%%%%%%%%%%%%%%%%%%%%%%%%%%%%%%%%%%%%%%%%%%%%%%%%%%

\subsection{RXJ1347 (z=0.45)}
\label{sec:results_rxj1347}

%%%%%%%%%%%%%%%%%%%%%%%%%%%%%%%%%%%%%%%%%%%%%%%%%%%%%%%%%%%%%%

RXJ1347 is one of the most luminous X-ray clusters, and has been well studied in radio, SZ, lensing, optical
spectroscopy, and X-rays \citep[e.g.][]{schindler1995,allen2002, pointecouteau1999,komatsu2001,kitayama2004,
gitti2007b,ota2008,bradac2008,miranda2008}. X-ray contours have long suggested RXJ1347 is a relaxed system
\citep[e.g.][]{schindler1997}, and it is classified as a cool core cluster \citep[e.g.][]{mann2012,sayers2013}. 

Indeed, the first sub-arcminute SZ observations \citep{komatsu2001,kitayama2004} saw an enhancement to
the southeast of the cluster X-ray peak, which was suggested as being due to shock heating. This enhancement
was confirmed by MUSTANG \citep{mason2010}. Further measurements were made with CARMA \citep{plagge2013},
which find the 9\% of the thermal energy in the cluster is in sub-arcminute substructure.
At low radio frequencies \citep[][237 MHz and 614 MHz]{ferrari2011},
\citep[][1.4 GHz]{gitti2007a} find evidence for a radio mini-halo in the core of RXJ1347. The cosmic ray electrons
are thought to be reaccelerated because of the shock and sloshing in the cluster \citep{ferrari2011}.

We observe a point source (coincident with the BCG) with flux density of $7.40 \pm 0.58$ mJy. Previous analysis of 
the MUSTANG data found the point source flux density as 5 mJy \citep{mason2010}. The difference in the flux 
densities is likely accounted by (1) the different modeling of point sources; primarily that we filter the double 
Gaussian, (2) we simultaneously fit the components, and (3) we assume a steeper profile in the core than the beta 
model assumed in \citet{mason2010}. Lower frequency radio observations found the flux density of the source to be 
$10.81 \pm 0.19$ mJy at 28.5 GHz \citep{reese2002}, and $47.6 \pm 1.9$ mJy at 1.4 GHz \citep{condon1998}. The BCG 
is observed to have a UV excess\citep{donahue2015}. 

Despite
the classification of being a cool core cluster, it is also observed that there are hot regions, intially
constrained as $kT > 10$ keV \citep[e.g.][]{allen2002,bradac2008}, and more recently constrained to even
hotter temperatures \citep[$kT > 20$ keV][]{johnson2012}, indicative of an unrelaxed cluster. 
\citet{johnson2012} also interpret the two cold fronts as being due to sloshing, where a subscluster has returned
for a second passage.

%\afterpage{
%\clearpage
%\thispagestyle{empty}
%\begin{figure}
%  \centering
%  \includegraphics[width=0.85\textwidth]{analysis_figures/cres/JF_Conf_Intervals_2params_both_default_speedy_9_Feb_2015_rxj1347.eps}
%  \includegraphics[width=0.85\textwidth]{analysis_figures/cres/PPP_arnaud_v3_log-log_3_Feb_2015_rxj1347.eps}
%  \caption{RXJ1347}
%  \label{fig:rxj1347params}
%\end{figure}
%\clearpage
%}

%%%%%%%%%%%%%%%%%%%%%%%%%%%%%%%%%%%%%%%%%%%%%%%%%%%%%%%%%%%%%%

\subsection{MACS 1311 (z=0.49)}
\label{sec:results_m1311}

%%%%%%%%%%%%%%%%%%%%%%%%%%%%%%%%%%%%%%%%%%%%%%%%%%%%%%%%%%%%%%


MACS 1311 is listed as a cool core cluster \citep[e.g.][]{sayers2013}, and appears to have quite circular
contours in the X-ray and lensing images, yet has evidence for some disturbance, given its classification
in \citet{mann2012}. However, the SZ contours from Bolocam show some enhancement  to the west, and has
a notable centroid shift ($27.7$\asec) westward from the X-ray centroid. When fitting pressure profiles
to this cluster, it appears that the enhanced SZ pressure at moderate radii ($r \sim 100$\asec) is due
to this enhancement, especially when noting that we use the X-ray centroid. Adopting the Bolocam centroid
does not change the pressure profile much, and we still observe a pressure enhancement at moderate radii.

%\afterpage{
%\clearpage
%\thispagestyle{empty}
%\begin{figure}
%  \centering
%  \includegraphics[width=0.85\textwidth]{analysis_figures/cres/JF_Conf_Intervals_2params_both_default_speedy_3_May_2015_m1311.eps}
%  \includegraphics[width=0.85\textwidth]{analysis_figures/cres/PPP_arnaud_v3_log-log_26_Apr_2015_m1311.eps}
%  \caption{MACS 1311}
%  \label{fig:macs_1311params}
%\end{figure}
%\clearpage
%}

%%%%%%%%%%%%%%%%%%%%%%%%%%%%%%%%%%%%%%%%%%%%%%%%%%%%%%%%%%%%%%

\subsection{MACS 1423 (z=0.54)}
\label{sec:results_m1423}

%%%%%%%%%%%%%%%%%%%%%%%%%%%%%%%%%%%%%%%%%%%%%%%%%%%%%%%%%%%%%%

MACS 1423 is a cool core cluster \citep{mann2012,sayers2013}. While the Bolocam contours are quite concentric,
and suggestive of a relaxed cluster, the centroid is still offset from the X-ray peak by an appreciable angle 
($19.8$\asec). Similar to MACS 1311, the pressure is slightly less than the ACCEPT X-ray derived pressure in the
core, and slightly greater at moderate radii. While this is expected for a centroid offset, we find that adopting
the Bolocam centroid again yields no substantial difference in the SZ pressure profile. We observe a point source 
(the cluster BCG) with flux density of $1.36 \pm 0.13$ mJy, which is also observed to have a UV excess 
\citep{donahue2015}. 

%\afterpage{
%\clearpage
%\thispagestyle{empty}
%\begin{figure}
%  \centering
%  \includegraphics[width=0.85\textwidth]{analysis_figures/cres/JF_Conf_Intervals_2params_both_default_speedy_9_Feb_2015_m1423.eps}
%  \includegraphics[width=0.85\textwidth]{analysis_figures/cres/PPP_arnaud_v3_log-log_24_Feb_2015_m1423.eps}
%  \caption{MACS 1423}
%  \label{fig:macs_1423params}
%\end{figure}
%\clearpage
%}

%%%%%%%%%%%%%%%%%%%%%%%%%%%%%%%%%%%%%%%%%%%%%%%%%%%%%%%%%%%%%%

\subsection{MACS 1149 (z=0.54)}
\label{sec:results_m1149}

%%%%%%%%%%%%%%%%%%%%%%%%%%%%%%%%%%%%%%%%%%%%%%%%%%%%%%%%%%%%%%

MACS 1149 is classified as a disturbed cluster \citep[e.g.][]{mann2012,sayers2013}, and lensing studies have found
that a single DM halo does not describe the cluster well, but rather at least four large-scale DM hales are used to
describe the cluster \citep{smith2009}. A large radial velocity dispersion \citep[1800 km s$^{-1}$][]{ebeling2007} is 
observed, indicative of merger activity along the line of sight. X-ray, SZ, and lensing (particularly 
strong lensing) all show elongation in the northwest-southeast direction. We see a $3\sigma$ feature to the east of
the centroids, but it is not clear that this is associated with any particular feature.

The SZ derived pressure profile roughly matches the shape of the X-ray derived pressure profile (Figure
\ref{fig:CI_all}), with the SZ pressure consistently greater than the X-ray pressure.We calculate
that the axis along the line of sight is $2.08 \pm 0.52$ (Section~\ref{sec:ellgeo}) times greater than the axes in the plane
of the sky. Although we do not find previous analysis of the elongation in the plane of the sky, we would certainly
expect this given (1) the inferred merger activity along the line of sight, and (2) the lensing strength of the cluster.

%\afterpage{
%\clearpage
%\thispagestyle{empty}
%\begin{figure}
%  \centering
%  \includegraphics[width=0.85\textwidth]{analysis_figures/cres/JF_Conf_Intervals_2params_both_default_speedy_9_Feb_2015_m1149.eps}
%  \includegraphics[width=0.85\textwidth]{analysis_figures/cres/PPP_arnaud_v3_log-log_22_Jan_2015_m1149.eps}
%  \caption{MACS 1149}
%  \label{fig:macs_1149params}
%\end{figure}
%\clearpage
%}

%%%%%%%%%%%%%%%%%%%%%%%%%%%%%%%%%%%%%%%%%%%%%%%%%%%%%%%%%%%%%%

\subsection{MACS 0717 (z=0.55)}
\label{sec:results_m0717}

%%%%%%%%%%%%%%%%%%%%%%%%%%%%%%%%%%%%%%%%%%%%%%%%%%%%%%%%%%%%%%

%\begin{figure}
%  \centering
%  \includegraphics[width=0.85\textwidth]{analysis_figures/M0717_mroczkowski_fig1.eps}
%  \caption{From \citet{mroczkowski2012}.}
%  \label{fig:m0717_mroczkowski}
%\end{figure}

Despite MACS 1149's impressive merging activity, MACS 0717 holds the title as the most disturbed massive cluster at $z> 0.5$
\citep{ebeling2007}, which appears to be accreting matter along a 6-Mpc-long filament \citep{ebeling2004}, and has the
largest known Einstein radius \citep[$\theta_e \sim 55$\asec;][]{zitrin2009}. Four distinct components are identified
from X-ray and optical analyses \citep{ma2009}, and the lensing analyses \citep{zitrin2009,limousin2012} find agreement
in the location of these four mass peaks with those from the X-ray and optical. 

%%% Rework (subclusters, but without figure...?)
%The four components are labelled in Figure~\ref{fig:m0717_mroczkowski}. 
There are four identified subclusters \citep[labeled A through D][]{mroczkowski2012}. \citet{ma2009} find that subcluster C is the
most massive component, while subcluster A is the least massive, and subclusters B and D are likely remnant cores. The
velocities of the components from spectroscopy are found to be $(v_A, v_B, v_C, v_D) = (+278_{-339}^{+295},+3238_{-242}^{+252},
-733_{-478}^{+486},+831_{-800}^{+843})$ km s$^{-1}$ \citep{ma2009}. 

MACS 0717 has also been observed at 610 MHz with the Giant Metrewave Radio Telescope (GMRT) which reveals both a radio
halo and a radio relic \citep{vanweeren2009}. This is interpreted as likely being due to a diffuse shock acceleration
(DSA).

We observe a foreground radio galaxy, modeled as a point source here, with flux density of $2.08 \pm 0.25$ mJy at 90 GHz. 
This was previously reported with an integrated flux density of $2.8 \pm 0.2$ mJy and an extended shape 14.\asec4 $\times$ 16.\asec1 
\citep{mroczkowski2012}. However, an improved beam modeling has allowed us to model the foreground galaxy given a known beam shape. 
It is also worth
noting that the MUSTANG data itself has been processed slightly different from that presented in \citet{mroczkowski2012};
in this work the map is produced with a common calculated as the mean across detectors.

%\afterpage{
%\clearpage
%\thispagestyle{empty}
%\begin{figure}
%  \centering
%  \includegraphics[width=0.85\textwidth]{analysis_figures/cres/JF_Conf_Intervals_2params_both_default_speedy_9_Feb_2015_m0717.eps}
%  \includegraphics[width=0.85\textwidth]{analysis_figures/cres/PPP_arnaud_v3_log-log_12_Mar_2015_m0717.eps}
%  \caption{MACS 0717}
%  \label{fig:macs_0717params}
%\end{figure}
%\clearpage
%}

%%%%%%%%%%%%%%%%%%%%%%%%%%%%%%%%%%%%%%%%%%%%%%%%%%%%%%%%%%%%%%

\subsection{MACS 0647 (z=0.59)}
\label{sec:results_m0647}

%%%%%%%%%%%%%%%%%%%%%%%%%%%%%%%%%%%%%%%%%%%%%%%%%%%%%%%%%%%%%%

MACS 0647 is at $z = 0.591$ and is classified as neither a cool core nor a disturbed cluster \citep{sayers2013}. 
It was included in the CLASH sample due to its strong lensing properties \citep{postman2012}.
Gravitational lensing \citep{zitrin2011}, X-ray surface brightness \citep{mann2012}, 
and SZ effect (MUSTANG, see Figure \ref{fig:mustang_maps_sample}, and Bolocam) maps all
show elongation in an east-west direction. 
In the joint analysis presented here, we see that the spherical model provides an adequate fit to both datasets and we note 
that the spherical assumption allows for a easier interpretation of the mass profile of the cluster.

%%%%%%%%%%%%%%%%%%%%%%%%%%%%%%%%%%%%%%%%%%%%%%%%%%%%%%%%%%%%%%

\subsection{MACS 0744 (z=0.70)}
\label{sec:results_m0744}

%%%%%%%%%%%%%%%%%%%%%%%%%%%%%%%%%%%%%%%%%%%%%%%%%%%%%%%%%%%%%%

MACS 0744 is neither classified as a cool core cluster nor a disturbed cluster \citep{mann2012,sayers2013}, but qualifies
as a relaxed cluster \citep{mann2012}. There is a dense X-ray core, and a doubly peaked red sequence of galaxies as found
by \citet{kartaltepe2008}. The gas is also found to be rather hot: $k_B T_e = 17.9_{-3.4}^{+10.8}$ keV, as determined by combining
SZ and X-ray data \citep{laroque2003}. 

The data presented here is the same as in \citet{korngut2011}, but has been processed differently: again, the primary difference
is in the treatment of the common mode. Additionally, \citet{korngut2011} optimize over the low-pass filtering of the common mode
and do not implement a correction factor for the SNR map. The surface brightness significance of the shock feature is the same, 
but is perhaps less
bowed than the kidney bean shape seen previously.  The excess in \citet{korngut2011} was an exciting results for MUSTANG, 
as it marked the first clear detection of a shock in the SZ that had not been previously been known from X-ray observations. 
\citet{korngut2011}
reanalyze the X-ray data with the knowledge of the shocked region from MUSTANG, and calculate the Mach number of the shock
based on (1) the shock density jump, (2) stagnation condition between the pressures at the edge of the cold front and just
ahead of the shock, and (3) temperature jump across the shock, and find Mach numbers between 1.2 and 2.1, with
a velocity of $1827_{-195}^{+267}$ km s$^{-1}$.

%\afterpage{
%\clearpage
%\thispagestyle{empty}
%\begin{figure}
%  \centering
%  \includegraphics[width=0.85\textwidth]{analysis_figures/cres/JF_Conf_Intervals_2params_both_default_speedy_9_Feb_2015_m0744.eps}
%  \includegraphics[width=0.85\textwidth]{analysis_figures/cres/PPP_arnaud_v3_log-log_26_Feb_2015_m0744.eps}
%  \caption{MACS 0744}
%  \label{fig:macs_0744params}
%\end{figure}
%\clearpage
%}

%%%%%%%%%%%%%%%%%%%%%%%%%%%%%%%%%%%%%%%%%%%%%%%%%%%%%%%%%%%%%%

%%%%%%%%%%%%%%%%%%%%%%%%%%%%%%%%%%%%%%%%%%%%%%%%%%%%%%%%%%%%%%

\subsection{CLJ 1226 (z=0.89)}
\label{sec:results_clj1226}

%%%%%%%%%%%%%%%%%%%%%%%%%%%%%%%%%%%%%%%%%%%%%%%%%%%%%%%%%%%%%%

CLJ 1226 is a well studied high redshift cluster \citep[e.g.][]{mroczkowski2009,bulbul2010,adam2015}. 
\citet{adam2015} find a point source at RA 12:12:00.01 and Dec +33:32:42 with a flux density of 
$6.8 \pm 0.7 \text{ (stat.)} \pm 1.0 \text{ (cal.)}$ mJy at 260 GHz and $1.9 \pm 0.2 \text{ (stat.)}$ at 150 GHz. 
This is not the same point source seen in \citet{korngut2011}, which is reported as a point source
with $4.6\sigma$ significance in surface brightness, and can be fit in our current analysis as a point source 
with a flux density of $0.33 \pm 0.13$ mJy. In contrast, we fit  the point source found in \citet{adam2015} with
a flux density of $0.36 \pm 0.11$ mJy. We therefore adopt the point source location presented in \citet{adam2015}
for our pressure profile analysis of CLJ 1226.
A short VLA filler observation (VLA-12A-340, D-array, at 7 GHz) was performed to follow up
this potential source. To a limit of $\sim 50 {\rm \mu Jy}$ nothing is seen, other than the clearly spatially
distinct radio source associated with the BCG at the cluster center (1 mJy at 7 GHz and 3.2 mJy in NVSS).
 
%%% K09 flux is 0.34 +/- 0.13 mJy in our maps. Now I've written it in. Jan 2016.

In the previous analysis of the MUSTANG data, \citet{korngut2011} find a ridge of significant substructure after 
subtracting a bulk SZ profile (N07, fitted to SZA data). They find that this ridge, southwest of the cluster
center, alongside X-ray profiles, are consistent with a merger scenario. 
%\textcolor{red}{[I will do an analysis with the point source found in Korngut+2011. The question is if I already 
%have the point source modelled (and to find it).}

%%%%%%%%%%%%%%%%%%%%%%%%%%%%%%%%%%%%%%%%%%%%%%%%%%%%%%%%%%%%%%%%%%%%%%%%%%%%%%%%%%%%%%%%%%%%%%%%%%%%%%%%%%%%%%%%
%%%%%%%%%%%%%%%%%%%%%%%%                    CONCLUSIONS!!!                           %%%%%%%%%%%%%%%%%%%%%%%%%%%
%%%%%%%%%%%%%%%%%%%%%%%%%%%%%%%%%%%%%%%%%%%%%%%%%%%%%%%%%%%%%%%%%%%%%%%%%%%%%%%%%%%%%%%%%%%%%%%%%%%%%%%%%%%%%%%%

\section{Conclusions}
\label{sec:conclusions}

We developed an algorithm to jointly fit gNFW pressure profiles to clusters observed via the SZ
effect with MUSTANG and Bolocam. We apply this algorithm to 14 clusters and find the profiles are 
consistent with a universal pressure profile found in \citet{arnaud2010}. Specifically, the 
pressure profile is of the form:
\begin{equation*}
  \Tilde{P_e} = \frac{P_0}{(C_{500} X)^{\gamma} [1 + (C_{500} X)^{\alpha}]^{(\beta - \gamma)/\alpha}},
%  \label{eqn:norm_gnfw}
\end{equation*}
where we fixed $\alpha$ and $\beta$ to values found in \citet{arnaud2010}. A comparison to previous
determinations of pressure profiles is shown in Figure~\ref{fig:pp_sets}. Within the radii where we
have the greatest constraints ($0.03 R_{500} \lesssim r \lesssim R_{500}$), the pressure profile from this
work is comparable to the other pressure profiles. This is further evidenced in the parameters themselves,
as seen in Table~\ref{tbl:pressure_profile_results}, especially in comparison to A10 parameter values.

Despite agreement for the ensemble constraint, we found discrepancies between the SZ and X-ray derived
pressure profiles for individual clusters and investigated the potential to explain these discrepancies
as being due to cluster geometry and ICM temperature (Section~\ref{sec:xray_comp}). We investigate cluster
geometry by looking at the ratio between spherically derived pressure profiles as fit to SZ and ACCEPT and
we find that the clusters have plane of sky-to-line of sight axis ratios, $\eta$, tabulated in 
Table~\ref{tbl:accept_gnfw}, which are generally less than unity, implying that most of these clusters are 
elongated along the line of sight. We extend this analysis to attempt to find the cluster geometry
in the core (from MUSTANG) relative to the larger scale ICM (from Bolocam) and find some 
hint that the cores tend to be more spherical than the ICM at larger radii. 
%\textcolor{red}{larger scale ICM}. However, our assumption is that the
%cluster geometry is one (or two) ellipsoids and this only explains a scalar offset in the pressure profiles.

In contrast, discrepancies due to electron temperature can account for differences with cluster-centric radii.
To explain the discrepancies we see, the ICM temperatures would need to be higher than derived by 
spectroscopic X-ray data by ACCEPT. These higher temperatures are also biased towards larger radii, which is 
plausible observationally as X-ray measurements would be less sensitive in regions of lower density.

We conclude that cluster geometry and ICM temperature are likely critical in accounting for the differences
between SZ and X-ray derived pressure profiles. However, we do not believe that either of these (with
strict ellipsoids) are sufficient to explain the differences observed. We postulate that deviations from
the ellipsoidal geometry, such as geometry in the plane of the sky as seen, for example, in MACS 1115
(Figure~\ref{fig:mustang_maps_sample}), will also be important in explaining the discrepancies observed.

%We tabulate two temperatures derived for the 
%clusters from ACCEPT \citep{cavagnolo2009} and \citet{morandi2015}: $T_X^1$ is calculated from a single spectrum over 
%$0.15 R_{500} < r < R_{500}$ for each cluster. $T_X^2$ is from \citet{morandi2015} and is calculated over 
%$0.15 R_{500} < r < 0.75 R_{500}$.  Additionally, we calculate $T_{mg}$, by fitting the ACCEPT 
%temperature profiles to the profile found in \citet{vikhlinin2006}. We then tabulate
%$T_{spec}$ as $T_{spec} = 1.11 \times T_{mg}$ given the ratio found in \citet{vikhlinin2006} between $T_{spec}$
%and $T_{mg}$. We may then compare either $T_x^1$ and $T_x^2$ to $T_{spec}$. Thus, we note that generally 
%$T_{spec} < T_{X}^1 \sim T_{X}^2$, which may indicate the ACCEPT pressure profiles may be biased low, relative 
%to the SZ pressure profiles by the deprojected (and interpolated) ACCEPT temperature profiles.

%These investigations of cluster geometry and electron temperatures may account for the discrepancy between
%SZ and X-ray pressure profiles. However, these investigations have been simplistic and are indicative of
%the global geometry and properties of the cluster (at all radii), with the exception that we have also found
%some indication of a difference in geometry between moderate-to-large radii and core. Yet, the discrepancy
%in pressure profiles is not simply a scalar offset, but that the shapes often differ, especially in that
%the SZ pressure profile is lower than X-ray in the cluster core, and SZ pressure is generally higher at
%large radii. Geometry in the plane of the sky may be able to account for this offset, as evidenced by
%MACS 1115, where an extended decrement to the north of the cluster center is seen by Bolocam, and drives the
%radial pressure up, whereas this extension does not appear in the X-ray surface brightness image.

%Finally, many of these clusters with discrepancies in pressure profiles have weak detections with 
%MUSTANG, and would benefit from additional high resolution observations. While this may not resolve the
%pressure difference at large radii, it could bring the pressure profile shape into greater alignment. 

%%%%%%%%%%%%%%%%%%%%%%%%%%%%%%%%%%%%%%%%%%%%%%%%%%%%%%%%%%%%%%%%%%%%%%%%%%%%%%%%%%%%%%%%%%%%%%%%%%%%%%%%%%%%%%%%
%%%%%%%%%%%%%%%%%%%%%%%%                    BIBLIOGRAPHY!!!                          %%%%%%%%%%%%%%%%%%%%%%%%%%%
%%%%%%%%%%%%%%%%%%%%%%%%%%%%%%%%%%%%%%%%%%%%%%%%%%%%%%%%%%%%%%%%%%%%%%%%%%%%%%%%%%%%%%%%%%%%%%%%%%%%%%%%%%%%%%%%

\bibliographystyle{apj}
\bibliography{mycluster}
\label{references}

\end{document}


