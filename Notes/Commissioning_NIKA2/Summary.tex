%----------------------------------------------------------------------------------------
%	SUMMARY
%----------------------------------------------------------------------------------------
The main measured parameters that define the actual NIKA2 performances are gathered in Table~\ref{nika2summary}.

\begin{table}[h]
\begin{center}    
  \begin{threeparttable}
    \begin{tabular}{|r|c|c|c|c|}
      \hline
      & Array 1 & Array 3  & Array 1\&3 & Array 2 \\
      \hline
      \hline
      Reference Wavelength  [mm]  &  1.2   &  1.2  & 1.2 & 2.0   \\
      Reference Frequency  [GHz]  &  260   &  260  & 260 & 150  \\
      Central Frequency [GHz]     &  255.5  &    257.8     &     &   151.6  \\
      Bandwidth         [GHz]     &   47.8  &     45.7     &     &    42.1  \\
      \hline
      Number of designed detectors       & 1140      &  1140    &    &    616  \\
      Number of valid detectors          &  952      &   961    &    &    553  \\
      Fraction of valid detectors [$\%$] &           &          &    &         \\
      Effective FOV\tnote{a}\hspace{1mm} [arcmin]    &   5.39    &   5.61    &    &   4.9  \\
      \hline
      Pixel size in beam sampling unit [F$\lambda$]  &    &   &    &   \\
      \hline
      FWHM\tnote{b}\hspace{1mm} [arcsec]  &  $11.3 \pm 0.2$   &  $11.2 \pm 0.2$  &   $11.2 \pm 0.1$  &  $17.7 \pm 0.1$ \\
      Beam efficiency\tnote{c}\hspace{1mm} [\% ]    &        &    &     &    \\
      rms of the FWHM on the FOV [$\%$]   &   &    &   &  \\
      \hline 
      rms calibration error [\%]            & 4.5  & 6.6  &   & 5  \\
      \hline
      Absolute calibration uncertainty [\%] &  \multicolumn{4}{|c|}{5} \\
      \hline
      $\alpha$ noise integration in time\tnote{d}\hspace{1mm}  &   &   &   &  \\
      \hline
      rms pointing error    [arcsec]    & \multicolumn{4}{|c|}{$<3$}  \\
      \hline
      NEFD\tnote{e}\hspace{1mm}   [$\rm{mJy} \cdot \rm{s}^{1/2}/\rm{beam}$]  &  30 (15)   & 30 (15)  &  30 (15)  & 20 (10) \\
      Mapping speed\tnote{f}\hspace{1mm} [arcmin$^2$/h/mJy$^2$] & 302  & 454  & 775 (1184)  & 7542 (10861)  \\
\hline

\end{tabular}
  \begin{tablenotes}
{\small     
  \item[(a)] Equivalent FOV covered by the valid detectors
  \item[(b)] Full-width at half-maximum of the main beam modelled as a two-dimensional Gaussian fitted from sidelobe-masked beam maps.
  \item[(c)]  Ratio between the main beam power and the total beam power up to a radius of XXX arcsec
  \item[(d)] Effective power law of noise reduction with integration time
  \item[(e)] NEFD in typical IRAM good sky opacity condition: 2mm pwv, $60^o$ elevation
  \item[(f)] Average (best) mapping speed at zero opacity for the February 2017 observation campaign. 
}
  \end{tablenotes}
\end{threeparttable}
\caption{Summary of the main characteristics describing the measured 
performances of NIKA2. {\bf FM : 260 GHz is 1.15 mm}.}
\label{tab:nika2summary}
\end{center}  
\end{table}


The performance parameters given in Table~\ref{nika2summary} are splitted in two different lists: first, the main characteristics, as defined in the MOU, are listed in Table~\ref{nika2summary_main}, second, other parameters, which are derived from the instrument characteristics described in the MOU, and that need to be characterized to complete the commissioning phase are given in Table~\ref{nika2summary_second}. Table~\ref{nika2summary_second} is constructed from the 'secondary' and 'tertiary' tables of Samuel's summary document. 

\begin{table}[h]
  \caption{Summary of the main characteristics describing the measured performances of NIKA2, as listed in MoU}
  \label{nika2summary_main}
  \begin{threeparttable}
    \begin{tabular}{|r|c|c|c|c|}
      \hline
      & Array 1 & Array 3  & Array 1\&3 & Array 2 \\
      \hline
      \hline
      NEFD\tnote{a}\hspace{1mm}   [$\rm{mJy} \cdot \rm{s}^{1/2}/\rm{beam}$]  &  30 (15)   & 30 (15)  &  30 (15)  & 20 (10) \\
      Number of designed detectors       & 1140      &  1140    &    &    616  \\
      Number of valid detectors          &  952      &   961    &    &    553  \\
      Fraction of valid detectors [$\%$] &           &          &    &         \\
      \hline
      FWHM\tnote{b}\hspace{1mm} [arcsec]  &  $11.3 \pm 0.2$   &  $11.2 \pm 0.2$  &   $11.2 \pm 0.1$  &  $17.7 \pm 0.1$ \\
      \hline
      Effective FOV\tnote{c}\hspace{1mm} [arcmin]    &   5.39    &   5.61    &    &   4.9  \\
      \hline
      Pixel size in beam sampling unit [F$\lambda$]  &    &   &    &   \\
      \hline      
\end{tabular}
  \begin{tablenotes}
  \item[(a)] NEFD in typical IRAM good sky opacity condition: 2mm pwv, $60^o$ elevation
  \item[(b)] Full-width at half-maximum of the main beam modelled as a two-dimensional Gaussian fitted from sidelobe-masked beam maps.
  \item[(c)] Equivalent FOV covered by the valid detectors
    \end{tablenotes}
\end{threeparttable}
\end{table} 


\begin{table}[h]
  \caption{Summary of other NIKA2 performance characteristics either defined in the MoU or extracted from SL's summary document}
  \label{nika2summary_second}
  \begin{threeparttable}
    \begin{tabular}{|r|c|c|c|c|}
      \hline
      & Array 1 & Array 3  & Array 1\&3 & Array 2 \\
      \hline
      \hline
      Reference Wavelength  [mm]  &  1.2   &  1.2  & 1.2 & 2.0   \\
      Reference Frequency  [GHz]  &  260   &  260  & 260 & 150  \\
      Central Frequency [GHz]     &  255.5  &    257.8     &     &   151.6  \\
      Bandwidth         [GHz]     &   47.8  &     45.7     &     &    42.1  \\
      \hline
      Beam efficiency\tnote{a}\hspace{1mm} [\% ]    &        &    &     &    \\
      rms of the FWHM on the FOV [$\%$]   &   &    &   &  \\
      \hline 
      rms calibration error [\%]            & 4.5  & 6.6  &   & 5  \\
      \hline
      Absolute calibration uncertainty [\%] &  \multicolumn{4}{|c|}{5} \\
      \hline
      $\alpha$ noise integration in time\tnote{d}\hspace{1mm}  &   &   &   &  \\
      \hline
      rms pointing error    [arcsec]    & \multicolumn{4}{|c|}{$<3$}  \\
      \hline
      Mapping speed\tnote{b}\hspace{1mm} [arcmin$^2$/h/mJy$^2$] & 302  & 454  & 775 (1184)  & 7542 (10861)  \\
\hline

\end{tabular}
  \begin{tablenotes}
  \item[(a)] Ratio between the main beam power and the total beam power up to a radius of XXX arcsec
  \item[(b)] Average (best) mapping speed at zero opacity for the February 2017 observation campaign. 
  \end{tablenotes}
\end{threeparttable}
\end{table}
