%\chapter{Résumé des performances}

Ce chapitre vient clore la partie de ce document consacrée au
\emph{commissioning} de NIKA2 et la caractérisation de ses
performances.
L'essentiel du travail d'analyse a été réalisé sur une période
d'environ deux ans, entre septembre 2016, soit après l'intervention
majeure sur l'instrument, et décembre 2018. En septembre 2017, la
revue des performances de NIKA2 par l'IRAM a
marqué la fin de la phase de \emph{commissioning} en intensité, et la
livraison de l'instrument a été parachevée en décembre 2018 avec la
remise d'un document détaillant la calibration et la caractérisation des
performances.

Responsable de la caractérisation des performances de NIKA2 depuis
septembre 2016, j'ai dirigé toutes les phases de ce travail, jusqu'à
sa concrétisation par la remise du document à l'IRAM et l'écriture de
l'article de référence~\citep{Perotto2019}. Cette activité
d'encadrement a été rythmée par l'organisation des réunions du
consortium NIKA2, d'abord hebdomadaires puis bi-mensuelles; la création et
l'animation de la \emph{Commissioning Tiger Team}, au sein de laquelle
a été réalisée la caractérisation des performances; la présentation
des performances lors de la revue de l'IRAM; la formation d'une
personne de l'IRAM à la calibration \emph{Baseline}; la coordination
de l'écriture du document \emph{Commissioning\&Performance}, puis celle
de l'article. Il s'agit de l'expérience d'encadrement de la recherche
la plus complète que j'ai effectuée jusque-là.    

Parmi nos réalisations les plus marquantes, deux ont vocation à avoir
un impact important. La première est la méthode de calibration de
référence que nous avons développée pour NIKA2, lui assurant
robustesse aux effets systématiques et stabilité aux conditions
d'observation. Ensuite, l'article détaillant nos méthodes et nos
résultats vise à faire référence pour les utilisateurs de NIKA2 et
pour la calibration de futurs instruments millimétriques, en
particulier utilisant la technologie des KID.

Ici, nous résumons les principaux résultats décrits dans cet article,
tels qu'ils ont été détaillés au chapitre précédent, puis nous
proposerons quelques pistes d'amélioration des performances de
NIKA2 pour le futur.  


\section{Les performances}

Les performances de NIKA2 au télescope de 30-m de l'IRAM ont été
caractérisées à partir d'un vaste ensemble d'observation de sources de
calibration, de sources de pointage et de sources faibles, effectuées
lors de trois campagnes d'observation réparties au cours d'une année. Cet
ensemble forme un échantillon représentatif des conditions
d'observation sur le site.

La caractérisation des performances repose sur une calibration de
référence, la calibration \emph{Baseline}, que nous avons développée
et intensivement testée. Cette calibration est fondée sur quatre
choix méthodologiques principaux. La soustraction du bruit corrélé des
détecteurs repose sur une méthode de décorrelation de \emph{modes
  communs} mesurés dans les données, qui a été largement testée pour
le traitement des sources ponctuelles. Le flux des sources ponctuelles
est mesuré par l'ajustement d'une gaussienne de référence de FWHM fixe
et pour une fréquence de référence. L'atténuation de l'atmosphère est
corrigée en utilisant une version modifiée de la méthode employée dans
NIKA~\citep{Catalano2014}, basée sur la mesure de l'opacité
atmosphérique par les KID. L'élargissement apparent du
lobe de l'instrument, lié aux variations de température extérieure,
est traité par une sélection des scans retenant 16 heures
d'observations par jour.

Les performances sont définies par l'évaluation d'un ensemble de
caractéristiques instrumentales. Les principaux résultats de cette
mesure sont regroupés dans le tableau~\ref{tab:nika2summary} et
résumés ci-dessous.

\begin{table}[!thbp]
  \caption{Résumé des principales caractéristiques instrumentales de NIKA2}
  \label{tab:nika2summary}
  \centering    
  \begin{tabular}{rrrcl}
  \hline\hline
  \noalign{\smallskip}
  & Array 1\&3 & Array 2 & & Commentaire \\
  \noalign{\smallskip}
  \hline
  \noalign{\smallskip}
  Fréquence de référence [GHz]  & 260  & 150   &  & \footnotesize{voir Sect.~\ref{se:systeme_photo}}  \\
  Fréquence centrale [GHz]      &  254.7\&257.4  & 150.9 &  & \footnotesize{à atmosphère nulle}   \\
  Largeur de bande passante  [GHz]     &   49.2\&48.0   & 40.7  &  & \footnotesize{à atmosphère nulle} \\
  \hline
  \noalign{\smallskip}
  Diamètre du champ de vue     [arcmin] &   6.5       &   6.5   &  &  \\  
  Nombre total de détecteurs            &  1140\&1140 &    616  & & \footnotesize{voir Sect.~\ref{se:matrices}}\\
  Fraction de détecteurs valides [$\%$] &  84         &     90  & & \footnotesize{voir Sect.~\ref{se:KID_selection}} \\
  Échantillonnage du champ de vue \hspace{3mm} [$\lambda/D$] & 1.1 &  0.87 & & \footnotesize{voir Sect.~\ref{se:KID_pointing}} \\
  \hline
  \noalign{\smallskip}
  FWHM de référence\hspace{3mm} [arcsec]          & $12.5$     &   $18.5$  &  & \footnotesize{voir Sect.~\ref{se:systeme_photo}}\\
  FWHM$^c$\hspace{3mm} [arcsec]    &  $11.1 \pm 0.2$  &  $17.6 \pm 0.1$  & & \footnotesize{voir Sect.~\ref{se:mainbeam}}\\
  Efficacité du lobe principal \hspace{3mm} [$\%$] &  $47 \pm 3$   & $64 \pm 3$  &  & \footnotesize{méthode hybride}\\
  Dispersion de la FWHM [arcsec]  &    0.6        &      0.6        & & \footnotesize{\citet{Adam2018}} \\
  \hline
  \noalign{\smallskip}
  Erreur de calibration absolue [\%]      &   5         & 5 & &  \footnotesize{voir Sect.~\ref{se:syste}}\\
  Erreur de calibration systématique\hspace{3mm}  [\%]      &    0.6        & 0.3 & & \footnotesize{pwv = 2\,mm, $\elev = 60\degree$} \\
  Erreur de calibration RMS [\%]          &   5.7       &     3.0       & & \footnotesize{voir Sect.~\ref{se:rms_error}} \\
  Erreur de pointage    [arcsec]          & $<3$ &  $<3$  & & \footnotesize{\citet{Greve1996}} \\
  \hline
  \noalign{\smallskip}
  NEFD \hspace{3mm} [$\rm{mJy} \cdot \rm{s}^{1/2}$]  & $30 \pm 3$  & $9 \pm 1$ &  & \footnotesize{voir Sect.~\ref{se:nefd_mesures}}\\
  M$_{\rm{s}}$\hspace{3mm} [arcmin$^2 \cdot
    \rm{mJy}^{-2} \cdot \rm{h}^{-1}$] & $111 \pm 11$  &  $1388 \pm 174$ &  & \footnotesize{\emph{Mapping speed}}\\
  \hline
  \end{tabular}
\end{table}

Les 2900 KID qui composent les matrices de NIKA2 détectent un signal
astrophysique au moins lors de certaines observations. Nous
définissons des critères pour sélectionner les KID les plus
stables. Nous trouvons une fraction de KID valides (utilisables pour
l'exploitation scientifique) de 84\% dans la bande à 1\,mm et 90\% à
2\,mm. Les détecteurs non-sélectionnés se répartissent aléatoirement
dans le plan focal, de sorte que le champ de vue de 6,5' de diamètre
est bien couvert. \`A partir de la taille réel des détecteurs
(2.75\,mm et 2.0\,mm) et du diamètre de la pupille d'entrée (27\,m),
nous trouvons une taille des détecteurs en unité d'échantillonnage du
lobe de 1,1 et 0,9 $\lambda$/D dans les bandes à 1 et 2\,mm.   

Le lobe principal est bien modélisé par une gaussienne de FWHM
17,6''$\pm$0,1'' à 150\,GHz et 11,1''$\pm$0,2'' à 260\,GHz. Son
efficacité, évaluée par une méthode hybride utilisant les mesures
dédiées au télescope de 30-m, est de 64$\pm$3\% et 47$\pm$3\% à 150 et
260\,GHz, respectivement. Une fraction significative du signal est
donc reçue au-delà du lobe principal, dans la structure complexe du lobe
total, formée de lobes d'erreur et diverses figures de diffraction. La
dispersion de la FWHM dans le champ de vue a été mesurée
dans~\citet{Adam2018} à partir de cartes individuelles par
détecteur. Elle est de 0,6'' dans les deux
canaux de fréquence, en accord avec la dispersion attendue d'après la
mesure des surfaces focales (voir Sect.~\ref{se:focus}).   

\`A partir d'un large échantillon de scans de sources
brillantes, bien représentatif des conditions d'observation au
télescope de 30-m, nous vérifions la stabilité des mesures de flux aux
conditions d'observation et évaluons l'erreur de calibration RMS. Nous
trouvons environ 3\% à 150\,GHz et 6\% à 260\,GHz. Ces résultats
indiquent une stabilité remarquable pour un instrument millimétrique
au sol et valident les choix méthodologiques de notre
calibration. Pour évaluer l'incertitude de calibration, l'erreur RMS
doit être combinée à l'erreur de calibration absolue, qui est de 5\%
dans les deux bandes, et à l'erreur systématique. La contribution
dominante à l'erreur systématique est l'incertitude sur le facteur
correctif appliqué aux mesures d'opacité atmosphérique. \'Evaluée pour
les conditions d'observation d'hiver de référence de l'IRAM (pwv=2\,mm
et $\elev = 60\degree$), cette erreur est inférieure au pour-cent dans
les deux bandes.

\`A partir d'observations longue durée d'une source faible, nous avons
vérifié que le bruit sur le flux diminuait comme l'inverse de la
racine carré du temps d'intégration. La sensibilité de l'instrument a
été évaluée en comparant les résultats de deux méthodes et en
utilisant un échantillon de plus de 1000 scans, afin de
tester sa robustesse aux choix d'analyse et aux conditions
d'observation.
Nous trouvons une densité de flux équivalente au bruit (NEFD) de
$9\pm1$\,mJy.s$^{1/2}$ à 150\,GHz et de $30\pm3$\,mJy.s$^{1/2}$ à
260\,GHz. \`A partir de ces estimations, nous avons dérivé la vitesse
de cartographie (\emph{mapping speed}), c'est-à-dire la zone du ciel
qui peut être cartographiée à une profondeur de 1\,mJy en une heure
d'intégration. Nous trouvons une \emph{mapping speed} de 1388$\pm$174
et 111$\pm$11 arcmin$^2$/mJy$^2$/h à 150 et 260\,GHz. Ces capacités de
cartographie placent NIKA2 parmi les meilleures instruments
millimétriques haute résolution de sa génération~\citep{Tony2019}.


%Les observations de NIKA2 devraient susciter des avancées
%significatives dans plusieurs domaines de l'astrophysique et la cosmologie.

\section{Perspectives d'amélioration}

NIKA2 satisfait tous les critères de performance requis par l'IRAM et
conditionnant son installation définitive au télescope de 30-m. Pour
certaines caractéristiques, l'objectif ambitieux vers lequel tendre,
après des améliorations successives, est déjà atteint, voire
surpassé. C'est le cas de la sensibilité dans le canal d'observation à
150\,GHz. Par exemple, une carte à 150\,GHz, couvrant tout le champ de
vue et à une profondeur de quelques centaines de $\mu$Jy peut être
obtenue en une heure de temps d'observation. Ce canal de fréquence
bénéficie d'une combinaison de conditions favorables, allant d'une plus grande
facilité de construction des matrices de KID à un moindre impact de
l'opacité atmosphérique, en passant par de meilleures performances du
système optique jusqu'au miroir primaire du télescope. Ainsi NIKA2 est
un instrument idéal pour cartographier l'émission diffuse à 150\,GHz
(e.g. les amas de galaxies via l'effet Sunyaev-Zel'dovich). En
revanche, la situation s'inverse dans la bande à 260\,GHz, où se
cumulent les facteurs limitants. Ainsi, la sensibilité à 260\,GHz est
juste acceptable, avec une RMS du bruit dans le champ de vue d'environ
0.7\,mJy en une heure d'observation dans les conditions de référence
de l'IRAM. Plusieurs pistes d'amélioration existent, dont certaines
font d'ores et déjà l'objet de développements. Ces améliorations
concernent le dispositif instrumental, la connaissance de l'instrument
et la méthode de calibration, et la méthode d'analyse des données.

\subsubsection{Améliorations instrumentales}

Le principal facteur limitant la sensibilité à 260\,GHz est la
transmission sous-optimale de la lame dichroïque qui affecte
principalement la composante de polarisation illuminant la matrice A1
mais a aussi un impact sur la matrice A3 (voir
Sect.~\ref{se:gains}). Des développements sont en cours à Cardiff pour
fournir les lames dichroïques combinant grande taille et haute
performance nécessaires à NIKA2 et aux futurs instruments
millimétriques. En parallèle, la construction d'un banc test, incluant
un système optique refroidi à environ 100\,mK, est prévue à Grenoble.

Les bandes passantes des matrices à 1\,mm sont actuellement limitées à
la zone centrale de la fenêtre permise par l'atmosphérique, leur
conférant une robustesse aux variations des conditions atmosphériques.  
Cette caractéristique permet à NIKA2 d'effectuer des observations
exploitables à 260\,GHz même dans des conditions atmosphériques
défavorables, aux prix d'une réduction de la sensibilité dans ce
canal. Des matrices à 1\,mm acceptant une bande passante plus
large ont été développées à l'Institut Néel. 

La bande fréquentielle de l'électronique de lecture NIKEL-AMC limite
le nombre de KID qui peuvent être connectés à une même ligne de
base. Une nouvelle électronique est en cours de développement au LPSC
afin de doubler la largeur de bande. Cette nouvelle capacité pourrait
être exploitée de plusieurs manières : réduire la taille des KID des
matrices à 1\,mm pour un meilleur échantillonnage du champ de vue et à
la clef, un gain en résolution angulaire, espacer les fréquences de
résonance des KID afin de réduire la diaphonie et \emph{in fine}
augmenter le nombre de détecteurs utilisables, résoudre le problème de
diminution du gain des détecteurs en fonction de leur place dans la
bande fréquentielle. 

NIKA2 bénéficierait également d'améliorations du télescope de
30-m. Une rénovation de la surface du miroir primaire limiterait
l'effet d'élargissement apparent du lobe dû aux variations de
température externe. Elle se traduirait aussi par une meilleure
efficacité du lobe principal à 260\,GHz, ce qui contribuerait à
améliorer la sensibilité. 

\subsubsection{Améliorations de la calibration et l'analyse}

Suite aux interventions sur l'instrument et aux améliorations
apportées pendant la période de commissioning, les bandes passantes
ont pu légèrement varier par rapport à celles du dispositif
instrumental avant l'installation au télescope, telles qu'elles
avaient été précisément mesurées au laboratoire. Les bandes passantes
sont actuellement en cours de caractérisation \emph{in situ} au moyen
d'un interféromètre Martin-Pupplett dédié, construit à l'Institut
Néel, et venant se placer devant la fenêtre d'entrée du
cryostat. Cette caractérisation précise permettra de réduire les
incertitudes liées aux corrections de couleur sur les
densité de flux. Plus indirectement et avec sans doute plus d'impact
encore, elle permettra de simuler l'effet de l'atmosphère intégré dans
les bandes passantes à partir de modèles physiques de type
ATM~\citep{Pardo2001, ATM}. Ces simulations, ainsi que la possibilité
d'extrapoler les mesures issues du tau-mètre résident aux fréquences
de NIKA2, seront cruciales pour tester nos méthodes d'estimation de
l'opacité atmosphérique pour chaque scan. \`A la clef, l'objectif est
d'améliorer la méthode \emph{skydip} pour l'estimation de l'opacité
atmosphérique, afin d'obtenir des densités de flux robustes à
l'atténuation de l'atmosphère, sans avoir recours à un facteur
correctif.

La déformation résiduelle du miroir primaire sous l'effet de son poids
a un impact sur le flux dépendant de l'élévation à laquelle il
observe. Cet effet est connu sous le nom de corrélation
gain-élévation~\citep{Greve1998b}. Une mesure plus précise de
l'atténuation atmosphérique, elle aussi dépendante de l'élévation, 
permettrait de mesurer l'effet gain-élévation avec une précision
suffisante pour corriger les densités de flux, améliorant encore
l'erreur RMS de calibration.

Une autre piste d'amélioration de la calibration concerne les
lobes. En particulier, la combinaison de cartes, sur laquelle est
mesurée le lobe total, est affectée par le filtrage des grandes
échelles angulaires dû au traitement du bruit corrélé. Ce filtrage,
qui devient non négligeable à des distances radiales $\gtrsim 90''$,
nous impose le recours à une méthode hybride pour évaluer les
contributions au lobe au-delà de cette distance. Or le grand champ de
vue de NIKA2 pourrait être mieux mis à profit pour fournir une carte
du lobe du télescope de 30-m plus précise. Pour cela, deux méthodes sont
envisageables. Une observation dédiée vers une source compacte très
brillante, telle Mars, dans des conditions d'observation favorables,
permettrait de construire une carte étendue sans soustraction de
l'atmosphère et donc avec un filtrage minimale des grandes échelles
angulaires. La deuxième piste requerrait le développement de
méthodes de traitement du bruit corrélé qui préservent le signal aux
plus grandes échelles angulaires.


Plus généralement, l'amélioration de la décorrélation du bruit et la
caractérisation fine des effets de filtrages spatiaux constituent des
développements cruciaux et prioritaires de l'analyse de données. Là
encore, plusieurs pistes sont à l'étude, dont certaines mobilisent
déjà d'importants efforts. C'est le cas des méthodes visant à
l'amélioration du \emph{template} soustrait au signal de chacun des
détecteurs~\citep{Ponthieu2020}. Par exemple, dans \citet{Ruppin2019c},
chaque \emph{template} résulte d'une combinaison d'un mode commun par
matrice, d'un mode commun par sous-bandes électroniques et d'une
composante dépendante de la trajectoire effectuée par le détecteur sur
le ciel. Une autre piste actuellement explorée consiste à adapter des
méthodes existantes de fabrication de cartes aux observations de
NIKA2. Par exemple, une version dédiée à NIKA2 de
\emph{Scanamorphos}~\citep{Roussel2013, Roussel2018}, un outil de
correction du bruit basse-fréquence ou \emph{destriage} est
actuellement en cours de tests. Enfin, une réflexion est menée pour
développer des méthodes de type \emph{destrieur} spécifiques pour les
observations avec des KID.

\section{Conclusion}
Les performances de NIKA2 sont conformes aux attentes
et exceptionnelles pour certaines (sensibilité à 150\,GHz). Ces
performances ont motivé la décision de l'IRAM de
l'ouverture de NIKA2 à la communauté scientifique dès octobre 2017.
Il enregistre déjà une vingtaine de campagnes d'observation scientifique et restera
un instrument à demeure du télescope de 30-m de l'IRAM pour
au moins une décennie. L'exploitation scientifique a d'ores et déjà
commencée.

NIKA2 installé au 30-m de l'IRAM est actuellement l'unique opportunité
pour cartographier un champ de vue aussi large (6,5' de diamètre) à
une résolution angulaire aussi fine (mieux que 18'') et simultanément
dans deux bandes de fréquence du domaine millimétrique. Il offre donc
une vue sans précédent sur l'univers millimétrique. Ainsi, il fournira
des observations critiques, aussi bien aux échelles planétaires et
galactiques qu'à hauts \emph{redshifts}, pour permettre des avancées
majeures dans de nombreux domaines de l'astrophysique et la
cosmologie~\citep[\emph{e.g.}][]{Rigby2018, Bracco2017, Bethermin2017_simu, Mancuso2016,
  Ruppin2019a}.
%Par exemple, les observations de NIKA2 seront clef pour
%quantifier l'impact de l'environment sur les propriétés de la
%poussières et le processus de formation planétaire, pour comprendre la
%formation stellaire et sonder son évolution à toutes les époques, pour
%améliorer notre connaissance de la formation des structures à grande
%échelle et leur évolution à travers les ages cosmiques et sonder
%l'univers avec les amas de galaxies.   


NIKA2 est actuellement le meilleur instrument millimétrique pour la cartographie
haute-résolution des amas de galaxies à redshifts intermédiaires à
élevés ($z\in[0.5, 0.9]$) via l'effet Sunyaev-Zel'dovich. Son champ de
vue englobe un lobe de \emph{Planck}, tandis que sa résolution
angulaire est 17 fois meilleure. En alliant une haute résolution
angulaire et une grande vitesse de cartographie, il permet la
caractérisation des amas depuis leur cœur jusqu'à leur
périphérie. Les deux bandes de fréquence permettent de capturer
la signature spectrale de l'effet SZ pour une séparation
efficace des contaminants : la bande à 150\,GHz inclut le
maximum du signal SZ négatif, tandis que la bande à 260\,GHz mesure un
signal SZ très légèrement positif. L'exploitation des données de NIKA2
pour l'étude des amas de galaxie en tant que sonde cosmologique
constituera une importante partie de mon activité de recherche dans les
années à venir. La thématique de la cosmologie avec les amas de
galaxie, au cœur de mon projet de recherche, sera détaillée dans la
dernière partie de ce manuscrit.

