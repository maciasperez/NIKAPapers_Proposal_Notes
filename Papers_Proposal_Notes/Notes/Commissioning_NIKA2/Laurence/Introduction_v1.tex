%\chapter*{Introduction}


\emph{Le Cadre théorique.} La cosmologie a vu émerger un modèle
cohérent et efficace pour décrire essentiellement toutes les
observations cosmologiques, tout en laissant de larges zones d'ombre
quant à son interprétation théorique. Ce \emph{modèle
standard de la cosmologie} est fondé sur le paradigme de perturbations
primordiales gaussiennes, adiabatiques et quasi-invariantes d'échelle
générées à l'issue d'une phase d'inflation et croissant par
instabilité gravitationnelle dans un Univers en
expansion. Son contenu en énergie est
dominé par l'\emph{énergie noire}, de nature encore inconnue mais
compatible avec une constante cosmologique $\Lambda$, tandis
que la matière existe principalement sous la forme de
\emph{matière noire froide} (CDM) -- la forme de matière telle qu'elle
est décrite par le modèle standard de la physique des particules ne
constituant que quelques pourcents du contenu
total~\citep[pour des mesures récentes,
voir \emph{e.g}][]{Planck2018_cosmo, BOSS2017, eBOSS2019, DES2019,
SNLS2014, PANSTARR2018}. L'Univers est également empli d'un fond de
neutrinos qui, en se basant sur les indications d'une
masse non-nulle fournies par les expériences en laboratoire,
contribue pour une petite fraction (quelques dixièmes de pourcent) à
la densité de matière sombre~\citep[voir pour une
revue][]{Lesgourgues2006, Lesgourgues_Book}. C'est le modèle $\Lambda$CDM le plus
simple, qui offre un cadre à la plupart des analyses cosmologiques.\\


\emph{Le fond diffus cosmologique.} La mesure du fond diffus
cosmologique (CMB par la suite), rayonnement relique, émis lors du
découplage matière-radiation environs 380\,000
ans après le \emph{Big-Bang}, est l'une des observations fondatrices
de ce modèle cosmologique. L'étude des anisotropies de la température
du CMB et de la polarisation, e.g. \citet{MaBertschinger1995} pour
une approche technique, donne accès à une riche phénoménologie,
\emph{e. g.}~\citet{HuNature} pour une approche imagée. En effet, les
anisotropies du CMB nous fournissent une image unique des
perturbations de la métrique dans
l'Univers primordial (voir \emph{e.g.} \citet{Peebles1970}
ou \citet{Zeldovich1970} pour un article "historique'').
%~\reference{White, Silk, Bond, Peebles}.
Ces perturbations apparaissent à l'issue d'une phase d'expansion
exponentielle de l'Univers, l'inflation~\citep{Guth1981}. Une des
signatures de l'inflation est la production de modes tensorielles de
perturbations, généralement associés à un fond d'ondes gravitationnelles
primordiales, et qui auraient laissé une empreinte dans la
polarisation du CMB~\reference{Kamionkowsky1997}. Celle-ci est décrite par deux
quantités scalaires qui diffèrent par leurs propriétés de parité, le
mode E (pair) et le mode B (impair), définies de sorte que le mode B
primordial soit uniquement généré par les modes
tensorielles~\reference{Lewis}. Sa mesure offrirait donc une
opportunité unique d'accéder à la physique de l'inflation. Quant à la
mesure du mode E, elle est complémentaire de celle de la température
pour contraindre la cosmologie~\reference{Galli} et est, de plus, très sensible
à la physique de la réionisation de l'univers~\reference{??}.\\

%%%%%%%%%%%%%%%%%%%%%%%%%%%%%%%%%%%%%%%%%%%%%%%%%%%%%%%%%%%%%%%%%%%
\emph{Les résultats de Planck.} Le domaine du CMB a récemment été
marqué par les résultats du satellite \emph{Planck} de l'agence
spatiale européenne (ESA). \emph{Planck} a observé cinq fois l'ensemble du ciel entre
2009 et 2013 en température et en polarisation avec une résolution
angulaire allant jusqu'à cinq minutes d'arc et dans neuf bandes de
fréquence centrées entre 30\,GHz à 857\,GHz~\reference{??}. Il est en
outre, le premier satellite à avoir exploiter l'effet de lentille
gravitationnelle que les grandes structures de l'univers imprime sur
le CMB en tant que sonde cosmologique~\reference{lensing2013,
  lensing2015}. Depuis les premiers résultats cosmologiques basés sur
deux observations complètes du ciel en température en
2013~\reference{cosmo2013} jusqu'aux plus récents résultats utilisant
l'ensemble de la mission en 2015~\reference{cosmo2015} et incluant les
plus grandes échelles angulaires en polarisation en
2018~\reference{cosmo2018}, il a confirmé que le modèle cosmologique
$\Lambda$CDM le plus simple suffisait pour décrire les données. Ce
modèle est paramétré par six quantités uniquement, qui fixent le
contenu (les densités physiques de CDM et baryons), les conditions
initiales (l'amplitude et l'indice spectral du spectre de puissance
des perturbations de densité primordiales), la géometrie (l'échelle
angulaire de l'horizon du son au moment du découplage) et l'époque de
la réionisation de l'Univers. Les extensions à ce modèle sont
fortement contraintes en combinant les données du CMB avec des mesures
des oscillations acoustiques des baryons (BAO). Cette approche est
particulièrement puissante car 
l'échelle angulaire des oscillations acoustiques, qui ont eu lieu dans
le plasma primordial après l'égalité matière-radiation, est mesurée à
la fois à la surface de dernière diffusion (CMB) et imprimée dans la
distribution de matière de l'univers plus récent (BAO). Combinant les
données CMB de \emph{Planck} et les BAO mesurées par XXX, la platitude
de l'Univers est confirmée à quelques pour mille. Tandis qu'en
combinant le CMB avec les mesures de luminosité des supernovae de type
Ia (SNIa), qui sont des sondes de l'évolution de l'expansion de
l'univers, une énergie noire en accord avec une constante cosmologique
est confirmée avec une équation d'état égale à l'unité à quelques pour
mille près. Finalement, dans le secteur des neutrinos, le nombre
d'espèces relativistes (e.g. neutrinos) présentes bien avant la
recombinaison est compatible avec trois, défavorisant l'hypothèse de
l'existence d'une espèce de neutrinos stériles comme invoquée pour expliquer
l'anomalie dans les mesures d'oscillation de neutrinos en provenance
de réacteur~\reference{SLB}, et l'échelle absolue de masse des
neutrinos est inférieure à 0.12\,eV (95\%), très proche de la limite
la plus favorable permise par les mesures d'oscillation des neutrinos
(hiérarchie de masse inversée).\\

\emph{Les expériences CMB au sol.} En parallèle, les dix dernières
années ont vu l'avènement d'expériences CMB au sol accroissant leur
sensibilité en multipliant le nombre de détecteurs jusqu'à atteindre plusieurs dizaines de
milliers, et utilisant de grands télescopes pour atteindre des
résolutions angulaires de l'ordre de la minute d'arc~\reference{Keck
  Array, SPT, ACT, POLARBEAR} ou de la dizaine de secondes
d'arc~\reference{NIKA2}. En plus des anisotropies primaires du CMB,
ces expériences exploitent au mieux les anisotropies secondaires, les
empreintes laissées par la matière sur le CMB bien après le découplage
via divers processus physiques~\reference{Aghanim}. Via l'effet
Sunyaev-Zel'dovich thermique (tSZ)~\reference{SZ70}, plusieurs
expériences sol ont produit des catalogues d'amas de
galaxies~\reference{SPT-SZ, ACT}. Cet effet décrit la distortion
spectrale due à la diffusion Compton inverse des photons du CMB sur le
plasma chaud ionisé présent en particulier au sein des amas de
galaxies~\reference{revue?}. Par ailleurs, la première mesure du
potentiel gravitationnel intégré le long de la ligne de visée $\phi$,
reconstruit via l'effet de lentille gravitationnelle sur le CMB, est
publiée en 2013 par une équipe du \emph{South Pole
Telescope}~\reference{SPT2013}. Cette première mesure sera améliorée en
utilisant la reconstruction de l'effet de lentille sur la polarisation
du CMB~\reference{SPTpol}, et complètée par les mesures issues d'autres
expériences~\reference{ACT, ACTpol, Polarbear}. Finalement, les expériences
sol (ou embarquées en ballon) exploitent leur complémentarité pour
mesurer efficacement la polarisation. Des expériences visant une très
haute sensibilité pour une résolution angulaire relativement faible,
qui ciblent une détection du mode B primordial de polarisation, telles
BICEP, sont complétées par des expériences à plus haute résolution
angulaire, telles Keck Array, assurant une mesure des émissions
astrophysiques d'avant-plan et de l'effet de lentille
gravitationnelle. Cette stratégie est généralisée dans la convergence
des développements instrumentaux au sol (et en ballon) dans la
méta-collaboration S4~\reference{S4}, engagée dans la course à la
détection du mode B primordial, qui ouvrirait une fenêtre unique sur
la physique de la phase d'inflation cosmique. Les meilleures
contraintes actuelles sur les modes B primordiaux, paramétrées par le
rapport entre l'amplitude du spectre de puissance des modes
tensorielles et scalaires des perturbations primordiales à une échelle
angulaire pivot $k = 0.002\,\rm{Mpc}^{-1}$, sont $r_{0.002} < 0.07$,
obtenues dans une étude conjointe des données de \emph{Planck},
BICEP2, Keck Array en combinaison avec les
BAO~\reference{Planck2018}. Pour la mesure des modes B primordiaux,
qui constitue l'objectif scientifique principal de nombre de futurs projets
expérimentaux ambitieux (par ex. LiteBIRD), l'effet de lentille sur le
CMB est une thématique centrale. Une mesure précise du mode B induit
par effet de lentille, qui domine le mode B primordial sauf aux plus
grandes échelles angulaires, est cruciale.\\

\emph{L'effet SZ à haute résolution angulaire.} L'un des résultats les plus
intriguants de \emph{Planck} est le désaccord, ou la \og tension \fg
pour utiliser le terme préféré par la collaboration \emph{Planck},
entre le modèle cosmologique favorisé par le CMB et celui favorisé
par les amas de galaxies sélectionnés via l'effet
tSZ~\reference{cosmo2015, cosmo2018}. Cette tension est observée aussi
bien lorsque le modèle est dérivé du comptage des amas du
catalogue \emph{Planck}~\reference{SZ2015, SZ2018}, que lorsqu'il est 
contraint avec le spectre de puissance et les corrélations d'ordre
supérieure de la carte du paramètre de Compton produite
par \emph{Planck}~\reference{ymap2015}. Ces contraintes sont également
confirmées par les étude du comptage des amas des catalogues produits
par ACT~\reference{ACT-SZ} et SPT~\reference{SPT-SZ}. En outre, la
ré-interprétation des résultats cosmologiques dans le cadre
d’extension au modèle $\Lambda$CDM minimal
ne parvient à réconcilier les deux mesures (CMB primaire et effet SZ)
qu’au prix d’accroître les incohérences avec d’autres sondes cosmologiques~\reference{??}.
L’interprétation favorisée à ce jour pour cette tension est un effet
systématique entâchant l’exploitation des amas de galaxie en tant que
sonde cosmologique, et qui serait lié à l’incertitude sur la relation
entre l’observable et la masse totale des amas. Une meilleure
caractérisation de cette relation, et plus généralement une meilleure
connaissance des phénomènes physiques complexes mis en jeu dans le
milieu intra-amas, est nécessaire pour leur exploitation en cosmologie. 
Il s’agit en sus, de repérer d’éventuels effets d’évolution, qui
biaiseraient de manière subtantielle les amas à plus haut redshift,
puisque les relations masse-observable sont calibrées sur des amas
proches. Ce programme nécessite une résolution angulaire bien
meilleure que celle de \emph{Planck}, ou même que celle des
expériences au sol sus-citées, de résolution angulaire de l'ordre de
la minute d'arc. Une étude détaillée du milieu intra-amas à haut
redshift nécessite des expériences capables de cartographier un grand
champ de vue avec une résolution angulaire bien meilleure que la
minute d'arc~\reference{Tony2019}. C’est là, l’un des programmes
cosmologiques phare de l’expérience NIKA2, qui permettra la
cartographie d’un échantillon représentatif des amas de galaxies
détectés par \emph{Planck} et ACT, avec l’objectif d’améliorer les
résultats cosmologiques fondés sur ses sondes.\\

\emph{L'expérience NIKA2.} NIKA2 est une expérience installée au
télescope de 30 mètres de l'Institut de RadioAstronomie Millimétrique
(IRAM) capable d'observer un champ de vue de 6,5 minutes d'arc, avec
une résolution angulaire de l'ordre de la dizaine de secondes d'arc,
simultanément dans deux bandes de fréquence centrées à 150 et
260\,GHz et sensible à la polarisation dans le canal à
260\,GHz~\reference{Adam2018}. Une de ses particularités est
d'utiliser une technologie de détection alternative aux bolomètres qui
équipent habituellement les expériences CMB. NIKA2 comprend 2\,900
détecteurs à inductance cinétiques ou KID (\emph{kinetic inductance
detectors}), qui sont des résonateurs
superconducteurs~\reference{??}. Après la première lumière de NIKA2 au
télescope de 30-m de l'IRAM en octobre 2015 avec une électronique de
lecture encore incomplète, la phase de mise en service
(\emph{commissioning}) dans la configuration instrumentale finale a
débuté en janvier 2017. Elle s'est achevée en avril 2017 après une
phase de vérification scientifique probante. NIKA2 est ouverte à la
communauté scientifique depuis octobre 2017 et restera un instrument à
demeure du télescope de 30-m de l'IRAM pour au moins les dix
prochaines années. En sus de la cartographie haute-résolution des amas
de galaxies via l'effet SZ, NIKA2 fournira des observations sans
précédent, aussi bien à l'échelle des systèmes planétaires que des
grandes structures de l'univers, d'intérêt crucial en astrophysique et
cosmologie. NIKA2 constitue aussi une prouesse technologique et
instrumentale. La calibration des données issues de plusieurs milliers
de KID opérant dans le domaine millimétrique est une première
mondiale. Les méthodes observationnelles et de traitement de données
développées pour la calibration de NIKA2 et l'évaluation de ses
performances sont décrites dans \reference{Perotto2019}. Une partie
significative de ce manuscrit est consacrée aux méthodes et résultats
du \emph{commissioning} de NIKA2, dont j'ai eu la responsabilité.\\ 

\emph{Cosmologie avec les amas de galaxies dans Euclid.}
Les amas de galaxies, qui forment le stade le plus évolué
de la formation hiérarchique des structures de l’Univers, constituent la
sonde cosmologique ultime, permettant d’accéder à toute l’histoire de
l’Univers~\reference{Allen2011}. Ils contiennent toute l’information
cosmologique depuis les conditions initiales qui prévalaient dans
l’univers primordial, jusqu’à son évolution plus récente dominée par
l’énergie noire. Ils sont particulièrement sensibles au facteur de
croissance des structures et à tout ce qui peut l’affecter (énergie
noire, masse des neutrinos, etc.).
L’information cosmologique est principalement extraite par comptage
des amas en fonction de leur masse et de leur redshift. Mais aussi,
dans le contexte des futurs grands rélévés cosmologiques, l’étude des
corrélations de leur distribution spatiale (\emph{clustering}) est une
approche à vocation à se développer.
Comme précédemment évoqué, les amas sont le lieu d’une physique baryonique
complexe, qu’il convient de bien comprendre pour une exploitation
précise en Cosmologie, et qui les rend observables dans une vaste
gamme de fréquence, dans les domaines radio, millimétrique, X, optique
et infrarouge, en plus de leurs effets purement gravitationnels (effets de
lentille). A son tour, cette propriété permet de déployer les
approches multi-longueur d’onde, multi-sondes qui seront nécessaires
aux futurs grands relevés de galaxies pour la calibration en masse des
amas à plus hauts redshifts.
A moyen terme, le futur des grands relevés de galaxies s'incarne dans
deux projets phares, le \emph{Large Synoptique Survey Telescope}
(LSST) au sol~\reference{LSST} et le satellite de
l'ESA \emph{Euclid}~\reference{euclid}. [DESI, WFIRST, KIDS ?]. J'ai
rejoint le consortium \emph{Euclid} en 2016 avec pour projet de
participer à l'exploitation des amas de galaxies en cosmologie.\\


Ce manuscrit s'articule en trois parties. Dans la première partie, je
donne un rapide coup de projecteur à deux résultats marquants obtenus
avec PLANCK sur l'effet de lentille gravitationnelle sur le CMB. Cette
partie me permet d'introduire des notions de cosmologie générale et de
présenter le domaine du CMB, avant de me focaliser brièvement sur l'effet de
lentille sur le CMB. La deuxième partie est consacrée à la mise en
service de l'expérience NIKA2 et à l'évaluation de ses performances au
télescope de 30 mètres de l'IRAM. Cette partie, qui donne son titre au
manuscrit sera plus étoffée. On y trouvera une rapide présentation de
l'expérience NIKA2, une description de la procédure développée pour la
calibration, ainsi que la méthodologie et les résultats de
l'évaluation des performances. La dernière partie est consacrée aux
développements récents de mon activité de recherche et à mes projets à
court et moyen termes. Elle comporte un volet sur
le programme cosmologique de NIKA2 à partir de l'échantillon d'amas de
galaxies observés via l'effet SZ. J'y souligne aussi les synergies
avec d'autres sondes des amas de galaxies, en particulier le
rayonnement X. Ce volet, pourrait faire l'objet d'un sujet de thèse à
brève échéance. Dans un deuxième volet, je présente
l'expérience \emph{Euclid} et la préparation à l'exploitation des amas
de galaxies pour la cosmologie. 



