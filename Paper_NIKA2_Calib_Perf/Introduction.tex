%----------------------------------------------------------------------------------------
%	INTRODUCTION
%----------------------------------------------------------------------------------------
%\section{Introduction}
%\label{se:intro}

%Submillimetric domain: a unique thorough view of the Universe from
%planetary surfaces in the Solar System to early galaxies,
%high-redshift galaxy clusters and the Cosmic Microwave
%Background (CMB).\\
%Ground-based submillimeter experiments have made tremendous progress in the last
%decades thanks to the advent of large detector arrays, challenging the
%motivation for spatial experiments.\\
%The current experimental efforts are driven by two complementary
%challenges: the improvement of the polarisation sensitivity to detect
%primordial polarisation B-mode of the CMB and measure CMB lensing on
%one hand and on the other hand the increase the angular resolution to
%reveal the inner structure of astrophysical objects, such as
%high-redshift galaxies and galaxy clusters.\\

Submillimetric and millimetric domains offer a unique view on the
Universe from nearby astrophysical objects, including planets,
planetary systems, galactic sources, nearby galaxies, to high-redshift
cosmological probes, \emph{e.g.} distant dusty star-forming galaxies,
clusters of galaxies, Cosmic Infrared Background (CIB), Cosmic Microwave
Background (CMB).
%for scientific purposes that ranges from planetary science to cosmology.

Ground-based millimeter experiments have made spectacular progress in
the past-two decades thanks to the advent of large arrays of
high-sensitivity detectors~\citep{Wilson2008_AZTEC,
  Siringo2009_LABOCA, Swetz2011_ACT, Bleem2012_SPT, Monfardini2011_NIKA, Holland2013_SCUBA2,
  Dicker2014_MUSTANG2, Adam2017}. %, challenging the
%motivation for spatial experiments.
This fast grow will continue as experimental efforts are
driven by two challenges: improving the sensitivity to the
polarisation to detect the signature of
the end-of-inflation gravitational waves in the CMB, and reaching
arcminute angular resolution to exploit the CMB secondary anisotropies
as cosmological probes, and sub-arcminute resolution to unveil the
inner structure of faint or complex astrophysical objects and to map
the early universe down to the confusion limit. Furthermore, new
generation of sub-arcminute millimetric experiments, like NIKA2, which
combines high resolution with a high mapping speed and a large spectral
coverage, will revolutionize our detailed understanding of the
formation and evolution of structures throughout the Universe.
%
%In particular, it would allow us to study the inner structure of
%galaxy clusters (detected by Planck), and to get a new view on star
%formation at low and high redshift, by studying the role of magnetic
%field at a sub parsec scale in our Galaxy, and mapping the dusty star
%forming galaxies up to redshifts equal to 6.
%
%NIKA2 is a high-resolution large field-of-view multi-thousand detector
%camera in operation at the 30-m IRAM telescope since October 2017. ,
%after a successful commissioning of the instrument. 
%Fourth generation of
%imagers installed at the IRAM 30-m: after the single beam
%imagers, the 7-pixel camera of the late 90's and several versions of
%MAMBO. Unlike its predecessors, NIKA2 is based on KIDs. Validated with
%NIKA pathfinder.\\
%Brief historic of the commissioning. \\
%The instrument commissioning was successfully completed in September
%2017.
%NIKA2 is now open to the scientific community and will operate
%for the next decade.

The New IRAM KID Array Two, NIKA2, is a subarcminute-resolution
high-mapping speed camera observing simultaneously a 6.5' diameter
field of view (FOV) in intensity in two
frequency channels centered at 150 and $260\,\rm{GHz}$ and in
polarisation at $260\, \rm{GHz}$. NIKA2 was first installed at the
30-meter IRAM telescope in October 2015 with a partial readout
electronics, and operated in the final instrumental configuration since
January 2017. After a successful
commissioning phase that ended in October 2017, NIKA2 is
open to the community for science-purpose observations for the next
decade. NIKA2 will provide key observations both at the galactic scale
and at high redshifts to address plethora of open questions, including
the environment impact on dust properties, the star formation process
at low and high redshifts, the evolution of the large-scale structures
and the exploitation of galaxy clusters for accurate cosmology.
%
%NIKA2, which will provide unprecedented maps at low and high
%redshifts, ranging from the interstellar medium to the dusty star
%forming galaxies up to redshifts equal to 6, will be a key experiment
%to address plethora of open questions in astrophysic and cosmology,
%ranging from the understanding of the star formation process at low
%and high redshift to  and accurate cosmology with galaxy clusters 
%

At the galactic scale, progress in understanding the star formation
process relates to an accurate characterisation of dust properties in
the interstellar medium (ISM). NIKA2 will provide the high-resolution
high-mapping speed dual-wavelength millimeter observations that are
required for the determination of the mass and temperature of
statistically significant samples of dense, cold, star-forming
molecular clouds~\citep{Rigby2018}.
Deep multi-wavelengths surveys of nearby galaxies and of large areas
of the galactic plane also allows for setting constraints on
environmental-related variations of the dust properties.
Furthermore, NIKA2 observations are needed for a
detailed study of the inner molecular cloud filamentary structure that
hosts the Solar-mass star progenitors~\citep{Bracco2017}, to
understand the evolution process that culminates in star
formation (see \emph{e.g.}~\citet{Andre2014} for a review). Ultimately, these
observations are also helpful to understand planet formation within
protoplanetary disks.

For cosmology, NIKA2 observations will have two major
implications. On the one hand, they represent an unique opportunity to
constrain the evolution of the galaxy cluster mass calibration with
redshift and morphology for their accurate exploitation as cosmological probe. 
Galaxy clusters are efficiently detected via the thermal
Sunyaev-Zel'dovich (tSZ) effect up to high redshifts, as was recently
proven by recent CMB
experiments~\citep{Hasselfield2013_ACT_SZ, Reichardt2013_SPT_SZ, Planck2016_SZcat}.
The exploitation of the vast SZ-selected galaxy cluster catalogues is
currently the most powerful approach for cosmology with galaxy
clusters~\citep{Planck_2016_SZ_cosmo}. However, the accuracy of the tSZ cluster
cosmology relies on the calibration of the relation between the tSZ
observable and the cluster mass and on the assessment of both the redshift
evolution and the impact of the complex cluster physics. 
%Cosmological constraints can be drawn from tSZ-selected cluster
%samples by using either the cluster number counts or the angular power
%spectrum. It requires a high purity sample to avoid bias, and in
%particular the understanding of the distribution of matter within the
%cluster, and the evaluation of the scatter induced by disturbed
%systems in the relation between the tSZ integrated flux and the total
%cluster mass.
Previous arcminute resolution experiments only allowed detailed studies
of the intra cluster medium morphology for low redshift clusters (z <
0.2). Sub-arcminute resolution high mapping speed experiments are
required to extend our understanding of galaxy cluster towards high
redshift. \citet{Ruppin2018} has reported the first high-resolution
tSZ mapping of a galaxy cluster with NIKA2. Furthermore, NIKA2
capabilities for the characterization of high-redshift galaxy clusters
has been demonstrated using simulation~\citep{Ruppin2019}.

On the other hand, in-depth mapping of large extragalactic fields with
sub-arcminute resolution with NIKA2 will provide unprecedented insight
on the distant universe. %Combining high mapping speed and sub-arminute
%resolution is mandatory to
Indeed, hundreds of dust-obscured optically-faint galaxies will be
detected up to high redshift
($z>4$) during their major episodes of star formation. For this
purpose, the high-angular resolution, combined with high mapping
speed, is key to reduce the confusion noise, which is the ultimate
limit of cosmological surveys~\citep{Bethermin2017_simu}. 
This will help quantifying the star formation at $z>4$, probing the whole
star formation evolution across cosmic ages and clarifying the
role of the dusty galaxies in the reionization of the
universe~\citep{Mancuso2016}. Moreover, NIKA2 observations will be critical for our
understanding of galaxy formation and evolution.

%[ATTENTION COPIE]
%Similarly, distant universe studies via deep surveys will
%benefit from a large instantaneous field-of-view and sensitivity
%to cover sky regions at the confusion limit. This results
%in detecting dust-obscured optically faint galaxies during
%their major episodes of star formation in the early universe
%(B\'ethermin et al. 2017, Geach et al. 2017).\\

The current generation of sub-arcminute resolution experiments also
include the Large APEX Bolometer Camera
(LABOCA~\citep{Siringo2009_LABOCA}) at the Atacama
Pathfinder Experiment (APEX) 12-meter telescope, which covers a
12' diameter FOV at $345\, \rm{GHz}$; AzTEC at the
50-meter Large Millimeter Telescope, which operates with a
single bandpass centered at either 150, 220 or
$270\,\rm{GHz}$~\citep{Wilson2008_AZTEC}; the Submillimeter Common User Bolometer
Array Two (SCUBA-2~\citep{Holland2013_SCUBA2,Dempsey2013_SCUBA2}) on the
15-meter James Clerk Maxwell Telescope, which simultaneously
images a FOV of about 7' at $850\, \mu\rm{m}$ ($353\,
\rm{GHz}$) and $450\, \mu\rm{m}$
($666\,\rm{GHz}$); MUSTANG-2 at the 100-meter Green Bank telescope,
which maps a 6' FOV at
$90\,\rm{GHz}$~\citep{Dicker2014_MUSTANG2, Stanchfield2016_MUSTANG2}.
Therefore, NIKA2 is unique
in combining a resolution better than 20'', an instantaneous FOV of a
diameter of 6.5' and multi-band observation capabilities at $150$ and
$260\,\rm{GHz}$.


Most of these instruments consists of bolometric cameras. By contrast,
NIKA2 is based on the Kinetic Inductance Detectors (KID)
technology~\citep{Day2003, Doyle2008_LEKID, Shu2018_LEKID}. This concept has been
first demonstrated with a pathfinder instrument,
NIKA~\citep{Monfardini2010_NIKA, Monfardini2011_NIKA}.
Installed at the IRAM 30-m telescope until 2015, NIKA demonstrated
state-of-the-art performance~\citep{Catalano2014}, and obtained
breakthrough results
(see \emph{e.g.},~\citet{Adam2014, Adam2017_kSZ}.
NIKA has been crucial in optimizing the NIKA2 instrument and data
analysis. 

%Key tool for a large variety of astrophysical and cosmological
%purposes, from Kuiper Belt objects or moons in the Solar system to
%high-redshift galaxies and galaxy clusters. The main scientific
%obsjectives are articulated around five ambitious guaranteed-time
%large programs [add a small paragraph per LP]\\
%
%These scientific program requires high performance standard in term of
%mapping capabilities, sensitivity, angular resolution and stability of
%the system. We assess this performance using a large amount of
%technical and science-purpose data acquired since the full
%installation of NIKA2.\\

A thorough description of the NIKA2 instrument is found in \citet{Adam2018},
along with the results of the commissioning in intensity based on the
data acquired during the two technical campaigns of 2017. In this paper, we
detail the dedicated procedures that we have developed for the
calibration. We also assess the performance in intensity using a large
amount of data acquired between January 2017 and February 2018. The performance
assessment of the polarization capabilities will be addressed in a
forthcoming paper.
To achieve a reliable and high-accuracy estimation
of the camera performance we test stability
against observing conditions and analysis
 methodological choices. We compare several data analysis approaches,
and we check the stability of the result with a large number of
independent data sets corresponding to variable observing conditions.
%To achieve
%high-accuracy estimation of each key performance characteristics,
%we test their stability against the methodological choices by
%comparing various analysis approaches, and we check their stability
%againts the observating conditions using vast independent data sets.
Specifically, most of the performance assessment relies on data
acquired during the February 2017 technical campaign (N2R9) and the
October 2017 (N2R12) and January 2018 (N2R14) first and second
scientific-purpose observation pools. Each campaign consists of about
1300 observation scans lasting between two and twenty minutes for a
total observation time of about 150 hours.
%This large data set
%acquired through a year enables for testing against time-evolution and
%observing-dependency of the performance.

%Outlines
This paper constitutes a review of NIKA2 calibration and
performance assessment in intensity. It is intended to be a reference
for the observations with NIKA2, which will last at least ten years. 
The outline is as follows:
Sect.~\ref{se:instru}~to~\ref{se:dataproc} give short summaries of the
instrumental set up, the observational modes and the data analysis methods
that have been used for the calibration and the performance
assessment. Sect.~\ref{se:geometry}~to~\ref{se:sensitivity} detail the
dedicated calibration methods, extract the key characteristic results
and discuss their accuracy and robustness. The field-of-view
reconstruction and the KID selection for science purpose are discussed
in Sect.~\ref{se:geometry}. The beam pattern is characterised in
Sect.~\ref{se:beam}, along with the main beam
full-width at half maximum and the beam
efficiency. Sect.~\ref{se:opacity} is dedicated to the derivation of
the atmospheric opacity. The methods that we have proposed to
calibrate are gathered in Sect.~\ref{se:calibration}, while
Sect.~\ref{se:photometry} presents the validation of these methods and
the calibration accuracy and stability assessment. The noise
characteristics and the sensitivity are discussed in
Sect.~\ref{se:sensitivity}. Finally, Sect.~\ref{se:summary} summarizes
the main measured performance characteristics and offers perspectives
for their improvements. 















