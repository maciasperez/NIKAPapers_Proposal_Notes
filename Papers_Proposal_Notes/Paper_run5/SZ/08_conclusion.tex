The cluster RX~J1347.5-1145 is an ongoing merger, among the most-studied galaxy clusters at arcmin angular scales, making it a good target for the first tSZ observations with the {\it NIKA} prototype camera. Using a dual-band decorrelation with a high resolution instrument, we have imaged the tSZ morphology of the cluster from the core to its outer region. The detailed data analysis is specific to KIDs and to tSZ observations and has been validated on simulations. The observation of \mbox{RX~J1347.5-1145} constitutes the first tSZ observations with an instrument based on KIDs.

The reconstructed tSZ map of \mbox{RX~J1347.5-1145} is reliable on scales going from about 20 to 200~arcsec and shows a strong southeast extension that corresponds to the merger shock, as expected from the overpressure caused by the ongoing merger. We detect the non-alignment of the tSZ maximum and the \mbox{X-ray} center, which agrees with other single-dish data but disagrees with {\it CARMA} interferometric data. The tSZ extension is also observed in the radial flux profile of the cluster and the residual of the map with respect to the modeling of the relaxed part of the cluster. The generalized NFW fit of the NW region enables us to constrain the cluster pressure profile parameters $\theta_s$ and $P_0$. The pressure profile derived from \mbox{X-ray} agrees with this tSZ best-fit model.

The tSZ map and the radial profile measured with {\it NIKA} have been compared to {\it DIABOLO} observations at the same telescope with similar resolution and frequency coverage. The agreement between the two maps validates the tSZ observations presented in this work. In addition, the {\it NIKA} prototype map agrees with state-of-the-art sub-arcmin resolution tSZ observations, {\it MUSTANG} (90~GHz and 8~arcsec resolution) and {\it CARMA} (30 -- 90~GHz and $\sim$ 15~arcsec resolution) except for the tSZ peak position. The comparison shows that it is complementary to these experiments.

In this paper, KID arrays of the {\it NIKA} prototype have been proven to be competitive detectors for millimeter wave astronomy and in particular for the observation of galaxy clusters via the tSZ effect. The next generation instrument, {\it NIKA2}, consists of about 1000 detectors at 140~GHz and 4000 at 240~GHz with a field of view of  $\sim 6.5$ arcmin. With these characteristics, {\it NIKA2} is able to provide large high-resolution mapping of clusterss making it an ideal instrument for high-resolution observations of intermediate to large distance clusters of galaxies. The instrument {\it NIKA2} will be well adapted for a follow-up of unresolved sources in the {\it Planck} cluster sample \citep{Planck_survey}. 
