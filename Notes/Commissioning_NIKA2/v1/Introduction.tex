

\subsection{Goals of the total power commissioning}
The main goal of the NIKA2 commissioning runs, described below, is to 
characterize the instrument and check its 
performance with respect to the specifications 
described in Table \ref{nika2specs}. 
We also characterize the performance stability 
against various observing conditions and 
assess the methods and the precision with which the performance parameter are measured.


\begin{table}[ht]
\begin{center}
\begin{threeparttable}
\begin{tabular}{|r|c|c|}
  \hline
  \hline
Reference Wavelength  [mm]  &  1.2 & 2.0  \\
Reference Frequency  [GHz]  &  260 & 150  \\
\hline  
\hline
FOV diameter [arcmin]       &  5 (6.5)    &  5 (6.5)   \\
Pixel size in beam sampling unit [F$\lambda$]  &  0.9 (0.6)   &   0.9 (0.6)  \\
FWHM  [arcsec]              &  12 (10)   &  18 (16) \\
Fraction of valid detectors [$\%$] &  50 (90)   &  50 (90) \\
NEFD\tnote{a}\hspace{1mm}   [$\rm{mJy} \cdot \rm{s}^{1/2}/\rm{beam}$]  &  30 (15)   &  20 (10) \\
\hline
NEFD [$\rm{mJy} \cdot \rm{s}^{1/2}/\rm{beam}$] goal on $90\%$ of the pixels  &  15  & 10 \\
NEFD [$\rm{mJy} \cdot \rm{s}^{1/2}/\rm{beam}$] specification on $50\%$ of the pixels  &  30  &  20  \\
\hline
\end{tabular}
\begin{tablenotes}
{\small
  \item[(a)] NEFD in typical IRAM good sky opacity condition: 2 mm pwv, $60^\circ$ elevation
%  \tablefoottext{a}{NEFD in typical IRAM good sky opacity condition: 2mm pwv, $60^o$ elevation}
}
\end{tablenotes}
\end{threeparttable}
\caption[Main characteristics defining the expected performances of NIKA2.]{Each parameter is associated with two values: the first one indicates the \emph{specifications}, i.e. the requirements to be met by the instrument, while the second bracketed one gives the \emph{goals}, i. e. the values targeted by the collaboration.}
\label{nika2specs}
\end{center}
\end{table} 


\subsection{Technical campaigns and open pools}
We had 10 commissioning runs for NIKA2 as described in Table~\ref{nika2runs}.

\begin{table}[h]
\small
\begin{center} 
\begin{tabular}{|c|c|c|c|c|}
\hline 
RUN  & NIKA Run & Starting date    & End date         &  General comments \\
\hline
N2R1     & 13       & 29-October-2015   & 10-November-2015 & Not full instrumentation        \\
N2R2     & 14       & 24-November-2015  & 02-December-2015 & 13 NIKEL boards working         \\
N2R3     & 15       & 12-January-2016   & 01-February-2016 & 20 NIKEL boards                 \\
N2R4     & 16       & 1-March-2016      & 15-March-2016    & 	                               \\
Dark run & 17       & 4-May-2016        & 4-May-2016       & Dark tests with N2R4 conditions  \\
\hline
N2R5     & 18       & 16-September-2016 & 11-October-2016  & New dichroic, corrugated lenses, \\
         &          &                   &                  &  new array 2 mm, new electronics \\
N2R6     & 19       & 25-October-2016   & 1-November-2016  &                                  \\
N2R7     & 20       & 6-December-2016   & 13-December-2016 & Test external calibrator         \\
N2R8     & 21       & 9-January-2017    & 13-January-2017  & Replace array 1 lens by smooth one, \\
         &          &                   &                  &  adjust the alignement of internal optics \\ 
         &          & 24-January-2017   & 25-January-2017  & Tests on the sky   \\
N2R9     & 22       & 21-February-2017  & 28-February-2017 &                                   \\
N2R10    & 23       & 18-April-2017     & 25-April-2017    & End of commisioning phase 1,     \\
         &          &                   &                  & Science verification  \\  
% HA: 24
N2R11  & 24       &  8-June-2017   & 13-June-2017  &  polarization commissioning  \\
\hline
\end{tabular}
\caption[Commissioning campaigns, dates and general comments.]{
\label{nika2runs}}
\end{center} 
\end{table} 


\subsection{Data set for the performance assessment}

The performance assessment relies mainly on data acquired during N2R9, that is the first complete technical campaign for which the instrumental set up was in the final configuration. \\
For stability check through several campaigns, these data are complemented with the two first Open Pools data. 
