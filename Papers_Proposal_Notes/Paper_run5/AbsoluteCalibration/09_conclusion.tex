In this paper we have shown that \NIKA\ is a competitive instrument for
millimeter wave astronomy using KID detector technology.
We presented several instrumental and data analysis improvements including:
\begin{itemize}
\item a reliable optimization of the detectors working point that significantly increased the number of valid detectors and their responsivity;
\item an automatic self atmospheric absorption correction along the line of sight;
\item a data analysis pipeline adapted to KID specifics. 
\end{itemize}
These lead to a significant improvement of the performance in terms of measured NFED
and to accurate photometry on point sources.
%. In
%particular some of the major aspects that limited the calibration accuracy of
%\NIKA\ were the real time optimisation of the detectors working point and the
%atmospheric absorption correction. These improvements have been described in
%this work with a discussion about how these technical improvements permitted
%to reach astrophysical quality data by presenting results obtained for several
%astrophysical point like and extended sources. 

Table \ref{tab:table_fin} summarizes the main \NIKA\ 
characteristics and performance as measured on the sky. We obtained a 
sensitivity (averaged over all valid detectors) of 40 and 14~mJy.s$^{1/2}$
for the best weather conditions for the 1.25~mm and 2.14~mm arrays, respectively,
estimated on point like sources.
Additionally, the camera performance 
can be quantified with its mapping speed: the area that can be observed per unit time at a given sensitivity.    
\\
\\
The future \NIKAii\ will be made of about 1000 detectors at 2.14~mm and 2 $\times$ 2000 at
1.25~mm with a circular field of view of $\sim 6.5$ arcmin diameter. \NIKAii\
will be commissioned at the end of 2015. In addition the \NIKAii\ instrument
will have linear polarization capabilities at 1.25~mm. The performance in
polarization will be tested in the \NIKA\ camera during the 2014 observation
campaigns.

