%\chapter*{Introduction}


NIKA2 -- et plus largement les expériences du domaine
millimétrique capables de résoudre l'effet Sunyaev-Zel'dovich (SZ) au
sein des amas de galaxies -- aura un apport crucial pour la cosmologie
avec les amas de galaxies. Les performances de NIKA2, et les
développements pour leur caractérisation, font l'objet de la première
partie du manuscrit, tandis que son programme en cosmologie fondée sur
les amas de galaxies constitue le sujet principal de la deuxième
partie. Forte d'une riche histoire qui a
participé de la fondation du modèle standard de la
cosmologie~\citep[voir \emph{e.g.}][pour une revue]{Allen2011},
l'utilisation des amas de galaxies via des analyses multi-longueur
d'onde sera une thématique centrale et structurante de la cosmologie pour les dix
années à venir. Pour l'appréhender, je présente d'abord un état des
lieux de la cosmologie actuelle et présente les développements
expérimentaux phares de la décennie qui s'ouvre.\\



\emph{Le Modèle Standard.} La cosmologie a vu émerger un modèle
cohérent et efficace pour décrire essentiellement toutes les
observations cosmologiques, tout en laissant de larges zones d'ombre
quant à son interprétation théorique. Ce \emph{modèle
standard de la cosmologie} est fondé sur le paradigme de perturbations
primordiales gaussiennes, adiabatiques et quasi-invariantes d'échelle
générées à l'issue d'une phase d'inflation et croissant par
instabilité gravitationnelle dans un Univers spatialement plat et en
expansion pour former les grandes structures de l'Univers. Son contenu
en énergie est dominé par l'\emph{énergie noire}, de nature encore
inconnue mais compatible avec une constante cosmologique $\Lambda$, tandis
que la matière existe principalement sous la forme de
\emph{matière noire froide} (CDM) -- la forme de matière telle qu'elle
est décrite par le modèle standard de la physique des particules ne
constituant que quelques pour cents du contenu
total~\citep[pour des mesures récentes,
voir \emph{e.g}][]{Planck_2018_cosmo, BOSS2017, eBOSS2019,
DES2018_cosmo, DES2019_cosmo, SNLS2014, PANSTARR2018}. L'Univers est également empli d'un fond de
neutrinos qui, en se basant sur les indications d'une
masse non-nulle fournies par les expériences en laboratoire,
contribue pour une petite fraction (quelques dixièmes de pour cent pour
la masse minimale permise $\Sigma m_\nu = 0.06\,\rm{eV}$) à
la densité de matière sombre~\citep[voir pour une
revue][]{Lesgourgues2006, Lesgourgues_Book}. C'est le modèle
$\Lambda$CDM le plus simple, qui aujourd'hui suffit pour décrire les
données cosmologiques.\\

\emph{Les mesures actuelles.} Ce résultat a récemment été
confirmé dans l'analyse cosmologique finale
de \emph{Planck}~\citep{Planck_2018_cosmo}. Ce satellite de l'agence
spatiale européenne (ESA) a observé cinq fois l'ensemble du ciel entre
2009 et 2013 avec une résolution angulaire allant jusqu'à cinq minutes
d'arc et dans neuf bandes de fréquence centrées entre 30\,GHz à
857\,GHz pour cartographier les anisotropies de la
température du CMB et de la polarisation~\citep{Planck_2013_mission,
Planck_2015_mission, Planck_2018_mission}. Ainsi depuis les premiers
résultats cosmologiques basés sur deux observations complètes du ciel
en température en 2013~\citep{Planck_2013_cosmo} jusqu'aux plus récents
résultats utilisant l'ensemble de la mission en
2015~\citep{Planck_2015_cosmo} et incluant les plus grandes échelles
angulaires en polarisation en 2018~\citep{Planck_2018_cosmo}, le modèle
cosmologique $\Lambda$CDM le plus simple, déterminé par six
paramètres, fournit la meilleure description des données CMB seules ou
en combinaison avec d'autres sondes cosmologiques, comparée à des
modèles étendus. Ces six paramètres fixent le contenu (les densités
physiques de CDM et baryons), les conditions initiales (l'amplitude et
l'indice spectral du spectre de puissance des perturbations de densité
primordiales), la géométrie (l'échelle angulaire de l'horizon du son
au moment du découplage) et l'époque de la réionisation de l'Univers. 

Les extensions à ce modèle sont fortement contraintes en combinant
plusieurs sondes cosmologiques sensibles aux conditions initiales, à
la géométrie et à la croissance des structures. 
Une approche puissante consiste à combiner les données du CMB avec des mesures des
oscillations acoustiques des baryons (BAO). L'échelle angulaire des
oscillations acoustiques, qui ont eu lieu dans
le plasma primordial après l'égalité matière-radiation, est mesurée à
la fois à la surface de dernière diffusion (CMB) et imprimée dans la
distribution de matière de l'univers plus récent (BAO), brisant la
dégénérescence géométrique. Combinant les
données CMB de \emph{Planck} et les BAO mesurées dans les données
complètes du Baryon Oscillation Spectroscopic Survey (BOSS) de
SDSS-III, la platitude de l'Univers est confirmée à quelques pour
mille~\citep{BOSS2017, Planck_2018_cosmo}. Tandis qu'en
combinant le CMB avec les mesures de luminosité des supernovæ de type
Ia (SNIa), qui sont des sondes de l'évolution de l'expansion de
l'univers, une énergie noire en accord avec une constante cosmologique
est confirmée avec une équation d'état égale à l'unité à quelques pour
mille près~\citep{SNLS2014, PANSTARR2018, Planck_2018_cosmo}.
Finalement, dans le secteur des neutrinos, le nombre
d'espèces relativistes (e.g. neutrinos) présentes bien avant la
recombinaison est compatible avec trois, défavorisant l'hypothèse de
l'existence d'une espèce de neutrinos stériles comme invoquée pour expliquer
l'anomalie dans les mesures d'oscillation de neutrinos en provenance
de réacteur~\citep{LSND2001, STEREO2019}, et l'échelle absolue de masse des
neutrinos est inférieure à 0.12\,eV (95\%), très proche de la limite
la plus favorable permise par les mesures d'oscillation des neutrinos
(hiérarchie de masse inversée).\\


\emph{Ce paysage est susceptible de radicalement changer dans la décennie
2020 avec l'avènement de grands projets dédié à la cosmologie,
promettant un progrès spectaculaire des sondes cosmologiques les plus
à même de questionner le modèle cosmologique actuel.} \\

\emph{Les futurs développements expérimentaux.}
Dans le domaine du CMB, les expériences au sol s'appuient sur les
succès de la génération précédente qui comprend BICEP et Keck
Array~\citep{BK2018}, SPT-SZ et SPTpol~\citep{deHaan2016, SPTpol2019},
ACT et ACTPol~\citep{Swetz2011_ACT, ACTpol2017} ou
Polarbear~\citep{Polarbear2017}. Les résultats cosmologiques les plus
marquants concernent la recherche de mode B primordial pour contraindre les scénario
d'inflation~\citep{BKP2015, BK2016}, la mesure de l'effet de lentille
gravitationnel sur le CMB~\citep[\emph{e.g.}][]{ACT2011_lensing, POLARBEAR2019_lensing,
SPTpol2019_lensing, SPTpol2019_clusterlensing} et l'exploitation de l'effet
Sunyaev-Zel'dovich (SZ) pour la construction de catalogues de galaxies
et de cartes du paramètre de Compton~\citep[\emph{e.g.}][]{Hasselfield2013_ACT_SZ, Dunkley2013, ACTpol2018_SZ, Reichardt2013_SPT_SZ,
George2015, Bleem2015, deHaan2016, Bleem2019, Bocquet2019}. Fortes de ces succès, elles
continuent de se développer rapidement
avec BICEP3~\citep{BICEP3_2018}, \emph{Advanced} ACTPol~\citep{AdvACT2018},
SPT-3G~\citep{SPT3G_2018} ou \emph{Simons Array}~\citep{SA_2016}.
La stratégie consistant à exploiter la synergie
entre les expériences optimisées pour la mesure des grandes échelles
angulaires et la course au mode B primordial, et les expériences
installées au foyer de grands télescopes visant des mesures à haute
résolution angulaire leur permettant d'exploiter les anisotropies
secondaires -- et en premier lieu l'effet de lentille et le SZ --
comme sondes cosmologiques. Cette stratégie s'incarne dans la
méta-collaboration multi-sites CMB-S4, dont le précurseur,
le \emph{Simons Observatory}~\citep{SO2019} vise une première lumière
en 2021.

Du coté de l'optique et du proche infrarouge (IR), le début de la
décennie va voir les résultats cosmologiques finaux des expériences
actuelles, telles \emph{Hyper Suprime-Cam} au télescope de 8.2 mètres
Subaru~\citep[HSC, ][]{SUBARU2018}, le \emph{Kilo-Degree Survey} au
VLT \emph{Survey Telescope}~\citep[KiDS, ][]{KiDS2013} ou le \emph{Dark Energy
Survey} qui utilise la camera DECam installée au télescope Blanco de 4 mètres
de l'ESO~\citep[DES, ][]{DES2016}. En parallèle, le
spectrographe \emph{Dark Energy Spectroscopic instrument}~\citep[DESI,
][]{DESI2016}
récemment installé au télescope de 4 mètres Mayall fournira ses
premières données et seront mis en
service les deux projets phares que sont l'observatoire au sol de la \emph{National Science Foundation} \emph{Vera C. Rubin Observatory}~\citep[VRO, anciennement LSST, ][]{LSST2009}, équipé de la plus grande caméra CCD jamais construite %piloté par les Etats-Unis
et le satellite de l'ESA \emph{Euclid}, qui embarque
un imageur optique et un spectro-imageur IR~\citep{EUCLID2011}. Le relevé du VRO, appelé \emph{Legacy Survey of Space and Time} (LSST) couvrira 18 000 dégrés carrés dans six bandes optiques, tandis que le relevé d'\emph{Euclid} couvrira l'ensemble du ciel (soit 15 000 degrés carrés hors du plan galactique et des pôles écliptiques) dans une large bande optique et trois bandes dans le proche infrarouge.    
Ces grands relevés optiques/NIR combineront de multiples sondes de la
croissance des structures et de l'histoire de l'expansion de l'Univers.
Ils utiliseront aussi bien les propriétés statistiques de la distribution
des galaxies, les effets de lentille gravitationnelle, les supernovæ
(DESI, LSST), les comptages d'amas de galaxies ou la statistiques des vides
cosmiques. 

La prochaine génération de grands observatoires en rayon X ont un
important programme en cosmologie basé sur l'exploitation des amas de
galaxies~\citep{eROSITA_cosmo}. Pour la décennie 2020, le satellite
eROSITA~\citep{eROSITA2014} qui a été lancé en juillet 2019 effectuera
un relevé très haute sensibilité sur tout le ciel. La décennie 2030
sera marquée par le futur grand satellite de l'ESA,
Athena~\citep{ATHENA2013},
équipé à la fois d'une caméra à grand champ de vue et d'un calorimètre
à haute-résolution spectrale pour une compréhension fine de la
physique des baryons dans les amas de galaxies. 
%adjoindre au relevé d'amas une
%cartographie de la vitesse propre du gaz dans les amas.

Plus récemment, et suivant les résultats de
EDGES~\citep{EDGES2018Natur}, pléthore d'expériences se développent
autour de la mesure de la transition hyperfine de l'hydrogène neutre
afin de sonder la distribution spatiale de matière à très hauts
redshifts et les débuts
de l'époque de la réionisation. Ces projets suivent deux voies
parallèles, l'une visant une haute résolution angulaire avec le projet
\emph{Square Kilometer Array}~\citep[SKA][]{SKA2018} à l'horizon 2030 et ses
précurseurs, et l'autre se développant autour de la cartographie en
intensité~\citep[par exemple][]{TIANLAI2018, HIRAX2016} avec en perspective le
futur projet \emph{Packed Ultra-wideband Mapping
Array}~\citep[PUMA,][]{PUMA2019}.\\

Ces programmes vont pouvoir tester les propriétés de l'énergie
noire afin d'en comprendre la nature; mesurer l'échelle absolue de la
masse des neutrinos, aussi petite soit-elle; contraindre des
extensions à la Relativité Générale; tester les propriétés de la
matière noire et rechercher la présence de nouvelles particules.

Par ailleurs, les tensions entre sondes de l'univers primordial (CMB)
et sonde de l'univers plus évolué (les structures à grandes échelles)
révélées dans les résultats de Planck sont peut-être les prémisses
d'une déviation au modèle $\Lambda$CDM de base.
La tension à 3.6$\sigma$ entre les mesures directes de $H_{0}$ utilisant
des supernovæ proches calibrées par les
Céphéides~\citep[\emph{e.g.}][]{Riess2018}, confirmée avec une
signifiance moindre par les mesures du retard de la lumière des
quasars fortement lentillés, et entre la mesure dérivée du
meilleur ajustement du modèle $\Lambda$CDM
de \emph{Planck}~\citep{Planck_2018_cosmo} a suscité une abondante
littérature~\citep[voir par exemple][pour un
résumé]{Verde2019}. Par ailleurs, en moyenne, les sondes des grandes
structures semblent favoriser une amplitude des fluctuations et une
densité de matière ($\sigma_8$ et $\Omega_{\rm{m}}$) plus faibles que
celles dérivées indirectement du modèle $\Lambda$CDM de
Planck~\citep[voir][pour une revue]{Pratt2019}. Cette tension d'une
signifiance $\lesssim 2\sigma$ est observée entre les résultats de
\emph{Planck} et les analyses multi-sondes des relevés optiques DES ou
HSC~\citep[\emph{e.g.}][]{DES2019_cosmo},
mais aussi plusieurs analyses fondées sur les amas de
galaxies~\citep[\emph{e.g.}][]{Bocquet2019}.
De plus, l'effet de lentille sur le CMB reconstruit dans les cartes
de \emph{Planck} est lui-même en désaccord à $2\sigma$ avec les
spectres de puissance (lentillés) du CMB
de \emph{Planck}~\citep{Planck_2018_cosmo}, tandis qu'il est en accord 
avec l'analyse multi-sondes de DES, utilisant toutes les
corrélations à deux points possibles entre la distribution spatiale
des galaxies, le cisaillement gravitationnel et l'effet de lentille
sur le CMB, à l'exception du spectre de puissance de ce
dernier\citet{DES2019_cosmo}.
Pour finir, le désaccord à $2\sigma$ entre les modèles préférés par le
CMB et par l'effet SZ de \emph{Planck}, dans les analyses de comptage
des amas ou d'étude de leur distribution spatiale, tel
qu'annoncé dans \citet{Planck_2014_SZ_Cosmo, Planck_2014_ymap,
Planck_2016_SZ_cosmo, Planck2016_ymap}, s'est sensiblement résorbé
dans l'analyse finale~\citep{Planck_2018_cosmo, Zulbedia2019,
Salvati2018}, au prix de favoriser des propriétés des amas de galaxies
en tension avec notre connaissance actuel du milieu intra-amas, comme
détaillé dans la deuxième partie du manuscrit. Tandis que les tensions
entre divers sondes cosmologiques sont actuellement en général peu
significatives et peuvent être imputées à une méconnaissance des
effets systématiques, elles peuvent aussi être les premières
indications d'une déviation au modèle $\Lambda$CDM le plus simple.\\


\emph{L'amélioration de la prise en compte des effets systématiques
qui affectent les sondes cosmologiques et les tests des propriétés de
l'Univers au-delà de $\Lambda$CDM seront les deux grands axes de la
cosmologie de la décennie 2020. Les amas de galaxies en tant que sonde
cosmologique auront un rôle de premier plan à jouer dans ce programme.}\\


%%%%%%%%%%%%%%%%%%%%%%%%%%%%%%%%%%%%%%%%%%%%%%%%%%%%%%%%%%%%%%%%%%%%%%%%%%%
%%%%%%%%%%%%%%%%%%%%%%%%%%%%%%%%%%%%%%%%%%%%%%%%%%%%%%%%%%%%%%%%%%%%%%%%%%%
\emph{Cosmologie avec les amas de galaxies.}
Les amas de galaxies, qui forment le stade le plus évolué
de la formation hiérarchique des structures de l’Univers, constituent
une sonde cosmologique complète, permettant d’accéder à toute l’histoire de
l’Univers~\citep{Allen2011}. Ils contiennent toute l’information
cosmologique depuis les conditions initiales qui prévalaient dans
l’univers primordial, jusqu'à son évolution plus récente dominée par
l’énergie noire. Ils sont particulièrement sensibles au facteur de
croissance des structures et à tout ce qui peut l’affecter, tel l'énergie
noire, les propriétés des neutrinos, une modification de la Relativité
Générale~\citep{Haiman2001, Wang2005, Bolliet2019, Mohr2003,
Hagstotz2019}.
%Combinées avec une sonde de l’univers primordial, tel le CMB, elles
%vont permettre de contraindre les extensions du modèle ΛCDM standard
%qui affectent la croissance des structures, dans le secteur des
%neutrinos (Wang et al. 2005; Bolliet et al. 2019), de l’énergie noire
%(Haiman et al. 2001) ou des modifications de la gravité (Mohr et
%al. 2003; Hagstotz et al. 2019).
L’information cosmologique est principalement extraite par comptage
des amas en fonction de leur masse et de leur redshift. Mais aussi,
dans le contexte des futurs grands projets cosmologiques, l’étude des
corrélations de leur distribution spatiale (\emph{clustering}) est une
approche à vocation à se développer~\citep[e.g.][]{Mana2013}. Les
simulations numériques fournissent la modélisation de ces deux
quantités, l'évolution de la fonction de masse et les fonctions de
corrélation spatiale. Les deux propriétés fondamentales pour leur
utilisation en cosmologie sont donc leur redshift et leur masse
totale. Cette dernière, qui
n'est pas une observable, doit être soit reconstruite à partir des
effets de lentilles gravitationnels, soit inférée à partir
d'observables qui lui sont fortement corrélées. Les amas de galaxies,
qui sont essentiellement des halos de matière noire (85$\%$) ayant piégés du
gaz chaud (12$\%$) et des galaxies (3$\%$), sont dominés par la
gravité et à ce titre, approximativement invariants d'échelle, ce qui
permet d'extrapoler à l'ensemble des amas, les propriétés mesurées sur
un sous-échantillon d'amas proches, en particulier la relation liant
une observable à la masse totale. Ils sont aussi le lieu d’une
physique baryonique complexe, qu’il convient de bien comprendre pour
une exploitation précise en cosmologie, et qui les rend
observables dans une vaste gamme de fréquence, dans les domaines
radio, millimétrique, X, optique, infrarouge, et même en rayonnement
gamma, en plus de leurs effets purement gravitationnels (effets de
lentille). A son tour, cette richesse observationnelle permet de
déployer les approches multi-longueur d’onde, multi-sondes
qui seront nécessaires aux futurs grands relevés de galaxies pour la
calibration en masse des amas jusqu'aux plus hauts redshifts
observables.\\


Plusieurs des grands projets qui vont marquer la décennie 2020, évoqués
précédemment, placent la cosmologie avec les amas de galaxies au cœur
de leur programme scientifique.

Les deux grands relevés optiques/NIR qui seront en opération au début
des années 2020, LSST et \emph{Euclid} détecteront plusieurs centaines
de milliers d'amas de galaxies via leur richesse,
c'est-à-dire le nombre de galaxies qu'ils contiennent. Plus
précisément, le VRO-LSST permettra la construction d'un catalogue de plus de 300 000 amas
détectés jusqu'à des redshifts $\lesssim 1.4$~\citep{LSST2019}, tandis
que \emph{Euclid} détectera environ 200 000 amas jusqu'à hauts
redshifts, avec en particulier environ 40 000 amas à redshift
$>1$~\citep{Sartoris2016, Adam2019_euclid}. Une reconstruction de la
masse des amas sera possible en exploitant leur propres mesures de l'effet de
lentille gravitationnelle sur les galaxies d'arrière-plan, permettant
ainsi une calibration en interne de la relation
masse-richesse. Les contraintes sur l'évolution et la mesure de la
dispersion de cette relation, en particulier à haut redshifts,
pourront être affinés par des analyses multi-sondes de données
externes fortement corrélées à la masse, telles une combinaison de
données dans le domaine X et d'observation à haute résolution
angulaire de l'effet SZ, par exemple.

Le \emph{Simons Observatory}, en opération autour de 2021, détectera
un nombre d'amas de galaxies via l'effet SZ surpassant d'un ordre de
grandeur celui de \emph{Planck}~\citep{SO2019}, tandis que dans le
cadre de CMB-S4, prévu pour la fin de la décennie, ce nombre sera
porté à 70 000~\citep{CMB-S4}. Les analyses cosmologiques exploiteront
aussi bien le comptage des amas que les propriétés statistiques de la
carte de l'effet SZ thermique. De plus, ces expériences permettront
une reconstruction de l'effet de lentille que les amas impriment sur
le CMB, fournissant ainsi une calibration interne de la relation
masse-observable~\citep{Melin2015, Louis2017}. Finalement, d'autres
effets SZ du second ordre pourront être exploités, à la fois pour une
compréhension plus fine de la physique du milieu intra-amas, mais
aussi comme sondes cosmologiques supplémentaires. C'est le cas en
particulier, de l'effet SZ cinétique, dû au mouvement des amas dans le
référentiel du CMB, qui permettra de contraindre les vitesses propres
de la matière~\citep{SO2019}.


Le satellite eROSITA, lancé en juillet 2019, détectera de l'ordre de
100 000 amas de galaxies via leur luminosité en X jusqu'à un redshift de
l'ordre de 1~\citep{eROSITA_cosmo}. De plus,
pour un sous-échantillon de quelques milliers, la température du
milieu intra-amas sera également mesurée par spectroscopie, permettant
une mesure de la masse totale de l'amas sous l'hypothèse d'équilibre
hydrostatique~\citep{Hofmann2017}. \\  

Dans le cadre de ces futurs programmes, les analyses cosmologiques
basées sur les amas de galaxies fourniront des contraintes
indépendantes des autres sondes cosmologiques et compétitives pour
tester les extensions du modèle $\Lambda$CDM minimal, pourvues que les
effets systématiques susceptibles d'affectés la mesure de la masse
totale soient bien contrôlés~\citep{Mana2013, Krause2017,
Sartoris2016, SO2019, eROSITA_cosmo}.
%L’interprétation favorisée à ce jour pour cette tension est un effet
%systématique entâchant l’exploitation des amas de galaxie en tant que
%sonde cosmologique, et qui serait lié à l’incertitude sur la relation
%entre l’observable et la masse totale des amas.
Une caractérisation précise de la relation entre l’observable et la
masse totale des amas, et plus généralement une meilleure
connaissance des phénomènes physiques complexes mis en jeu dans le
milieu intra-amas, est nécessaire pour leur exploitation en cosmologie. 
Il s'agit en sus, de repérer d’éventuels effets d’évolution, qui
biaiseraient de manière substantielle les amas à plus haut redshift,
puisque les relations masse-observable sont calibrées sur des amas
proches. Pour la calibration de la masse des amas distants, la
reconstruction de la masse par effet de lentille devient plus
hasardeuse avec un nombre moindre de galaxies d'arrière-plan
disponibles. Une voie prometteuse, qui s'est récemment ouverte avec
l'avènement d'expériences SZ à haute-résolution angulaire, est
l'analyse combinée d'observation en X et en SZ à une résolution
angulaire comparable, d'un échantillon d'amas de galaxies
représentatif de la population d'amas générale.
%Ce programme nécessite une résolution angulaire bien
%meilleure que celle de \emph{Planck}, ou même que celle des
%expériences au sol sus-citées, de résolution angulaire de l'ordre de
%la minute d'arc.
%Une étude détaillée du milieu intra-amas à haut
%redshift nécessite des expériences capables de cartographier un grand
%champ de vue avec une résolution angulaire bien meilleure que la
%minute d'arc~\citep{Tony2019}.
C'est là, l'un des programmes cosmologiques phare de l’expérience
NIKA2, qui permettra la cartographie SZ à haute-résolution angulaire
d’un échantillon représentatif des amas de galaxies détectés
par \emph{Planck} et ACT jusqu'à des redshifts de 0.9, avec l'objectif
de contraindre les propriétés des amas fondamentales pour leur
utilisation comme sonde cosmologique.\\


\emph{L'expérience NIKA2.} NIKA2 est une expérience installée au
télescope de 30 mètres de l'Institut de Radio Astronomie Millimétrique
(IRAM) capable d'observer un champ de vue de 6,5 minutes d'arc, avec
une résolution angulaire de l'ordre de la dizaine de secondes d'arc,
simultanément dans deux bandes de fréquence centrées à 150 et
260\,GHz et sensible à la polarisation dans le canal à
260\,GHz~\citep{Adam2018, Perotto2019}. Une de ses particularités est
d'utiliser une technologie de détection alternative aux bolomètres qui
équipent habituellement les expériences CMB. NIKA2 comprend 2\,900
détecteurs à inductance cinétiques ou KID (\emph{kinetic inductance
detectors}), qui sont des résonateurs
supraconducteurs~\citep{Day2003, Doyle2008_LEKID, Roesch2012_LEKID}. Après la première lumière de NIKA2 au
télescope de 30-m de l'IRAM en octobre 2015 avec une électronique de
lecture encore incomplète, la phase de mise en service
(\emph{commissioning}) dans la configuration instrumentale finale a
débuté en janvier 2017. Elle s'est achevée en avril 2017 après une
phase de vérification scientifique probante. NIKA2 est ouverte à la
communauté scientifique depuis octobre 2017 et restera un instrument à
demeure du télescope de 30-m de l'IRAM pour au moins la décennie 2020-2030. En sus de la cartographie haute-résolution des amas
de galaxies via l'effet SZ, NIKA2 fournira des observations sans
précédent, aussi bien à l'échelle des systèmes planétaires que des
grandes structures de l'univers, d'intérêt crucial en astrophysique et
cosmologie~\citep[\emph{e.g.}][]{Rigby2018, Bracco2017, Bethermin2017_simu, Mancuso2016,
  Ruppin2019a}. NIKA2 constitue aussi une prouesse technologique et
instrumentale. La calibration des données issues de plusieurs milliers
de KID opérant dans le domaine millimétrique est une première
mondiale. Les méthodes observationnelles et de traitement de données
développées pour la calibration de NIKA2 et l'évaluation de ses
performances sont décrites dans \citet{Perotto2019}.\\
%Une partie
%significative de ce manuscrit est consacrée aux méthodes et résultats
%du \emph{commissioning} de NIKA2, dont j'ai eu la responsabilité.\\ 


%\emph{Les expériences CMB au sol.} En parallèle, les dix dernières
%années ont vu l'avènement d'expériences CMB au sol accroissant leur
%sensibilité en multipliant le nombre de détecteurs jusqu'à atteindre
%plusieurs dizaines de milliers, et utilisant de grands télescopes pour
%atteindre des résolutions angulaires de l'ordre de la minute d'arc~\reference{Keck
%  Array, SPT, ACT, POLARBEAR} ou de la dizaine de secondes
%d'arc~\reference{NIKA2}. En plus des anisotropies primaires du CMB,
%ces expériences exploitent au mieux les anisotropies secondaires, les
%empreintes laissées par la matière sur le CMB bien après le découplage
%via divers processus physiques~\reference{Aghanim}. Via l'effet
%Sunyaev-Zel'dovich thermique (tSZ)~\reference{SZ70}, plusieurs
%expériences sol ont produit des catalogues d'amas de
%galaxies~\reference{SPT-SZ, ACT}. Cet effet décrit la distortion
%spectrale due à la diffusion Compton inverse des photons du CMB sur le
%plasma chaud ionisé présent en particulier au sein des amas de
%galaxies~\reference{revue?}. Par ailleurs, la première mesure du
%potentiel gravitationnel intégré le long de la ligne de visée $\phi$,
%reconstruit via l'effet de lentille gravitationnelle sur le CMB, est
%publiée en 2013 par une équipe du \emph{South Pole
%Telescope}~\reference{SPT2013}. Cette première mesure sera améliorée en
%utilisant la reconstruction de l'effet de lentille sur la polarisation
%du CMB~\reference{SPTpol}, et complètée par les mesures issues d'autres
%expériences~\reference{ACT, ACTpol, Polarbear}. Finalement, les expériences
%sol (ou embarquées en ballon) exploitent leur complémentarité pour
%mesurer efficacement la polarisation. Des expériences visant une très
%haute sensibilité pour une résolution angulaire relativement faible,
%qui ciblent une détection du mode B primordial de polarisation, telles
%BICEP, sont complétées par des expériences à plus haute résolution
%angulaire, telles Keck Array, assurant une mesure des émissions
%astrophysiques d'avant-plan et de l'effet de lentille
%gravitationnelle. Cette stratégie est généralisée dans la convergence
%des développements instrumentaux au sol (et en ballon) dans la
%méta-collaboration S4~\reference{S4}, engagée dans la course à la
%détection du mode B primordial, qui ouvrirait une fenêtre unique sur
%la physique de la phase d'inflation cosmique. Les meilleures
%contraintes actuelles sur les modes B primordiaux, paramétrées par le
%rapport entre l'amplitude du spectre de puissance des modes
%tensorielles et scalaires des perturbations primordiales à une échelle
%angulaire pivot $k = 0.002\,\rm{Mpc}^{-1}$, sont $r_{0.002} < 0.07$,
%obtenues dans une étude conjointe des données de \emph{Planck},
%BICEP2, Keck Array en combinaison avec les
%BAO~\reference{Planck2018}. Pour la mesure des modes B primordiaux,
%qui constitue l'objectif scientifique principal de nombre de futurs projets
%expérimentaux ambitieux (par ex. LiteBIRD), l'effet de lentille sur le
%CMB est une thématique centrale. Une mesure précise du mode B induit
%par effet de lentille, qui domine le mode B primordial sauf aux plus
%grandes échelles angulaires, est cruciale.\\

%\emph{L'effet SZ à haute résolution angulaire.} L'un des résultats les plus
%intriguants de \emph{Planck} est le désaccord, ou la \og tension \fg
%pour utiliser le terme préféré par la collaboration \emph{Planck},
%entre le modèle cosmologique favorisé par le CMB et celui favorisé
%par les amas de galaxies sélectionnés via l'effet
%tSZ~\reference{cosmo2015, cosmo2018}. Cette tension est observée aussi
%bien lorsque le modèle est dérivé du comptage des amas du
%catalogue \emph{Planck}~\reference{SZ2015, SZ2018}, que lorsqu'il est 
%contraint avec le spectre de puissance et les corrélations d'ordre
%supérieure de la carte du paramètre de Compton produite
%par \emph{Planck}~\reference{ymap2015}. Ces contraintes sont également
%confirmées par les étude du comptage des amas des catalogues produits
%par ACT~\reference{ACT-SZ} et SPT~\reference{SPT-SZ}. En outre, la
%ré-interprétation des résultats cosmologiques dans le cadre
%d’extension au modèle $\Lambda$CDM minimal
%ne parvient à réconcilier les deux mesures (CMB primaire et effet SZ)
%qu’au prix d’accroître les incohérences avec d’autres sondes cosmologiques~\reference{??}.


%A moyen terme, le futur des grands relevés de galaxies s'incarne dans
%deux projets phares, le \emph{Large Synoptique Survey Telescope}
%(LSST) au sol~\reference{LSST} et le satellite de
%l'ESA \emph{Euclid}~\reference{euclid}. [DESI, WFIRST, KIDS ?]. J'ai
%rejoint le consortium \emph{Euclid} en 2016 avec pour projet de
%participer à l'exploitation des amas de galaxies en cosmologie.\\


Ce manuscrit s'articule en deux grandes parties. La première est
consacrée à la mise en service de l'expérience NIKA2 et à l'évaluation
de ses performances au télescope de 30 mètres de l'IRAM. Cette partie,
la plus étoffée, décrit mes travaux de recherche en tant que
responsable du \emph{commissioning} de NIKA2 depuis 2016, qui se sont
récemment concrétisés par la livraison de l'instrument à l'IRAM et la
publication de référence des performances. On y trouvera une rapide
présentation de l'expérience NIKA2, une description de la procédure
développée pour la calibration, ainsi que la méthodologie et les
résultats de l'évaluation des performances.
La deuxième partie est consacrée aux développements les plus récents
de mon activité de recherche dans NIKA2 et à mes projets à court et
moyen termes autour de la cosmologie avec les amas de galaxies. Elle
s'ouvre sur une introduction à cette thématique riche, puis développe
le programme cosmologique de NIKA2 à partir de l'échantillon d'amas de
galaxies observés via l'effet SZ. J'y souligne aussi les synergies avec
d'autres sondes des amas de galaxies, en particulier le rayonnement
X. Les projets à courts termes qui y sont décrits pourraient faire
l'objet d'un sujet de thèse à brève échéance. Des projets plus
prospectifs et des perspectives d'extensions du programme amas de
galaxies de NIKA2 à plus long terme sont également
discutés. La conclusion synthétise ces travaux et projets dans NIKA2
et les inscrit dans une perspective plus large de la cosmologie avec
les amas de galaxies dans le cadre des futurs grands relevés dans les domaines de l'optique, du rayonnement X et du CMB.  


%dans l'expérience \emph{Euclid} et la combinaison de sondes.   


%Dans un deuxième
%volet, je présente l'expérience \emph{Euclid} et la préparation à
%l'exploitation des amas de galaxies pour la cosmologie. 



