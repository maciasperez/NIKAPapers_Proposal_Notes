%----------------------------------------------------------------------------------------
%	SUMMARY
%----------------------------------------------------------------------------------------

\subsection*{Scope $\&$ objectives}
NIKA2 performance in intensity have been assessed, as defined in the
MoU, after the end of the commissioning in intensity (phase 1). This
commissioning phase ended after the April 2017 technical campaign,
which included a science verification phase, and was officialised at
the 'End-of-commissioning' review that took place at the IRAM in September
2017. This document, which comes along the instrument delivery,
presents the performance, as well as the methods we developped
for their assessement and the robustness tests we performed.

\subsection*{Data set}
The performance assessment is based on observations acquired at
the February 2017 technical campaign (a.k.a. N2R9), during which NIKA2
instrumental set-up was in final configuration for the first time, and
on observations taken during the first science pools of October 2017
(a.k.a. N2R12) and January 2018 (a.k.a. N2R14). Data acquired during
April 2017 technical campaign (a.k.a. N2R10) were not considered for
the sensitivity assessement due to extra noise induced by an
end-of-life turbo pump, which disappears once the turbo pump was
replaced. The considered data set allow us to check the performance
statibility for one year and a large range of atmospheric conditions. 

\subsection*{Methods}
The performance assessment utilizes an IDL-based data reduction
pipeline that was internally developed in the NIKA2 collaboration
to go from raw time-ordered data to calibrated maps. We give a
synthetic description of the entire pipeline in
Sect.~\ref{se:pipeline_overview} for the reader convenience. The
calibration and performance assessment methods are briefly presented
in Sect.~\ref{se:pipeline_overview} and further detailed in Chapters 3
to 8, which are outlined below.

In Chapt.~\ref{se:opacities}, the atmospheric attenuation is corrected
using two-step method: first, the line-of-sight opacity of each
observing scans is estimated using a series skydip observations
performed with the NIKA2 instrument, then a correction coefficient is
applied to the skydip opacity to ensure the flus density stablity
against the atmospheric condition. Opacity derivation based on the
resident IRAM taumeter measures is also tested for consistency checks.

In Chapt.~\ref{se:fp_reconstruction}, \bm\ scans are used to derive
the position of each KIDs in the FoV and perform a first selection to
discard noisy and cross-talking KIDs. The KIDs are further selected on
the comparison between their measured position and the expected
designed position. Finally, series of defocused \bm\ scans are used to
reconstruct the focus surfaces and estimate the telescope focus
setting that optimizes the focus across the arrays.   

In Chapt.~\ref{se:beams}, noticeable features of the beam pattern are
identified using \bm\ scans. Radial profile of the full beam are
measured up to a radius of $180''$ and are fitted using a
three-Gaussian model. The main beam of each array is evaluated using
three complementary methods: the first Gaussian that best fit the full
beam profile, a single Gaussian fit on a side-lobe-masked version of
the profile and a 2D Gaussian fit on a side-lobe-masked version of the
beam map. Then we use the knowledge we gain on the full beam and main
beam to derive the main beam efficiency up to $180''$. A more precise
description of the beam pattern above $180''$, which is useful for
diffuse emission studies, is let for a future dedicated study that
will be conducted as part of the diffuse emission NIKA2 Large
Programs.

Chapter~\ref{se:calibration}





\subsection*{Results}
The main measured parameters that define the actual NIKA2 performances
are gathered in Table~\ref{nika2summary}.

\begin{table}[h]
\begin{center}    
  \begin{threeparttable}
    \begin{tabular}{|r|c|c|c|c|}
      \hline
      & Array 1 & Array 3  & Array 1\&3 & Array 2 \\
      \hline
      \hline
      Reference Wavelength  [mm]  &  1.15   &  1.15  & 1.15 & 2.0   \\
      Reference Frequency  [GHz]  &  260    &  260   & 260  & 150  \\
      Central Frequency [GHz]     &  255.5  &    257.8     &     &   151.6  \\
      Bandwidth         [GHz]     &   47.8  &     45.7     &     &    42.1  \\
      \hline
      Number of designed detectors       & 1140      &  1140    &    &    616  \\
      Number of valid detectors          &  952      &   961    &    &    553  \\
      Fraction of valid detectors [$\%$] &           &          &    &         \\
      Effective FOV\tnote{a}\hspace{1mm} [arcmin]    &   5.39    &   5.61    &    &   4.9  \\
      \hline
      Pixel size in beam sampling unit [F$\lambda$]  &    &   &    &   \\
      \hline
      FWHM\tnote{b}\hspace{1mm} [arcsec]  &  $11.3 \pm 0.2$   &  $11.2 \pm 0.2$  &   $11.2 \pm 0.1$  &  $17.7 \pm 0.1$ \\
      Beam efficiency\tnote{c}\hspace{1mm} [\% ]    &        &    &     &    \\
      rms of the FWHM on the FOV [$\%$]   &   &    &   &  \\
      \hline 
      rms calibration error [\%]            & 4.5  & 6.6  &   & 5  \\
      \hline
      Absolute calibration uncertainty [\%] &  \multicolumn{4}{|c|}{5} \\
      \hline
      $\alpha$ noise integration in time\tnote{d}\hspace{1mm}  &   &   &   &  \\
      \hline
      rms pointing error    [arcsec]    & \multicolumn{4}{|c|}{$<3$}  \\
      \hline
      NEFD\tnote{e}\hspace{1mm}   [$\rm{mJy} \cdot \rm{s}^{1/2}/\rm{beam}$]  &  30 (15)   & 30 (15)  &  30 (15)  & 20 (10) \\
      Mapping speed\tnote{f}\hspace{1mm} [arcmin$^2$/h/mJy$^2$] & 302  & 454  & 775 (1184)  & 7542 (10861)  \\
\hline

\end{tabular}
  \begin{tablenotes}
{\small     
  \item[(a)] Equivalent FOV covered by the valid detectors
  \item[(b)] Full-width at half-maximum of the main beam modelled as a two-dimensional Gaussian fitted from sidelobe-masked beam maps.
  \item[(c)] Ratio between the main beam power and the total beam power up to a radius of 180 arcsec
  \item[(d)] Effective power law of noise reduction with integration time
  \item[(e)] NEFD in typical IRAM good sky opacity condition: 2mm pwv, $60^o$ elevation
  \item[(f)] Average (best) mapping speed at zero opacity for the February 2017 observation campaign. 
}
  \end{tablenotes}
\end{threeparttable}
\caption[Main performance measurements]{Summary of the main characteristics describing the measured 
performances of NIKA2.}
\label{tab:nika2summary}
\end{center}  
\end{table}


The performance parameters given in Table~\ref{nika2summary} are
splitted in two different lists: first, the main characteristics, as
defined in the MOU, are listed in Table~\ref{nika2summary_main},
second, other parameters, which are derived from the instrument
characteristics described in the MOU, and that need to be
characterized to complete the commissioning phase are given in
Table~\ref{nika2summary_second}. Table~\ref{nika2summary_second} is
constructed from the 'secondary' and 'tertiary' tables of Samuel's
summary document.


\begin{table}[h]
  \caption[Main performance requirement]{Summary of the main characteristics describing the measured performances of NIKA2, as listed in MoU}
  \label{nika2summary_main}
  \begin{threeparttable}
    \begin{tabular}{|r|c|c|c|c|}
      \hline
      & Array 1 & Array 3  & Array 1\&3 & Array 2 \\
      \hline
      \hline
      NEFD\tnote{a}\hspace{1mm}   [$\rm{mJy} \cdot \rm{s}^{1/2}/\rm{beam}$]  &  30 (15)   & 30 (15)  &  30 (15)  & 20 (10) \\
      Number of designed detectors       & 1140      &  1140    &    &    616  \\
      Number of valid detectors          &  952      &   961    &    &    553  \\
      Fraction of valid detectors [$\%$] &           &          &    &         \\
      \hline
      FWHM\tnote{b}\hspace{1mm} [arcsec]  &  $11.3 \pm 0.2$   &  $11.2 \pm 0.2$  &   $11.2 \pm 0.1$  &  $17.7 \pm 0.1$ \\
      \hline
      Effective FOV\tnote{c}\hspace{1mm} [arcmin]    &   5.39    &   5.61    &    &   4.9  \\
      \hline
      Pixel size in beam sampling unit [F$\lambda$]  &    &   &    &   \\
      \hline      
\end{tabular}
  \begin{tablenotes}
  \item[(a)] NEFD in typical IRAM good sky opacity condition: 2mm pwv, $60^o$ elevation
  \item[(b)] Full-width at half-maximum of the main beam modelled as a two-dimensional Gaussian fitted from sidelobe-masked beam maps.
  \item[(c)] Equivalent FOV covered by the valid detectors
    \end{tablenotes}
\end{threeparttable}
\end{table} 


\begin{table}[h]
  \caption[Secondary performance measurements]{Summary of other NIKA2 performance characteristics either defined in the MoU or extracted from SL's summary document}
  \label{nika2summary_second}
  \begin{threeparttable}
    \begin{tabular}{|r|c|c|c|c|}
      \hline
      & Array 1 & Array 3  & Array 1\&3 & Array 2 \\
      \hline
      \hline
      Reference Wavelength  [mm]  &  1.2   &  1.2  & 1.2 & 2.0   \\
      Reference Frequency  [GHz]  &  260   &  260  & 260 & 150  \\
      Central Frequency [GHz]     &  255.5  &    257.8     &     &   151.6  \\
      Bandwidth         [GHz]     &   47.8  &     45.7     &     &    42.1  \\
      \hline
      Beam efficiency\tnote{a}\hspace{1mm} [\% ]    &        &    &     &    \\
      rms of the FWHM on the FOV [$\%$]   &   &    &   &  \\
      \hline 
      rms calibration error [\%]            & 4.5  & 6.6  &   & 5  \\
      \hline
      Absolute calibration uncertainty [\%] &  \multicolumn{4}{|c|}{5} \\
      \hline
      $\alpha$ noise integration in time\tnote{d}\hspace{1mm}  &   &   &   &  \\
      \hline
      rms pointing error    [arcsec]    & \multicolumn{4}{|c|}{$<3$}  \\
      \hline
      Mapping speed\tnote{b}\hspace{1mm} [arcmin$^2$/h/mJy$^2$] & 302  & 454  & 775 (1184)  & 7542 (10861)  \\
\hline

\end{tabular}
  \begin{tablenotes}
  \item[(a)] Ratio between the main beam power and the total beam power up to a radius of XXX arcsec
  \item[(b)] Average (best) mapping speed at zero opacity for the February 2017 observation campaign. 
  \end{tablenotes}
\end{threeparttable}
\end{table}
