Galaxy clusters are the largest gravitationally bound objects in the universe and thus serve as ideal cosmological probes 
and astrophysical laboratories. Because the formation of galaxy clusters stems from overdensities of matter
and depends on the cosmic composition of the universe, one can constrain cosmological parameters such as the matter 
density of the universe, $\Omega_m$, the matter power spectrum normalization, $\sigma_8$,
and the equation of state for dark energy density $\Omega_{\Lambda}$, $w$ \citep[e.g.][]{carlstrom2002}.

Within a galaxy cluster, the gas in the intracluster medium (ICM) constitutes 90\% of the
baryonic mass \citep{vikhlinin2006b} and is directly observable in the X-ray due to bremsstrahlung emission. 
At millimeter and sub-millimeter wavelengths, the ICM is observable via the Sunyaev-Zel'dovich effect (SZE) 
\citep{sunyaev1972}: the inverse Compton scattering of cosmic microwave background (CMB) photons off of
the hot ICM electrons. The thermal SZE is observed as an intensity decrement relative to the CMB at wavelengths longer 
than $\sim$1.4 mm (frequencies less than $\sim$220 GHz).
At longer radio wavelengths, if relativistic electrons are present, parts of the ICM may emit synchrotron emission.

In the core of a galaxy cluster, baryonic physics are non-negligible and non-trivial. Some observed
physical processes in the core include shocks and cold fronts \citep[e.g.][]{markevitch2007}, sloshing
\citep[e.g.][]{fabian2006}, and X-ray cavities \citep{mcnamara2007}. It is also theorized that helium sedimentation
should occur, most noticeably in low redshift, dynamically-relaxed clusters \citep{abramopoulos1981, gilfanov1984} 
and recently the expected helium enhancement via sedimentation has been numerically simulated \citep{peng2009}. 
This would result in an offset between X-ray and SZE derived pressure profiles.

At large radii ($R \gtrsim R_{500}$),\footnote{$R_{500}$
is the radius at which the enclosed average mass density is 500 times the critical density, 
$\rho_c(z)$, of the universe} equilibration timescales are longer, accretion is ongoing, 
and hydrostatic equilibrium (HSE) is a poor approximation. 
Several numerical simulations show that the fractional contribution
 from non-thermal pressure increases with radius \citep{shaw2010,battaglia2012,nelson2014}. 
For all three studies, non thermal pressure fractions between 15\% and 30\% are found at ($R \sim R_{500}$)
for redshifts $0 < z < 1$. However, the intermediate radii, between the core and outer regions of the 
galaxy cluster, offer a region where self-similar scalings derived from HSE can be used to describe simulations 
and observations \citep[e.g.][]{kravtsov2012}. Moreover, both simulations and observations find low
cluster-to-cluster scatter in pressure profiles within this intermediate radial range \citep[e.g.][]{borgani2004,
nagai2007,arnaud2010,bonamente2012,planck2013a,sayers2013}.

While many telescopes capable of making SZE observations are already operational or are being built, most have
angular resolutions (full width, half maximum - FWHM) of one arcminute or larger. The MUSTANG camera \citep{dicker2008}
on the 100 meter Robert C. Byrd Green Bank Telescope \citep[GBT, ][]{jewell2004} with its angular resolution of 9\asec 
(FWHM) and sensitivity up to the limit of MUSTANG's instantaneous field of view, $1$\amin, 
is one of only a few SZE instruments with sub-arcminute resolution.
To probe a wider range of scales we complement our MUSTANG data with SZE data from Bolocam \citep{glenn1998}. 
Bolocam is a 144-element bolometer
array on the Caltech Submillimeter Observatory (CSO) with a beam FWHM of 58\asecs at 140 GHz and circular FOV with 8\amins 
diameter, which is well matched to the angular size of $R_{500}$ ($\sim 4$\amin) for both of the clusters in our sample. 

%In this paper, we extend the map fitting technique used in \citet{young2014}, to simultaneously
%fit 3D pressure profiles to Bolocam and MUSTANG data. With MUSTANG's
%high resolution, this is the first analysis to make use of SZE observations that cover similar scales 
%($0.03 R_{500} <r \lesssim R_{500}$) to those probed by X-ray studies
%\citep[][; hereafter N07 and A10 respectively]{nagai2007,arnaud2010} which have constrained the average cluster pressure profile. 
%N07 compared X-ray and simulation results over radial scales ($0.1 R_{500} \lesssim r \lesssim R_{500} $), whereas
%A10 used X-ray determined pressure profiles for $0.03 R_{500} \lesssim r < R_{500}$, and simulation results for
%$R_{500} < r$. More recently, the Planck collaboration has published an analysis combining \xmm\ observations,
%which span ranges $0.02 R_{500} < r < R_{500}$ with \emph{Planck} observations, which span radial ranges 
%$0.1 R_{500} < r < 3 R_{500}$ \citep{planck2013a}. Additionally, a sample of clusters studied by Bolocam  
%has been analyzed using solely the SZE data, which spans radial ranges of $0.07 R_{500} < r < 3.5 R_{500}$
%\citep{sayers2013}.

This paper is organized as follows. In Section~\ref{sec:obs} we describe the MUSTANG and Bolocam observations and reduction. 
In Section~\ref{sec:jointfitting} we review the method used to jointly fit pressure profiles to MUSTANG and Bolocam data. We
present results from the joint fits in Section~\ref{sec:pp_constraints} and compare our results to X-ray derived pressures 
in Section~\ref{sec:xray_comp}. 
Throughout this paper we assume a flat $\Lambda$CDM cosmology with $\Omega_m = 0.3$, $\Omega_{\lambda} = 0.7$, and $H_0 = 70$ 
km s$^{-1}$ Mpc$^{-1}$, consistent with the 9-year \emph{Wilkinson Microwave Anisotropy Probe} (WMAP) results reported in
 \cite{hinshaw2013}.
