\section{Calibration reference software}

The reference software used for NIKA2 calibration are available under
SVN in the directory {\tt Processing/Scr/Reference/}, refered to as
the calibration directory. Once a first beam map scan is reduced, the
subsequent steps of the calibration are implemented using the main
script in the calibration directory 
{\tt Calibration/baseline$\_$calibration.pro}. 
This script, which must be edited to perform the calibration of each
campaign, calls the following daughter scripts:


\subsubsection*{FOV geometry and relative calibration}
In the calibration directory, {\tt Geometry/reduce$\_$beammap.pro}

\subsubsection*{Opacity}
Calibration directory then {\tt
  Opacity/reduce$\_$skydips$\_$reference.pro}.
This reference script calls the routine {\tt all$\_$skydips.pro}, which relies on
the fitting routine available in {\tt
  Processing/Realtime/nk$\_$test$\_$allskd4.pro}

\subsubsection*{Absolute calibration}
Calibration directory then {\tt
  Photometry/calibration$\_$uranus$\_$reference.pro}.
For the beam-hardened calibration test, the photometric
correction is implemented in {\tt
  Photometry/photometric$\_$correction.pro}



\subsubsection*{Photometric checks}
Once all these steps are done, the calibration can be checked using
the script {\tt Photometry/validate$\_$calibration$\_$reference.pro}.
