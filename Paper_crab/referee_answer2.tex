
%\documentclass{article}
\documentclass[12pt]{article}
\usepackage[utf8]{inputenc}
\usepackage[english]{babel}

\title{Referee report: NIKA polarization observations of the Crab nebula}
\author{A.Ritacco, J.F. Macías-Pérez, N. Ponthieu}
%\date{March 2018}

\begin{document}
\maketitle

\noindent Dear Referee,\\
Thank you again for your attentive reading. 
In the following we answer (in boldface) to your questions and comments.
The sentences as rewritten in the paper are given in italic.
We have also included in the text your comments to ease the reading. 
In the last version of the paper we have boldfaced the changes. \\  \\
Best regards,\\
A. Ritacco, J.F. Macías-Pérez, N. Ponthieu on behalf of the authors.


\begin{enumerate}
    \item 
{
As described in my previous report, the measurements and analysis 
presented in this paper are of important value to the community. I am 
pleased to see that the latest version of the paper addresses all the 
main issues I raised. It is exciting to see the new Planck 
measurements. There are some minor issues that I believe 
should be resolved. 
}

Equation (2), for the uncertainty on p appears to be incorrect, based 
on dimensional analysis. I believe equation (3) is also incorrect; 
the terms under the square root should be cross-terms, such as 
$Q^2\sigma_U^2$. Whatever the final correct equations are, I ask the 
authors to guide me in how these results were obtained from the 3 
references given (i.e. what equations under what assumptions).

{\bf Please have a look to the Appendix A \ref{appendix} where we detail the arithmetics of the uncertainties derivation on p and $\psi$. Actually, dimensional analysis does not pose any problem: Q, U, $\sigma_Q$ and $\sigma_U$ are homogeneous to intensity, like $\sigma_I$. So the term under the square root goes like I$^4$, hence the total ratio is dimensionless. For the same reasons, equation (3) gives an error on the angle that is also dimensionless, as it should for an angle.}

{\bf For Eq. 3 you are right, thank you. We have corrected the equation in the text and also recalculated the errors for the angles. Please find in the paper (Tab. 1,2) the new values boldfaced. Figures 6, 7, 8 have been updated.}

{\bf Furtheremore referring to {\it Montier et al. 2015 I} we have demonstrated (see Ritacco PhD thesis 2016, paragraph 6.3.4) that in our case the full covariance matrix (Eq. 3) reduces to the diagonal terms. Moreover Ritacco et al. 2017 A$\&$A, 599, A34 show a noise spectrum in Stokes Q and U compatible with a white noise like spectrum. So in our derivation of the uncertainties we assume that: (i) correlations between the total and polarized intensities are negligible; (ii) correlated noise between Q and U is also negligible; and (iii) equal noise is observed on Q and U measurements.}


In section 3.1, a "zero level" subtraction is described, with 
particular reference to the annulus between 4 and 4.5 arcminute 
radius. Later, "aperture photometry" is used to estimate fluxes with 
"radius of 5', 7' and 10' FWHM from the center of the map". What is 
meant by "FWHM" here? If this is a circular aperture then the FWHM 
would mean the diameter, not the radius. Are the radii 5', 7', and 
10', or is that the diameter? If it's the diameter, doesn't that go 
outside the annulus at which the "zero level" was computed and 
removed? The authors need to clarify whether these are radii or 
diameters, and I would encourage them to not use "FWHM" in this 
context. 

\textbf{Indeed this section is a bit confusing. Some typos remained from the previous version.
  We consider circular apertures with diameters 5',7' and 10'.
  Indeed for the 10' values we are outside the annulus where the zero level is calculated. So we now compute the zero level in an annular ring between radius 4.5'\textless R \textless 5'. We have decided to limit the analysis to the maximum of the map size avoiding the edges and considering that all the flux is concentrated in 7.6' of aperture. Values are now given in the text for apertures of 5', 7', 9'. Figures 5,6 have also been updated. In 3.2 we add a sentence to explain this point: {\it All the listed values are estimated using aperture photometry inside apertures of 5', 7', and 9' from the center
of the map. We limit the analysis to the maximum of the map
size avoiding the edges and considering that all the flux is
concentrated in 7.6’ of aperture, see Fig. 5. 
}}

At various points, the color correction of 1.05 is mentioned. A few 
more words are warranted here... what assumptions (spectral index, 
pass band) go into the color correction? How big is the uncertainty? 
After color correction, are the quoted flux densities valid at the 
nominal frequency of 150 GHz, or some other frequency? 

\textbf{The color correction factor has been estimated using the spectral index of {\it Macias et al. 2010}, namely $\beta$ = -0.296 and the NIKA bandpasses. These bandpasses are shown in {\it Catalano et al. 2014}, and the uncertainty to the 2mm one is 1$\%$. Once the color correction is applied the flux obtained corresponds to the integrated flux in the bandpass.}

{\bf  Sentence added:
 {\it This factor has been estimated using the spectral index $\beta$ measured by {\it Macias et al. 2010} and considering the NIKA bandpass at 2 mm as given in {\it Catalano et al. 2014}. The final flux is thus the integrated value in the bandpass.}}

In section 4.1, it is claimed that the flux at radio and millimeter 
wavelengths is due to synchrotron from a single population of 
particles in the "same" magnetic field. This leads to the statement 
that "The direction of the polarization is therefore expected to be 
constant across the frequency range 30-300 GHz while the polarization 
degree may vary." I do not see how the polarization degree, and 
angle, could be decoupled like this. Either different parts of the 
nebula could scale differently with frequency (producing varying 
degree and angle of polarization), or they all scale together, locking 
all polarization properties to the total intensity.

\textbf{Indeed the phrasing is incorrect, thank you for noting this. We mean that the direction of polarization and the degree of polarization on small scales accross the nebula, but this pattern should be the same at different frequencies of observations. We have rephrased it as:
 {\it The direction and degree of the polarization is therefore expected to be constant across the frequency range 30-300 GHz.}}

\end{enumerate}

\appendix
\label{appendix}
\section{Derivation of uncertainties on the degree of polarization and the angle}

Let's define the polarization degree p:

\begin{equation}
 p    = \frac{\sqrt{Q^2 + U^2}}{I} \nonumber 
\end{equation}

so

\begin{eqnarray}
\frac{\partial p}{\partial Q} &=& \frac{Q}{I\sqrt{Q^2+U^2}}, \\
\frac{\partial p}{\partial U} &=& \frac{U}{I\sqrt{Q^2+U^2}}, \\
\frac{\partial p}{\partial I} &=& -\frac{\sqrt{Q^2+U^2}}{I^2},
\end{eqnarray}

thus

\begin{eqnarray}
\sigma_p^2 &=& \left(\frac{Q}{I\sqrt{Q^2+U^2}}\right)^2\sigma_Q^2 + 
\left(\frac{U}{I\sqrt{Q^2+U^2}}\right)^2\sigma_U^2 + 
\left(\frac{\sqrt{Q^2+U^2}}{I^2}\right)^2\sigma_I^2\\
&=&\frac{(Q^2\sigma_Q^2 + U^2\sigma_U^2)I^2 +
  (Q^2+U^2)^2\sigma_I^2}{I^4(Q^2+U^2)} \\
&=&\frac{(Q^2\sigma_Q^2 + U^2\sigma_U^2)I^2 +
  (p^2I^2)^2\sigma_I^2}{I^4(Q^2+U^2)} \\
&=&\frac{(Q^2\sigma_Q^2+U^2\sigma_U^2)+p^4I^2\sigma_I^2}{p^2I^4}
\end{eqnarray}

hence

\begin{equation}
\sigma_p = \frac{\sqrt{Q^2\sigma_Q^2+U^2\sigma_U^2+p^4I^2\sigma_I^2}}{pI^2}.
\end{equation}

As far as the angle is concerned:

\begin{eqnarray}
\psi & = & \frac{1}{2}\arctan{\frac{U}{Q}}\\
\frac{\partial \psi}{\partial Q} &=& \frac{1}{2}\frac{-U}{Q^2+U^2} \\
\frac{\partial \psi}{\partial U} &=& \frac{1}{2}\frac{Q}{Q^2+U^2}
\end{eqnarray}

\begin{eqnarray}
\sigma_\psi^2 &=& \frac{1}{4}\frac{U^2\sigma_Q^2+Q^2\sigma_U^2}{(Q^2+U^2)^2}\\
&=&\frac{1}{4}\frac{U^2\sigma_Q^2+Q^2\sigma_U^2}{p^4I^4}
\end{eqnarray}

so

\begin{equation}
\sigma_\psi = \frac{1}{2}\frac{\sqrt{U^2\sigma_Q^2+Q^2\sigma_U^2}}{(pI)^2}
\end{equation}



\end{document}
