%
%  Created by LP from a copy a photometry_HA.tex and a merge of cal_JFL.tex
%


\section{Reference flux densities of the primary calibrators }%{\color{blue} Herv\'e}}
\label{se:ref_flux_primaries}

% HA text
The two main calibrators of NIKA2 are the giant planets Uranus and
Neptune. Mars can also be used as primary calibrator, but care must be
taken to use a flux corresponding to the date of the observations.  

\subsection{Uranus and Neptune}
\label{se:ref_flux_uranus_neptune}
For the flux densities of the giant planets, we use the ESA model from
\cite{ESAmodel}: Version 5 for Neptune and Version 4 for Uranus. 
Both models provide the planet brightness temperature in the
Rayleigh-Jeans approximation as a function of the frequency. The
resulting flux is therefore: 
\begin{equation}
S_{\nu} = \Omega \times \frac{2 \nu^{2} k T_{RJ}}{c^2}
\end{equation}
where $\Omega$ is the solid angle of the planet on the sky. Following
Bendo et al. (2013) and correcting their equation 12 we have:
\begin{equation}
\Omega = \pi \frac{r_{e} r_{p-a}}{D^{2}} 
\label{eq:omega}
\end{equation}
where $r_{e}$ is the equatorial radius of the planet and $r_{p-a}$ is
its apparent polar radius, and $D$ the distance to the
planet. $r_{p-a}$ can be computed from the sub-observer latitude $\phi$
({\it e.g.} the latitude of the observed {\bf ?}  as seen from the planet in the
planet equatorial reference frame) and $r_{p}$ the polar radius of the
planet as:
\begin{equation}
r_{p-a} = \sqrt{r_{p}^2 \cos^{2}\phi + r_{e}^2 \sin^{2} \phi}
\end{equation}
All quantities to compute the planet flux are obtained from the NASA
Horizons web site \cite{NASAHorizon}, and are
listed in table~\ref{tab:planetphysparam}. To compute the planet fluxes for a given date, we use the python
photometry package available at \cite{gith-Haussel}.

\begin{table}[ht]
\begin{center}
\begin{tabular}{|c|c|c|}
\hline
     & Uranus & Neptune \\
\hline
$r_{e}$ [km]  & 25559 & 24764 \\ 
\hline
$r_{p}$ [km]  & 24973 & 24341  \\
\hline
$\phi$         & Ob-lat & Ob-lat \\
\hline
$D$   [AU]    & delta   & delta \\
\hline
\end{tabular}
\end{center}
\caption[Primary calibrator flux models]{Physical quantities used for the Uranus and Neptune fluxes
  computation (equation~\ref{eq:omega}. Ob-lat and delta are quantities 
  tabulated by NASA Horizons system \cite{NASAHorizon} as a function of the date}
\label{tab:planetphysparam}
\end{table}


The model spectra are linearly interpolated in log space at the
reference frequencies of the NIKA2 bandpasses. Fluxes for all NIKA2
calibration runs are listed in table~\ref{tab:fluxPred}, together with
the expected variation between the start and end of a run. 

The Uranus and Neptune models have been compared to Planck
observations of these planets \cite{PLCK-LII}. For Uranus, the model used in the comparison
is the ESA V2, and it is found to overpredict by 4 K (about 4\%) the
observed RJ temperature at 143~GHz, to agree at 217 GHz, and
to underpredict at 353 GHz. We use for NIKA2 calibration ESA model V4,
that predict a flux respectively -3.3\%, 0.3\% and 4.7\% higher in the
the 143, 217 and 353 GHz, that would lead to a percent
accuracy with respect to Planck observations. 

For Neptune, the same study compared Planck observation with the ESA V5
model, {\it i. e.} the same one used for NIKA2 calibration. For this
planet, temperatures are found to disagree at most by 5 K, i.e 4.1\%,
with the same trend with frequency as observed for Uranus. All thing
considered, this study confirm that Uranus ESA V4 and Neptune ESA V5
models are accurate to 5\% for predicting planet fluxes. Calibration
values tabulated in table~\ref{tab:flucPred} show that the variations
of Uranus and Neptune over the duration of a typical NIKA2 run are
negligible compared to the model accuracy. On the other hand, not
taking into account the planet shape and orientation with respect to
the observer in the computations of its solid angle can lead to errors
between 1 and 2\% as illustrated in the Python notebook
\cite{gith-Haussel-Note}
distributed with the software. 



\begin{table}[p]
\centering
\begin{tabular}{|l|r|r|r|r|r|r|}
\hline
NR$^{a}$  & JD$^{b}$ & $\Delta t$ $^{c}$ & $S_{\nu}$(260 GHz)  $^{d}$& $S_{\nu}$(150  GHz)$^{e}$  & $\Delta S_{\nu}/  S_{\nu} ^{f}$  \\
\hline
         & d  &  d        & Jy               & Jy                 &                                                                    \%  \\
\hline
         &    &            & \multicolumn{3}{|c|}{Uranus}\\
\hline
13 & 2457330.5 &  12 & 45.59 & 17.65 & -0.89\\
14 & 2457354.5 &  8 & 44.44 & 17.21 & -1.07\\
15 & 2457409.5 &  20 & 40.62 & 15.73 & -3.22\\
16 & 2457455.5 &  14 & 38.27 & 14.82 & -1.16\\
18 & 2457660.0 &  25 & 46.06 & 17.83 & +1.25\\
19 & 2457690.0 &  7 & 46.09 & 17.85 & -0.32\\
20 & 2457732.0 &  7 & 44.14 & 17.09 & -1.04\\
21 & 2457764.5 &  4 & 41.82 & 16.19 & -0.69\\
22 & 2457809.0 &  7 & 39.08 & 15.13 & -0.83\\
23 & 2457865.0 &  7 & 37.96 & 14.70 & +0.14\\
24 & 2457915.4 &  5 &  39.49 & 15.29 & +0.66 \\
\hline
         &    &            & \multicolumn{3}{|c|}{Neptune}\\
\hline
13 & 2457330.5 &  12 & 17.09 & 7.18 & -1.26\\
14 & 2457354.5 &  8 & 16.64 & 6.99 & -0.92\\
15 & 2457409.5 &  20 & 15.76 & 6.62 & -1.35\\
16 & 2457455.5 &  14 & 15.55 & 6.53 & +0.19\\
18 & 2457660.0 &  25 & 17.65 & 7.41 & -1.30\\
19 & 2457690.0 &  7 & 17.24 & 7.24 & -0.68\\
20 & 2457732.0 &  7 & 16.46 & 6.91 & -0.79\\
21 & 2457764.5 &  4 & 15.92 & 6.68 & -0.34\\
22 & 2457809.0 &  7 & 15.56 & 6.53 & -0.08\\
23 & 2457865.0 &  7 & 15.89 & 6.67 & +0.57\\
24 & 2457915.4 &  5 & 16.73 & 7.02 & +0.56 \\
\hline
         &    &            & \multicolumn{3}{|c|}{Mars}\\
\hline
13 & 2457330.5 &  12 & 146.19 & 48.30 & +7.75\\
14 & 2457354.5 &  8 & 175.88 & 58.14 & +8.70\\
15 & 2457409.5 &  20 & 319.71 & 105.62 & +27.68\\
16 & 2457455.5 &  14 & 666.46 & 218.49 & +30.37\\
18 & 2457660.0 &  25 & 597.17 & 199.44 & -21.61\\
19 & 2457690.0 &  7 & 439.23 & 146.24 & -4.82\\
20 & 2457732.0 &  7 & 311.78 & 103.98 & -4.89\\
21 & 2457764.5 &  4 & 239.37 & 79.54 & -2.12\\
22 & 2457809.0 &  7 & 174.99 & 57.94 & -4.94\\
23 & 2457865.0 &  7 & 123.61 & 40.61 & -5.44\\
24 & 2457915.4 &  5 & 102.08 & 33.68 & +0.59 \\
\hline
\end{tabular}
\caption[NIKA2 Planet flux expectations]{ a: Nika Run, b: Julian Date when the
  model are computed, c: Run duration, d, e: total fluxes at 260 and
  150 GHz, f: varition of the 150 GHz flux density over the duration
  of the run}
\label{tab:fluxPred}
\end{table}


\subsection{Mars}
For Mars, we use the model of Belloche \&  Amri (2006) available at
\cite{beloche},
with default parameters. Model output is computed at the two reference
frequencies of NIKA2, 150 and 260~GHz.

Fluxes of Mars are tabulated in table~\ref{tab:fluxPred}. In many
cases, the variations of Mars flux during the course of a run are
larger than the model uncertainty (5\%), and should be recomputed at
more frequent times. 

%{\bf FM: conclusion for Mars ?}\\

