%----------------------------------------------------------------------------------------
%	FOCAL PLANE RECONSTRUCTION
%----------------------------------------------------------------------------------------
%\section{Focal Plane Geometry}
%\label{se:geometry}


%   Methods
%----------------------------------------------------------------------------------------
\subsection{FOV position reconstruction}
\label{se:fov_geometry}

In order to be able to produce a map, one needs to associate a pointing
direction to any data sample of the system. The telescope provides us with
various pointing information for a reference position in the focal plane. We
then need to know the relative pointing offsets of each detector. We use
\bms\ for this purpose (see Sect.~\ref{se:beammaps}). The determination of the
KID offsets in the focal plane proceeds in two steps.

\paragraph{Step 1.} We apply a median filter per
KID timeline whose width is of 31 samples, that is equivalent to about 
5~FWHM at 65 arcsec/s given that the sampling frequency is 23.84~Hz,
and we project one map per KID in Nasmyth
coordinates. This median filter removes
efficiently most of the low frequency atmospheric and electronic
noise, albeit with a slight ringing and flux loss on the
source. However, at this stage, we are only interested in the location
of the observed planet.
To derive the Nasmyth coordinates from the
provided $(\alpha_t,\delta_t)$ and $(\Delta\alpha_t,\Delta\delta_t)$
coordinates, we build the following quantities at time~$t$:

\begin{eqnarray}
\Delta x_t &=& \cos\delta_t \Delta\alpha_t - \sin \delta_t\Delta \delta_t \nonumber \\
\Delta y_t &=& \sin\delta_t \Delta\alpha_t + \cos \delta_t\Delta \delta_t \nonumber
\end{eqnarray}

Note that $\Delta\alpha_t$ is already corrected by the $\cos\delta_t$ factor to
have orthonormal coordinates in the tangent plane of the sky and be immune to
the geodesic convergence at the poles. Moreover, before projecting the
time-ordered data onto maps, we check
the accuracy of the time-stamping and the consistency between the
telescope time and NIKA2 time. First, we test the
synchronisation of the electronic boxes and the regularity of the
time increase. In case of detection of an anomaly, the
time-stamping of the impacted data samples is
recalculated by interpolating from the accurately stamped
neighbour samples. Then, a so-called \emph{zigzag}
procedure is used to test for delay between the telescope pointing
information and NIKA2 timelines: we fit a constant shift between the
source positions estimated using the subscans in one direction and
using the subscans in the opposite direction. The data timelines are
then projected onto maps. 
We fit a 2D elliptical Gaussian on
each KID map. The centroid of this Gaussian is a first estimate of the KID
offsets, FWHM's, ellipticity and sensitivity. We apply a first KID selection by
removing outliers to the statistics on these parameters. We also discard
manually KIDs that show a cross-talk counterpart on their map.

\paragraph{Step 2.} With these offsets, it is already possible to produce maps
from the combination of all detectors. However, in order to go up to
the calibration stage, we must correct for the flux loss induced by
the median filter and ensure that each timeline is treated in the same
way as the final observation will. For this, we apply the pointing
reconstruction presented in Sect.~\ref{se:ptg} and the data reduction
presented in Sect.~\ref{se:toi_proc}. We still do not have absolute
calibration for each KID (no opacity correction yet) but the amplitude
of the fitted centroid on the same planet provides the required
cross-calibration between KIDs. The final absolute calibration will be
presented in Sect.~\ref{se:calibration}.\\

This analysis is repeated on all \bms\ to obtain statistics and
precision on each KID parameter, together with estimates on KID
performance stability, as discussed in the next sections.

\subsection{FOV grid distortion}
\label{se:grid_distortion}

We studied the matching of the KID position on the sky to the
design position. The global result is scaling, rotation, and shift
parameters for each array. They are described in Table~\ref{ta:gridmatch}.

\begin{table}[ht]
\label{ta:gridmatch}
\begin{center}
\begin{tabular}{|c|c|c|l|}
\hline
Array 1  &	Array 3   &	Array 2   &	Comment \\
\hline
1.25     &      1.25      &     2.05     &     \small{$\lambda$ in mm} \\
1140 	 &      1140 	   &        616  &     \small{Total of designed KIDs (TDK)} \\
673/736  &	734/758  &	437/444  &     \small{Well-placed KIDs (WPK)/Found KIDs (FK)} \\
91/59 	 &    96/64 	 &      98/71 	 & \small{Ratio [\%] of WPK/FK and WPK/TDK} \\
0.87 	 &     0.84 	  & 0.66     &	\small{Median deviation [arcsec] for detectors deviating by less than 5~arcsec} \\
0.52 	 &     0.69 	 &        0.68 	 & \small{Mean distortion across the FoV in arcsec} \\
2.3 -4.5  &	2.0 -5.8  &	9.3 -7.5  &	\small{Array center in Nasmyth coordinates [arcsec]} \\
4.90  &	4.88  &	4.88  &	\small{Plate scaling [arcsec/mm] in the Design x and y (averaged)} \\
77.3  &	76.4  &	78.2  &	\small{Plate rotation angle [degree] from the Design to Nasmyth coordinates} \\
6.6  &	6.6  &	6.6  &	\small{FOV [arcmin] (Total KIDs)} \\
9.8/2.00  &	9.7/2.00  &	13.3/2.75  &	\small{Distance between near detectors [arcsec, mm]} \\
% Old values, rechecked (difference comes from lambda and D)
%   1.24  &	1.22  &	0.97  &	Distance between near detectors with $30\,\rm{m}$ diameter aperture [in $\lambda$/D] \\
% here new values checked FXD 30 Nov 2018:
1.11  &	1.10  &	0.87  &	\small{Measured Distance between near
  detectors [in $\lambda$/D] } \\
%\new{with $27\,\rm{m}$ effective diameter aperture} [in $\lambda$/D]} \\
1.09  & 1.09  & 0.93  & \small{Modeled Distance with ZEMAX}\\
\hline
\end{tabular}
\end{center}
\caption[Field-of-view deformations]{Linear 2D fit of the observed
  position of the detectors in the sky for N2R9 against their mechanically
  designed position. The initial table of Found KIDs is given
  by the focal plane geometry procedure, as described in
  Sect.~\ref{se:fov_geometry},
  applied to N2R9 \bm\ scans. More than 90\% of the detectors (WPK/FK) are
  within less than 5 arcseconds of their expected position. }
\end{table}

It shows that on average the position of each detector is known to better than
an arcsecond. The 1\,mm arrays have almost the same center but this center
differs by 7 and 2\,arcsec in Nasmyth coordinates from the
2\,mm array center. The sampling is above
$\lambda/D$ at 1\,mm, assuming a 27\,m effective diameter aperture. Note that
the plate rotation angle was designed as 76.2\,degrees, less than 2
degrees from what is observed. We find that array 1
has some of the most deviant detectors (above 4\,arcsec from their expected
position). These detectors should be excluded from further analysis. We call
distortion (in the table) the $x.y$ term in the polynomial fitting between the
design grid and the observed position (the fitting is done with the $x$ and
$y$ linear terms and the $x.y$ term). 

This has been compared to expectations obtained using ZEMAX
simulation. 
%The grid diagram generated using ZEMAX provides us with
%the maximum dispersion in the field defined by
%
%\begin{equation}
%P = \frac{\sqrt{(x_p - x_r)^2 + (y_p - y_r)^2}}{\sqrt{x_p^2 + y_p^2}},
%\end{equation}
%
%where $(x_p, y_p)$ and $(x_r, y_r)$ are respectivelly the predicted
%and real coordinates on the image surface relative to the reference
%field position image location (see page 170 of the ZEMAX manual, 2007).
%The predicted coordinates for the whole field are obtained using a
%linear interpolation of a small area in the field central part,
%whereas the real coordinates are calculated by ray tracing through the
%optical system.
%
%\begin{figure}[ht] 
%\begin{center}
%  \includegraphics[width=0.9\textwidth]{Figures/NIKA2_Grid-distortion.png}
%  \caption[Simulated FOV grid]{NIKA2 grid diagram simulated using
%    ZEMAX. \new{The step of the grid is of 30 arcsec and the side width
%      of the black square is of seven arcmin. The red circle corresponds to the 
%      $6.5\,\rm{arcmin}$ diameter FOV. Blue crosses show the grid
%      distorsions in az-el coordinates and in the case of an observing
%      elevation of $14^o$. These grid distorsions depends on the
%      telescope elevation, whereas in the image plane (the plane of the
%      KID arrays) the distorsions are invariant. This allows for
%      concluding that the grid distorsions are caused by the optical
%      elements located between M4 and the dichroic plate.} 
%  }
% \label{fig:fov_grid_distortion_zemax}
%\end{center}
%\end{figure}
%Figure \ref{fig:fov_grid_distortion_zemax} shows the ZEMAX grid diagram for
%NIKA2 simulated optic system.
%
We generated a grid diagram for NIKA2 optic system and found a maximum
grid distortion of $2.7\%$ in the $6.5'$ FOV. We notice that the
strongest distortion appear in the upper right corner of the Nasmyth plan, which is
also the area of the largest defocus w.r.t. to the center (see
Sect.~\ref{sec:focus_surfaces}).
An expected distortion of $2.7\%$ is at most a 5'' shift from the
center to the outside of the array. The quoted distortions between the
measured and designed positions are consistent with the expected
maximum distortions from the NIKA2 optics.

%Auxillary information on this work can be found in this wiki post\footnote{\tiny
%  {\tt http$://$www.iram.fr$/$wiki$/$nika2$/$index.php$/$April$\_$19,$\_$2017,$\_$FXD,$\_$KID$\_$position$\_$mapping$\_$and$\_$Field$\_$distortion$\_$for$\_$Run9}}.
% FXD: this would need to be more ascertained. Lack of time to go further.


\subsection{KID selection and average geometry}
\label{avg_kidpar}

\begin{figure*}[!tp]
\begin{center}
\includegraphics[trim=2cm 14cm 4cm 4cm, clip=true,width=0.45\linewidth]{Figures/A1_fwhm_color_count.pdf}
%\includegraphics[trim=2cm 14cm 5cm 4cm, clip=true,width=0.45\linewidth]{Figures/A1_positions.pdf}
\includegraphics[trim=2cm 14cm 4cm 4cm, clip=true,width=0.45\linewidth]{Figures/A3_fwhm_color_count.pdf}
%\includegraphics[trim=2cm 14cm 5cm 4cm, clip=true,width=0.45\linewidth]{Figures/A3_positions.pdf}
\includegraphics[trim=2cm 14cm 4cm 4cm, clip=true,width=0.45\linewidth]{Figures/A2_fwhm_color_count.pdf}
%\includegraphics[trim=2cm 14cm 5cm 4cm, clip=true,width=0.45\linewidth]{Figures/A2_positions.pdf}
\caption[KID selection in the FOV]{Average detector positions
  for arrays A1, A3, and A2. The three plots show the detectors that have seen
  the sky and passed the quality criteria for at least two \bms\ during Run10, 9
  and 8: 952, 961, and 553 for A1, A3 and A2, respectively. The color
  indicates, from green to red, how many times a KID was
  identified as valid on a \bm. The inner and outer
  dash-line circles correspond to a FOV of 5.5\,arcmin and 6.5\,arcmin,
  respectively. Units are arcseconds.}
\label{fig:avg_fov_color}
\end{center}
\end{figure}

In order to identify the most stable KIDs, we compare the KID parameters
obtained with several \bms.  In the following, we show results as obtained using
seven \bms\ from Run10, two from Run9 and one from Run8.  For each KID we
compute the average position on the focal plane and the average FWHM, counting
the times that it has been considered as valid and at the same position. Indeed,
a few KIDs have close resonances and can be tuned and switched on some scans. A
few others must also be discarded because they appear identical numerically due
to \samu{the fact that a same (noisy) KID can sometimes be associated
  to two different frequency tone in the acquisition.}
%a remaining artefact in the acquisition.
These KIDs are flagged out (less than 1\% of the designed KIDs).

In Fig.~\ref{fig:avg_fov_color} we show the
average focal plane reconstruction, from green to red depending on the number of
times that the KID has been considered as valid.
\new{The eight internal readout feed lines that connect each of the two
  $1\,\rm{mm}$ arrays are also noticeable in this figure: first, slightly
  larger spaces are seen between KID rows connected to different
  feed-lines than between KID rows of the same feed line and second, KIDs at the end of a
  feed line are less often valid than others (see e.g. the FOV of
  Array 3). For A1, this end-of-feedline effect is mixed with the
  effect of the KID gain variation across the FOV, which mainly affects
  the South-West third of the array, as discussed in Sect.~\ref{se:flatfields}.}

For A1, A3 and A2, respectively, we have 952, 961, and 553 KIDs that
have been considered as valid at least twice (840, 508, 868 valid at
least five times). Using this criterion, we deduce the fraction of
valid detectors over the designed ones, as given in Table~\ref{tab:number_of_kids}.

%% \begin{figure}[htp]
%% \begin{center}
%% \includegraphics[trim=2cm 14cm 5cm 4cm, clip=true,width=0.55\linewidth]{Figures/A1_positions.pdf}
%% \includegraphics[trim=2cm 14cm 5cm 4cm, clip=true,width=0.55\linewidth]{Figures/A3_positions.pdf}
%% \includegraphics[trim=2cm 14cm 5cm 4cm, clip=true,width=0.55\linewidth]{Figures/A2_positions.pdf}
%% \caption[Stability of KID positions in the field-of-view]{For the valid
%%   detectors, we show the positions of each pixel, as obtained from each beam
%%   map. Some of them are not found at the same position for all the \bms.
%% Units are arcseconds. \todo{{\bf FM : color code : same as on the 1st
%%       maps of validity}}}
%% \label{fig:jumping_kids}
%% \end{center}
%% \end{figure}

%\begin{figure}[htp]
%\begin{center}
%\includegraphics[trim=2cm 14cm 5cm 4cm, clip=true,width=0.55\linewidth]{Figures/A1_test_positions.pdf}
%\includegraphics[trim=2cm 14cm 5cm 4cm, clip=true,width=0.55\linewidth]{Figures/A3_test_positions.pdf}
%\includegraphics[trim=2cm 14cm 5cm 4cm, clip=true,width=0.55\linewidth]{Figures/A2_test_positions.pdf}
%\caption[Average KID positions]{For the valid detectors,
%  we show the mean (red crosses) and the median (black squares)
%  positions of each pixel, as obtained from each \bm.
%  Units are arcseconds. \todo{FM : color code ? same as on the 1st maps of validity}}
%\label{fig:mean_vs_median}
%\end{center}
%\end{figure}

\begin{table}[ht]
\begin{center}  
  \begin{tabular}{|c|c|c|c|}
    \hline
    Array & Designed detectors &  Valid detectors & Fraction\\
    \hline\hline
    A1 & 1140 & 952 &  84\%\\
    A3 & 1140 & 961 &  84\%\\
    A2 & 616  & 553 &  90\%\\
    \hline
  \end{tabular}
  \caption[Number of detectors]{Summary of the number of valid detectors per array.}
  \label{tab:number_of_kids}
\end{center}    
\end{table}

