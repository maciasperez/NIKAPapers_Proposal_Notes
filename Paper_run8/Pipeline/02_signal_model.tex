
\begin{itemize}
\item list the contributions to kid timelines : signal, atmosphere noise,
  electronic correlated noise, white noise
\item Mention coupling between some kids (doubles) and plateau correction
\item explain how we ``decorrelate''
\item ...
\end{itemize}
~\\

At each time $t$, a kid data sample $d$ is the sum of astrophysical signal $s$ ,
atmosphere emission $a$ , white noise $n$ and electronic noise $c$. Each of
these components is weighted by its own calibration factor. The electronic noise
is partially correlated.

\begin{equation}
d_k(t) = \alpha_k s(t) + \beta_k a(t) + \gamma_k c(t) + n(t)
\end{equation}

Some kids ({\bf give fraction number}) also show ghost images of other kids and
we introduce the mixing matrix $M$ to account for this effect. In practice, the
raw timelines are therefore better described by

\begin{equation}
d_k'(t) = M_{kk_1}d_{k_1}(t)
\end{equation}

