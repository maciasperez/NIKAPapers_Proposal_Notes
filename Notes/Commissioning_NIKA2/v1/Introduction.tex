

%\section{Goals of the total power commissioning}

\begin{table}[ht]
\begin{center}
\begin{threeparttable}
\begin{tabular}{|r|c|c|}
  \hline
  \hline
Reference Wavelength  [mm]  &  1.2 & 2.0  \\
Reference Frequency  [GHz]  &  260 & 150  \\
\hline  
\hline
FOV diameter [arcmin]       &  5 (6.5)    &  5 (6.5)   \\
Pixel size in beam sampling unit [F$\lambda$]  &  0.9 (0.6)   &   0.9 (0.6)  \\
FWHM  [arcsec]              &  12 (10)   &  18 (16) \\
Fraction of valid detectors [$\%$] &  50 (90)   &  50 (90) \\
NEFD\tnote{a}\hspace{1mm}   [$\rm{mJy} \cdot \rm{s}^{1/2}/\rm{beam}$]  &  30 (15)   &  20 (10) \\
\hline
NEFD [$\rm{mJy} \cdot \rm{s}^{1/2}/\rm{beam}$] goal on $90\%$ of the pixels  &  15  & 10 \\
NEFD [$\rm{mJy} \cdot \rm{s}^{1/2}/\rm{beam}$] specification on $50\%$ of the pixels  &  30  &  20  \\
\hline
\end{tabular}
\begin{tablenotes}
{\small
  \item[(a)] NEFD in typical IRAM good sky opacity condition: 2 mm pwv, $60^\circ$ elevation
%  \tablefoottext{a}{NEFD in typical IRAM good sky opacity condition: 2mm pwv, $60^o$ elevation}
}
\end{tablenotes}
\end{threeparttable}
\caption[Main characteristics defining the expected performances of NIKA2.]{Each parameter is associated with two values: the first one indicates the \emph{specifications}, i.e. the requirements to be met by the instrument, while the second bracketed one gives the \emph{goals}, i. e. the values targeted by the collaboration.}
\label{nika2specs}
\end{center}
\end{table} 

\begin{table}[h]
\small
\begin{center} 
\begin{tabular}{|c|c|c|c|c|}
\hline 
RUN  & NIKA2 Run & Starting date    & End date         &  General comments \\
\hline
N2R1     & 13       & 29-October-2015   & 10-November-2015 & Not full instrumentation        \\
N2R2     & 14       & 24-November-2015  & 02-December-2015 & 13 NIKEL boards working         \\
N2R3     & 15       & 12-January-2016   & 01-February-2016 & 20 NIKEL boards                 \\
N2R4     & 16       & 1-March-2016      & 15-March-2016    & 	                               \\
Dark run & 17       & 4-May-2016        & 4-May-2016       & Dark tests with N2R4 conditions  \\
\hline
N2R5     & 18       & 16-September-2016 & 11-October-2016  & New dichroic, corrugated lenses (lenses with Anti-Reflection Structures), \\
         &          &                   &                  &  New 2 mm array, new electronics \\
N2R6     & 19       & 25-October-2016   & 1-November-2016  &                                  \\
N2R7     & 20       & 6-December-2016   & 13-December-2016 & Test external calibrator         \\
N2R8     & 21       & 9-January-2017    & 13-January-2017  & Replace array 1 lens by smooth one, \\
         &          &                   &                  &  adjust the alignement of internal optics \\ 
         &          & 24-January-2017   & 25-January-2017  & Tests on the sky   \\
N2R9     & 22       & 21-February-2017  & 28-February-2017 &                                   \\
N2R10    & 23       & 18-April-2017     & 25-April-2017    & End of commisioning phase 1,     \\
         &          &                   &                  & Science verification  \\  
% HA: 24
N2R11    & 24       &  8-June-2017      & 13-June-2017     &  polarization commissioning  \\
\hline
\end{tabular}
\caption[Commissioning campaigns, dates and general comments.]{
Brief summary of the NIKA2 commissioning campaigns indicating the starting and end dates as well as
some general comments.
\label{nika2runs}}
\end{center} 
\end{table} 

This document describes the work accomplished by the NIKA2 consortium in order
to validate the performances of NIKA2. More specifically, the MoU defined
specifications and goals on some key parameters. They are summarized in
Tab~\ref{nika2specs}. Additional information on the instrument and ancillary
parameters were also requested by IRAM, together with a document demonstrating
how the performances have been assessed, both in terms of observations and data
reduction. This document is the answer.\\

Table~\ref{nika2runs} summarizes the timeline of NIKA2 integration and tests
since the replacement of NIKA1 by NIKA2 in Oct.~2015. Based on this overview and
the observing conditions, we have focused our analysis on data from
the N2R9 calibration campaign and the N2R12 and N2R14 science pools.
The final results of this commissioning are given in the last section
essentially via a summary table~\ref{tab:nika2summary} with links to the
relevant sections for detailed demonstration of the results.

In order to be self-consistent, the document is also meant to be read in chronological
order. Chapter~\ref{se:chap_instru_obs_methods} summarizes the main features of
the instrument and of the data reduction that will be used in the remaining of
the document for all performance derivations. It also lists and introduces the necessary inputs
that are required for the full data reduction and that are discussed in details
in the following sections.
We start with atmospheric opacity in
chapter~\ref{se:opacities}, then proceed to detector positional offsets in
chapter~\ref{se:fp_reconstruction} and beam characterization in chapter~\ref{se:beams}.
Chapter~\ref{se:calibration} addresses the overall calibration of
observations. At last, chapters~\ref{se:photometry}~and~\ref{se:nefd} address
the final performances of the instrument in terms of photometry and sensitivity,
before the final summary section~\ref{se:summary}.

