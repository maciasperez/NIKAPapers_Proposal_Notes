\documentclass[a4paper,10pt]{article}
\usepackage{epsfig}
\usepackage{latexsym}
\usepackage{graphicx}
\usepackage{amsfonts}
\usepackage{amsmath}
\usepackage{xcolor}

%\topmargin=-3cm
\topmargin=-1cm
\oddsidemargin=-1cm
\evensidemargin=-1cm
\textwidth=17cm
%\textheight=27cm
\textheight=25cm
\raggedbottom
\sloppy

\definecolor{Blue}{rgb}{0.,0.,1.}
\definecolor{LightSkyBlue}{rgb}{0.691,0.827,1.}
\definecolor{Red}{rgb}{1.,0.,0.}
\definecolor{Green}{rgb}{0.,1.,0.}
\definecolor{Purple}{rgb}{0.5, 0., 0.5}
\definecolor{Try}{rgb}{0.15,0.,1}
\definecolor{Black}{rgb}{0., 0., 0.}

%To get DRAFT accross all pages
\usepackage{draftcopy}
%To replace ``DRAFT'' by ``ON GOING''
%\draftcopyName{ON GOING}{150}


%%%%%%%%%%%%%%%%%%%%%%%%%%%%%%%%%%%%%%%%%%%%%%
%%%%%%%%%%%%%%%%%%%%%%%%%%%%%%%%%%%%%%%%%%%%%% 

\title{NIKA Pipeline output products, OpenPool2, Version0}
\author{The NIKA collaboration and IRAM}

\begin{document}
\maketitle

\abstract{This note describes the data products that the NIKA collaboration
  provides to external observers for the second NIKA open time pool
  observations in November 2014 for the version 0.  These products are the
  calibrated maps.  Other products will be made available in later versions.}
% [vA.1: clarify the clean TOI product, vB.0
%   concerns OP2 Data release V0]}

\section{Presentation}

This version V0 of the NIKA second open pool products has been made by the
NIKA team by the 5 December 2014.  It is distributed by IRAM. It follows the
observing run by 2 weeks which were needed for the data processing.

The products are described the following Sect.~\ref{sec:maps}.

For each project, there are fits files containing the maps of each scan and
combined scans of each object. The data processing has been oriented towards
the diffuse emission.

There are also illustrative pdf files. You will find that for each object,
there is a directory that contains:

- The fits files (RaDec2000 projection) for target sources

- Figures: many pdf files were generated out of the fits files to get a quick
look at them. Units in Jy/beam unless otherwise stated. A smoothing has been
done with a Gaussian FWHM of 10 arcseconds (only for the figures). 

{\sl JK} pdf are jackknife maps representing the half difference of a random split
of scans in two halves.

{\sl SNR} pdf is the signal-to-noise maps assuming Gaussian white noise.

{\sl flux} pdf is the brightness map display

{\sl stddev} pdf is the standard deviation map display

{\sl time} pdf is the total integration time per 2 arcsecond square pixel.

{\sl scan} pdf is a brightness display per scan


The main beam calibration was done assuming a Gaussian main beam of 12.5 and
18.5 arcseconds (FWHM). The primary calibrator is Uranus with fluxes of 43.0
and 18.0~Jy.

The main beam to full beam correction is by 1.56$\pm$0.10 at 1~mm and
1.35$\pm$0.10 at 2~mm. It has not been applied to the maps.

At this stage, the offline products cannot be used for scientific analysis as
some photometric uncertainties are still there (mostly opacity effects). The
point-source photometry may probably correct at the 20\% level at 1mm and 15\%
level at 2mm. These products must just be used to evaluate the potential
return of each observed source.

We hope to deliver a final V1 version of the second NIKA open pool data
products in several months including our best strategy of systematic removals.
 
Contact your NIKA friend of project to give us your feedback. 

%%%%%%%%%%%%%%%%%%%%%%%%%%%%%%%%%%%%%%%%%%%%%%
\section{Maps}\label{sec:maps}
We provide maps as FITS files for all the individual scans and a combination
of scans per source (one fits file per source and per wavelength). Combined
maps and individual scan maps are named as \textcolor{blue}{{\tt
    MAPS\_1mm\_source\_v1.fits}} and \textcolor{blue}{{\tt
    MAPS\_2mm\_source\_v1.fits}}.  The signal map (surface brightness map
(opacity corrected)), is provided in extension 0 of the FITS files then other
extensions contain the standard deviation maps, and exposure time per pixel
(hit map multiplied by the sample duration) and then all scans brightness,
standard deviation maps.  More detailed information is given in the header. We
highlight in the following the main issues concerning the map making:
\begin{itemize}
\item The standard coordinate system used is Ra.--Dec. (tangential projection:
  RA---TAN, DEC--TAN).

\item We use a nearest grid point projection with a pixel size of 2~arcsec
  (this can be adjusted by request). 

\item A predefined header for map projection can be used upon request if
  provided by the observers

\item Pixels of the maps that have not been sampled are set to 0 for the
  flux maps and zero for the time per pixel maps. Pixels with less than two measurements are set to
  0 for the standard deviation maps.
% (see Tab.~\ref{tab:table_map})

\item Zero level is set in all detectors timelines outside the source before
  combining them.

\item Timelines are weighted by the inverse variance of the noise, which is
  computed outside the source.

\item In the case of point source data, electronic and atmospheric
  contributions to the data are decorrelated using the standard method
  described in \cite{NIKA_abs_calib}. Basically, a common-mode timeline is
  built by averaging all timelines and avoiding on-source detectors at any
  sample.  The common-mode is first scaled to each detector and then
  subtracted from the timelines using a simple regression procedure.

\item For extended source data two decorrelation methods can be used. The
  first of them minimizes the noise but removes large scale structures. It is
  based on the same common-mode method described above but no masking the
  source. The second one is based in an iterative procedure. First, a
  simplified map is constructed using the former method so that the location
  of the source can be infered. Then this information is used to mask the
  source when computing the common-mode. This method preserves large scales
  (up to the size of the array) but is noisier. Both methods are described in
  \cite{NIKA_abs_calib}. For this release, the two released iterations (when
  given) correspond to the latter method.

\end{itemize}

	% \begin{table}[ht]
	% \begin{center}
	% \begin{tabular}{|c|c|c|c|c|}
        %  \hline
	% Extension & Axis & Content & Units & Comment \\
	% \hline
	% 0 & 0 & Flux density & Jansky/beam & Opacity corrected \\
	% 0 & 1 & Standard deviation & Jansky/beam & Estimated form the TOI \\
	% 0 & 2 & Observing time per pixel & second & Same as hit map normalized by sampling frequency \\
	% 1 &    & Info structure & -- & See Tab.~\ref{tab:info_map} \\
	% \hline
	% \end{tabular}
	% \end{center}
	% \caption{Map FITS files extension 0 }
	% \label{tab:table_map}
	% \end{table}

	% \begin{table}[ht]
	% \begin{center}
	% \begin{tabular}{|c|c|c|c|}
        %  \hline
	% Structure row & Content & Units & Comment \\
	% \hline
        % 1 & Number of scan used & none & \\
        % 2 & List of the scan used & none & Only one for individual scan maps \\
        % 3 & List of the KIDs used & none & Labeled by the detector number (numdet) \\
        % 4 & List of opacities of the given channel & none & Computed from skydips \\
        % 5 & List of the integration times & second & \\
        % 6 & Scan type & none & {\it e.g.} azimuth, elevation, lissajous \\
	% \hline
	% \end{tabular}
	% \end{center}
	% \caption{Map FITS files extension 1.}
	% \label{tab:info_map}
	% \end{table}
	
%%%%%%%%%%%%%%%%%%%%%%%%%%%%%%%%%%%%%%%%%%%%%%

\section{Data delivery}

The IMBFITS files and (the clean calibrated TOI Not Done Yet) are archived by
IRAM and can be provided on request. Calibration products will be available on
the NIKA wiki page: {\tt http://www.iram.es/IRAMES/mainWiki/NIKA/Main}

The maps, associated figures and logfiles, are delivered to each project
account under :

/vis/xxx-14/observationData/nika 

where xxx-14 is your project number.  Just after the run, the products of a
preliminary offline reduction are provided as version v0. Then a clean version
v1 (including calibration products) will be delivered within several
months. Observers may contact their respective NIKA instrument friend of
project for information regarding the offline processing ({\tt
  http://www.iram.es/IRAMES/mainWiki/Continuum/PoolOrganization/2ndNIKApool}). Depending
on feedbacks, a version v2 may be needed.
 

%----------------------------------------------------------------------------------------
\begin{thebibliography}{}
   \bibitem{NIKA_abs_calib} Performance and calibration of the NIKA camera at
     the IRAM 30 m telescope, A. Catalano et al., 2014, A\&A, 569
   \bibitem{calvo_2012} Calvo, M., Roesch, M., D\'esert, F. X., et al., Improved mm-wave photometry for kinetic inductance detectors, 2013, A\&A, 551, L12
\end{thebibliography}

\end{document}
