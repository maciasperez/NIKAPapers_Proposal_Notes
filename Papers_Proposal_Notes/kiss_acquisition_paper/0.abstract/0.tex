\color{black}

\emph{Context.} Major features demanded for instruments in contemporary cosmology are: high mapping speed to cover large portion of the sky and multi-frequency capability for the component separation. In order to simultaneously satisfy the two conditions you can design large arrays of photo-noise-limited detectors adopting a spectroscopy technique that allows wide instantaneous Field of View, FoV. 

\emph{Method.} We exploit a Fourier Transform Spectrometer, FTS. Instead of distribute the pixels to the different frequencies (e.g. on-chip and grating spectroscopy) we developed a double array sensitive to all the interested frequencies. Thank to the FTS we can, then, separate and reconstruct the spectrum. In the ground-based case like KISS, the drawback from such a technique is that we need to be fast to avoid signal contamination from the atmospheric background fluctuations: the single interferometric figure must be acquired with a stable background to properly convert it in a spectrum where the noise does not become indistinguishable from the signal. It is necessary, thus, to use low time-constant detectors and fast FTS technology. KISS adopts, for such target, two arrays of Kinetic Inductance Detector, KID.

\emph{Results.} We developed a new technique to calibrate the KID raw data. We first demonstrate the feasibility of this technique with a simulation. We, thus, study the performance of such technique for the modulation and the tuning, applying it to real on-sky observations. The acquisition technique has been qualified during the first commissioning campaign. It represents a solution for fast multi-frequency acquisition that exploits KIDs based Fourier transform spectrometry.