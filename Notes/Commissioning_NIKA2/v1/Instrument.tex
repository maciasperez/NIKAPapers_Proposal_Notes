
\section{Overview of the NIKA2 instrument}

\subsection{Optics}

NIKA2 is a millimeter camera able to simultaneously image a
field-of-view of 6.5\,arcmin at 150 and 260~GHz, with polarimetric
capabilities at 260~GHz using KIDS. A 30-centimeters diameter air-gap
dichroic splits the 150 GHz (reflection) from the 260 GHz
(transmission) beams.  A grid polarizer ensures then the separation of
the two linear polarisations on the 260 GHz channel.  There are
therefore three KIDS array in NIKA2, one at 2mm, two at 1mm. 

%-------------------------
% HERVE
\input{bandpass}
%-------------------------


\subsection{Cryogenics}
Further technical details on the NIKA2 instrument can be found in the
technical paper {\bf REF !}



\subsection{KIDs and electronics}

NIKA2 focal plane is equipped with arrays of Kinetic Inductance
Detectors (KIDs hereafter), which are high-quality factor
superconducting resonators. They are operated at 150~mK that is well
below the critical temperature, which ensures their responsivity
properties ... 

\addparag{LUMPED KIDS}

The 2mm array (150~GHz) consist of 616 pixels
(KIDS), disposed to cover a circle with a 78 mm diameter. Each pixel
has a size of $2.8 \time 2.8 mm^{2}$ . This is the maximum pixel size that can
be adopted without significantly degrading the telescope resolution,
as it corresponds roughly to a 1F$\lambda$ sampling of the focal
plane.  The array is labelled A2 in all the following.

At 1mm, there are two arrays, labelled A1 and A3, consisting of 1140
KIDS with a size of $2 \time 2 mm^{2}$ to ensure a comparable 1F$\lambda$
sampling.


\addparag{NIKEL}


\subsection{KID tuning}

\addparag{$\ftone$ definition}




