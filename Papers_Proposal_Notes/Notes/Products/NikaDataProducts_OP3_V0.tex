\documentclass[a4paper,10pt]{article}
\usepackage{epsfig}
\usepackage{latexsym}
\usepackage{graphicx}
\usepackage{amsfonts}
\usepackage{amsmath}
\usepackage{xcolor}

%\topmargin=-3cm
\topmargin=-1cm
\oddsidemargin=-1cm
\evensidemargin=-1cm
\textwidth=17cm
%\textheight=27cm
\textheight=25cm
\raggedbottom
\sloppy

\definecolor{Blue}{rgb}{0.,0.,1.}
\definecolor{LightSkyBlue}{rgb}{0.691,0.827,1.}
\definecolor{Red}{rgb}{1.,0.,0.}
\definecolor{Green}{rgb}{0.,1.,0.}
\definecolor{Purple}{rgb}{0.5, 0., 0.5}
\definecolor{Try}{rgb}{0.15,0.,1}
\definecolor{Black}{rgb}{0., 0., 0.}

%To get DRAFT accross all pages
%\usepackage{draftcopy}
%To replace ``DRAFT'' by ``ON GOING''
%\draftcopyName{ON GOING}{150}


%%%%%%%%%%%%%%%%%%%%%%%%%%%%%%%%%%%%%%%%%%%%%%
%%%%%%%%%%%%%%%%%%%%%%%%%%%%%%%%%%%%%%%%%%%%%% 

\title{NIKA Pipeline output products, OpenPool3, Version0}
\author{The NIKA collaboration and IRAM}

\begin{document}
\maketitle

\abstract{This note describes the data products that the NIKA collaboration
  provides to external observers for the third NIKA open time pool
  observations in January-February 2015 for the version 0.  These products are the
  calibrated maps.  Other products will be made available in later versions upon
  request.}

\section{Presentation}

This version V0 of the NIKA third open pool products has been made by the
NIKA team by April 18th, 2015.  It is distributed by IRAM. It follows the
observing run by 2 months (this includes the polarization run that followed the
Open Pool).

The products are described the following Sect.~\ref{sec:maps}.

The data processing has been oriented towards the diffuse emission. For each
project, there is a fits file for the 1mm channel, one for the 2mm
channel. There are also a .csv file and a summary pdf per object.

The main beam calibration was done assuming a Gaussian main beam of 12 and
17 arcseconds (FWHM). The primary calibrator is Uranus with fluxes of 38.0
and 15.9~Jy.

The main beam to full beam correction is by 1.56$\pm$0.10 at 1~mm and
1.35$\pm$0.10 at 2~mm. It has not been applied to the maps.

At this stage, the point-source photometry may probably be correct at the 15\%
level at 1 and 2mm. These products should just be used to evaluate the potential
return of each observed source. Consistency between scans and readiness for
scientific use has not been extensively checked by our team and each observer is
responsible for his use of these data.

An improved version V1 will be delivered upon request based on feedback from observers.
 
Contact your NIKA friend of project to give us your feedback. 

%%%%%%%%%%%%%%%%%%%%%%%%%%%%%%%%%%%%%%%%%%%%%%
\section{Maps}\label{sec:maps}
Some more details on maps. Each map contains the combined scans of each object
(RaDec2000 projection):

\begin{itemize}
\item The standard coordinate system used is Ra.--Dec. (tangential projection:
  RA---GSL, DEC--GSL).

\item extension 0 is empty
\item extension 1 contains the flux map (in Jy/Beam)
\item extension 2 contains the noise map (in Jy/Beam)
\item extension 3 contains the integration time per map pixel (in seconds)
\item extension 4 contains average information.

\item We use a nearest grid point projection with a pixel size of 2~arcsec
  (this can be adjusted by request). 

\item Pixels of the maps that have not been sampled are set to 0 for the
  flux maps and zero for the time per pixel maps. Pixels with less than two measurements are set to
  0 for the standard deviation maps.
% (see Tab.~\ref{tab:table_map})

\item Zero level is set in all detectors timelines outside the source before
  combining them.

\item Timelines are weighted by the inverse variance of the noise, which is
  computed outside the source.

\item In the case of point source data, electronic and atmospheric
  contributions to the data are decorrelated using the standard method
  described in \cite{NIKA_abs_calib}. Basically, a common-mode timeline is
  built by averaging all timelines and avoiding on-source detectors at any
  sample.  The common-mode is first scaled to each detector and then
  subtracted from the timelines using a simple regression procedure.

\item For extended source data we use an iterative map making. We decorrelate
  kids from a common mode (one per wavelength) computed on the entire scan
  (hence including the source) and project data onto a map. This map is then
  read and subtracted from the timelines to improve the estimation of the common
  mode for another decorrelation. This process can be iterated several
  times. For this release of the data, one iteration was performed only.

\end{itemize}

\section{The .csv file}
A .csv file that can be easily parsed by any spreadsheet editor summarizes the
main points of the analysis of point sources. This file is produced by default,
even for diffuse or extended sources and has therefore little meaning in these
cases. The .csv file gathers flux measurements at the center and on the source
fitted position, for each scan and for the combination of all scans (last
line). {\tt flux\_center} is measured at the map center position, while {\tt
  flux} is measured where a gaussian centroid is fit on the map. Note, however,
that both fluxes are estimated with fixed FWHM of 12 and 17 arcsec (1 and 2mm
resp.) to ensure point-source calibration.

\section{The .pdf file}
For each object, the associated .pdf file presents the combined map and the maps per
scan, together with examples of timelines before/after decorrelation and their
associated power spectra, and the scanning strategy.

On each map, two fluxes are provided. The first one, {\tt flux} is computed
around a fitted gaussian centroid (displayed in black). The second one, {\tt flux\_center} is
computed at the center of the map (displayed in green). Both fluxes are estimated with fixed FWHM of
12 and 17 arcsec (1 and 2mm resp.) to ensure point-source calibration. The upper
caption gives the free fit parameters and should not be used.

%% \section{Data delivery}
%% 
%% The IMBFITS files and (the clean calibrated TOI Not Done Yet) are archived by
%% IRAM and can be provided on request. Calibration products will be available on
%% the NIKA wiki page: {\tt http://www.iram.es/IRAMES/mainWiki/NIKA/Main}
%% 
%% The maps, associated figures and logfiles, are delivered to each project
%% account under :
%% 
%% /vis/xxx-14/observationData/nika 
%% 
%% where xxx-14 is your project number.  Just after the run, the products of a
%% preliminary offline reduction are provided as version v0. Then a clean version
%% v1 (including calibration products) will be delivered within several
%% months. Observers may contact their respective NIKA instrument friend of
%% project for information regarding the offline processing ({\tt
%%   http://www.iram.es/IRAMES/mainWiki/Continuum/PoolOrganization/2ndNIKApool}). Depending
%% on feedbacks, a version v2 may be needed.
 

%----------------------------------------------------------------------------------------
\begin{thebibliography}{}
   \bibitem{NIKA_abs_calib} Performance and calibration of the NIKA camera at
     the IRAM 30 m telescope, A. Catalano et al., 2014, A\&A, 569
   \bibitem{calvo_2012} Calvo, M., Roesch, M., D\'esert, F. X., et al., Improved mm-wave photometry for kinetic inductance detectors, 2013, A\&A, 551, L12
\end{thebibliography}

\end{document}
