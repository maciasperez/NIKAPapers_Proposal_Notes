
\section{Conclusion}
\label{conclusion}

\todo{Recall that we are in fact interested in the unknown difference between
  the model and the measure, not specifically its linearity that can be
  considered as a first order approximation to any differnce between an a priori
  known calibration curve}\\

\todo{aussi faire remarquer que l'eq.~(\ref{eq:eq-NL}) is linear in $\epsilon$,
  so we have further mitigation by the average of many detectors around their
  known calibration curve.}\\

This paper presents the study of KIDs systematic effects such as the non-linearity and an application to the CMB polarization. 
KIDs are a new kind of detectors based on superconducting technology that provides high sensitivity. They have been developed for the construction of NIKA and NIKA2 since 2007, which is now the first operational instrument using KIDs. One of the key asset of KIDs is their natural multiplexing capability which makes them one of the best candidates for future space mission that needs large size detector array. In this context, in a first part, we studied the response of a KID by using two methods to reconstruct the signal (\rf and \cf) and its systematic effects caracterized by its non-linearity coefficient \eps. For an incoming source consisting of a planet of 500 Jy, the dipole and a HWP, and for different scan speed, we found $\varepsilon_{rf}$ $\simeq 10^{-5}$ and $\varepsilon_{cf}$ $\simeq 10^{-7}$. The non-linearity depends on the way that we reconstruct the signal, and even if \rf can reconstruct the signal very well, \cf is better at it and generates less non-linearities. Another good point is that the modulation of the HWP does not bias the measurement by inducing large non-linearities. We have seen that in order to have less non-linearities, we need to put constraints on the scanning speed, so in a second part, we did more realistic simulations, by scanning a map of the Galaxy and dipole by using satellite pointing strategies. We found, that depending on the incoming sources, $\varepsilon_{rf}$ varies between $10^{-7}$ (Galaxy only) and $10^{-3}$ (Galaxy, dipole, HWP), and $\varepsilon_{cf}$ varies between $10^{-8}$ (Galaxy only) and $10^{-4}$ (Galaxy, dipole, HWP) (POLSAT, VOIR PB PLANCK). The results show that in a space context KIDs are capable of accurately reproducing the signal and that as in the precedent simulations \cf is slightly better than \rf.\\
The measure of CMB $B$ modes polarization is a major goal of observational cosmology, as their detection would sign the presence of primordial gravitational waves and provide a confirmation of inflation. Observations of CMB polarization demand a high control of systematics effect and in light of this, we demonstrated the capabilities of KIDs arrays to detect B modes polarization by comparing systematic effects coming from the detector and the leakage of dust temperature into polarization maps. With HEALPix, we simulated spurious signals from a map of the Galaxy and generated modified $C_{l}$. From $C_{l}^{TE}$, we calculated the coefficient ($\varepsilon'$) related to the leakage of temperature into polarization. For a tensor-to-scalar ratio (T/S) = 0.1, 0.01, 0.001 we respectively found $\varepsilon' \simeq 2.51$x$10^{-2}$, 7.95x$10^{-3}$, 2.51x$10^{-3}$. Because the leakage of temperature into polarization maps is the systematic effect that is most likely to contaminate the detection of $B$ modes , to be able to measure them, the non-linear coefficient $\varepsilon$ related to the detector must be lower than $\varepsilon'$. In all of the precedent simulations, $\varepsilon $ varried between $10^{-3}$ and $10^{-8}$, so $\varepsilon < \varepsilon'$, therefore we can say that the KID is capable of detecting $B$ mode polarization. Finally, because of the KID multiplexing ability and its capability of detecting $B$ mode polarization, we can say that KID technology is developping toward becoming one of the best candidates to space mission and study of the CMB polarization.
