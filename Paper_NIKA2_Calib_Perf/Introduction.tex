%----------------------------------------------------------------------------------------
%	INTRODUCTION
%----------------------------------------------------------------------------------------
%\section{Introduction}
%\label{se:intro}

Submillimetric domain: a unique thorough view of the Universe from
planetary surfaces in the Solar System to early galaxies,
high-redshift galaxy clusters and the Cosmic Microwave
Background (CMB).\\

Ground-based submillimeter experiments have made tremendous progress in the last
decades thanks to the advent of large detector arrays, challenging the
motivation for spatial experiments.\\

The current experimental efforts are driven by two complementary
challenges: the improvement of the polarisation sensitivity to detect
primordial polarisation B-mode of the CMB and measure CMB lensing on
one hand and on the other hand the increase the angular resolution to
reveal the structure of poorly known or faint objects, such as
high-redshift galaxies and galaxy clusters.\\

NIKA2 is a high-resolution large field-of-view multi-thousand detector
camera operated at the 30-m IRAM telescope. Fourth generation of
imagers installed at the IRAM 30-m: after the single beam
imagers, the 7-pixel camera of the late 90's and several versions of
MAMBO. Unlike its predecessors, NIKA2 is based on KIDs. Validated with
NIKA pathfinder.\\

Brief historic of the commissioning. \\

Key tool for a large variety of astrophysical and cosmological
purposes, from Kuiper Belt objects or moons in the Solar system to
high-redshift galaxies and galaxy clusters. The main scientific
obsjectives are articulated around five ambitious guaranteed-time
large programs [add a small paragraph per LP]\\

These scientific program requires high performance standard in term of
mapping capabilities, sensitivity, angular resolution and stability of
the system. We assess this performance using a large amoiunt of
technical and sciencce-purpose data acquired since the full
installation of NIKA2.\\

Outlines













