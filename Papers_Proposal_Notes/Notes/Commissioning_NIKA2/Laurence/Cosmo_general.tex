%\chapter{Cosmologie avec les amas de galaxies}
%\label{se:cosmo_general}

%----------------------------------------------------------------------------------------
%
%
%
%
%                COSMO-AMAS
%                 
%
%
%
%----------------------------------------------------------------------------------------

\emph{La cosmologie avec les amas de galaxies occupe une place centrale de
mon projet de recherche à court et moyen termes. Je commence la
seconde partie de ce manuscrit par un résumé succint de ce domaine,
qui n'a pas vocation à constituer une revue mais a deux
objectifs. D'abord, dans le cadre d'une mobilité thématique depuis
l'effet de lentille sur le CMB aux amas de galaxies, l'exercice a son
intérêt propre. Ensuite, ce résumé constitue l'introduction, fournit
la justification et situe le contexte de mon projet de recherche.}\\


Les amas de galaxies forment une puissante sonde cosmologique.
Ce sont les objets les plus massifs résultant du processus de formation
hiérarchique des structures à partir des maxima de
densité dans la distribution de matière primordiale. Ils contiennent
donc l'information sur les conditions initiales, le contenu et
l'évolution de l'Univers. Ce sont des traceurs aussi bien de la
distribution des grandes stuctures de l'univers que de leur
croissance~\citep[pour une revue]{Allen2011}. En mesurant une quantité
standardisable, telle la fraction de gaz qu'ils contiennent, les amas
sont utilisés pour mesurer les distances et ainsi sonder
l'histoire de l'expansion de l'Univers~\citep{Sasaki1996, Ettori2009}.  
L'abondance des amas et leur distribution spatiale sont sensibles en
particulier à la densité de matière $\Omega_{\rm{m}}$ et à l'amplitude
des fluctuations linéaires de densité spatialement moyennées sur un
rayon de 8\,Mpc\,$h^{-1}$, $\sigma_8$. Combinées
avec une sonde de l'univers primordial, tel le CMB, elles vont
permettre de contraindre les extensions du modèle
$\Lambda$CDM standard qui affectent la croissance des structures, dans
le secteur des neutrinos~\citep{Wang2005, Bolliet2019}, de l'énergie
noire~\citep{Haiman2001} ou des modifications de la
gravité~\citep{Mohr2003, Hagstotz2019}. Combiner le CMB et
la distribution d'amas de galaxie,
c'est aussi un formidable test de cohérence du modèle cosmologique.%,
%qui est alors
%contraint avec les deux extrêmes de la formation hierarchique des
%structures.
Le modèle doit décrire aussi bien la distribution des perturbations de
densité qui laissent leur empreinte sur le rayonnement émis au
découplage que la distribution statistique et spatiale des plus gros
objets qui se forment aux maxima de cette distribution de matière
initiale, à l'issu d'un long processus d'effondrement gravitationnel
dans un univers en expansion. 

\section{Des objets multi-sondes}
\label{se:multisondes}

Les amas de galaxies sont des objets gravitationnellement liés et
virialisés, dominés par la matière noire, qui représente environ 80\%
de leur contenu. Ce sont donc des halos de matière noire de
masse entre $10^{14}$ et $10^{15}$\,M$_{\odot}$, qui ont piégé du gaz,
chauffé ($\approx 10^{8}\,\rm{K}$) et ionisé, et des galaxies. Ce
plasma diffus, chaud et magnétisé domine
le contenu baryonique des amas, le milieu
intra-amas, représentant environ 17\% de la masse totale, tandis que
les galaxies constituent les 3\% restants. Gravitationellement
relaxés, ce sont des objets auto-similaires, décrits les uns par
rapport aux autres par des relations d'échelle. S'ils sont dominés par
des processus gravitationnels, ils sont aussi le lieu de processus
physiques baryoniques complexes, dont la plupart sont encore
imparfaitement compris, tels la formation des galaxies, le
refroidissement du gaz, le rôle de la rétro-action des supernovae et
noyaux actifs de galaxies (voir par exemple, \citet{Voit2005}). 

\'Etant multi-composantes, ils sont de façon inhérente multi-sondes.
La lumière stellaire des galaxies et la lumière diffuse du
milieu intra-amas sont détectables dans l'optique et l'infrarouge (IR)
proche. Dans ce domaine de longueur d'onde, les amas de galaxies sont
revélés par la distribution des galaxies qu'ils enferment, leur
dispersion de vitesse et le profil de luminosité du milieu
intra-amas. Leur potentiel gravitationnel intégré sur la ligne de
visée peut être reconstruit en exploitant les effets de lentille
gravitationnelle fort et faible sur la lumière des galaxies situées en
arrière-plan~\citep[ pour une revue]{Bartelmann2010}. Dans le domaine des
rayons X, le milieu intra-amas est détectable par l'effet Bremsstrahlung des
électrons et les raies d'émission des éléments plus lourds que
l'hydrogène~\citep{Sarazin1986} La densité électronique du milieu intra-amas peut être
cartographiée par photométrie en rayon X, tandis que la température
électronique peut être inferrée par spectroscopie. Dans le domaine
millimétrique, les amas de galaxies sont détectés par effet
Sunyaev-Zel'dovich, qui décrit la distortion spectrale induite par la
diffusion Compton inverse des photons du CMB sur les électrons du milieu
intra-amas~\citep{SZ1970}.
L'effet de lentille gravitationelle de l'amas sur les photon du CMB
laisse aussi une signature dans les cartes des anisotropies du
CMB~\citep{Seljak2000}. Finalement, les amas émettent par des
processus non-thermiques dans les domaines radio, ainsi que parfois
dans les domaines X énergétique (\emph{Hard X-ray}) et gamma, via les
phénomènes violents (supernovae, AGN, chocs) qu'ils
abritent~\citep{Rephaeli2008}.

Le caractère multi-composantes et complexe des amas est à la fois une
difficulté, car leur exploitation en cosmologie ne peut pas faire
l'économie de mesurer l'impact de ces processus et de leur évolution
sur les observables cosmologiques, et à la fois une force, car ces
observables sont multiples et peuvent etre combinées dans une analyse
cosmologique conjointe.

\section{Exploitation des amas de galaxies pour la cosmologie}
\label{se:sondecosmo}

Réprésentant les plus grands halos de matière noire formés
relativement récemment à l'échelle cosmologique ($0<z<3$), les amas de
galaxies dévient encore peu du régime linéaire de croissance des structures.
Leur croissance non-linéaire et sa dépendence au modèle cosmologique
est formalisée dans le cadre d'un modèle de halos~\citep[, par exemple
  pour une revue]{Cooray2002}. La fonction de masse des halos,
c'est-à-dire leur densité numérique par intervalle de masse et de
redshift, ainsi que les propriétés statistiques de leur distribution
spatiale, en particulier la fonction d'autocorrélation des halos, et
de leur distribution de vitesse encodent l'information
cosmologique. Des modèles analytiques de la fonction de masse des
halos ont été dérivés dans le cas sphérique~\citep{Press-Schechter1974}
et elliptique~\citep{Sheth2001}. Pour mieux capturer les effets
non-linéaires et l'évolution en redshift, la fonction de masse des
halos est paramétrisée et calibrée à partir de simulations
N-corps~\citep{Tinker2008}.

L'exploitation des amas de galaxies pour la cosmologie repose donc en
premier lieu, sur la mesure précise de leur masse
totale~\citep[voir \emph{e.g.}][pour une revue récente sur la mesure
de la masse des amas]{Pratt2019}. Les sondes qui se fondent sur l'effet
gravitationnel de l'amas, telles les effets
de lentilles sur les galaxies ou le CMB, reconstruisent sa masse
totale indépendament des processus baryoniques en jeu en son
sein. Dans ce cas, l'enjeu est le contrôle des effets systématiques qui
impactent la reconstruction de la masse (triaxialité, effets
d'alignement, avant-plans, par exemple). D'autres sondes utilisent le
contenu baryonique de l'amas comme traceur de masse. La correlation
entre l'observable baryonique, e. g. la densité numérique de galaxies
membres, l'émission X, l'effet SZ, et la masse doit être
caractérisée. Dans le scenario auto-similaire~\citep{Kaiser1986}, les
amas ont une structure universelle définie par leur masse et leur
redshift et se déduisent les uns des autres par des lois d'échelle en
masse et redshift. Ainsi, à partir d'hypothèses sur l'état
thermodynamique du milieu intra-amas, des relations d'échelle entre
les observables baryoniques et la masse peuvent être établies et des
profils universels pour les quantités thermodynamiques
paramétrisés. Les écarts aux hypothèses sur la morphologie,
l'évolution et l'état thermodynamique des amas induisent une
modification de la forme et une dispersion dans les relations
d'échelle et les profils thermodynamiques universels qu'il faut
précisément mesurer~\citep{NFW1996, Nagai2007, Arnaud2010}. C'est là le
principal facteur limitant les analyses cosmologiques avec les amas de
galaxies. L'amélioration de ces mesures mobilise d'importants efforts,
qui se basent tant sur des simulations hydrodynamiques de plus en plus
réalistes~\citep[comme exemple récent]{Henden2019} que sur des programmes
observationnels dédiées. 
En particulier, c'est l'objectif premier du programme cosmologique
avec les amas de galaxies observés avec NIKA2, comme ce sera discuté à
la Sect.~\ref{se:LP-SZ}.

Enfin, l'échantillon d'amas inclus dans l'analyse cosmologique doit
pouvoir être relié à la population d'amas complète sous-jacente. La
fonction de sélection de l'échantillon, qui caractérise sa pureté et
sa complétude, doit être précisément estimée. Elle dépend des
caractéristiques instrumentales et observationelles du relevé et des
algorithmes de détection mis en oeuvre pour la construction de
l'échantillon~\citep[par exemple]{Melin2005}. \\

En suivant ce programme, et
grâce à l'avénement d'expériences permettant la construction de larges
catalogues bien caractérisés, des résultats cosmologiques compétitifs
et complémentaires d'autres sondes ont été récemment obtenus dans les
domaines du rayonnement X~\citep{Vikhlinin2009, Bohringer2014, Mantz2015, Marulli2018,
  Pacaud2018}, de l'optique et proche infrarouge~\citep{Rozo2010,
  Mana2013, Costanzi2019}, et du millimétrique via l'exploitation de
l'effet SZ~\citep{Hasselfield2013_ACT_SZ, Planck2016_ymap,
  Planck_2016_SZ_cosmo, Bolliet2018, Bolliet2019, deHaan2016,
  Bocquet2019, Salvati2018, Zulbedia2019}.

Dans la prochaine décennie, les amas de galaxies vont devenir l'une
des sondes cosmologiques principales avec la mise en service de grands
relevés dans l'optique et le proche infrarouge, tels l'expérience au
sol \emph{Large Synoptic Survey Telescope} (LSST) et le satellite
\emph{Euclid}, l'exploitation scientifique des grands relevés en X
(eRosita, Athena) et le développement continu d'expériences sol à
haute résolution angulaire dans le domaine millimétrique (Advanced
ACT, SPT-3G, Simons Array, Simons Observatory). 


\section{Cosmologie avec l'effet SZ}
\label{se:cosmo_sz}

L'exploitation de l'effet SZ pour la cosmologie recèle de spécificités
et avantages, décrits dans~\citet{Carlstrom2002}, par exemple, et que
l'on résume ici. D'abord, l'effet SZ thermique (tSZ)
étant une distortion du spectre de corps noir du CMB, il est
théoriquement indépendent du redshift. C'est là un avantage unique
pour l'étude d'amas de galaxies jusqu'à hauts redshifts, en
particulier pour bien capturer la dépendance dans le modèle cosmologique de
leur abondance en fonction du redshift. Expérimentalement, le signal SZ est toutefois dillué
dans le lobe de l'instrument, ce qui ré-introduit une limite en
redshift à la détection des amas par cette méthode. Ensuite,
l'amplitude du tSZ, quantifiée par le paramètre de Compton $y$, dépend
de l'intégrale le long de la ligne de visée de la pression des
électrons du milieu intra-amas (ICM). %Or, pour un amas à
%l'équilibre hydrostatique dont l'ICM est modélisé par un gaz parfait,
%la pression de l'ICM est directement corrélée à la masse hydrostatique
%de l'amas, qui à son tour, trace la masse totale.
Comparé à la photométrie dans le domaine des X, qui dépend du carré de la densité
électronique de l'ICM et donc trace préférentiellement les régions
denses au coeur des amas, le paramètre de Compton sonde l'ICM jusque
dans la périphérie. Cette propriété rend l'effet tSZ plus immune aux
effets thermodynamiques complexes qui ont lieu au coeur des
amas. Aussi, les amas détectés via le tSZ sont en principe directement
sélectionnés en fonction de leur masse, simplifiant la caractérisation
de la fonction de sélection pour un échantillon de ces
amas~\citep{Holder2000}. En pratique, la détection des amas n'est pas
simplement déterminée par un seuil en signal tSZ et la fonction de
sélection d'un échantillon d'amas détecté en tSZ va aussi dépendre des
caractéristiques expérimentales et algorithmiques de sa
construction~\citep{Melin2005}.

En plus du tSZ, la vitesse particulière de l'amas le long
de la ligne de visée par rapport au référentiel du CMB se manifeste
par une seconde composante à l'effet SZ, appelée SZ
cinétique~\citep[kSZ;][]{Sunyaev1980}. La mesure du kSZ, dont
l'amplitude est en moyenne un ordre de grandeur en dessous du tSZ, est
un défi expérimental (comme discuté à la
Sect.~\ref{se:NIKA_SZ}). C'est aussi une opportunité unique de mesurer
le champ de vitesses des grandes structures de l'univers. Une telle
mesure aurait des implications multiples pour la cosmologie,
permettant de sonder la croissance des
structures, les non-gaussianités primordiales ou la
réionisation~\citep[voir][par exemple]{SO2019}. 
Les corrections relativistes constituent un autre effet du second
ordre, dépendant des ordres supérieurs de la température de
l'ICM~\citep[voir \emph{e.\,g.}][]{Chluba2012}. Cet effet peut s'envisager comme une composante
supplémentaire, le rSZ. Une estimation de la
température électronique de l'ICM pourrait être
extraite d'une mesure du rSZ, qui à son tour permettrait d'affiner les
relations d'échelle masse-observable établies dans le domaine
X~\citep[voir \emph{e.\,g.}][]{Pratt2009}, largement utilisées pour la
calibration en masse des amas dans les analyses cosmologiques. Finalement, en tant
qu'anisotropies secondaires du CMB, l'effet SZ est mesurable par les
expériences CMB et fournit une sonde cosmologique supplémentaire
compétitive avec des exigences instrumentales compatibles avec celles
de la reconstruction de l'effet de lentille sur le CMB. 

Des expériences CMB, telles le satellite \emph{Planck}~\citep{Planck2016_SZcat}, et les
expériences sol \emph{South Pole Telescope}~\citep[SPT;][]{Bleem2015, Bleem2019}
et \emph{Atacama Cosmology Telescope}~\citep[ACT;][]{Hasselfield2013_ACT_SZ} ont récemment
publiés des catalogues d'environ 2000 amas de galaxies détectés via
l'effet tSZ. De plus, une carte du paramètre de Compton a été
construite par séparation de composantes dans les cartes par fréquence
sur tout le ciel de \emph{Planck}~\citep{Planck2016_ymap} et la
puissance angulaire des effets tSZ et kSZ a été mesurée dans les
spectres de puissance angulaires de ACT et
SPT~\citep[\emph{e.g.}]{Dunkley2013, George2015}.
Ces observables ont été utilisées pour contraindre les modèles
cosmologiques via l'estimation de l'abondance des amas dans les
catalogues~\citep{Planck_2016_SZ_cosmo, Hasselfield2013_ACT_SZ,
deHaan2016, Bocquet2019, Zulbedia2019}, et via la mesure du spectre de puissance
angulaire et des moments d'ordres supérieurs dans la carte de
paramètre de Compton~\citep{Planck2016_ymap, Hurier2017, Bolliet2018, Bolliet2019,
Salvati2018} 


\section{Cohérence Univers proche et lointain}
\label{se:cosmo_tensions}

Dès les premiers résultats de 2013, la
collaboration \emph{Planck} annonçait un désaccord
d'une significance $>2\sigma$ entre les modèles cosmologiques mesurés
à partir des anisotropies primaires de température du CMB et à partir de
l'échantillon d'environ 200 amas de galaxies détectés via le
tSZ~\citep{Planck_2014_SZ_Cosmo, Planck_2014_ymap}. Ce désaccord a été
ensuite confirmé par l'analyse cosmologique avec le tSZ
de \emph{Planck} en 2015, incluant plus de deux fois plus
d'amas~\citep{Planck_2016_SZ_cosmo, Planck2016_ymap}, ainsi que par
les analyses d'autres expériences mesurant le
tSZ~\citep[\emph{e.g.}][]{Hasselfield2013_ACT_SZ,
deHaan2016}. D'autres sondes des grandes structures, utilisant aussi
bien les amas de galaxies~\citep[\emph{e.g.}][]{Bohringer2014,
Pacaud2018} que d'autres observables, favorisent également des valeurs
de $\sigma_8$ plus basse que celle estimée par le CMB. Pour une
compilation des résultats ont pourra se réferer à~\citet{Salvati2018}.
Ce désaccord entre le CMB et les amas de galaxies a été
abondamment discuté et a suscité une intense littérature autour de
deux interprétations~\citep[voir \emph{e.g.}][pour un
résumé]{Planck_2016_SZ_cosmo, Salvati2018}. Il peut d'une part
indiquer un effet physique au-delà du modèle $\Lambda$CDM minimal
impactant la croissance des structures (masse des neutrinos
non-minimale, écart de l'énergie noire d'une constante cosmologique,
gravité modifiée), ou d'autre part résulter de l'impact de la
physique complexe en jeu dans les amas sur l'estimation de leur masse
(écart au scenario auto-similaire, pression d'origine non-thermique,
évolution en redshift, par exemple).    

Dans \citet{Planck_2014_SZ_Cosmo} et \citet{Planck_2016_SZ_cosmo},
l'observable tSZ est reliée à la masse totale, définie comme
$M_{500}$, la masse contenue dans une sphère de rayon $R_{500}$ au
sein de laquelle la densité moyenne de matière est 500 fois la
densité critique de l'Univers au redshift de l'amas, via des
observations en X. 
%la méthode de calibration décrite dans \citet{Arnaud2010} est utilisée
%pour relier l'observable tSZ à la masse totale, définie comme
%$M_{500}$, la masse contenue dans une sphère de rayon $R_{500}$ au
%sein de laquelle la densité moyenne de matière est 500 fois la
%densité critique de l'Univers au redshift de l'amas.
Une relation entre l'observable tSZ intégrée dans une sphère de rayon
$R_{500}$, $Y_{500}$ et l'observable X utilisée comme indicateur de 
$M_{500}$, $Y^{X}$, est estimée dans un sous-échantillon de 71 amas à
redshift $<0.45$, correspondant aux amas de l'échantillon cosmologique
(439 amas) qui ont été observés avec une bonne précision par le satellite
X \emph{XMM-Newton}. L'observable $Y^{X}$ est fortement corrélée à la
masse de l'amas sous l'hypothèse que le milieu intra-amas est bien
modélisé par un gaz parfait à l'équilibre
hydrostatique~\citep{Kravtsov2006}. Cette relation entre $Y^{X}$ et
masse de l'amas a été mesurée dans un échantillon de 20 amas pour lesquels
l'hypothèse d'équilibre hydrostatique a été validée dans \citet{Arnaud2010}.
Enfin, la masse supposant l'équilibre hydrostatique, $M_{\rm{HE}}$,
est reliée à la masse totale telle que $M_{\rm{HE}} = (1-b)M_{500}$,
où $(1-b)$ paramétrise le biais hydrostratique. Ce biais, qui englobe les écarts à
l'équilibre hydrostatique et les effets de sélections des amas en X, doit à son
tour être calibré. Pour cela, le biais hydrostatique attendu peut être
estimé dans les simulations hydrodynamiques. Les simulations récentes
prédisent un biais hydrostatique de 10 à 20\%, motivant la valeur
centrale $(1 - b) = 0.8$ adoptée dans~\citet{Planck_2014_SZ_Cosmo}. Le
biais peut aussi être mesuré dans un sous-échantillon d'amas par
comparaison de la masse dérivée du tSZ et de la masse reconstruite via
l'effet de lentille gravitationnelle dans les données
optiques. Suivant cette méthode et en utilisant un échantillon de 22
amas appartenant au programme \emph{Weighting the Giants}
(WtG), \citet{vonderLinden2014} ont mesuré $(1-b) = 0.688 \pm
0.072$. De même, à partir de 50 amas du programme \emph{Canadian
Cluster Comparison Project} (CCCP) au \emph{Canada–France–Hawaii
Telescope} (CFHT), \citet{Hoekstra2015} annoncent
$(1-b) = 0.76 \pm 0.05\rm{(stat)} \pm 0.06\rm{(syst)}$. Un biais
hydrostatique plus faible encore a été mesuré à partir d'un
échantillon de 50 amas à redshift $<0.3$ du \emph{Local Cluster
Substructure Survey} (LoCuSS) avec le télescope
Subaru~\citep{Smith2016}. Récemment, en corrélant la carte de
paramètre de Compton de \emph{Planck} avec le champ de cisaillement
gravitationnel reconstruit avec l'instrument \emph{Hyper Suprime-Cam}
de \emph{Subaru}, \citet{Osato2019} ont mesuré un biais hydrostatique
en accord avec les autres calibrations en masse utilisant les lentilles
faibles. \`A partir d'une compilation des estimations récentes 
du biais hydrostatique, \citet{Salvati2018}
donnent une estimation moyenne de $(1-b) \sim 0.8 \pm 0.08$. 
Une autre approche pour la calibration en masse des amas SZ, fondée
sur l'exploitation de l'effet de lentille gravitationnelle sur le CMB,
a aussi été testée pour la première fois dans la données
de \emph{Planck}~\citep{Melin2015, Planck_2016_SZ_cosmo}. Connaissant
le redshift de l'amas, la masse est estimée à partir du potentiel
gravitationnel intégré sur la ligne de visée en direction de l'amas
reconstruit dans les cartes de CMB.
Cette méthode, applicable à l'ensemble de l'échantillon cosmologique a été
revisitée dans \citet{Zulbedia2019}. Par une estimation conjointe des
paramètres cosmologiques et du biais, ils trouvent $(1-b) = 0.71 \pm
0.10$, en accord avec les mesures issu de l'effet de lentille sur les
galaxies.


Un test de cohérence entre les contraintes cosmologiques dérivées du
CMB et des amas de galaxies, consiste à estimer le biais hydrostatique
à partir des anisotropies primaires de CMB et des observables SZ. Dans
\citet{Planck_2016_SZ_cosmo}, le biais hydrostatique mesuré par cette
méthode et qui permet de réconcilier les deux observables est
$(1-b) = 0.58 \pm 0.04$, en désaccord avec les autres estimations. Ce
désaccord s'est amoindri dans l'analyse cosmologique de 2018, incluant
l'ensemble des données de Planck, y compris la polarisation aux
grandes échelles angulaires mesurées par le \emph{High Frequency
Instrument}. Dans \citet{Planck_2018_cosmo}, la polarisation aux
grandes échelles angulaires favorise une valeur de la profondeur
optique depuis la réionisation $tau$ plus basse que celle estimée
dans \citet{Planck_2016_cosmo}. La statistique des anisotropies
primaires de température est sensible à la combinaison de paramètres
$A_{\rm{s}} \exp{(-2\tau)}$, où $A_{\rm{s}}$ est l'amplitude du spectre
de puissance des fluctuations de densité primordiales. Ainsi, pour
garder $C_\ell^{TT}$ inchangé, une valeur de $\tau$ plus faible
favorise aussi une diminution de $A_{\rm{s}}$, qui implique une valeur
plus faible de $\sigma_8$. Ainsi, dans une analyse combinant le
comptage des amas et la distribution spatiale du paramètre de
Compton, \citet{Salvati2018} annoncent un accord (à $1.8\sigma$) entre
les modèles cosmologiques inférés à partir du CMB primaire et de
l'effet tSZ. Ces résultats sont également confirmés
dans \citet{Zulbedia2019}. Toutefois, le bias hydrostatique estimé en
combinant le CMB et le tSZ est $(1-b) = 0.65 \pm 0.04$ dans le modèle
$\Lambda$CDM minimal et reste $\lesssim 0.67$ quelque soit les
extensions à ce modèle considérées. Le désaccord avec les
mesures du biais hydrostatique via l'effet de lentille ou les
prédictions des simulations est moindre, mais non entièrement
résorbé. 

D'autres effets systématiques sont susceptibles d'affecter les
paramètres cosmologiques inférés à partir des amas de galaxies, de
sorte qu'il n'y a plus aucun désaccord entre modèles cosmologiques
favorisés par le CMB ou les amas lorsque l'on relache les \emph{a
priori} sur la physique des amas. Ce peut être en considérant une
déviation aux hypothèses sur la fonction de masse des
halos~\citep{Bocquet2016}, les lois d'échelle entre masse et
observable~\citep{Planck_2016_SZ_cosmo}, ou le profile de pression
utilisé pour estimer l'observable~\citep{Ruppin2019b}. Donc,
l'exploitation des amas de galaxies en cosmologie, en particulier pour
contraindre les déviations au modèle $\Lambda$CMB minimal, requiert
une amélioration de notre connaissance de leur propriétés
physiques. En particulier, des efforts sont encore nécessaires pour
mieux caractériser la forme du profil de pression et de la relation
d'échelle entre masse et observable, ainsi que leur dispersion
intrinsèque et leur évolution en redshift.  
Pour cela, une voie privilégiée consiste à observer des amas de
galaxie à haute résolution angulaire et jusqu'à haut redshicht. C'est
là, l'un des principaux objectifs scientifiques de NIKA2. 



% Ade et al 2019 [Simons Observatory]
% The ability to use cluster abundances to constrain cosmological
%parameters is limited by uncertainties in the
%observable-to-mass scaling relation (e.g., Vikhlinin et al.
%2009; Vanderlinde et al. 2010; Sehgal et al. 2011; Benson
%et al. 2013; Hasselfield et al. 2013b; Planck Collaboration
%2014c; Mantz et al. 2014; Planck Collaboration
%2016j; Mantz et al. 2015; de Haan et al. 2016). Aussi Pratt 2019
%
%
%
%
