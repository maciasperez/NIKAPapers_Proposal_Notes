\abstract {Clusters of galaxies provide valuable information on the evolution of the Universe and large scale structures. Recent cluster observations via the thermal Sunyaev-Zel'dovich (tSZ) effect have proven to be a powerful tool to detect and study them. In this context, high resolution tSZ observations ($\sim$ tens of arcsec) are of particular interest to probe intermediate and high redshift clusters.}
{Observations of the tSZ effect will be carried out with the millimeter dual-band {\it NIKA2 }camera, based on Kinetic Inductance Detectors (KIDs) to be installed at the IRAM 30-meter telescope in 2015. To demonstrate the potential of such an instrument, we present tSZ observations with the {\it NIKA }camera prototype, consisting of two arrays of 132 and 224 detectors that observe at 140 and 240~GHz with a 18.5 and 12.5~arcsec angular resolution, respectively.} 
{The cluster \mbox{RX~J1347.5-1145} was observed simultaneously at 140 and 240~GHz. We used a spectral decorrelation technique to remove the atmospheric noise and obtain a map of the cluster at 140~GHz. The efficiency of this procedure has been characterized through realistic simulations of the observations.}
{The observed 140~GHz map presents a decrement at the cluster position consistent with the tSZ nature of the signal. We used this map to study the pressure distribution of the cluster by fitting a gNFW model to the data. Subtracting this model from the map, we confirm that \mbox{RX~J1347.5-1145} is an ongoing merger, which confirms and complements previous tSZ and X-ray observations.}
{For the first time, we demonstrate the tSZ capability of KID based instruments. The {\it NIKA2 }camera with $\sim 5000$ detectors and a $6.5$~arcmin field of view will be well-suited for in-depth studies of the intra cluster medium in intermediate to high redshifts, which enables the characterization of  recently detected clusters by the {\it Planck }satellite.}