
\nico{
\begin{itemize}
%\item we concentrate our efforts on timeline processing and pre-whitening and
%  use a naive co-additive map making
\item we are currently working on optimal map makers for our future deep field
  integrations and with NIKA2 in perspective
\item pointing accuray and monitoring
\item Show transfer function and photometry reconstruction on point sources and
  diffuse emission (simulations)
\item Mention the difference between noise on TOIs and noise estimated on the
  map background
\end{itemize}
}
~\\

The problem of map making has been extensively discussed in the literature, with
pros and cons of various methods depending on the kind of signal the instrument
is after and at what stage of the data processing the various sources of noise
are addressed. As a side question, one could ask what exactly belongs to map
making or to TOI processing, and to what extent the two overlap. In this work,
we have put efforts in atmosphere and noise subtraction at the TOI level but
this involved spatial information obtained from maps built in the early stages
of the TOI processing. While different in principle from an integrated
iterative map making {\it e.g.} \cite{sanepic}, this process does involve interations and
exchange of information between time and map domains. This being said, in this
section, we describe what goes beyond the standard TOI processing already
presented in sect.~\ref{se:toi_proc} and how TOIs are combined into maps.

The general map making equations reads:

\begin{equation}
d_t = A_{tp}S_p + n_t
\end{equation}

where $d$ is the vector that gathers all detector timelines, $S_p$ is the sky
map considered as a vector and $A_{tp}$ is the pointing matrix that encodes the
address of the observed pixel $p$ by a sample at time $t$. $n_t$ is the total
noise vector. The maximum likelihood solution of this equations is ({\it e.g. }
\cite{madmap}):

\begin{equation}
\hat{S} = \left(A^TN^{-1}A\right)^{-1}A^TN^{-1}d
\label{eq:map_making}
\end{equation}

where $N$ is the full noise covariance matrix $<nn^T>$, whose size is
$(N_{kids}\times N_{samples})^2$ elements. Rather than estimating this matrix
and compute eq.~(\ref{eq:map_making}), we shall assume that after the TOI
processing described in sect.~\ref{se:toi_proc}, the detector timelines are
dominated by their own white noise and that they are independent of one
another. $N$ therefore reduces to a diagonal matrix and its diagonal elements
are simply the variance of each kid noise. Solving eq.~(\ref{eq:map_making})
reduces to a co-addition of samples with optimal weighting by the inverse
variance of the noise. Errors on the final map induced by this approximation are
estimated via full Monte-Carlo simulations of the observation (see
sect.~\ref{se:validation}).


% In this work, rather than
%determining the complete noise covariance matrix $N$, using it in this form and
%then estimate how uncertainties propagate to the final map,
%we... \nico{Generaliser l'equation en incluant un matrice F de filtrage et
%  expliquer que notre approche ayant consiste a faire du pre-whitening des
%  timelines, on se restreint ici a de la coaddition mais qu'on simule toute la
%  chaine pour valider le processus et les erreurs induites. Revoir comment Eric
%  amene ca aussi dans le papier Master.}
