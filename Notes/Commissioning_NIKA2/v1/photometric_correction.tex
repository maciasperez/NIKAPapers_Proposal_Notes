\subsection{Photometric correction}

The effect of the telescope-driven beam size variations, as discussed
in Sect.~\ref{se:obsdate_variations}, is mitigated by correcting the measured flux
densities using a beam-size dependent function, referred to as the
photometric correction.


In our reference photometric system, as presented in
Sect.~\ref{se:flux_density_equation} and
Sect.~\ref{ap:flux_density_equation},
the flux density is estimated by fitting an amplitude of a fixed-width
Gaussian beam. Using Eqs.~\ref{eq:calfwhm0}, \ref{eq:pointsourcephot}
and \ref{eq:pointsourcemap}, observed maps are modeled as:
\begin{equation}
  M(\theta, \phi) = \frac{A}{2 \pi \sigma_{0}^{2}} e^{-\frac{\theta^{2}}{2\sigma_{0}^{2}}}, 
\end{equation}
where $\sigma_{0}$ is the Gaussian beam size corresponding to the
reference FWHM, which is $12.5''$ for the 1mm arrays and $18.5''$ for
the 2mm array, as defined in Table~\ref{tab:definitions}. The
amplitude of the fixed-FWHM Gaussian provides us with an estimator of
the flux density 
\begin{equation}
  \hat{S}  = 2 \int \int M(\theta, \phi)\, e^{-\frac{\theta^{2}}{2\sigma_{0}^{2}}} \sin \theta d\theta d\phi.
  \label{eq:flux_density_estimator}
\end{equation}

If the beam size varies so that the Gaussian part has a size given by
$\sigma ’$, the map model rewrites  
\begin{equation}
  M(\theta, \phi) = \frac{A'}{2 \pi \sigma'^{2}} e^{-\frac{\theta^{2}}{2\sigma'^{2}}}.
  \label{eq:broader_beam_map}
\end{equation}

Ingesting Eq.~\ref{eq:broader_beam_map} in
Eq.~\ref{eq:flux_density_estimator}, we find that the flux density
estimator depends on the actual beam size as
\begin{equation}
  \hat{S}  = A' \frac{2 \sigma_0^2}{(\sigma'^2 + \sigma_0^2)}
\end{equation}


We use a photometric correcting function $f(\sigma')$ to evaluate the corrected flux density estimate $\hat{S}_{pc}$ from the
uncorrected flux density estimate $\hat{S}$ as
\begin{equation}
  \hat{S}_{pc} = f(\sigma')\hat{S},
\end{equation} 
where
\begin{equation}
  f(\sigma') = \frac{(\sigma'^2 + \sigma_0^2)}{2 \sigma_0^2}. 
\end{equation} 



