%
%%
%%
%%

%% LP: first synthetic version of Appendix D. 

%% LP modif : to be validated by HA :
%% - 'spectral irradiance' remplac\'ee par 'flux density', qu'on utilise plus souvent
%% - 'response' remplac\'e par 'map' [ une occurence]
%% - '\Omega_{b}' remplac\'e par 'B_{k}', plus \'evocateur du beam et souligne la dependance dans le KID k
%% - '\Omega_{s}' remplac\'e par '\Omega_{p}' pour \'evoquer angle solide sur la pupille
%% - A VERIFIER: les quantit\'es pas connues précisement
%%   'Because both $A_{p}$ and $ \Omega_{p} $ are not known with good accuracy', correct ? 



%% [intro]
The calibration of the NIKA2 instrument in its final configuration of
January 24th 2017 is studied in this section in using Uranus as the
main primary calibrator. Neptune and Mars are also considered as
valuable alternatives to calibrate when Uranus is not visible.

% HA 
\section{Photometric System {\color{blue} Herv\'e}}
\label{se:cal_HA}

\begin{table}[h]
\begin{center}
\begin{tabular}{|c|c|c|}
\hline
     & 1 mm & 2 mm \\
\hline
Reference frequency $\nu_{0}$ & 260 GHz & 150 GHz \\
\hline
Reference FWHM                      & 12.5  '' & 18.5 '' \\
\hline
\end{tabular}
\caption{NIKA2 reference frequencies and FWHM}
\end{center}
\label{tab:definitions}
\end{table}

\section{Main photometric equation}

Starting from the general equation for the response of a detector to
an astronomical source, we derive the main photometric equation as
detailed in Sect.~\ref{ap:cal_HA}.
The map (in units of the KID resonance frequency shift or $\rm Hz$) of a source using a single KID $k$
after correction of the atmospheric absorption is
\begin{equation}
R^{k}(\theta, \phi) \simeq G_{k}  A_{p}\Omega_{p} \int_{0}^{+\infty} I(\nu)
T(\nu) T_{atm}(\nu) B_{k}(\nu_{0}, \theta \times \frac{\nu}{\nu_{0}},
\phi) d\nu 
\label{eq:mainphot}
\end{equation}

where we have considered a source observed at airmass $\am$ under
$mm_{H_{2}O}$ of precipitable water, with specific intensity $I_{\nu}$ (in units
of  ${\rm W/m^{2}/sr/Hz}$) in the direction $\theta, \phi$, where $\theta$
is the off-axis distance and $\phi$ the position angle, illuminating the KID $k$. 
The dependance on elevation and opacity is corrected, as discussed in
section~\ref{se:opacities}, so that Eq.~\ref{eq:mainphot} is the KID
response outside of atmosphere (in terms of airmass, but not in terms
of transmission).

The integral in the right-hand part of Eq.~\ref{eq:mainphot} gives the total power (units of $\rm W$)
falling on the KID, where the factors are:
\begin{itemize}
\item $T(\nu)$, NIKA2 bandpasses, which are measured and corrected for the Raighleigh-Jeans term as discussed in Sect.~\ref{se:bandpasses}
\item $T_{atm}(\nu)$, the transmission of the atmosphere at zenith. 
\item $B_{k} (\theta, \phi, \nu)$, the fraction of source signal illuminating the KID, that is the beam function. It has been parametrized as a function of the effective frequency as defined in eq~\ref{eq:nueff} of Sect.~\ref{ap:cal_HA},
considering that its frequency dependency is only due to the diffraction law, hence a variation as $1/\nu$ from a reference frequency $\nu_0$.  
\end{itemize}

The gain of the KID $k$ $G_{k}$ (units of  $\rm Hz \cdot \rm W^{-1}$) converts the total power in $\rm W$
to the frequency shift in $\rm Hz$. $A_{p}$ is the area of the entrance pupil ({\it i.e.} the
dish collecting area), and $\Omega_{p}$ is the solid angle of the source seen from the
entrance pupil.

Because both $A_{p}$ and $ \Omega_{p} $ are not known with good
accuracy, it is not possible to compute all the terms of
Eq.~\ref{eq:mainphot} from first principles, and a practical way of
calibrating the system must be used: it is done by observing a primary
calibrator.


\subsection{Calibrator map in the reference system}

A primary calibrator is a source whose flux density (or spectral irradiance) is
known. For NIKA2, we use two planets as primary calibrators, Uranus
and Neptune.

The specific intensity $I_{c}(\nu)$ of the
calibrator is:
\begin{equation}
I_{c}(\nu) =  \frac{S_{c}(\nu)}{\Omega_{p}} =\frac{ S_{c}
  (\nu_{0})}{\Omega_{p}} f(\frac{\nu}{\nu_{0}})
\label{eq:intensity_calibrator}
\end{equation}
where $S_{c}(\nu)$ is the flux density of the calibrator (units
of $\rm W/m^{2}/Hz$ or Jy). We parametrize the source flux density
as a function of a reference frequency $\nu_{0}$ that we choose
arbitrarily to be: $\nu_{0} = 150$~GHz for the 2mm array and $\nu_{0}
= 260$~GHz for both 1mm arrays. 

Ingesting Eq.~\ref{eq:intensity_calibrator} in the main photometric
equation (Eq.~\ref{eq:mainphot}), the map (in units of $\rm Hz$) of a
calibrator observed with the KID $k$ is:
\begin{equation}
R_{c}^{k}(\theta, \phi) =  G_{k} A_{p} S_{c} (\nu_{0})  \int_{0}^{+\infty}
f(\frac{\nu}{\nu_{0}}) B_{k}(\nu_{0}, \theta \times \frac{\nu}{\nu_{0}},
\phi) T(\nu) T_{atm}(\nu) d\nu
\label{eq:mainbeammap}
\end{equation}

This observed map is modelled with a fixed-width Gaussian as 
\begin{equation} 
R_{c}(\theta, \phi)  = \frac{A_{c}}{2 \pi \sigma_{0}^{2}}
e^{-\frac{\theta^{2}}{2\sigma_{0}^{2}}}.
\label{eq:calfwhm0}
\end{equation}
The reference FWHM, labelled FWHM$_{0}$, which we recall, is related
to $\sigma_{0}$ by $2 \sqrt{2\ln{2}} \sigma_{0} = {\rm FWHM}_{0}$),
are $12.5''$ for the 1mm arrays and $18.5''$ for the 2mm
array, as defined in Table~\ref{tab:definitions}. These values have
been chosen sizably larger than the main beam values, as reported in
Sect.~\ref{se:beams}, to account for a fraction of the signal smeared 
in the error beam.


We form an estimator of the FWHM$_{0}$ Gaussian amplitude:
\begin{equation} 
\hat{A_{c}}  = 2 \int \int R_{c}(\theta, \phi)e^{-\frac{\theta^{2}}{2\sigma_{0}^{2}}} \sin \theta d\theta d\phi
\label{eq:fixed-width-gaussian-estimator}
\end{equation}

But we also know that the integral of the observed map should give the power
emitted by the source. Therefore, we form the map:
\begin{equation}
M_{c}(\theta, \phi) = R_{c}(\theta, \phi)   S_{c} (\nu_{0}) / \hat{A}_{c}
\end{equation}
where  $S_{c} (\nu_{0})$ is the flux density of the calibrator
at a reference frequency $\nu_{0}$ given in Table~\ref{tab:definitions}.

This map has units of $\rm Jy$. By construction, integrating over the map we have:
\begin{equation}
\int\int M_{c}(\theta, \phi) \sin \theta d\theta d\phi = S_{c}(\nu_{0}).
\end{equation}



\subsection{Calibration in FWHM$_{0}$ beam}
\label{se:flux_density_equation}

Similarly, a point source with flux density $S_{s}(\nu)$ will
generate a response at position $(\theta, \phi)$
\begin{equation}
R_{s}^{k}(\theta, \phi) =  G_{k} A_{p}  \int_{0}^{+\infty}
S_{s}(\nu) B_{k}(\nu_{0}, \theta \times \frac{\nu}{\nu_{0}},
\phi) T(\nu) T_{atm}(\nu) d\nu
\label{eq:pointsourceresponse}
\end{equation}

Note here that the effective frequency for the beam is not necessarily
the same as the one for the primary calibrator, as it depends on the
source spectrum.

We fit the amplitude of a Gaussian of FWHM$_0$ width $\hat{A_{s}}$.
The flux density estimate for the source is then:
\begin{equation}
\hat{S}(\nu_{0})  = \frac{S_{c}(\nu_{0})}{A_{c}} \times \hat{A_{s}}
\label{eq:pointsourcephot}
\end{equation}


In other words, the flux estimate is the flux that should have the
calibrator in order to generate a response that would be fitted with a FWHM$_0$
Gaussian of the same amplitude as the source.

Let us form the map:
\begin{equation}
M_{s}(\theta, \phi) = \frac{S_{c} (\nu_{0})}{A_{c}}  R_{s}(\theta,
\phi)
\label{eq:pointsourcemap}
\end{equation}

The map $M_{s}(\theta, \phi)$ is said to be calibrated in Jy / FWHM$_{0}$
beam. The factor $S_{c} (\nu_{0})/A_{c}$ is referred to as the
absolute calibration factor.

If we have a single point source in $M$, we have when we fit a Gaussian
of fixed width:
\begin{equation}
\int \int M_{s}(\theta, \phi) \sin \theta d\theta d\phi = \hat{A_{s}}  S_{c} (\nu_{0}) /
A_{c} = \hat{S}(\nu_{0})
\end{equation}

Note that the flux density estimate $\hat{S}(\nu_{0})$ {\em is not} the flux of the source at the
reference frequency. In order to find the flux of the source at the
reference frequency, a color correction has to be applied
\begin{equation}
S_{s}(\nu_{0}) = \hat{S}(\nu_{0})  C_{s}
\end{equation}


%The color correction $C_{s}$ is derived in Sect.~\ref{ap:cal_HA}. 
%For aperture photometry, the map calibrated in Jy / $FWHM_{0}$ must be
%converted in a map in Jy / pixel. 

