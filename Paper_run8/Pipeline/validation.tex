\nico{
\begin{itemize}
\item Validation on simulations and estimation of remaining systematic errors
\item Transfer function
\item Transfer function in the case of undetected weak sources : how much do we
  loose on the flux of an undetected weak source during the common mode
  subtraction and how to we correct the upper limit we place on this flux ?
\item Faire une estimation des erreurs dues au beam a la FICSBell pour ne pas
  avoir a convoluer le meme ciel par $N_{kids}$ beam elliptiques ou un truc a la
  asymfast ?
\item Traiter du map making SZ ici ou le garder pour le papier SZ ?
\item Mentionner les comparaisons au bruit theorique et a celui mesure en labo
  pour montrer la consistance ?
\end{itemize}
}

In this section, we describe the simulation pipeline that we have developed to
assess the performances of the data reduction pipeline. We do simulate the full
physical process of radiation absorption by a kid and its conversion into a
final ``scientific'' sample, this has been addressed elsewhere
(\cite{xx,xx1,xx2} \nico{TBC}). The starting point of our simulations is
therefore a sky image, that is scanned according to the actual \nika~pointing,
to which we add simulations of atmosphere and electronic noise, cosmic
rays... \nico{ complete list TBC}. We describe \nico{XX} simulations : a bright
point source to illustrate the quality of our photometry reconstruction, a faint
point source to show our sensitivity, an extended source. We also the transfer
function of the complete data reduction chain.

\subsection{Bright point sources}

\subsection{Faint point sources}

\subsection{Extended emission}

\subsection{Diffuse emission}


