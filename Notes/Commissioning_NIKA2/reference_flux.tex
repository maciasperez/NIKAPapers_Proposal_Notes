%
%  Created by LP from a copy a photometry_HA.tex and a merge of cal_JFL.tex
%


\subsection{Reference flux densities of the calibrators}
\label{se:ref_flux}

% HA text
The two main calibrators of NIKA2 are the giant planets Uranus and
Neptune. Mars can also be used as primary calibrator, but care must be
taken to use a flux corresponding to the date of the
observations. Secondary calibrators were also observed during the
commissioning campaign. 

\subsubsection{Uranus and Neptune}
For the flux densities of the giant planets, we use the ESA model from
\cite{ESAmodel}: Version 5 for Neptune and Version 4 for Uranus. 
Both models provide the planet brightness temperature in the
Rayleigh-Jeans approximation as a function of the frequency. The
resulting flux is therefore: 
\begin{equation}
S_{\nu} = \Omega \times \frac{2 \nu^{2} k T_{RJ}}{c^2}
\end{equation}
where $\Omega$ is the solid angle of the planet on the sky. Following
Bendo et al. (2013) {\bf REF} and correcting their equation 12 we have:
\begin{equation}
\Omega = \pi \frac{r_{e} r_{p-a}}{D^{2}} 
\label{eq:omega}
\end{equation}
where $r_{e}$ is the equatorial radius of the planet and $r_{p-a}$ is
its apparent polar radius, and $D$ the distance to the
planet. $r_{p-a}$ can be computed from the sub-observer latitude $\phi$
({\it e.g.} the latitude of the observed {\bf ?}  as seen from the planet in the
planet equatorial reference frame) and $r_{p}$ the polar radius of the
planet as:
\begin{equation}
r_{p-a} = \sqrt{r_{p}^2 \cos^{2}\phi + r_{e}^2 \sin^{2} \phi}
\end{equation}
All quantities to compute the planet flux are obtained from the NASA
Horizons web site \cite{NASAHorizon}, and are
listed in table~\ref{tab:planetphysparam}. To compute the planet fluxes for a given date, we use the python
photometry package available at \cite{gith-Haussel}.

\begin{table}[ht]
\begin{center}
\begin{tabular}{|c|c|c|}
\hline
     & Uranus & Neptune \\
\hline
$r_{e}$ [km]  & 25559 & 24764 \\ 
\hline
$r_{p}$ [km]  & 24973 & 24341  \\
\hline
$\phi$         & Ob-lat & Ob-lat \\
\hline
$D$   [AU]    & delta   & delta \\
\hline
\end{tabular}
\end{center}
\caption{Physical quantities used for the Uranus and Neptune fluxes
  computation (equation~\ref{eq:omega}. Ob-lat and delta are quantities 
  tabulated by NASA Horizons system \cite{NASAHorizon} as a function of the date}
\label{tab:planetphysparam}
\end{table}



 
The model spectra are linarly interpolated in log space at the
reference frequencies of the NIKA2 bandpasses. Fluxes for all NIKA2
calibration runs are listed in table~\ref{tab:fluxPred}, together with
the expected variation between the start and end of a run. 

The Uranus and Neptune models have been compared to Planck
observations of these planets \cite{PLCK-LII}. For Uranus, the model used in the comparison
is the ESA V2, and it is found to overpredict by 4 K (about 4\%) the
observed RJ temperature at 143~GHz, to agree at 217 GHz, and
to underpredict at 353 GHz. We use for NIKA2 calibration ESA model V4,
that predict a flux respectively -3.3\%, 0.3\% and 4.7\% higher in the
the 143, 217 and 353 GHz, that would lead to a percent
accuracy with respect to Planck observations. 

For Neptune, the same study compared Planck observation with the ESA V5
model, {\it i. e.} the same one used for NIKA2 calibration. For this
planet, temperatures are found to disagree at most by 5 K, i.e 4.1\%,
with the same trend with frequency as observed for Uranus. All thing
considered, this study confirm that Uranus ESA V4 and Neptune ESA V5
models are accurate to 5\% for predicting planet fluxes. Calibration
values tabulated in table~\ref{tab:flucPred} show that the variations
of Uranus and Neptune over the duration of a typical NIKA2 run are
negligible compared to the model accuracy. On the other hand, not
taking into account the planet shape and orientation with respect to
the observer in the computations of its solid angle can lead to errors
between 1 and 2\% as illustrated in the Python notebook
\cite{gith-Haussel-Note}
distributed with the software. 



\begin{table}[p]
\centering
\begin{tabular}{|l|r|r|r|r|r|r|}
\hline
NR$^{a}$  & JD$^{b}$ & $\Delta t$ $^{c}$ & $S_{\nu}$(260 GHz)  $^{d}$& $S_{\nu}$(150  GHz)$^{e}$  & $\Delta S_{\nu}/  S_{\nu} ^{f}$  \\
\hline
         & d  &  d        & Jy               & Jy                 &                                                                    \%  \\
\hline
         &    &            & \multicolumn{3}{|c|}{Uranus}\\
\hline
13 & 2457330.5 &  12 & 45.59 & 17.65 & -0.89\\
14 & 2457354.5 &  8 & 44.44 & 17.21 & -1.07\\
15 & 2457409.5 &  20 & 40.62 & 15.73 & -3.22\\
16 & 2457455.5 &  14 & 38.27 & 14.82 & -1.16\\
18 & 2457660.0 &  25 & 46.06 & 17.83 & +1.25\\
19 & 2457690.0 &  7 & 46.09 & 17.85 & -0.32\\
20 & 2457732.0 &  7 & 44.14 & 17.09 & -1.04\\
21 & 2457764.5 &  4 & 41.82 & 16.19 & -0.69\\
22 & 2457809.0 &  7 & 39.08 & 15.13 & -0.83\\
23 & 2457865.0 &  7 & 37.96 & 14.70 & +0.14\\
24 & 2457915.4 &  5 &  39.49 & 15.29 & +0.66 \\
\hline
         &    &            & \multicolumn{3}{|c|}{Neptune}\\
\hline
13 & 2457330.5 &  12 & 17.09 & 7.18 & -1.26\\
14 & 2457354.5 &  8 & 16.64 & 6.99 & -0.92\\
15 & 2457409.5 &  20 & 15.76 & 6.62 & -1.35\\
16 & 2457455.5 &  14 & 15.55 & 6.53 & +0.19\\
18 & 2457660.0 &  25 & 17.65 & 7.41 & -1.30\\
19 & 2457690.0 &  7 & 17.24 & 7.24 & -0.68\\
20 & 2457732.0 &  7 & 16.46 & 6.91 & -0.79\\
21 & 2457764.5 &  4 & 15.92 & 6.68 & -0.34\\
22 & 2457809.0 &  7 & 15.56 & 6.53 & -0.08\\
23 & 2457865.0 &  7 & 15.89 & 6.67 & +0.57\\
24 & 2457915.4 &  5 & 16.73 & 7.02 & +0.56 \\
\hline
         &    &            & \multicolumn{3}{|c|}{Mars}\\
\hline
13 & 2457330.5 &  12 & 146.19 & 48.30 & +7.75\\
14 & 2457354.5 &  8 & 175.88 & 58.14 & +8.70\\
15 & 2457409.5 &  20 & 319.71 & 105.62 & +27.68\\
16 & 2457455.5 &  14 & 666.46 & 218.49 & +30.37\\
18 & 2457660.0 &  25 & 597.17 & 199.44 & -21.61\\
19 & 2457690.0 &  7 & 439.23 & 146.24 & -4.82\\
20 & 2457732.0 &  7 & 311.78 & 103.98 & -4.89\\
21 & 2457764.5 &  4 & 239.37 & 79.54 & -2.12\\
22 & 2457809.0 &  7 & 174.99 & 57.94 & -4.94\\
23 & 2457865.0 &  7 & 123.61 & 40.61 & -5.44\\
24 & 2457915.4 &  5 & 102.08 & 33.68 & +0.59 \\
\hline
\end{tabular}
\caption{NIKA2 Planet fluxes. a: Nika Run, b: Julian Date when the
  model are computed, c: Run duration, d, e: total fluxes at 260 and
  150 GHz, f: varition of the 150 GHz flux density over the duration
  of the run}
\label{tab:fluxPred}
\end{table}


\subsubsection{Mars}
For Mars, we use the model of Belloche \&  Amri (2006) available at
\cite{beloche},
with default parameters. Model output is computed at the two reference
frequencies of NIKA2, 150 and 260~GHz.

Fluxes of Mars are tabulated in table~\ref{tab:fluxPred}. In many
cases, the variations of Mars flux during the course of a run are
larger than the model uncertainty (5\%), and should be recomputed at
more frequent times. 


{\bf FM: conclusion for Mars ?}\\


% End HA edit

%\subsubsection{Secondary calibrators: asteroids}
\subsubsection{Secondary calibrators}

%
% LP: commented the obsolete text below (see Jean-Francois's email on
% August, 17)
%


%The asteroids Ceres and Vesta have been modeled by Muller et al (2014) in accounting for 
%size, shape, spin-properties, albedo, and thermal properties and in adjusting to PACS, SPIRE and HIFI observations
%of Herschel with an accuracy of 5\%. 
%Thomas Mueller has tabulated flux densities at different wavelengths, in particular at 1300$\mu$m, every five days
%until 2020 \footnote{http://www.iram.es/IRAMES/mainWiki/Continuum/Calibrators}.
%We have used the prediction at  1300$\mu$m made for  23rd february 2017
%and extrapolated it  to the central frequencies of the arrays in using a Rayleigh-Jeans
%spectrum expected for Ceres and Vesta. Their flux densities in
%Table~\ref{tab:fluxPred} are for this date. Over the five days of  run 9 (february 23 - 28), the
%flux densities  at 1300$\mu$m  have decreased by  3\% 
%for Ceres and  by 6\%  for Vesta in Muller's tables but we have not corrected for this effect in our analysis below.  
%
%The secondary calibrator MWC349A is a young Be star, part of a stellar binary system, surrounded by a disk. Its radio
%continuum emission originates in an ionized bipolar outflow (Tafoya et al 2004).
%MWC349 has been monitored with the  Plateau de Bure interferometer
%and shown to be only slightly angularly resolved, making it a point source for the 30-metre telescope. We have adopted
%its flux densities from this monitoring \footnote{http://www.iram.fr/IRAMFR/IS/IS2012/presentations/krips-fluxcalibration.pdf}.
%The secondary calibrator CRL2688 is an Asympotic Giant Branch star. Its radio continuum emission is mostly from circumstellar dust and
%is somewhat extended  (Knapp et al 1994).
%Its flux densities at $850\mu$m  and $450\mu$m  have been stable at the 5\% level as monitored by SCUBA2 (Dempsey et al 2013).
%We have extrapolated their flux densities to the central frequencies
%of the arrays with a power law of index $\alpha=-2.47$ derived from these SCUBA2 measurements.
%The secondary calibrator NGC7027 is a young, dusty, carbon rich Planetary Nebula with an ionized core.
%It is extended in the continuum and molecular lines (Bieging et al 1991) and  is not a point source for the telescope.
%Its  most recent flux densities are reported at $1100\mu$m  and $2000\mu$m by Hoare et al (1992). It has been reported
%to decrease by $\sim$ 0.145 percent/yr in the optically thin part of its spectrum above  $6$ GHz from VLA
%observations (Zijlstra, van Hoof \& Perley 2008, and Hafez et al, 2008) that makes these flux densities uncertain by 3.6\%
%at present. Its SED from cm wavelengths to optical is also presented in Hafez, Y.A. et al (2008).
%The flux densities adopted at the central frequencies of the arrays for these three calibrators are in Table~\ref{tab:fluxPred}.
%
%
%
%
%
%\begin{table}
%\centering
%\label{tab:fluxPred}
%\caption[]{Flux densities of calibrators at the reference frequencies
% of arrays for Run9 (computed for 2017-02-24T00:00) and Run10
%(computed for 2017-04-21T00:00)}
%\begin{tabular}{|l|r|r|}
%\hline
%\multicolumn{1}{|c}{}  & \multicolumn{2}{|c|}{flux densities (Jy)}  \\
%\hline
%         &    A1, A3      &  A2   \\
%       &  260 GHz    & 150 GHz \\
%\hline
%\multicolumn{1}{|c}{}  & \multicolumn{2}{|c|}{Run9}  \\
%
%\hline
%Uranus   &  39.10 & 15.14 \\
%Neptune  & 15.56 & 6.53 \\
%\hline
%\multicolumn{1}{|c}{}  & \multicolumn{2}{|c|}{Run10}  \\
%Uranus   &  37.95 & 14.69 \\%
%Neptune  & 15.56 & 6.67  \\
%\end{tabular}
%\label{tab:fluxPred}
%\end{table}

%Vesta    &   0.99   &  0.35 &   1.01  \\
%Ceres    &   0.89   &  0.31 &   0.91   \\
%MWC349   &   2.2    &  1.6  &   2.2   \\
%NGC7027  &   3.61   &  4.42 &   3.61  \\
%CRL2688  &   3.03   &  0.83 &   3.03  \\
%\hline
%\end{tabular}
%\label{tab:fluxPred}
%\end{table}




%\begin{table}
%\centering
%\label{tab:fluxPred}
%\caption[]{Reference flux densities of calibrators at central frequencies of arrays.}
%\begin{tabular}{|l|c|c|c|}
%\hline
%\multicolumn{1}{|c}{}  & \multicolumn{3}{|c|}{flux densities (Jy)}  \\
%\hline
%         &    A1      &  A2   &   A3    \\
 %        &  255GHz    & 152GHz  &  258GHz \\
%\hline
%Uranus   &  37.12   & 16.35 &  37.810 \\
%Neptune  &  15.28   &  6.84 &  15.58  \\
%Vesta    &   0.99   &  0.35 &   1.01  \\
%Ceres    &   0.89   &  0.31 &   0.91   \\
%MWC349   &   2.2    &  1.6  &   2.2   \\
%NGC7027  &   3.61   &  4.42 &   3.61  \\
%CRL2688  &   3.03   &  0.83 &   3.03  \\
%\hline
%\end{tabular}
%\label{tab:fluxPred}
%\end{table}



%%%%%%%%%%%%%%%%%%%%%%%%%%%%%%%%%%%%%%%%%%%%%%%%%%%%%%%%%%%%%%%%%%%%%%%%%%
%
%   cp in cal_JFL.tex

%\subsubsection{Reference flux densities of secondary calibrators}
\label{se:fluxSec}
%(revised by JFL Aug 2017 - must remove the old version in photometry-HA.tex)
% --> done by LP following intruction in Jean-Francois's email (to be checked)

The secondary calibrator MWC349A is a young Be star, part of a stellar binary system, surrounded by a disk. Its radio
continuum emission originates in an ionized bipolar outflow \cite{Tafoya}.
MWC349A has been monitored with the  Plateau de Bure interferometer and VLA,
and shown to be stable in time and only slightly angularly resolved, making it a point source
for the 30-metre telescope. The SED of MWC349A   \cite{krips} is presented in Fig.~\ref{fig:Krips2017}. We have computed its flux densities at the NIKA2 reference frequencies 150 and 260 GHz with 
$S_\nu = 1.16\pm0.01 \times (\nu/100 \rm{GHz})^{0.60\pm0.01}$ provided by this monitoring \cite{krips}.


%\noindent {\bf FM: MWC349A or MWC349 ?}\\

The secondary calibrator CRL2688 is an Asympotic Giant Branch star. Its radio continuum emission is mostly
from circumstellar dust and is somewhat extended \cite{Knapp}.
Its flux densities at $850\ \mu$m  and $450 \ \mu$m  have been stable at the 5\% level as monitored by SCUBA2 in 2011-2012 
\cite{Dempsey}. We have extrapolated their flux densities to  150 and 260 GHz
with the power law $S_{\nu} \propto \nu^{\alpha}$ and index $\alpha=2.44\pm0.18$ derived from their SCUBA2 measurements.

\begin{table}[t]
\begin{center}    
  \begin{threeparttable}

\begin{tabular}{|l|c|c|c|l|}
\hline
\multicolumn{1}{|c}{}  & \multicolumn{3}{|c}{flux  densities (Jy)} & \multicolumn{1}{|c|}{}  \\
\hline
         &    A1 \& A3       &  A2             &          &   Ref. \\
         &  260 GHz           &  150 GHz         & $\alpha^1$ &      \\
\hline
MWC349A   &   $2.06\pm0.04$  &  $1.48\pm0.02$ &  $+0.60\pm0.01$      &  PdB \cite{krips}    \\
NGC7027  &   $3.46\pm0.11$   &  $4.26\pm0.24$  &  $-0.34\pm0.10$     &  Hoare et al 1992 \cite{Hoare}      \\
CRL2688  &   $2.91\pm0.23$   &  $0.76\pm0.14$  &  $+2.44\pm0.18$     &  Dempsey et al 2013  \cite{Dempsey} \\
\hline
\end{tabular}
  \begin{tablenotes}
{\small     
  \item[$^1$]  Spectral index is defined as $S_{\nu} \propto \nu^{\alpha}$. 
}
  \end{tablenotes}
\end{threeparttable}
\caption[]{Reference flux densities of secondary calibrators at the NIKA2 reference frequencies 150 and 260 GHz. Uncertainties of flux densities extrapolated
at 150 and 260 GHz include contribution of the uncertainty on $\alpha$.}
\label{tab:flux_ref_sec}
\end{center}
\end{table}
% cahier I p. 128 et 129.

The secondary calibrator NGC7027 is a young, dusty, carbon rich Planetary Nebula with an ionized core.
It is extended in the continuum and molecular lines (Bieging et al 1991), and  is not a point source
for the 30-metre telescope.
Its  most recent flux densities are reported at $1100\mu$m  and $2000\mu$m by Hoare et al (1992). It has been reported
to decrease by $\sim$ 0.145 percent/yr in the optically thin part of its spectrum above  $6$ GHz from VLA
observations (Zijlstra, van Hoof \& Perley 2008, and Hafez et al, 2008) that makes these flux densities uncertain
by 3.6\% currently. Its SED from cm wavelengths to optical is also presented in Hafez, Y.A. et al (2008).
Its flux densities have been extrapolated to 150 and 260 GHz and the modelled decrease since 1992 included.

All these extrapolated flux densities are in Table~\ref{tab:flux_ref_sec}.

\begin{figure}[h]
\begin{center}
  \includegraphics[clip, angle=0, scale=0.4]{Figures/MWC_349_pap-flux.eps}
  \caption{SED of MWC349A from its flux density monitoring at PdB and VLA \cite{krips}.
  Symbols are for primary calibrators used (Uranus, Neptune and Mars).}
\label{fig:Krips2017}
\end{center}
\end{figure}
