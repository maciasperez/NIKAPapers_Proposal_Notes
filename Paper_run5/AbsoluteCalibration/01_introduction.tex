% Astrophysical context 

New challenges in millimeter-wave astronomy require instruments with a combination of high sensitivity, angular resolution, and mapping speed.  These goals demand 
the development of   a new generation of arrays with large number of 
detectors (see for example \cite{2010SPIE.7741E...2B, gismo, 2008ACT, 2008SPIE.7020E..25G}). Current individual detectors, such as high
impedance bolometers \citep{2012MmSAI..83...72T}, are already photon-noise-limited both for space and the ground observations. 

One of the possible technologies suitable for scaling up to larger format arrays is the 
kinetic inductance detector (KID). These detectors are designed to be read out with frequency-division 
multiplexing with a large multiplex factor from a few hundred to several thousand per 
readout chain \cite{2012SPIE.8452E..0OB, Swenson2010}) and are relatively simple
to fabricate (see for example \cite{day2003, 2007stt..conf..170D}).
This technological solution has been selected
for the \NIKA\ project (\cite{NIKA_2010, NIKA_2011}) consisting of
a dual-band millimeter camera for observations at the 30~m Institut de RadioAstronomie Millim\'etrique (IRAM) telescope (Pico Veleta, Spain).
\\ 
The \NIKA\ camera is a 356-pixel instrument
conceived as a resident common user instrument and 
an early technology demonstrator of the competitiveness of KID arrays. \NIKA\ also serves
as preparation for \textit{NIKA2}, a full-scale instrument to be deployed in late 2015 and 
designed to fill the field of view of the IRAM 30~m telescope. 
The goals of the \NIKA\ instrument are to perform simultaneous observations in two bands (1.25\,mm and
2.14\,mm) of milliJansky point sources and to map faint extended
continuum emission with diffraction-limited
resolution and background-limited performance. Achieving these goals 
will enable the \NIKA\ instrument to be competitive in several
astrophysical fields, such as observations from the northern hemisphere of clusters of galaxies detected by PLANCK via the Sunyaev-Zel'dovich (SZ) effect (\cite{2013A&A...550A.128P}), observations of high redshift
galaxies and quasars, detection of early stages of star formation in molecular clouds in our galaxy 
and mapping of dust and free-free emission in nearby galaxies.  In a companion paper, we present the first observation of the thermal SZ effect on
clusters of galaxies \citep{2013arXiv1310.6237A}.  The \NIKA\ camera is now open to the public for observations as of January 2014.

%Why this paper
Previous observing campaigns with the \NIKA\ camera have revealed several technical aspects that limited the sensitivity of detectors. \cite{Calvo2013} show how to linearize the KID signal, while another important issue is the working point of the detectors that evolves with time, owing to varying atmospheric conditions,
thus producing a loss in response. A dynamical tuning of the readout
electronics was developed to optimize the KID working point
between two different sky observations. Atmospheric absorption correction is also required, and a dedicated procedure has been developed in order to use the \NIKA\ instrument as a tau-meter.


These improvements were implemented for the last two technical observing
campaigns that took place in November 2012 and June 2013. We report here  the overall
linearity, sensitivity, and absolute calibration of \NIKA\ .


This article is structured as follows: a review of the main characteristics of
the \NIKA\ instrument that focuses on the instrumental improvements with respect
to the previous observing campaigns, is presented in Sects.~\ref{instr} and~\ref{tuning}.
In Section~\ref{focal}, we describe the sky observations made with \NIKA\ that
allowed us to characterize the focal plane. Section~\ref{dataprocessing} provides a short description of the reduction pipeline used to analyse the data. In Section~\ref{abs_cal} we describe the atmospheric absorption
correction method. In Section~\ref{se:nefd}, the final performance
of \NIKA\ in terms of  instrumental noise equivalent flux density (NEFD) is
calculated on point-like sources. Finally in Section~\ref{obs} we {detail the} observations
of point sources and extended sources of the last November 2012 (run5) and June
2013 (run6) campaigns.
