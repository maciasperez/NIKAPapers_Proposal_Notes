
\section{Measurement of the CMB...}

\subsection{Introduction of the problem and main equations}
\begin{itemize}
\item equations pour montrer l'impact de la non linearite
\item plot qui montre le niveau de NL tolerable sur du CMB pur (les Cl actuels)
  et le niveau de residus de foregrounds qu'on peut tolerer aussi (TE, EB,
  TB...)
\item Do not forget the dipole (NL on it and induced NL)
\item Careful with the actual (single sky) correlation between foregrounds and
  CMB that does not cancel ou ``on stat average''...
\item discuss the maximum expected flux from the galaxy vs a few planet fluxes
  for which we may encounter stronger than spec non linearities.
\item careful not to fit the circle and a small signal (=> bad fit and false
  NL), but to fit it on ``decent signal'' (not too strong either) and THEN
  measure a weak signal. White noise should be enough ?
\end{itemize}

\subsection{Simulations}
\subsubsection{CMB NL residuals}

\subsubsection{Foregrounds residuals}

\begin{itemize}
\item compute foregrounds' alms (polarized) and propagate to power spectra =>
  take 353GHz dust polarized map and 30GHz (which is best) synchrotron
  template. Warning: ringing vs large scale domination
\end{itemize}
