%----------------------------------------------------------------------------------------
%
%   EUCLID
%
%----------------------------------------------------------------------------------------
%\chapter{Cosmologie avec les amas de galaxies dans \emph{Euclid}}

{\color{vert}\lipsum[2-3]}

\section{La mission \emph{Euclid}}
% 1 page
{\color{vert}\lipsum[2-5]}

\section{Comptage des amas et \emph{clustering}}
% 1 page
{\color{vert}\lipsum[2-5]}

\section{Analyses multi-sondes}

Relation entre richesse et masse en utilisant le tSZ comme traceur de
la masse\\
Utiliser les caractérisations croisées entre tSZ et richesse comme
input de la calibration en masse utilsant le Weak Lensing (Dietrich,
J. P., Bocquet, S., Schrabback, T., et al. 2019, MNRAS, 483, 2871)\\


``our work highlights the value of consistency checks between scaling
relations inferred from multi-wavelength observations, which should
lead to constraints with better understood systematic uncertainties.''~\citep{Bleem2019} 


% 1 page
{\color{vert}\lipsum[2-5]}
