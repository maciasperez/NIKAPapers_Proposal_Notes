

\section{Calibration accuracy test using secondary calibrators}
\label{se:calibration_bias}

The calibration accuracy is primarily assessed by checking
that the flux density measurement of known sources is unbiased.

This performance test also depends on the accuracy with which the
chosen secondary calibrator SED is known. Our main secondary
calibrator is MWC349, for which we have derived precise NIKA2 flux
density espectations as discussed in
Sect.~\ref{se:ref_flux_secondaries}.
These are mainly limited by the absolute calibration of the Plateau de
Bure interferometer and VLA.

We define the calibration bias $b$ for array $i$ as
the ratio between the measured flux density $\hat{S_{i}}$ using the
Gaussian fixed-width beam photometry as discussed in
Sect.~\ref{se:cal_HA} and the flux density expectations $S^{0}_{i}$ as
given in Sect.~\ref{se:ref_flux_secondaries}. From a series of
secondary calibrator scans, we evaluate the average calibration bias
per array $b_{\rm A_i}$, which by construction, should be equal to
unity within the precision with which the expected flux densities are
known. Moreover, the calibration bias stability against the observed
opacity provides us with a robustness test of the opacity correction,
and the stability againts the measured beam size, a test of the
photometric susceptibility to optical variations.
%(driven by the main dish distortions)

In Sect.~\ref{se:baseline_bias}, we test the stability of the
calibration bias using the baseline calibration method, which we
recall, relies on i) the \emph{corrected skydip} opacity correction method
and ii) the baseline selection of observation scans to mitigate the impact
of the beam distortion during afternoons. In
Sect.~\ref{se:other_bias}, the baseline results are compared to
results drawn using other calibration methods that either resort to different
opacity correction or include a photometric correction to correct for
the beam impact. 


\subsection{Baseline calibration accuracy}
\label{se:baseline_bias}


\emph{corrected skydip} opacity correction method in
Sect.~\ref{se:corrected-skydip}
 
baseline scan selection in Sect.\ref{se:data_selection} 


\subsection{Comparison with other calibration methods}
\label{se:other_bias}
