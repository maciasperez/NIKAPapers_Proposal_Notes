%
%
%
%
%\chapter{Le programme cosmologique de NIKA2}
%\label{se:cosmo_NIKA2}

%----------------------------------------------------------------------------------------
%
%
%
%
%                COSMO HAUTE-RESOLUTION
%                 
%
%
%
%----------------------------------------------------------------------------------------
\section{Cosmologie CMB à haute résolution angulaire}

NIKA2 s'inscrit dans l'effort expérimental dans le domaine du CMB. 
Après le succès des expériences mesurant les grandes échelles
angulaires~\citep[BICEP2 et Keck Array][]{BK2018} et celles à
résolution angulaire de 1 à 4', telles SPT-SZ et
SPTpol~\citep{deHaan2016, SPTpol2019}, ACT et
ACTpol~\citep{ACTpol2017, ACTpol2018} ou
POLARBEAR~\citep{Polarbear2017}, les expériences CMB au sol se
développent tout azimuth, avec BICEP3~\citep{BICEP3_2018},
\emph{Advanced ACTpol}~\citep{AdvACT2018}, SPT-3G~\citep{SPT3G_2018}
ou le \emph{Simons Array}~\citep{SA_2016}. Pour la future génération
d'expériences, les efforts se rassemblent dans des
meta-collaborations multi-sites, telles
CMB-S4~\footnote{\url{http://CMB-S4.org}}, dont les précurseurs, tel
le \emph{Simons Observatory}~\citep{SO2019} sont
en cours de construction. L'objectif étant de combiner très haute
sensibilité à la polarisation pour espérer détecter le mode B
primordial qui signe la fin de l'inflation et résolution angulaire de
l'ordre de la minute d'arc pour assurer une bonne mesure des
anisotropies secondaires, en particulier l'effet de lentille
gravitationnelle sur le CMB, qui a aussi un impact sur le mode B
primordial, et l'effet SZ pour sonder la matière baryonique à haut
redshift. En particulier, le \emph{Simons Observatory}, dont la
première lumière est prévue en 2021, détectera dix fois plus d'amas
de galaxies que \emph{Planck} et exploitera aussi bien l'abondance des
amas que les propriétés statistiques de la carte de paramètre de
Compton pour dériver des contraintes compétitives sur la cosmologie,
par exemple dans le secteur des neutrinos.  
%Les exigences observationnelles pour la mesure du
%SZ étant les memes que celles du CMB lensing, la cosmologie avec le SZ
%est aussi en tres fort développement.
De plus, avec la perspective d'exploiter l'effet SZ cinétique pour
mesurer les propriété thermodynamique des amas, les effets baryoniques
non-thermiques, la dispersion des vitesses et le début du processus de
réionisation, une nouvelle voie très prometteuse
s'ouvre~\citep{SO2019}. La cosmologie avec le SZ est donc un objectif
scientifique prioritaire des futures expériences CMB.

En parallèle, d'importants efforts instrumentaux sont déployés dans le
domaine CMB/millimétrique pour atteindre des résolutions angulaires en
deçà de la minute d'arc. NIKA2 participe de cet effort. D'autres
expériences à haute résolution angulaire dans le domaine
millimétrique, à fort impact sur la physique des amas
de galaxie via l'exploitation de l'effet SZ, sont actuellement en
fonctionnement ou en projet. Parmi celles-ci, on peut citer l'imageur
MUSTANG-2, installé au \emph{Green Bank telescope} de 100 mètres,
ouvert à la communauté depuis 2018, capable de carthographier un champ
de vue de 4.35’ à 90\,GHz avec une résolution angulaire de
9''~\citep{Dicker2014_MUSTANG2, Stanchfield2016_MUSTANG2}. En
particulier, l'utilité de MUSTANG-2 pour mesurer le profil de
pression des amas de galaxie via l'effet tSZ à été récemment
démontrée~\citep{Romero2019_SZ}.
Autre exemple, l'observatoire interférométrique \emph{Atacama Large
  Millimeter/Submillimeter Array} (ALMA), qui inclut trois réseaux
d'antennes de sept à douze mètres de diamètre, peut imager un champ de
vue jusqu'à environ 1.5' à une résolution de quelques seconde d'arc
dans la gamme de fréquence 84 à 950\,GHz~\citep{ALMA2008, Iguchi2009}.
Dans le domaine du SZ, ALMA a récemment fourni une cartographie à une
résolution de 3.5'' d'un choc dans un amas en
collision~\citep{Basu2016}. La future génération d'expériences
millimétriques/sub-millimétriques haute-résolution est d'ores et déjà en
construction. Ces projets incluent \emph{TolTEC}, au
\emph{Large Millimeter-wave Telescope} de 50 mètres de diamètre,
capable d'imager un champ de vue de 4' à une résolution de 10 à 5''
dans trois bandes de fréquence centrées à 150, 220 et
280\,GHz~\citet{bryan_optical_2018}; CONCERTO, un spectromètre qui
sera installé au télescope de 12-m de l'\emph{Atacama Pathfinder
  Experiment} (APEX) et couvrira la gamme de
fréquence de 125 à 360\,GHz avec un champ de vue d'environ 15' et une
résolution angulaire maximale de 20''~\citep{Lagache2018}; 
CCAT-prime est un télescope de 6-m de diamètre dont l'exploitation
devrait commencer en 2021 sur le site de \emph{Cerro Chajnantor}, pour
ouvrir la voie à la construction d'un télescope de 25-m (CCAT) sur ce
site exceptionnel~\citep{Stacey2018}; Il accueillera l'imageur
\emph{Prime-Cam}, dont les observations dans une gamme de fréquence où
le signal SZ est positif devrait permettre de séparer les contributions
tSZ, kSZ et rSZ dans les futurs relevés d'amas de
galaxies via l'effet SZ~\citep{Mittal2018}. On se reportera à
\citet{Tony2019} pour une revue récente de l'effet SZ à haute
résolution.

Pour la cosmologie, l'objectif des expériences millimétriques à très
haute résolution angulaire est double. Il s'agit d'une part d'obtenir
des cartographies de plus en plus profondes
de l'univers distant en reculant la limite de confusion, avec pour
objectif de contraindre l'évolution cosmique de la formation d'étoile
et les scenarii de la réionisation de l'Univers~\citep{Mancuso2016,
  Bethermin2017_simu}. D'autre part, ces expériences permettent de
sonder la structure interne des amas de galaxie et de contraindre
l'évolution de leur propriétés avec le redshift afin d'améliorer leur
exploitation en cosmologie, comme discuté à la
Sect.~\ref{se:cosmo_tensions}. Ces deux approches cosmologiques
constituent des objectifs prioritaires de NIKA2,
mobilisant chacun un large programme d'observation en temps
garanti. Mon projet de recherche dans NIKA2 s'inscrit dans le grand
programme dédié à la cosmologie avec les amas de galaxie, dont je suis
\emph{Principal Investigator} aux côtés de Frédéric Mayet. 



%----------------------------------------------------------------------------------------
%
%
%
%
%
%                 DESCRIPTION LP-SZ
%
%
%
%----------------------------------------------------------------------------------------
\section{Le grand programme d'observation d'amas de galaxies}
\label{se:LP-SZ}

La cosmologie avec les amas de galaxies est actuellement limitée par
la précision avec laquelle la masse des amas peut être déterminée, en
particulier à des redshifts $\gtrsim 0.4$, comme discuté à la
Sect.~\ref{se:cosmo_tensions}. Ainsi, une étude détaillée des propriétés
thermodynamiques du milieu intra-amas et de la morphologie des amas de
galaxies dans un large échantillon incluant des amas distants
($z\gtrsim 0.7$) permettrait d'améliorer les résultats cosmologiques se
fondant sur cette sonde. L'expérience NIKA2, décrite au
chapitre\,\ref{chap:nika2iram} et dont les performances ont
été détaillées aux chapitres\,\ref{chap:calib_perf} et
\,\ref{chap:nika2_resume}, nous offre une opportunité unique pour mener
une telle étude, via une cartographie haute-résolution de l'effet SZ
d'un échantillon d'amas couvrant une large gamme de redshifts. C'est
là l'objectif du grand programme SZ~\footnote{site web :
  \url{http://lpsc.in2p3.fr/NIKA2LPSZ/}} (LP-SZ) de NIKA2.


\subsection{L'échantillon d'amas}
Le LP-SZ est l'un des cinq grands programmes sélectionnés par le
consortium NIKA2-IRAM et à ce titre, bénéficie de 300 heures
d'observation (\emph{Guaranteed Time}) avec l'expérience NIKA2 au
télescope de 30-m de l'IRAM.

Le LP-SZ cible un échantillon représentatif de 50 amas de galaxies à
redshifts intermédiaires et à hauts redschifts ($0.5 < z < 0.9$),
sélectionnés dans les catalogues d'amas de \emph{Planck} et ACT. La
distribution spatiale, ainsi que la distribution en masse et redshift
des amas de l'échantillon est présentée à la Fig.~\ref{fig:LP-SZ}.
%
\begin{figure}
  \centering
  \includegraphics[width=0.49\textwidth]{Figures/NIKA2-SZ/Figure_LPSZ_dust_map_FR.jpg}
  \hspace{4mm}
  \includegraphics[width=0.44\textwidth, clip=true, trim=0cm -0.7cm 0cm 0cm]{Figures/NIKA2-SZ/LPSZ_M_z_grid.pdf}
  \caption{Left: Map of the kinetic SZ effect toward \mbox{MACS~J0717.5+3745} using NIKA pathfinder data, as discussed in Adam et al. (2017)$^{31}$. Right: NIKA2 Guaranteed-time cosmology program sample of galaxy clusters}
  \label{fig:LP-SZ}
\end{figure}
%
L'échantillon se distribue dans
deux intervalles de redshifts, les redshifts intermédiaires dans la
gamme 0.5 à 0.7 et les hauts redshifts entre 0.7 et 0.9. Il se découpe
en cinq intervalles en masse, définis à partir de la quantité
$E_{z}^{-2/3} D_{\rm{A}}^2 Y_{500}$ fortement corrélée à la masse. Cette
quantité dépend de l'observable tSZ intégré $Y_{500}$ et du modèle
cosmologique à travers $E_{z} = H(z)/H_0$ et $D_{\rm{A}}$, la distance
angulaire. Nous avons sélectionné cinq amas cibles dans chacun des
intervalles en redshift et en masse en utilisant des critères assurant
la représentativité de l'échantillon. Cette propriété est fondamentale
pour fournir des mesures des relations d'échelle et des profils
thermodynamiques applicables à l'ensemble des amas et donc utiles pour
les analyses cosmologiques. Comme discuté à la Sect.~\ref{se:cosmo_sz}, une
sélection basée sur le signal tSZ capture presque tous les amas
au-dessus d'un seuil en masse, et définit bien un échantillon
représentatif. Ainsi, les amas du LP-SZ sélectionnés sont ceux
appartenant à un catalogue d'amas détectés via le tSZ, dont le
redshift est dans la gamme 0.5-0.9 et qui sont observables depuis le
télescope de 30-m de l'IRAM. Pour ce dernier critère, nous retenons
les amas dont la déclinaison est $\delta > -11^{\circ}$, comme figuré
par le cercle pointillé sur le panneau de gauche de la
Fig.~\ref{fig:LP-SZ}. Les quatre intervalles de plus hautes masses sont
peuplés par des amas sélectionnés dans le premier catalogue de
\emph{Planck}~\citep[le seul disponible au moment de la définition de
l'échantillon][]{Planck2013_SZcat}, tandis que les amas de
l'intervalle de basse masse sont issus du catalogue
ACT~\citep{Hasselfield2013_ACT_SZ}.

Par ailleurs, les observations tSZ avec NIKA2 seront complétées par
des suivis avec d'autres sondes. En particulier, la complémentarité
avec les observations dans le domaine X avec le satellite
\emph{XMM-Newton} est l'une des forces du LP-SZ. Les amas du LP-SZ
également observés par \emph{XMM-Newton} sont repérés avec une étoile
dans le panneau de droite de la Fig.~\ref{fig:LP-SZ}. Des demandes de
suivi avec \emph{XMM-Newton}, portées par l'équipe du LP-SZ, sont en
cours.   

\subsection{Objectifs et livrables}

L'objectif premier du LP-SZ est de fournir à la communauté des cartes
à haute résolution angulaire de l'effet tSZ, ainsi que les profils de
pression reconstruits, pour un échantillon représentatif d'amas de
galaxies. Ces données permettront une étude approfondie des propriétés
des amas de galaxie jusqu'à haut redshift et de leur évolution
cosmologique. De même qu'un profil de pression universel a été mesuré
avec \emph{REXCESS}, un échantillon représentatif de 33 amas observés
avec \emph{XMM-Newton} dans l'univers proche
($z<0.2$)~\citep{Arnaud2010}, l'échantillon du LP-SZ permettra de
tester la régularité du profil de pression jusqu'à $z=0.9$ à partir
d'une observable sondant directement la pression. Aussi, l'impact des
sous-structures et de la morphologie des amas sur l'observable $Y_{500}$
pourra être quantifié. Des indicateurs de l'état dynamique des amas
pourront être défini à partir des observations tSZ. Ainsi, le LP-SZ
permettra de caractériser l'impact des déviations au
comportement auto-similaire sur le biais et la dispersion intrinsèque
de la relation d'échelle et du profil de pression moyen, et leur
évolution en redshift. Ces résultats attendus auront d'importantes retombées
pour la cosmologie avec les amas de galaxies.

Les objectifs scientifiques peuvent encore être étendus en exploitant
la complémentarité avec d'autres sondes des amas, et en premier lieu,
les observations dans le domaine des X. Gràce à l'expérience NIKA2, le
milieu intra-amas sera cartographié via l'effet tSZ avec le même
niveau de précision qu'en X, tant en résolution angulaire qu'en
sensibilité. Ainsi, une étude conjointe des données tSZ de NIKA2 et
des données X de \emph{XMM-Newton} permettra une caratérisation
complète des propriétés thermodynamiques des amas. Aux profils de
pression électronique $P_e(r)$ reconstruits dans les données tSZ, nous
adjoindrons les profils de densités $n_e(r)$ estimés dans les données
X. Une telle étude multi-sonde nous permettra de mesurer le profil de
température $k_{\rm{B}} T_e(r) = P_e(r)/n_e(r)$, sans avoir recourt
aux mesures spectroscopiques en X, ainsi que le profil
d'entropie $K(r) = P_e(r)/n_e(r)^{5/3}$, un bon indicateur de l'état
dynamique des amas. Ensuite, sous l'hypothèse de l'équilibre
hydrostatique, nous reconstruirons le profil de masse hydrostatique
des amas $M_{\rm{HE}} (r) \propto r^2/n_e(r) dP_e(r)/dr$.  
%\begin{equation}
%  M_{\rm{HE}} (r) = - \frac{r^2}{\mu_{\rm{gaz}}m_pn_e(r)}
%\end{equation}
L'étude statistique de ces profils thermodynamiques sera essentielle
pour caractériser les propriétés physique des amas, leur évolution en
redshift et la relation entre l'observable tSZ et la masse des amas.  


\subsection{L'équipe}
Le LP-SZ mobilise une équipe d'une vingtaine de chercheurs fortement
impliqués, comprennant des experts de l'effet SZ, qui ont joué un rôle
majeurs dans l'analyse des amas de galaxies dans \emph{Planck} et ont
mené des études tSZ à partir des observations de NIKA, des experts de
renommée mondiale des amas de galaxies observés en X, des experts de
simulations hydrodynamiques et des experts d'autres sondes d'amas
(observations optiques et radio). Par ailleurs, plusieurs membres ont
une connaissance très fine de l'instrument NIKA2. En tant que co-PI,
je coordonne les activités conjointement avec le PI.



%----------------------------------------------------------------------------------------
%
%
%
%
%
%                 ETUDES PILOTES ET Premiers resultats 
%
%
%
%----------------------------------------------------------------------------------------
\section{Les études pilotes et les premiers résultats}

\subsection{\'Etudes pilotes avec le prototype NIKA}
{\color{vert}\lipsum[5-7]}

\begin{figure}
  \centering
  \includegraphics[width=0.49\textwidth]{Figures/NIKA2-SZ/MACSJ0717_kSZ_map.pdf}
  \caption{Map of the kinetic SZ effect toward \mbox{MACS~J0717.5+3745} using NIKA pathfinder data, as discussed in Adam et al. (2017)$^{31}$.}
  \label{fig:nikanika2}
\end{figure}

\subsection{Premiers résultats de NIKA2}
{\color{vert}\lipsum[2-4]}

\begin{figure}
  \centering
  \includegraphics[width=0.49\textwidth]{Figures/NIKA2-SZ/Paper_NIKA2_Data.pdf}
  \includegraphics[width=0.44\textwidth]{Figures/NIKA2-SZ/Fig_PSZ2_G144_Scaling_relation.pdf}
  \caption{First SZ results with NIKA2. Left: The NIKA2 SZ map toward the galaxy cluster PSZ2-G0144.83+25.11. The high-resolution (20~arcsec) high-accuracy ($13.5\sigma $ measurement at peak) map covers the cluster from the core to the outskirts and reveals its morphology. An excess SZ signal is observed in the South-West region, indicating an overpressure within the intracluster medium (ICM). Right: Illustration of the impact of the ICM dynamics on the inner scatter of the SZ mass-observable relation. NIKA2 $Y_{500}$ estimates from the analysis with and without masking the over-pressure of PSZ2-G0144.83+25.11 are shown as a function of $M_{500}$, along with the cluster sample and the $Y_{500}-M_{500}$ scaling relation used in \emph{Planck} SZ-selected cluster count based cosmology analysis$^{13}$. These figures are extracted from Ruppin {\it et al.} (2018)$^{34}$. }
  \label{fig:nika2-sz}
\end{figure}



%----------------------------------------------------------------------------------------
%
%
%
%
%
%                 Prospectives ? 
%
%
%
%----------------------------------------------------------------------------------------
\section{Résultats attendus, implication pour la cosmologie et
  prospectives}


[Résultats principaux attendus] \\

[Amélioration des contraintes cosmologiques avec les amas de galaxies
  dasn Planck + pour les autres survey SZ]\\

[Etudes multisondes : contraintes sur le biais hydrostatic]\\

[synergie avec NOEMA]\\

[Calibration des futurs surveys optique/NIR]\\
