

NIKA2 photometric capabilities after the calibration, which we recall
comprizes the opacity correction as described in
Sect.~\ref{se:opacity_correction}, the KID inter-calibration as discussed in
Sect.~\ref{se:intercalibration} and the absolute calibration as
addressed in Sect.~\ref{se:calibration}, are assessed in this section using
secondary calibrators (planetary nebulae NGC7027, CRL2688, and
MWC349A).
%the two largest asteroids Ceres and Vesta were also observed. 
In Sect.~\ref{se:ref_flux_secondaries}, the flux density expectations
in NIKA2 bands for these calibrators are determined. Then, we evaluate
the flux density bias and statistical uncertainties for the baseline
calibration in Sect.~\ref{se:valid_baseline_cal} and for the
beam-variation hardened calibration in
Sect.~\ref{se:valid_photocorr_cal}. Finally, in
Sect.~\ref{se:cal_JFL}, we check the robustness of the calibration
results using an alternative calibration method that relies on
aperture photometry.


\subsection{Reference flux densities of the secondary calibrators {\color{blue} Jean-Fran\c cois}}
%\label{se:ref_flux_secondaries}

%
% LP: commented the obsolete text below (see Jean-Francois's email on
% August, 17)
%


%The asteroids Ceres and Vesta have been modeled by Muller et al (2014) in accounting for 
%size, shape, spin-properties, albedo, and thermal properties and in adjusting to PACS, SPIRE and HIFI observations
%of Herschel with an accuracy of 5\%. 
%Thomas Mueller has tabulated flux densities at different wavelengths, in particular at 1300$\mu$m, every five days
%until 2020 \footnote{http://www.iram.es/IRAMES/mainWiki/Continuum/Calibrators}.
%We have used the prediction at  1300$\mu$m made for  23rd february 2017
%and extrapolated it  to the central frequencies of the arrays in using a Rayleigh-Jeans
%spectrum expected for Ceres and Vesta. Their flux densities in
%Table~\ref{tab:fluxPred} are for this date. Over the five days of  run 9 (february 23 - 28), the
%flux densities  at 1300$\mu$m  have decreased by  3\% 
%for Ceres and  by 6\%  for Vesta in Muller's tables but we have not corrected for this effect in our analysis below.  
%
%The secondary calibrator MWC349A is a young Be star, part of a stellar binary system, surrounded by a disk. Its radio
%continuum emission originates in an ionized bipolar outflow (Tafoya et al 2004).
%MWC349 has been monitored with the  Plateau de Bure interferometer
%and shown to be only slightly angularly resolved, making it a point source for the 30-metre telescope. We have adopted
%its flux densities from this monitoring \footnote{http://www.iram.fr/IRAMFR/IS/IS2012/presentations/krips-fluxcalibration.pdf}.
%The secondary calibrator CRL2688 is an Asympotic Giant Branch star. Its radio continuum emission is mostly from circumstellar dust and
%is somewhat extended  (Knapp et al 1994).
%Its flux densities at $850\mu$m  and $450\mu$m  have been stable at the 5\% level as monitored by SCUBA2 (Dempsey et al 2013).
%We have extrapolated their flux densities to the central frequencies
%of the arrays with a power law of index $\alpha=-2.47$ derived from these SCUBA2 measurements.
%The secondary calibrator NGC7027 is a young, dusty, carbon rich Planetary Nebula with an ionized core.
%It is extended in the continuum and molecular lines (Bieging et al 1991) and  is not a point source for the telescope.
%Its  most recent flux densities are reported at $1100\mu$m  and $2000\mu$m by Hoare et al (1992). It has been reported
%to decrease by $\sim$ 0.145 percent/yr in the optically thin part of its spectrum above  $6$ GHz from VLA
%observations (Zijlstra, van Hoof \& Perley 2008, and Hafez et al, 2008) that makes these flux densities uncertain by 3.6\%
%at present. Its SED from cm wavelengths to optical is also presented in Hafez, Y.A. et al (2008).
%The flux densities adopted at the central frequencies of the arrays for these three calibrators are in Table~\ref{tab:fluxPred}.
%
%
%
%
%
%\begin{table}
%\centering
%\label{tab:fluxPred}
%\caption[]{Flux densities of calibrators at the reference frequencies
% of arrays for Run9 (computed for 2017-02-24T00:00) and Run10
%(computed for 2017-04-21T00:00)}
%\begin{tabular}{|l|r|r|}
%\hline
%\multicolumn{1}{|c}{}  & \multicolumn{2}{|c|}{flux densities (Jy)}  \\
%\hline
%         &    A1, A3      &  A2   \\
%       &  260 GHz    & 150 GHz \\
%\hline
%\multicolumn{1}{|c}{}  & \multicolumn{2}{|c|}{Run9}  \\
%
%\hline
%Uranus   &  39.10 & 15.14 \\
%Neptune  & 15.56 & 6.53 \\
%\hline
%\multicolumn{1}{|c}{}  & \multicolumn{2}{|c|}{Run10}  \\
%Uranus   &  37.95 & 14.69 \\%
%Neptune  & 15.56 & 6.67  \\
%\end{tabular}
%\label{tab:fluxPred}
%\end{table}

%Vesta    &   0.99   &  0.35 &   1.01  \\
%Ceres    &   0.89   &  0.31 &   0.91   \\
%MWC349   &   2.2    &  1.6  &   2.2   \\
%NGC7027  &   3.61   &  4.42 &   3.61  \\
%CRL2688  &   3.03   &  0.83 &   3.03  \\
%\hline
%\end{tabular}
%\label{tab:fluxPred}
%\end{table}




%\begin{table}
%\centering
%\label{tab:fluxPred}
%\caption[]{Reference flux densities of calibrators at central frequencies of arrays.}
%\begin{tabular}{|l|c|c|c|}
%\hline
%\multicolumn{1}{|c}{}  & \multicolumn{3}{|c|}{flux densities (Jy)}  \\
%\hline
%         &    A1      &  A2   &   A3    \\
 %        &  255GHz    & 152GHz  &  258GHz \\
%\hline
%Uranus   &  37.12   & 16.35 &  37.810 \\
%Neptune  &  15.28   &  6.84 &  15.58  \\
%Vesta    &   0.99   &  0.35 &   1.01  \\
%Ceres    &   0.89   &  0.31 &   0.91   \\
%MWC349   &   2.2    &  1.6  &   2.2   \\
%NGC7027  &   3.61   &  4.42 &   3.61  \\
%CRL2688  &   3.03   &  0.83 &   3.03  \\
%\hline
%\end{tabular}
%\label{tab:fluxPred}
%\end{table}



%%%%%%%%%%%%%%%%%%%%%%%%%%%%%%%%%%%%%%%%%%%%%%%%%%%%%%%%%%%%%%%%%%%%%%%%%%
%
%   cp in cal_JFL.tex

%\subsubsection{Reference flux densities of secondary calibrators}
\label{se:fluxSec}
%(revised by JFL Aug 2017 - must remove the old version in photometry-HA.tex)
% --> done by LP following intruction in Jean-Francois's email (to be checked)

The secondary calibrator MWC349A is a young Be star, part of a stellar
binary system, surrounded by a disk. Its radio continuum emission
originates in an ionized bipolar outflow \cite{Tafoya}.  MWC349A has
been monitored with the Plateau de Bure interferometer and VLA, and
shown to be stable in time and only slightly angularly resolved,
making it a point source for the 30-metre telescope. The SED of
MWC349A \cite{krips} is presented in Fig.~\ref{fig:Krips2017}. We have
computed its flux densities at the NIKA2 reference frequencies 150 and
260 GHz with $S_\nu = 1.16\pm0.01 \times
(\nu/100 \rm{GHz})^{0.60\pm0.01}$ provided by this
monitoring \cite{krips}.


%\noindent {\bf FM: MWC349A or MWC349 ?}\\

The secondary calibrator CRL2688 is an Asympotic Giant Branch
star. Its radio continuum emission is mostly from circumstellar dust
and is somewhat extended \cite{Knapp}.  Its flux densities at
$850\ \mu$m and $450 \ \mu$m have been stable at the 5\% level as
monitored by SCUBA2 in 2011-2012
\cite{Dempsey}. We have extrapolated their flux densities to  150 and 260 GHz
with the power law $S_{\nu} \propto \nu^{\alpha}$ and index
$\alpha=2.44\pm0.18$ derived from their SCUBA2 measurements.

\begin{table}[t]
\begin{center}    
  \begin{threeparttable}

\begin{tabular}{|l|c|c|c|l|}
\hline
\multicolumn{1}{|c}{}  & \multicolumn{3}{|c}{flux  densities (Jy)} & \multicolumn{1}{|c|}{}  \\
\hline
         &    A1 \& A3       &  A2             &          &   Ref. \\
         &  260 GHz           &  150 GHz         & $\alpha^1$ &      \\
\hline
MWC349A   &   $2.06\pm0.04$  &  $1.48\pm0.02$ &  $+0.60\pm0.01$      &  PdB \cite{krips}    \\
NGC7027  &   $3.46\pm0.11$   &  $4.26\pm0.24$  &  $-0.34\pm0.10$     &  Hoare et al 1992 \cite{Hoare}      \\
CRL2688  &   $2.91\pm0.23$   &  $0.76\pm0.14$  &  $+2.44\pm0.18$     &  Dempsey et al 2013  \cite{Dempsey} \\
\hline
\end{tabular}
  \begin{tablenotes}
{\small     
  \item[$^1$]  Spectral index is defined as $S_{\nu} \propto \nu^{\alpha}$. 
}
  \end{tablenotes}
\end{threeparttable}
\caption[]{Reference flux densities of secondary calibrators at the NIKA2 reference frequencies 150 and 260 GHz. Uncertainties of flux densities extrapolated
at 150 and 260 GHz include contribution of the uncertainty on $\alpha$.}
\label{tab:flux_ref_sec}
\end{center}
\end{table}
% cahier I p. 128 et 129.

The secondary calibrator NGC7027 is a young, dusty, carbon rich
Planetary Nebula with an ionized core.  It is extended in the
continuum and molecular lines (Bieging et al 1991), and is not a point
source for the 30-metre telescope.  Its most recent flux densities are
reported at $1100\mu$m and $2000\mu$m by Hoare et al (1992). It has
been reported to decrease by $\sim$ 0.145 percent/yr in the optically
thin part of its spectrum above $6$ GHz from VLA observations
(Zijlstra, van Hoof \& Perley 2008, and Hafez et al, 2008) that makes
these flux densities uncertain by 3.6\% currently. Its SED from cm
wavelengths to optical is also presented in Hafez, Y.A. et al (2008).
Its flux densities have been extrapolated to 150 and 260 GHz and the
modelled decrease since 1992 included.

All these extrapolated flux densities are in
Table~\ref{tab:flux_ref_sec}.

\begin{figure}[h]
\begin{center}
  \includegraphics[clip, angle=0, scale=0.4]{Figures/MWC_349_pap-flux.eps}
  \caption{SED of MWC349A from its flux density monitoring at PdB and VLA \cite{krips}.
  Symbols are for primary calibrators used (Uranus, Neptune and Mars).}
\label{fig:Krips2017}
\end{center}
\end{figure}
