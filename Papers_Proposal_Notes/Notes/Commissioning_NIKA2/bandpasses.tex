\begin{figure}[ht] % Inline image example
\begin{center}
\includegraphics[width=0.9\textwidth]{Figures/SpectralBands/atm_transmission.pdf}
\caption{Spectral transmission of the the three NIKA2 arrays as a function of frequency in GHz. For illustration we also plot the ATM atmospheric model for different values of pwv. \label{spectralband}}
\end{center}
\end{figure}


\begin{figure}[ht] % Inline image example
\begin{center}
\includegraphics[width=\textwidth]{Figures/SpectralBands/opacity_ratio_vs_tau1.pdf}
\caption{Expected atmospheric opacity ratio of the 2 and 1 mm channels as function of the opacity at 1 mm. \label{thopacities}}
\end{center}
\end{figure}



The NIKA2 spectral bands were measured in the laboratory using a Martin Pupplet interferometer.
Both arrays and filter bands were considered in the measurements. Theser were obtained from the difference of two black-bodies, hence they include a a $\nu^2$ RJ term. During the commissioning in Run 5 array 2 was replaced. The new array has a different spectral transmission. Figure shows the spectral transmissions for the three arrays. Notice that array A2 was replaced by a new on in N2R5 and that the spectral transmissions are not the same (red and cyan lines in the figure).


\begin{table}[h]
\caption{Spectral transmission characteristics for the NIKA2 arrays.%
\label{nika2runs}}
\begin{tabular}{|c|c|c|c|c|}
\hline 
  &     A1  &  A3 &  A2 2015 & A2 2016 \\ 
\hline 
Central Frequency [GHz] &   255.5  & 257.8   &   147.7  & 151.6 \\  
Bandwidth [GHz]         &   47.8   & 45,7    &   41.8   & 42.1 \\
\hline 
\end{tabular} 
\end{table} 


%What actually matters more than the ``central frequency'' that depends on many
%assumptions and definitions are the bandpasses. We should make available in a
%.fits file, clearly, our bandpasses to avoid future misunderstanding and propagation of
%false numbers. Official values should be 150 and 260~GHz. We should also clearly
%state that these measured bandpasses were done with the difference of two
%black-bodies, hence they include a $\nu^2$ RJ term.\\

Using the NIKA2 bandpasses for N2R9, we can integrate the ATM atmospheric model to compute the expected ration between the atmospheric opacity for the two NIKA2 channels. This shown in Figure~\ref{thopacities} where we present the atmospheric opacity ratio of the 2 and 1 mm channels as a function of the opacity for the 1 mm one.

