
%\chapter*{Introduction}

La cosmologie a vu émerger un modèle cohérent et efficace
pour décrire essentiellement toutes les observations cosmologiques,
tout en laissant de larges zones d'ombre quant à son interprétation
théorique~\reference{??}. Ce \emph{modèle standard de la cosmologie} est
fondé sur le paradigme de perturbations primordiales gaussiennes,
adiabatiques et quasi-invariantes d'échelle générées à l'issue d'une
phase d'inflation et croissant par instabilité gravitationnelle dans
un Univers en expansion~\reference{??}. Son contenu en énergie est
dominé par l'\emph{énergie noire}, de nature encore inconnue mais
compatible avec une constante cosmologique $\Lambda$, tandis
que la matière existe principalement sous la forme de
\emph{matière noire froide} (CDM) -- la forme de matière telle qu'elle
est décrite par le modèle standard de la physique des particules ne
constituant que quelques pourcents du contenu
total~\reference{??}. L'Univers est également empli d'un fond de
neutrinos~\reference{??} qui, en se basant sur les indications d'une
masse non-nulle fournies par les expériences en laboratoire~\reference{?},
contribue pour une petite fraction (quelques dixièmes de pourcent) à
la densité de matière sombre.
% Cette contribution pourrait être même plus
% importante, s'il existe un neutrino non-standard, dit \emph{stérile}, telle
% la particule invoquée pour expliquer les anomalies observées
% par les expériences mesurant certaines oscillations de
% neutrinos.

La mesure du fond diffus cosmologique (CMB par la suite), rayonnement
relique, émis lors du découplage matière-radiation environs 380\,000
ans après le \emph{Big-Bang}, est l'une des observations fondatrices
de ce modèle cosmologique. L'étude des anisotropies de la température
du CMB et de la polarisation, e.g. \reference{Ma\&Bertshinger} pour
une approche technique, donne accès à une riche phénoménologie,
e. g.~\reference{Wayne Hu} pour une approche imagée. En effet, les
anisotropies du CMB nous fournissent une image unique des
perturbations de la métrique dans
l'Univers primordial~\reference{White, Silk, Bond, Peebles}. Ces perturbations
apparaissent à l'issue d'une phase d'expansion exponentielle de
l'Univers, l'inflation~\reference{Guth}. Une des signatures de
l'inflation est la production de modes tensorielles de perturbations,
généralement associés à un fond d'ondes gravitationnelles
primordiales, et qui auraient laissé une empreinte dans la
polarisation du CMB~\reference{??}. Celle-ci est décrite par deux
quantités scalaires qui diffèrent par leurs propriétés de parité, le
mode E (pair) et le mode B (impair), définies de sorte que le mode B
primordial soit uniquement généré par les modes
tensorielles~\reference{Lewis}. Sa mesure offrirait donc une
opportunité unique d'accéder à la physique de l'inflation. Quant à la
mesure du mode E, elle est complémentaire de celle de la température
pour contraindre la cosmologie~\reference{Galli} et est, de plus, très sensible
à la physique de la réionisation de l'univers~\reference{??}.

Combinée à une sonde des distances cosmologiques (e.g. les mesures
directes de la constante de Hubble ou les relevés de supernovae de
type Ia), l'étude statistique des anisotropies du CMB nous permet de
contraindre les paramètres cosmologiques décrivant les conditions
initiales qui prévalaient dans l'univers primordial, le contenu et la
géométrie de l'univers actuel~\reference{White, Silk, Bond, Peebles}.
Une approche élégante consiste à combiner le CMB avec les
oscillations acoustiques des baryons (BAO), qui sont l'empreinte
laissée par les oscillations baryoniques au moment du découplage sur
la distribution de la matière~\reference{}. L'échelle angulaire des plus
grandes perturbations de densité ayant oscillées après l'égalité
matière-rayonnement, celles qui sont à l'origine des pics acoustiques
observés dans le spectre de puissance du CMB, est imprimée dans la
distribution des grandes structures dans l'Univers beaucoup plus
récent. Dans cette approche, l'étalon de distance est directement
connecté à une observable du CMB. 


L'étude des anisotropies primordiales n'épuise pas l'information
cosmologique du CMB. Après le découplage, le rayonnement CMB est
faiblement perturbé par interaction électromagnétique ou
gravitationnelle avec la matière rencontrée sur la ``ligne de visée''
depuis la surface de dernière diffusion jusqu'à nous. Les empreintes
laissées sur le CMB par ces processus physiques variés sont regroupées
sous le termes d'anisotropies secondaires du CMB~\reference{Aghanim}. Celles-ci
constituent d'intéressants traceurs des grandes structures de
l'Univers, sensibles à leur évolution récente. Une conséquence notable
est que, via l'exploitation des anisotropies secondaires, le CMB
devient une sonde autonome pour contraindre l'ensemble des paramètres
cosmologiques. Parmi les anisotropies secondaires, deux représentent
de puissantes sondes cosmologiques et retiendront notre attention dasn
ce manuscrit : l'effet de lentille gravitationnelle sur le CMB et
l'effet Sunyaev-Zel'dovich.

La première, l'effet de lentille sur le CMB, revêt une importance
particulière puisqu'en sus d'un intérêt cosmologique propre, sa mesure
a également un fort impact sur l'exploitation des anisotropies
primordiales. Cet effet vient faiblement déformer les anisotropies de
température et la polarisation du CMB, laissant plusieurs signatures
remarquables: i) il induit un mode B secondaire de polarisation qui
domine le mode B primaire, et qu'il va donc être nécessaire de
soustraire pour contraindre les scenarii d'inflation, ii) il perturbe
les propriétés statistiques des cartes de la température et de la
polarisation, en modifiant les spectres de puissance angulaires et en
introduisant des non-gaussianités~\reference{Lewis\&Challinor}. C'est par l'étude de ces
modifications statistiques qu'il est possible de mesurer l'effet de
lentille en reconstruisant le champ du potentiel gravitationnel
sous-jacent~\reference{Hu}. Cette mesure nous donne accès à la distribution de masse
de l'univers, de la surface de dernière diffusion jusqu'à nous, et ce
avec une sensibilité maximale à des décalages vers le rouge de l'ordre
de 2.5 et pour des structures évoluant encore dans un régime linéaire bien
modélisable~\reference{}. Ces caractéristiques en font une sonde bien adaptée pour
contraindre l'échelle absolue de la masse des
neutrinos~\reference{perotto2006}. Par ailleurs, une mesure précise de
la carte du potentiel gravitationnel permet d'accéder aux observables
primordiales de la polarisation du CMB, par application d'une
procédure d'inversion de l'effet de lentille (\emph{delensing})~\reference{Smith}.\\


L'effet Sunyaev-Zeldovich (SZ) décrit la diffusion Compton inverse du
rayonnement CMB sur les électrons énergétiques du gaz chaud ionisé
présent dans les grandes structures de l'Univers, et en particulier
dans les plus grands objets liés gravitationnellement, les amas de
galaxies. Il s'agit d'une distortion spectrale du CMB bien
modélisée~\reference{Sunyaev\&Zeldovich}, diminuant son intensité à
des fréquences inférieures à 217\,GHz au profit des fréquences
supérieures. L'effet SZ permet donc théroriquement de détecter les
amas de galaxies indépendamment de leur \emph{redshift}, nous donnant
accès aux premiers objets formés. De plus, l'amplitude de l'effet SZ,
donnée par le paramètre de Compton, est une mesure de la pression
intégrée le long de la ligne de visée du gaz intra-amas, qui forme
l'essentiel du contenu baryonique des amas de galaxies. 
[masse totale.] 


[PLANCK ]Le satellite de l'agence spatiale européenne (ESA) \emph{Planck} a
mesuré les anisotropies en température et en polarisation sur tout le
ciel avec une résolution angulaire atteignant cinq minutes d'arc et
une sensibilité  
a
fourni à la communauté la mesure définitive des anisotropies primaires
de température  
[RESULTATS DE PLANCK]. \\

[TENSION->NIKA2]\\


[NIKA2] \\


[EUCLID]  \\

[COSMO AVEC LES AMAS DE GALAXIES] \\



Ce manuscrit s'articule en trois parties. Dans la première partie, je
donne un rapide coup de projecteur à deux résultats marquants obtenus
avec PLANCK sur l'effet de lentille gravitationnelle sur le CMB. Cette
partie me permet d'introduire des notions de cosmologie générale et de
présenter le domaine du CMB, avant de me focaliser brièvement sur l'effet de
lentille sur le CMB. La deuxième partie est consacrée à la mise en
service de l'expérience NIKA2 et à l'évaluation de ses performances au
télescope de 30 mètres de l'IRAM. Cette partie, qui donne son titre au
manuscrit sera plus étoffée. On y trouvera une rapide présentation de
l'expérience NIKA2, une description de la procédure développée pour la
calibration, ainsi que la méthodologie et les résultats de
l'évaluation des performances. La dernière partie est consacrée aux
développements récents de mon activité de recherche et à mes projets à
court et moyen termes. Elle comporte un volet sur
le programme cosmologique de NIKA2 à partir de l'échantillon d'amas de
galaxies observés via l'effet SZ. J'y souligne aussi les synergies
avec d'autres sondes des amas de galaxies, en particulier le
rayonnement X. Ce volet, pourrait faire l'objet d'un sujet de thèse à
brève échéance. Dans un deuxième volet, je présente l'expérience
EUCLID et la préparation à l'exploitation des amas de galaxies
pour la cosmologie. 


