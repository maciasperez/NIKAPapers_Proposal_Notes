
%\documentclass{article}
\documentclass[12pt]{article}
\usepackage[utf8]{inputenc}
\usepackage[english]{babel}

\title{Referee report: NIKA polarization observations of the Crab nebula}
%\author{Alessia Ritacco}
%\date{March 2018}

\begin{document}
\maketitle
Dear Referee,\\
thank you very much for your attentive report. 
We went deeper in our analysis considering all the suggestions you have recommended. 
The paper has been entirely reviewed.
We also have collaborated with the Planck team to understand the behaviour of the SED shown by Planck high frequency data. These latter have been reanalyzed using Planck 2018 data that will be soon published. Please find below the answer (in boldface) to your questions. Most of the paper has been rewritten, only section 2 had tiny modifications. \\
Best regards,\\
A. Ritacco, J.F. Macías-Pérez, N. Ponthieu


\begin{enumerate}
    \item 
The paper presents an analysis of NIKA observations of the Crab
Nebula, at 150 GHz. The analysis includes measurements of the
integrated intensity, polarized intensity, and net polarization angle.
These measurements are combined with data from other instruments
(especially WMAP and Planck), and used to constrain simple empirical
models for the SED in intensity and polarized intensity in the
microwave regime.

The Crab Nebula is indeed an important polarized calibration source
for CMB instruments, and efforts to measure its properties and assess
consistency between different instruments across a broad frequency
range is very valuable to the community.

For a paper that aims to contribute to the delicate science of
polarization calibration, I would have expected more attention to be
paid to the accounting of systematic errors, and to astrometric
accuracy. In addition, the section on SEDs should have been
straight-forward but instead is marred by confusing discussion of the
models and apparent mistakes in the fitting.

In what follows, I discuss the most serious technical issues. I have
additional concerns with the interpretation of the results, but it
does not seem worth discussing these until the numerical reductions
have been revisited.


1. The handling of noise bias, and more generally the discussion of
errors and uncertainties, is not sufficiently sophisticated to meet the
stated goals of this paper (namely that the measurements might be used
for calibration of other instruments). There are two places where
this is most apparent:


In section 3, an effort is made to make statements about the spatially
integrated flux, polarized flux, and polarization angle of the nebula.
This information is potentially valuable for calibration of CMB
instruments, with larger beams. Ultimately the authors settle on a
method where pixels are excluded from the average if the polarized
flux in the pixel is below 3 times the noise level. This is
problematic for several reasons:

(a) Because the polarized intensity is distributed differently than
the unpolarized intensity, cutting on polarized brightness will bias
the measurement towards parts of the nebula with a pixels with a
higher polarization degree (p). This will inflate the "average" p
relative to what an experiment with a larger beam would be able to
observe.

(b) The reason given for the exclusion is to avoid "noise bias". The
noise bias is described earlier in the paper and is indeed important
when converting low S/N measurements of I,Q,U into estimates of p and
polarization angle (psi). But to compute integrated p and psi values
correctly, one should first obtain integrated values of I, Q, and U;
these will have quite high S/N, compared to individual pixel S/N, and
noise bias will not be a concern.

(c) This exclusion is hardly a minor technical point, since it changes
the inferred polarization angle by 3.5 degrees and the polarization
fraction by more than 1 percentage point. For the polarization angle,
at least, this change is much larger than the total uncertainty.

Although the authors perform this computation both with and without
the exclusion of S/N \textless 3 points, the values they carry forward, and
use in summary sentences and plots, are the biased ones (using S/N \textgreater 3
only). One suspects that the S/N \textgreater 3 results were ultimately chosen
because they agree better with the results from other experiments
(WMAP and Planck). The authors mention the need for an "additional
uncertainty at a level of 1\% on p and 4 deg on psi", but this
uncertainty is not mentioned again; it is not included in tables,
plots, or summary sentences.


Secondly, the authors often quote numbers with very small statistical
errors. For example, for one of the averages in section 3, the mean
polarization angle has an error of 0.0004 degrees. This suggests
either a numerical mistake, or that the map noise has been
treated as white when in fact it has significant correlated structure
on angular scales larger than the pixel. In the latter case, a
profitable approach is to check for consistency between different
subsets of the data (e.g., look at two co-adds of 12 maps instead of
the one co-add of 24 maps).

\textbf{In this section we have probably underevaluated the uncertainty related to the polarization estimates. We therefore accounted for your suggestions using jack-knife maps to estimate the uncertainty in addition to the statistical one. Moreover we also consider the difference in terms of polarization estimates between the two co-added maps, before and after leakage correction, which we now include in the systematic errors. The calibration error is also considered in the estimation of the flux densities. As a consequence, this section has been completely reviewed. We have used aperture photometry methods to compute flux densities and polarization degree and angle accounting for all pixels. We agree with the fact that this the only way to compare our results with low resolution experiments.}

\item  Basic errors in astrometry and coordinate handling.

The position of the Crab Pulsar is given, to sufficient precision, in
both the Introduction and the Conclusion. But the position of the
pulsar is not plotted correctly in any of the images that purport to
do so. Rather, the position appears to have been truncated to 0.01
degrees prior to plotting.

The position angle (PA) conversion between galactic and equatorial
coordinates is not done correctly. In tables 1 and 2 we find that the
Equatorial and Galactic PA differ by values ranging between 220 to 245
degrees, even though all of these points lie in a very small region of
the sky.

These issues undermine a reader's confidence that the authors have
performed the photometry as described, or that they understand the
tools used to convert between coordinate systems to a sufficient
degree that we can trust their results.

\textbf{The position of the Crab nebula was indeed plotted, by mistake, in a wrong position on the maps. The issue occurred in our tools when we have tried to zoom the plot in the inner part. This has been now solved. All the values in the tables have been recalculated. For the conversion between Equatorial and Galactic coordinates the main issue was that the polarization angle changes drastically sign for equatorial angle of 148deg. We have checked our conversion methods using different tools finding consistent values.}

\item Models are not explained clearly, and fits contain obvious errors.

(a) The intensity model described in equations (6) and (7) is not
completely described. It appears to be discontinuous, through 100
GHz, and the best-fit values are only given for 2 of the 4 model
parameters. Fig. 6 shows a very reasonable broken power-law fit to
the data, but the description in the text is inadequate and confusing.

(b) In 4.2, models are fit to the polarized intensity from WMAP and
Planck. It is alleged that only data below 100 GHz are used; this
raises the question of whether or not the Planck measurement at 100
GHz is included. The best-fit parameters for Planck, stated in
equation (9) and plotted in Fig. 7, do not appear to be the best fit
to these data. Indeed, if I take the Planck measurements and fit a
simple power-law, I find a very different best-fit line: A = 91.8 and
beta = -0.38.

These errors and/or omissions make it difficult to take any of the
model fitting seriously; they undermine the conclusions of the paper.

\textbf{Since Planck HFI data have been reanalysed because of a flux loss observed in previous results, this part has been totally reviewed. The fits have been recalculated using all the data sets. The polarization spectral index is found to be consistent with the index estimated in total intensity. The models fit both low and high resolution experiments observations. Hopefully the section is clearer than before.}

\end{enumerate}

\end{document}
