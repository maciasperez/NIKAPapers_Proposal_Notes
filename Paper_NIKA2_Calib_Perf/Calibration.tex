%----------------------------------------------------------------------------------------
%	8./ Calibration
%----------------------------------------------------------------------------------------
%\section{Calibration}
%\label{se:calibration}

We present the calibration of the absolute scale of the flux densities
for the NIKA2 instrument in this section using Uranus as the main
primary calibrator. Practically, at the stage of the FOV
reconstruction (see Sect.~\ref{se:geometry}), an absolute
coefficient factor per detectors is derived using a \bm\ scan of
Uranus. This step realizes also the inter-calibration of all the KID,
as the coefficient factors give the KID gains. Secondly, the flux
density absolute scale is further refined by monitoring the primary
calibrator all along the observation campaign to estimate an
absolute calibration correction.

We have evidenced a daily variation of the absolute calibration
coefficients as a function of the observation date, which is related
to temperature-induced variation of the beam size. If left uncorrected, this variation
induces a sizable increase of the calibration uncertainties. To
overcome this issue, we primarily flag the most impacted observation
dates and exclude the observations acquired during these periods. This
conservative approach constitutes the baseline calibration method,
which is further used for the performance assessment. For cross-check,
we also proposed an alternative method relying on a photometric
correction depending on the beam size. Both approaches require an
accurate monitoring of the beam size as a function of the observation
date.


First we describe the method for the absolute calibration in
Sect.~\ref{se:calibration_method}, then we present the
inter-calibration and the flat fields in
Sect.~\ref{se:flat_field}. The temperature-induced variation and the
beam size monitoring are then discussed in
Sect.~\ref{se:beam_variation}. Finally, the baseline calibration is
presented in Sect.~\ref{se:baseline_calibration} and the calibration
with a photometric correction in
Sect.~\ref{se:photometric_correction}.  



%---------------------------------------------------------------------
%	Method
%---------------------------------------------------------------------

\subsection{Absolute calibration procedure and photometric system}
\label{se:calibration_method}

We detail here the procedure for the calibrating the absolute scale of
the flux density and the chosen photometric system.

\subsubsection{Photometric system}

The main primary calibrators of NIKA2 are the giant planets Uranus and
Neptune. The latter is used when the former is not visible in the most
stable observing conditions. The flux density expectations of the
primary calibrators are derived in Sect.~\ref{se:flux_theo}. 

We parametrize the primary calibrator flux density
$S_{\rm{c}}(\nu) = S_{\rm{c}}(\nu_0)\, f(\nu/\nu_{0})$, where $f(\nu/\nu_{0})$
encloses the spectral dependence, 
as a function of a reference frequency $\nu_{0}$ that we choose
arbitrarily to be: $\nu_{0} = 150$~GHz for the 2mm array and
$\nu_{0}= 260$~GHz for both 1mm arrays. Projecting the raw data (in
units of the KID resonance frequency shift or $\rm Hz$) of a
calibrator, we model the raw map with a fixed-width Gaussian
\begin{equation}
  R_{\rm{c}}(\theta, \phi)  = \frac{A_{\rm{c}}}{2 \pi \sigma_{0}^{2}}
e^{-\frac{\theta^{2}}{2\sigma_{0}^{2}}},
\end{equation}
where $\sigma_{0}$ is derived from the
reference FWHM, labelled FWHM$_{0}$, which are $12.5''$ for the 1mm
arrays and $18.5''$ for the 2mm array. These values have
been chosen sizably larger than the main beam values, as reported in
Sect.~\ref{se:beam}, to account for a fraction of the signal smeared 
in the first error beam and first side lobes.
Both the reference frequency and FWHM, $\nu_0$ and FWHM$_{0}$, define our reference photometric system, as
summarized in Table~\ref{tab:definitions}.

\begin{table}[!htbp]
  \begin{center}
    \caption{NIKA2 reference frequencies and FWHM}
    \begin{tabular}{lcc}
      \hline\hline
      \noalign{\smallskip}
      & 1 mm & 2 mm \\
      \noalign{\smallskip}
      \hline
      \noalign{\smallskip}
      Reference frequency $\nu_{0}$ & 260 GHz & 150 GHz \\
      Reference FWHM  FWHM$_{0}$    & 12.5'' & 18.5'' \\
      \noalign{\smallskip}
      \hline
    \end{tabular}
  \end{center}
  \label{tab:definitions}
\end{table}

The absolute calibration coefficients are estimated as the ratio of
the flux density expectations at the reference frequency
$S_{\rm{c}}(\nu_0)$ and the amplitude estimate of the fixed-width reference
FWHM Gaussian $A_{\rm{c}}$. For any point-like source $\rm{s}$, the map
\begin{equation}
  M_{\rm{s}}(\theta, \phi) = \frac{S_{\rm{c}} (\nu_{0})}{A_{\rm{c}}}
  R_{\rm{s}}(\theta,\phi),
\end{equation}
where $R_{\rm{s}}(\theta,\phi)$ is the raw data projection, is calibrated in Jy/FWHM$_{0}$
beam. The flux density estimate for the source is then:
\begin{equation}
S(\nu_{0})  = \frac{S_{\rm{c}}(\nu_{0})}{A_{\rm{c}}} \, A_{\rm{s}},
\label{eq:pointsourcephot}
\end{equation}
where $A_{\rm{s}}$ is the amplitude estimate of a FWHM$_0$ Gaussian.

\subsubsection{Color correction}

The flux density estimate $S(\nu_{0})$ gives the
flux of the source at the reference frequency only if the source has
the same spectral index as the calibrator. In general, to retrieve the
flux of the source at the reference frequency, a color correction
$C_{\rm{s}}$ has to be applied
\begin{equation}
S_{\rm{s}}(\nu_{0}) = S(\nu_{0})  C_{\rm{s}}(\nu_{0}, \alpha_{\rm{s}}),
\end{equation}
which depends on the reference frequency $\nu_{0}$ and the source
spectral index $\alpha_{\rm{s}}$.
Neglecting the effect of the atmosphere on NIKA2 transmission, we compute the color correction
factor for target sources of spectral indices $\alpha_{\rm{s}}$ that are
different from Uranus using
\begin{equation}
  C_{\rm{s}}(\nu_{0}, \alpha_{\rm{s}}) = \frac{\int_{0}^{+\infty} (\nu/\nu_0)^{1.6} ~T({\nu}) d\nu}{ \int_{0}^{+\infty} (\nu
    /\nu_0)^{\alpha_S} ~ T({\nu}) d\nu}.
\end{equation}

Color correction factors for eight values of $\alpha_{\rm{s}}$, and in particular
for $\alpha_{\rm{s}}= 0.6$ which is the spectral index of MWC349, are
given in Table~\ref{tab:mod}. 

\begin{table*}[!h]
\caption{Color correction factor for a target source  $S \propto \nu ^{\alpha_{\rm{s}}}$}
\label{tab:mod}
\centering 
\begin{tabular}{l| c c c c c c c c}
\hline\hline
\noalign{\smallskip}
Array  & \multicolumn{8}{c}{$\alpha_{\rm{s}}$} \\
\noalign{\smallskip}
\hline
          &  -2 &  -1    &    0  & + 0.6 & +1  &  +2  & +3 & +4  \\
%            \noalign{\smallskip}
            \hline
%            \noalign{\smallskip}
          A1   & 0.876  &  0.916   &   0.951  & 0.969 &  0.981   &  1.005  &    1.024  &  1.037   \\
          A2   & 0.945  &  0.972   &   0.990  & 0.996 &  0.998   &  0.997  &    0.986  &  0.966      \\ 
          A3   & 0.907  &  0.940   &   0.967  & 0.980 &  0.987   &  1.001  &    1.009  &  1.011     \\
            \noalign{\smallskip}
            \hline
\multicolumn{8}{c}{Note : Uranus/Moreno model used for Uranus in this
  Table.}
\end{tabular}
\end{table*}


\subsubsection{Diffuse source}
For aperture photometry or for extended sources, the map calibrated in
Jy/FWHM$_{0}$ must be converted in a map in Jy/sr, that is corrected
with the reference beam efficiency, defined as the ratio of the solid
angle enclosed in the reference fixed-width Gaussian beam and the
solid angle of the total beam. The total solid angle estimates are
presented in Sect.~\ref{se:beam_efficiency}. The reference beam efficiencies
are given in Table~\ref{tab:reference_beam_efficiency}.


\begin{table}[!thbp]
  \caption[]{Reference beam efficiencies}
  \label{tab:reference_beam_efficiency}
  \centering    
  \begin{tabular}{lrrrr}
    \hline\hline
    \noalign{\smallskip}
    & Array 1 & Array 3  & Array 1\&3 & Array 2 \\
    \noalign{\smallskip}
    \hline
    \noalign{\smallskip}
    FWHM$_{0}$ [arcsec]          &  $12.5$   &  $12.5$  &   $12.5$  &   $18.5$  \\
    B.E$_{0}$\tablefootmark{a}\hspace{3mm}  [\% ] & $70 \pm 4$ & $72 \pm 4$ & $70 \pm 4$ & $85 \pm 3$ \\
    \hline
  \end{tabular}
  \tablefoot{ \\
    \tablefoottext{a}{Reference Beam Efficiency, estimated as the ratio between the reference FWHM beam power and the total beam power up to a radius of 180 arcsec} 
  }
\end{table}


\subsubsection{Practical calibration}



%---------------------------------------------------------------------
%	INTERCALIBRATION
%---------------------------------------------------------------------
\subsection{Relative calibration \& flat fields}
\label{se:flat_field}




%---------------------------------------------------------------------
%	TEMPERATURE-INDUCED VARIATION
%---------------------------------------------------------------------
\subsection{Temperature-induced variation}
\label{se:beam_variation}



%---------------------------------------------------------------------
%	BASELINE CALIBRATION
%---------------------------------------------------------------------
\subsection{Baseline calibration}
\label{se:baseline_calibration}



%---------------------------------------------------------------------
%	PHOTOMETRIC CORRECTION
%---------------------------------------------------------------------
\subsection{Photometric correction}
\label{se:photometric_correction}
