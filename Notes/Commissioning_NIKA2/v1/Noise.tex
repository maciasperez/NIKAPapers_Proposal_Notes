%%%%%%%%%%%%%%%%%%%%%%%%%%%%%%%%%%%%%%%%%%%%%%%%%%%%%%%%%%%%%%%%%%%%%%%%%%%%%%%%%%%%%%%%%
%
%
%       SECTION: NOISE DESCRIPTION
%
%
%%%%%%%%%%%%%%%%%%%%%%%%%%%%%%%%%%%%%%%%%%%%%%%%%%%%%%%%%%%%%%%%%%%%%%%%%%%%%%%%%%%%%%%%%


%----------------------------------------------------------------------
%
%    
%
%---------------------------------------------------------------------
\section{Time domain noise properties {\color{YellowGreen} Juan}}
\label{se:noise}
In this Section we describe the main properties of the time domain noise for
each of the NIKA2 arrays considering observations of faint sources in good
atmospheric conditions. For this purpose we have used several decorrelation
methods trying to identify possible multiple components in the noise:

\begin{itemize}
\item {\bf Common mode decorrelation (CM)}. We search for a common mode template
  using all detectors of the same array. To avoid bias from bad detectors we
  consider the median common mode.

\item {\bf Principal Component Analysis (PCA)}. For each NIKA2 array
  independently we decompose the covariance matrix in principal components. From
  those we derive up to 10 independent templates corresponding to the largest
  eigenvalues.

\item {\bf Most correlated pixels (\cmoneb)}. For each detector in a given array we
  identify the detectors that are most correlated to it (a minimum of
  14). Using those detectors we compute a common mode as in method CM.
\end{itemize}

Notice that in the following the atmospheric signal will be considered simply as
correlated sky noise. Furthermore, we assume astrophysical signals are
negligible in time domain and they do not dominate the observed correlation
features and power spectra discussed below. Finally, in here the decorrelation
methods are applied to the full data set for each NIKA2 scan and are not
optimized to preserve astrophysical signals.
 
\subsection{Detector-Detector correlation matrix}

In Figure~\ref{corrmatrix} we show the noise correlation matrices for the N2R9
faint source scan {\bf 20170228s151} for the raw data and for the decorrelated
data obtained by using the methods described above. As expected the raw data
noise correlation is dominated by atmospheric noise and we observe full
correlation between detectors but for badly behaving detectors which are removed
from the analysis. Significant residual correlation and anti-correlation is
observed after CM decorrelation. This is both due to spatial changes in the
atmospheric emission (overall residuals) and to instrumental and electronic
noise characteristics (correlation blocks that can be associated to electronic
boxes).  The PCA decorrelation leads to approximately block-diagonal correlation
matrices. These observed blocks in the correlation matrix can be associated to
first order to the different sub-bands in each of the electronic boxes. In the
case of the MCP decorrelation, for which only those pixels highly correlated to
the pixel of interest are used, we observe that the correlation matrix is more
diagonal as in the two other derivations of the common mode.


\begin{figure}[ht] % Inline image example
\begin{center}
\includegraphics[width=0.3\textwidth]{Figures/NoiseTests/corrmat_TOI_array_1_20170228s151.pdf}
\includegraphics[width=0.3\textwidth]{Figures/NoiseTests/corrmat_TOI_array_2_20170228s151.pdf}
\includegraphics[width=0.3\textwidth]{Figures/NoiseTests/corrmat_TOI_array_3_20170228s151.pdf}
\includegraphics[width=0.3\textwidth]{Figures/NoiseTests/corrmat_TOI_CM_array_1_20170228s151.pdf}
\includegraphics[width=0.3\textwidth]{Figures/NoiseTests/corrmat_TOI_CM_array_2_20170228s151.pdf}
\includegraphics[width=0.3\textwidth]{Figures/NoiseTests/corrmat_TOI_CM_array_3_20170228s151.pdf}
\includegraphics[width=0.3\textwidth]{Figures/NoiseTests/corrmat_TOI_PCA_array_1_20170228s151.pdf}
\includegraphics[width=0.3\textwidth]{Figures/NoiseTests/corrmat_TOI_PCA_array_2_20170228s151.pdf}
\includegraphics[width=0.3\textwidth]{Figures/NoiseTests/corrmat_TOI_PCA_array_3_20170228s151.pdf}
\includegraphics[width=0.3\textwidth]{Figures/NoiseTests/corrmat_TOI_BCP_array_1_20170228s151.pdf}
\includegraphics[width=0.3\textwidth]{Figures/NoiseTests/corrmat_TOI_BCP_array_2_20170228s151.pdf}
\includegraphics[width=0.3\textwidth]{Figures/NoiseTests/corrmat_TOI_BCP_array_3_20170228s151.pdf}
\end{center}
\caption[KID-to-KID correlation matrices]{From left to right correlation matrices for the three NIKA2 arrays (A1, A2, and A3) for scan 20170228s150. From top to bottom we present the correlation of the raw data, after CM, PCA and MCP decorrelation methods. \label{corrmatrix}}
\end{figure}



\subsection{RMS noise and power spectra}

\begin{figure}[ht] % Inline image example
\begin{center}
\includegraphics[width=0.3\textwidth]{Figures/NoiseTests/rms_TOI_array_1_20170228s151.pdf}
\includegraphics[width=0.3\textwidth]{Figures/NoiseTests/rms_TOI_array_2_20170228s151.pdf}
\includegraphics[width=0.3\textwidth]{Figures/NoiseTests/rms_TOI_array_3_20170228s151.pdf}
\includegraphics[width=0.3\textwidth]{Figures/NoiseTests/pws_TOI_array_1_20170228s151.pdf}
\includegraphics[width=0.3\textwidth]{Figures/NoiseTests/pws_TOI_array_2_20170228s151.pdf}
\includegraphics[width=0.3\textwidth]{Figures/NoiseTests/pws_TOI_array_3_20170228s151.pdf}
\end{center}
\caption[Noise RMS and power spectra]{From top to bottom and from left to right, we show the data rms and power spectra for the three NIKA2 arrays (A1, A2, and A3) for scan 20170228s150. The rms and power spectra are given for the raw data (blue), and for the CM (green), PCA (red) and MCP (cyan) decorrelated data. The vertical lines in the rms noise figures separate detectors from different readout electronic boxes. Whithin each electronic box the pixels are ordered with increasing resonant frequency across the electronic band. \label{rmspws}}
\end{figure}

In Figure \ref{rmspws} we present the rms noise per kid and the power spectra of
a typical raw datastream, and the CMB, PCA and MCP decorrelated data.  We observe that after
decorrelation we reduce significantly the rms of the noise. Equivalently, in
$1/f$-like noise in the power spectra (principally due to atmospheric emission)
is significantly reduced leading to nearly flat spectra down to 0.05 Hz, with
larger $1/f$-like residual noise for the CM decorrelation method at lower
frequencies. This is translated into a larger rms noise for this method with
respect to the others. For the three arrays we find increasing noise with
increasing resonant frequency withing each electronic box. This is probably
related to the difference of gains between subbands in the readout
electronics. We also find for the three arrays some noise bursts that are not
fully consistent from one decorrelation method to another. \\

A more detailed study of the noise properties for improve the signal reconstruction is taking place taking profits of the qualitative findings described here. Report on this work will be made available to IRAM when the methodology is fully stablished.
