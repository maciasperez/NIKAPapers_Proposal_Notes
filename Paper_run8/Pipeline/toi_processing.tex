
%% \begin{itemize}
%% \item list the contributions to kid timelines : signal, atmosphere noise,
%%   electronic correlated noise, white noise
%% \item Mention coupling between some kids (doubles) and plateau correction
%% \item explain how we ``decorrelate''
%% \item ...
%% \end{itemize}
%% ~\\

At each time $t$, a kid data sample $d$ is the sum of astrophysical signal $s$ ,
atmosphere emission $a$ , white noise $n$ and electronic noise $c$. Each of
these components is weighted by its own calibration factor. The electronic noise
is partially correlated.

\begin{equation}
d_k(t) = \alpha_k s(t) + \beta_k a(t) + \gamma_k c(t) + n(t)
\label{eq:signal_model_1}
\end{equation}

The telescope diameter etc... place the far field at... hence, the atmospheric
emission is seen by all detectors as a common signal.

\subsection{Deglitching}

The first operation involved in the TOI processing is the flagging and the
correction of cosmic ray induced signal, often refered to as ``glitches''. For
\nika, the rate of glitches is \nico{XXX}. Given the \nico{ XXX ms} time contant of
kids and our sampling rate of 23.8~Hz, glitches behave like a single data sample
in large excess compared to the neighbour samples. To detect such samples, we
compare the value of each sample to the median of the TOI on the 200 \nico{ TBC}
neighbour samples and flag it if the difference is larger than 6 \nico{ TBC} times
the r.m.s. of these neighbour samples. We exclude the flagged samples to fit a
straight baseline on the remaining samples and replace the flagged samples by
the value of the fit while keeping the flag information in order not to project
this fake value on maps.

\subsection{Atmosphere and correlated noise subtraction}
\label{se:point_source_decorr}

To subtract the atmosphere and electronic noise contributions of
eq.~(\ref{eq:signal_model_1}), we use the fact that they are not fixed in sky
coordinates and that they dominate at low temporal frequencies. For each kid, we
flag out samples when the kid is closer to the source than 3~FWHM's \nico{
  TBC}. With the remaining samples, we can cross-calibrate all kids far from the
source, {\it i.e.} on the atmosphere and the correlated electronic noise
component to derive a single average timeline (hereafter refered to as ``common
mode''). Each kid is then cross-calibrated on this common mode far from the
planet and the latter is subtracted from the entire kid TOI. This process
suppresses atmospheric noise and a significant fraction of the correlated
electronic noise while preserving point source photometry, since in pratice, no
filtering is applied (\nico{ give numbers from simulations and develop in the case
  of decorrelation from blocks of kids}).

This method is in principal suitable only for the observation of point sources
with known positions. In practice, it works for undetected weak sources. Indeed,
in this case, the source signal is subdominant compared to the common mode and
it can be ignored during the decorrelation process. If after integration, the
source shows up on the map, we can then determine its position and then repeat
the process while flagging it out. If no source is detected, we can only place
an upper limit on the source flux. The fraction of the flux potentially
subtracted with the common mode can then be assessed via Monte-Carlo simulations
(see Sect.~\ref{se:validation}). This method works also for sources that,
although not point like, are small enough compared to our matrix (1.5~arcmin
field of view \nico{ exact value TBC}). In this case, it is still possible to mask
out samples that are on source and to use those out of the source to
derive the common mode. \nico{ conlude on the other extrem case of diffuse
  emission for which our instrument is indeed not optimal, but address how much
  we can still recover (e.g. taurus, SZ)}.
