\section{Summary of results of aperture photometry}


Observations of Uranus and Neptune during runs 9, 12 and 14 have been used to measure the true beam, i.e. total beam incluing main and error beams, 
the beam efficiencies, and the ratio between the flux densities determined with the aperture photometry and gaussian fit.

In Table \ref{tab:solid}, the solid angle of the true beam  $\Omega_{true} (\nu,r_{max}) = \int_0^{r_{max}} B(\nu, r) 2 \pi r dr$
is estimated at $r_{max} =180''$ and
the gaussian solid angle $\Omega_{gauss}={{2\pi} \over {(2 \sqrt{2\ln2})^2}} (fwhm)^2$ is derived with the $fwhm$ determined
from a fit of a gaussian main beam within a radius of $0.65 \times fwhm$. 

\begin{table*}[!h]
\caption{Solid angle of true beam based on Uranus and Neptune observations}
\label{tab:solid}
\centering
\begin{tabular}{l| c | c c c | c c c}
\hline\hline
\noalign{\smallskip}
run  & Nber of scans & \multicolumn{3}{c}{$\Omega_{true}$ (arcsec$^{2}$)} & \multicolumn{3}{c}{$\Omega_{true}/\Omega_{gauss}$} \\
\hline
     &               &  A1    &    A2   &  A3  & A1  &  A2  & A3   \\
            \hline
r9    & 27  &  265$\pm$ 23    &  466$\pm$ 17 & 252 $\pm$ 23 &  1.80 $\pm$ 0.12    &  1.35 $\pm$ 0.05   &   1.74 $\pm$ 0.13   \\
r12   & 20  &  229$\pm$ 11    &  437$\pm$  9 & 221 $\pm$ 10 &  1.71 $\pm$ 0.06   &  1.30 $\pm$ 0.02   &   1.68 $\pm$ 0.06   \\
r14   & 28  &  251$\pm$ 16    &  457$\pm$ 15 & 245 $\pm$ 18 &  1.73 $\pm$ 0.08   &  1.32 $\pm$ 0.03   &   1.72 $\pm$ 0.08   \\
mean  &     &  248            &  453         &  239         &  1.74              &   1.32             &   1.71              \\
       \noalign{\smallskip}
            \hline
\end{tabular}
\end{table*}






In Table \ref{tab:MB}, the main beam efficiency is defined as the ratio between the power in the gaussian main beam fitted and 
the power out to $r_{max}=180''$ by summing intensities of all pixels within this radius. The level of the error beam 
is given relative to the main beam peak (we recall that -12dB as found is 6\%). 

\begin{table*}[!h]
\caption{Main beam efficiency and level of error beam}
\label{tab:MB}
\centering
\begin{tabular}{l| c | c c c | c c c}
\hline\hline
\noalign{\smallskip}
run  & Nber of scans & \multicolumn{3}{c}{Main beam efficiency } & \multicolumn{3}{c}{Error beam level} \\
\hline
     &               &  A1    &    A2   &  A3  & A1  &  A2  & A3   \\
            \hline
r9    & 27  &  54.1$\pm$ 3.2\%    &  74.7$\pm$ 2.9\% & 55.9 $\pm$ 3.7\%   &  -11.5 $\pm$ 0.8    &  -14.9 $\pm$ 0.6   &  -12.0 $\pm$ 0.6   \\
r12   & 20  &  55.7$\pm$ 2.0\%    &  77.4$\pm$ 1.0\% & 57.1 $\pm$ 2.0\%   &  -13.4 $\pm$ 0.3    &  -16.1 $\pm$ 0.3   &  -13.8 $\pm$ 0.3   \\
r14   & 28  &  55.0$\pm$ 2.7\%    &  76.0$\pm$ 1.8\% & 56.1 $\pm$ 2.6\%   &  -12.5 $\pm$ 0.6    &  -15.3 $\pm$ 0.6   &  -12.7 $\pm$ 0.8   \\
            \noalign{\smallskip}
            \hline
\end{tabular}
\end{table*}


Aperture photometry integrates the intensities of all pixels within radius $r_{max}=180''$. In Table \ref{tab:ratio}, we provide the ratio between the 
the flux density derived by aperture photometry and a simple Gaussian fit to the source (Uranus or Neptune) with fixed fwhms $12.5''$ and $18.0''$).
This ratio is larger than unity and indiques that aperture photometry recovers more flux density than the gaussian fit. Caveat : a full analysis
of this inconsistency would require to take into account the fact that KIDs (kidpars) have actually been calibrated by fitting a gaussian to Uranus 
or Neptune.    


\begin{table*}[!h]
\caption{ratio aperture photometry / gaussian fit flux densities   }
\label{tab:ratio}
\centering
\begin{tabular}{l| c | c c c }
\hline\hline
\noalign{\smallskip}
run     & Nber of scans  &  A1    &    A2   &  A3    \\
\hline
r9    & 27  &  1.08$\pm$ 0.03    &  1.04$\pm$ 0.02 & 1.09 $\pm$ 0.03     \\
r12   & 20  &  1.12$\pm$ 0.02    &  1.05$\pm$ 0.02 & 1.13 $\pm$ 0.02     \\
r14   & 28  &  1.09$\pm$ 0.03    &  1.03$\pm$ 0.02 & 1.09 $\pm$ 0.03     \\
\noalign{\smallskip}
\hline
\end{tabular}
\end{table*}
