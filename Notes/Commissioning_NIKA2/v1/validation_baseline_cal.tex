%
%
%
\subsection{Baseline calibration results}
%\label{se:valid_baseline_cal}

The baseline calibration accuracy and stability are primarily assessed
using two quantities. We define the calibration bias $b$ for array $i$ as
the ratio between the measured flux density $\hat{S_{i}}$ using the
Gaussian fixed-width beam photometry as discussed in
Sect.~\ref{se:cal_HA} and the flux density expectations $\hat{S}$ as
given in Sect.~\ref{se:ref_flux_secondaries}. From a series of
secondary calibrator scans, we evaluate
\begin{itemize}
\item[i)] the average calibration bias per array $b_{\rm A_i}$,
  which by construction, should be equal to unity within the precision
  with which the expected flux densities are known. Moreover, 
  the calibration bias stability against the observed opacity provides
  us with a robustness test of the opacity correction, and the stability
  againts the measured beam size, a test of the photometric
  susceptibility to optical variations. %(driven by the main dish
  %distortions)
\item[ii)] the standard deviation from the mean $\sigma_{\rm A_i}$,
  which consists in an estimate of the statistical calibration
  uncertainties that encloses errors of optical, atmospheric, noise
  and data processing origins. Added in quadrature with the model
  uncertainties reported in Moreno et al., it represents a
  conservative estimate of the total absolute calibration errors.
\end{itemize}
  
  

\addparag{ARRAY OF FIG: for A1, A3 and A2, col 1 : flux bias vs FWHM,
  col 2 : flux bias vs observed opacity}

