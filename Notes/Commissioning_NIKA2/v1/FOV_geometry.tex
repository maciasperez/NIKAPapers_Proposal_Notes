%----------------------------------------------------------------------------------------
%	FOCAL PLANE RECONSTRUCTION
%----------------------------------------------------------------------------------------
%\subsection{Methodology}
\label{se:fov}
 
\begin{figure}[p]
\begin{center}
 \includegraphics[angle=0, scale = 0.3]{Figures/FOV_A1.png}
 \includegraphics[angle=0, scale = 0.3]{Figures/FOV_A2.png}
 \includegraphics[angle=0, scale = 0.3]{Figures/FOV_A3.png}
\caption{Nasmyth offsets of each array, from beammap 20170226s415 on
  3C84 (N2R9).}
\label{fig:fov_ex}
\end{center}
\end{figure}

% moved to the average FOV section
%\begin{table}
%\begin{tabular}{|l|l|l|}
%\hline
%Array & Number of valid kids & Fraction of all kids\\
%\hline
%A1 & 793 & 0.75\\
%A2 & 481 & 0.83\\
%A3 & 872 & 0.83\\
%\hline
%\end{tabular}
%\end{table}

In order to determine the pointing offsets of each KID with respect to the reference sky
coordinates as commanded by the telescope tracking system, we perform a {\it beammap}, that is to say we map a bright and compact source, most of the time
a planet, with an elevation step small enough to meet Nyquist sampling at the 1-mm
beam scale, namely 4.8~arcsec. We observe this planet with a raster scan in
(az,el) coordinates, either with fixed elevation subscans or fixed azimuth
subscans. The former has the advantage of low air mass variation across a
subscan, the latter offers an orthogonal scan direction to the former: the
combination of both gives a more accurate determination of the far side
lobes. The data reduction proceeds in two steps.

\paragraph{Step 1.} We apply a median filter per
KID timeline whose width is 4~FWHM and we project one map per KID in Nasmyth
coordinates. This median filter removes efficiently most of the atmospheric and low frequency
electronic noise, albeit a slight ringing and flux loss on the
source. However, at this stage, we are only interested in the location of the
observed planet. To derive the Nasmyth coordinates from the provided (az,el)
coordinates, we build the following quantities at time~$t$ :

\begin{eqnarray}
dx_t &=& \cos el_t\, daz_t - \sin el_t\, del_t \nonumber \\
dy_t &=& \sin el_t\, daz_t + \cos el_t\, del_t \nonumber
\end{eqnarray}

\noindent {\bf FM: why not using $\delta$ as in the previous section ?}\\
\noindent {\bf FM: i don't think the $_t$ is useful}\\


where $el_t$ is the elevation of the reference pointing direction and $daz$ and
$del$ are the pointing offsets with respect to the source in azimuth and elevation as
provided by the tracking system. Note that $daz$ is already corrected by the
$\cos el_t$ factor to have orthonormal coordinates in the tangent plane of the sky
and be immune to the geodesic convergence at the poles. We then fit a 2D
elliptical gaussian on each kid map. The centroid of this gaussian is a first
estimate of the KID offsets, FWHM's, ellipticity and sensitivity. We apply a
first KID selection by removing outliers to the statistics on these
parameters. We also discard manually KIDs that show a cross-talk counter part on
their map. 
%At the end of this first step, we are ready to move to a second stage.

\paragraph{Step 2.} With the Nasmyth offsets derived in step 1, we are now able to
mask out the planet in each KID timeline. This mask is centered on the planet
location as seen by each kid, it is circular and has a radius of 60~arcsec. We
now build a template timeline (a.k.a. ``common mode'') in two steps. First, we
take the median of all samples of all KIDs that are outside this mask at a given
time $t$. This gives a first estimate of the common mode. Second, we
cross-calibrate each KID on this common mode when the KID is outside the mask
and we coadd all these KID cross-calibrated timelines when they are outside the
mask to have the final common mode. In this sum, each KID TOI is weighted by the
inverse of its variance outside the mask. Once we have this common mode in hand,
we cross-calibrate each TOI on it outside the mask and we subtract it to the
entire KID TOI. We then resume to the projection of each KID TOI in Nasmyth
coordinates like in step 1, and the 2D elliptical gaussian fit on the each kid
map. The centroid coordinates and the FWHM are now the final parameters that can
be derived on the current scan.

This analysis is repeated on all beam maps, which provides statistics and
precision on each KID parameter, together with estimates on KID performance stability.



{\bf FM : {\it precision} ? }\\
{\bf show a screen capture of Katana.}\\

% LP: j'ajoute une phrase pour referencer la Figure \ref{fig:fov}
We present an example of the FOV reconstruction in Fig.~\ref{fig:fov_ex}.

The FOV diameter is defined as 
\begin{equation}
FOV diameter = \sqrt{4 N_{tot. kids} \times gridstep^2/\pi}
\end{equation}

{\bf give the values of gridstep (pitch) both in mm and arcsec on the sky}\\

The same definition applies to ``Effective FOV'' to avoid extra multiplication
by the fraction of valid pixels

\begin{equation}
F\lambda = gridstep\times D(30m)/\lambda
\end{equation}

%\subsection{Average Focal Plane Reconstruction}
\label{avg_kidpar}

\noindent {\bf FM: Figures \ref{fig:avg_fov_color}, \ref{fig:jumping_kids} and \ref{fig:mean_vs_median} are too large.
captions cannot be read}
 

\begin{figure}[p]
\begin{center}
\includegraphics[trim=2cm 14cm 4cm 4cm, clip=true,width=0.55\linewidth]{Figures/A1_fwhm_color_count.pdf}
\includegraphics[trim=2cm 14cm 4cm 4cm, clip=true,width=0.55\linewidth]{Figures/A3_fwhm_color_count.pdf}
\includegraphics[trim=2cm 14cm 4cm 4cm, clip=true,width=0.55\linewidth]{Figures/A2_fwhm_color_count.pdf}
\caption{Average detectors positions for arrays A1, A3, and A2 (from
  green to red as a function of the number of times that a given pixel
  has been considered as valid). The three plots show the detectors
  that have seen the sky and passed the quality criteria for at least
  two beam maps during Run10, 9 and 8: 952, 961, and 553
  %925, 944, and 543
  for A1, A3 and A2, respectively. The inner and outer dash-line circles correspond to a 
  FOV of 5.5$\prime$ and 6.5$\prime$, respectively. Units are arcseconds. 
  The color (from green to red)  shows the number of times that a given pixel has been considered as valid.}
\label{fig:avg_fov_color}
\end{center}
\end{figure}

In order to identify the most stable pixels, we compare the KIDs parameter obtained with several beam maps. 
In the following, we   show results as obtained using seven beam maps from Run10, two from Run9 and one from Run8.
For each pixel we compute the average position on the focal plane and the average FWHM, counting the times that it has been considered as valid.

In Fig. \ref{fig:avg_fov_color} we show the average focal plane
reconstruction, from green to red depending on the number of times
that the pixel has been considered as valid. For A1, A3 and A2,
respectively, we have 952, 961, and 553 pixels that have been
considered as valid at least twice (840, 508, 868 valid at least five
times).
% LP: add a sentence to reference Table ``\ref{tab:number_of_kids}"
Using this criterion, we deduce the fraction of valid
detectors over the designed ones, as given in Table~\ref{tab:number_of_kids}. 
As a second step, we also flag pixels that move across the focal plane from a 
beam map to another (Fig. \ref{fig:jumping_kids} , jumping KIDs) and those 
who share the same position (twin KIDs). To identify the former, we look at the difference 
of the mean and median position of each KID (the red crosses and black squares in 
Fig. \ref{fig:mean_vs_median}). For the latter a criterion on the position is applied in 
order to find the pixels that are closer than the grid step.



\noindent {\bf FM: how many twins ? how many jumping kids ?}\\


\begin{figure}[htp]
\begin{center}
\includegraphics[trim=2cm 14cm 5cm 4cm, clip=true,width=0.55\linewidth]{Figures/A1_positions.pdf}
\includegraphics[trim=2cm 14cm 5cm 4cm, clip=true,width=0.55\linewidth]{Figures/A3_positions.pdf}
\includegraphics[trim=2cm 14cm 5cm 4cm, clip=true,width=0.55\linewidth]{Figures/A2_positions.pdf}
\caption{For the valid detectors, we show the positions of each pixel, as obtained from 
each beam map. Some of them are not found at the 
same position for all the beam maps. Units are arcseconds. {\bf FM : color code ?}}
\label{fig:jumping_kids}
\end{center}
\end{figure}

\begin{figure}[htp]
\begin{center}
\includegraphics[trim=2cm 14cm 5cm 4cm, clip=true,width=0.55\linewidth]{Figures/A1_test_positions.pdf}
\includegraphics[trim=2cm 14cm 5cm 4cm, clip=true,width=0.55\linewidth]{Figures/A3_test_positions.pdf}
\includegraphics[trim=2cm 14cm 5cm 4cm, clip=true,width=0.55\linewidth]{Figures/A2_test_positions.pdf}
\caption{For the valid detectors,
  we show the mean (red crosses) and the median (black squares)
  positions of each pixel, as obtained from each beam map.
  Units are arcseconds. {\bf FM : color code ?}}
\label{fig:mean_vs_median}
\end{center}
\end{figure}


% LP: copy from fov.tex + modif according to Samuel's comment
\begin{table}[ht]
\begin{center}  
  \begin{tabular}{|c|c|c|c|}
    \hline
    Array & Designed detectors &  Valid detectors & Fraction\\
    \hline\hline
    A1 & 1140 & 952 &  84\%\\
    A3 & 1140 & 961 &  84\%\\
    A2 & 616  & 553 &  90\%\\
    \hline
  \end{tabular}
  \caption{ CAPTION}
  \label{tab:number_of_kids}
\end{center}    
\end{table}

%\subsection{FOV grid distortion}
\label{se:grid_distortion}

{\bf LP edit using Samuel's comments, [TBC]}

We studied the matching of the KIDs position on the sky to the
\emph{design} position, as decribed in detail in this wiki post\footnote{see
  {\tt http$://$www.iram.fr$/$wiki$/$nika2$/$index.php$/$}
  
  {\tt April$\_$19,$\_$2017,$\_$FXD,$\_$KID$\_$position$\_$mapping$\_$and$\_$Field$\_$distortion$\_$for$\_$Run9}
}
The global result is scaling, rotation, and shift parameters for each
array. They are described in Table~\ref{ta:gridmatch}.

\begin{table}[ht]
\label{ta:gridmatch}
\begin{center}
\begin{tabular}{|c|c|c|c|}
\hline
Array 1  &	Array 3   &	Array 2   &	Comment \\
\hline
1 mm      &       1 mm     &        2 mm  & \\
1140 	 &      1140 	   &        616  &	Total of designed Kids \\
736/673  &	758/734  &	444/437  &	Found/Well-placed Kids \\
91/59 	 &    96/64 	 &      98/71 	 & Fraction [\%] of WPK/FoundKids and WPK/Total \\
0.87 	 &     0.84 	  & 0.66     &	Median deviation (arcsec) for pixels with a deviation smaller than 5 arcsec. \\
0.52 	 &     0.69 	 &        0.68 	 & Mean distortion across the FoV in arcsec \\
2.3 -4.5  &	2.0 -5.8  &	9.3 -7.5  &	Array center in Nasmyth coordinates (arcsec) \\
4.90  &	4.88  &	4.88  &	Plate scaling (arcsec/mm) in the Design x and y (averaged) \\
77.3  &	76.4  &	78.2  &	Plate rotation angle (degree) from the Design to Nasmyth coordinates \\
6.6  &	6.6  &	6.6  &	FOV (Total kids) \\
9.8/2.00  &	9.7/2.00  &	13.3/2.75  &	Distance between near detectors [arcsec, mm] \\
1.24  &	1.22  &	0.97  &	Distance between near detectors [in lambda/D] \\
\end{tabular}
\end{center}
\caption{Linear 2D fit of the observed position of the detectors in the sky
  against their mechanical designed position for N2R9. }
\end{table}

It shows that on average the position of each detector is known to better than
an arcsecond. The 1mm arrays have almost the same center but this center
differs by 7 and 2 arseconds from the 2mm array center. The sampling is above
$\lambda/D$ at 1 mm. Note that the plate rotation angle was designed as
76.2\,degrees, less than 2 degrees from what is observed. We find that array 1
has the most deviant detectors (above 4 arcseconds from their expected
position). These detectors should be excluded from further analysis. We call
distortion (in the table) the $x.y$ term in the polynomial fitting between the
design grid and the observed position (the fitting is done with the $x$ and
$y$ linear terms and $x.y$ term). 



This has been compared to expectations obtained using ZEMAX
simulation. The grid diagram generated using ZEMAX provides us with
the maximum dispersion in the field defined by

\begin{equation}
P = \frac{\sqrt{(x_p - x_r)^2 + (y_p - y_r)^2}}{\sqrt{x_p^2 + y_p^2}},
\end{equation}

where $(x_p, y_p)$ and $(x_r, y_r)$ are respectivelly the predicted
and real coordinates on the image surface relative to the reference
field position image location (see page 170 of the ZEMAX manual, 2007).
The predicted coordinates for the whole field are obtained using a
linear interpolation of a small area in the field central part,
whereas the real coordinates are calculated by ray tracing through the
optical system.

\begin{figure}[ht] 
\begin{center}
\includegraphics[width=0.9\textwidth]{Figures/NIKA2_Final_grid.png}
\caption{NIKA2 grid diagram simulated using ZEMAX. Crosses indicate
  the real coordinates on the Nasmyth image plan. {\bf question a
    Samuel: pourquoi les dimensions indiquees sont environs 4.5 arcmin
  de cote (et pas 6.5)?}}
 \label{fig:fov_grid_distortion_zemax}
\end{center}
\end{figure}

Figure \ref{fig:fov_grid_distortion_zemax} show the ZEMAX grid diagram for
NIKA2 simulated optic system. The maximum grid distortion is expected
to be of $2.7\%$ in NIKA2 $6.5'$ FOV. The distortion is the most
noticeable in the upper right corner of the Nasmyth plan, which is
also the area of the largest defocus w.r.t. to the center. 

An expected distortion of $2.7\%$ is at most a 5 arcsecond shift from the
center to the outside of the array.  The quoted measured distortions are not
too dissimilar once the different fitting methods have been taken into
account.

% FXD: this would need to be more ascertained. Lack of time to go further.



%   Geometry
%----------------------------------------------------------------------------------------
\subsection{Focal Plane Geometry}

In order to determine the pointing offsets of each KID with respect to the reference sky
coordinates as commanded by the telescope tracking system, we use beam-maps. 
The data reduction proceeds in two steps.

\paragraph{Step 1.} We apply a median filter per
KID timeline whose width is 4~FWHM and we project one map per KID in Nasmyth
coordinates. This median filter removes efficiently most of the atmospheric and low frequency
electronic noise, albeit a slight ringing and flux loss on the
source. However, at this stage, we are only interested in the location of the
observed planet. To derive the Nasmyth coordinates from the provided (az,el)
coordinates, we build the following quantities at time~$t$ :

\begin{eqnarray}
dx_t &=& \cos el_t\, daz_t - \sin el_t\, del_t \nonumber \\
dy_t &=& \sin el_t\, daz_t + \cos el_t\, del_t \nonumber
\end{eqnarray}

\noindent {\bf FM: why not using $\delta$ as in the previous section ?}\\
\noindent {\bf FM: i don't think the $_t$ is useful}\\


where $el_t$ is the elevation of the reference pointing direction and $daz$ and
$del$ are the pointing offsets with respect to the source in azimuth and elevation as
provided by the tracking system. Note that $daz$ is already corrected by the
$\cos el_t$ factor to have orthonormal coordinates in the tangent plane of the sky
and be immune to the geodesic convergence at the poles. We then fit a 2D
elliptical gaussian on each kid map. The centroid of this gaussian is a first
estimate of the KID offsets, FWHM's, ellipticity and sensitivity. We apply a
first KID selection by removing outliers to the statistics on these
parameters. We also discard manually KIDs that show a cross-talk counter part on
their map. 
%At the end of this first step, we are ready to move to a second stage.

\paragraph{Step 2.} With the Nasmyth offsets derived in step 1, we are now able to
mask out the planet in each KID timeline. This mask is centered on the planet
location as seen by each kid, it is circular and has a radius of 60~arcsec. We
now build a template timeline (a.k.a. ``common mode'') in two steps. First, we
take the median of all samples of all KIDs that are outside this mask at a given
time $t$. This gives a first estimate of the common mode. Second, we
cross-calibrate each KID on this common mode when the KID is outside the mask
and we coadd all these KID cross-calibrated timelines when they are outside the
mask to have the final common mode. In this sum, each KID TOI is weighted by the
inverse of its variance outside the mask. Once we have this common mode in hand,
we cross-calibrate each TOI on it outside the mask and we subtract it to the
entire KID TOI. We then resume to the projection of each KID TOI in Nasmyth
coordinates like in step 1, and the 2D elliptical gaussian fit on the each kid
map. The centroid coordinates and the FWHM are now the final parameters that can
be derived on the current scan.

This analysis is repeated on all beam maps, which provides statistics and
precision on each KID parameter, together with estimates on KID performance stability.



{\bf FM : {\it precision} ? }\\
{\bf show a screen capture of Katana.}\\

% LP: j'ajoute une phrase pour referencer la Figure \ref{fig:fov}
We present an example of the FOV reconstruction in Fig.~\ref{fig:fov_ex}.

The FOV diameter is defined as 
\begin{equation}
FOV diameter = \sqrt{4 N_{tot. kids} \times gridstep^2/\pi}
\end{equation}

{\bf give the values of gridstep (pitch) both in mm and arcsec on the sky}\\

The same definition applies to ``Effective FOV'' to avoid extra multiplication
by the fraction of valid pixels

\begin{equation}
F\lambda = gridstep\times D(30m)/\lambda
\end{equation}




%   KID selection
%----------------------------------------------------------------------------------------
\subsection{KID selection and average geometry}
\label{avg_kidpar}


\begin{figure}[p]
\begin{center}
\includegraphics[trim=2cm 14cm 4cm 4cm, clip=true,width=0.55\linewidth]{Figures/A1_fwhm_color_count.pdf}
\includegraphics[trim=2cm 14cm 4cm 4cm, clip=true,width=0.55\linewidth]{Figures/A3_fwhm_color_count.pdf}
\includegraphics[trim=2cm 14cm 4cm 4cm, clip=true,width=0.55\linewidth]{Figures/A2_fwhm_color_count.pdf}
\caption{Average detectors positions for arrays A1, A3, and A2 (from
  green to red as a function of the number of times that a given pixel
  has been considered as valid). The three plots show the detectors
  that have seen the sky and passed the quality criteria for at least
  two beam maps during Run10, 9 and 8: 952, 961, and 553
  %925, 944, and 543
  for A1, A3 and A2, respectively. The inner and outer dash-line circles correspond to a 
  FOV of 5.5$\prime$ and 6.5$\prime$, respectively. Units are arcseconds. 
  The color (from green to red)  shows the number of times that a given pixel has been considered as valid.}
\label{fig:avg_fov_color}
\end{center}
\end{figure}

In order to identify the most stable pixels, we compare the KIDs parameter obtained with several beam maps. 
In the following, we   show results as obtained using seven beam maps from Run10, two from Run9 and one from Run8.
For each pixel we compute the average position on the focal plane and the average FWHM, counting the times that it has been considered as valid.

In Fig. \ref{fig:avg_fov_color} we show the average focal plane
reconstruction, from green to red depending on the number of times
that the pixel has been considered as valid. For A1, A3 and A2,
respectively, we have 952, 961, and 553 pixels that have been
considered as valid at least twice (840, 508, 868 valid at least five
times).
% LP: add a sentence to reference Table ``\ref{tab:number_of_kids}"
Using this criterion, we deduce the fraction of valid
detectors over the designed ones, as given in Table~\ref{tab:number_of_kids}. 
As a second step, we also flag pixels that move across the focal plane from a 
beam map to another (Fig. \ref{fig:jumping_kids} , jumping KIDs) and those 
who share the same position (twin KIDs). To identify the former, we look at the difference 
of the mean and median position of each KID (the red crosses and black squares in 
Fig. \ref{fig:mean_vs_median}). For the latter a criterion on the position is applied in 
order to find the pixels that are closer than the grid step.



\noindent {\bf FM: how many twins ? how many jumping kids ?}\\


\begin{figure}[htp]
\begin{center}
\includegraphics[trim=2cm 14cm 5cm 4cm, clip=true,width=0.55\linewidth]{Figures/A1_positions.pdf}
\includegraphics[trim=2cm 14cm 5cm 4cm, clip=true,width=0.55\linewidth]{Figures/A3_positions.pdf}
\includegraphics[trim=2cm 14cm 5cm 4cm, clip=true,width=0.55\linewidth]{Figures/A2_positions.pdf}
\caption{For the valid detectors, we show the positions of each pixel, as obtained from 
each beam map. Some of them are not found at the 
same position for all the beam maps. Units are arcseconds. {\bf FM : color code ?}}
\label{fig:jumping_kids}
\end{center}
\end{figure}

\begin{figure}[htp]
\begin{center}
\includegraphics[trim=2cm 14cm 5cm 4cm, clip=true,width=0.55\linewidth]{Figures/A1_test_positions.pdf}
\includegraphics[trim=2cm 14cm 5cm 4cm, clip=true,width=0.55\linewidth]{Figures/A3_test_positions.pdf}
\includegraphics[trim=2cm 14cm 5cm 4cm, clip=true,width=0.55\linewidth]{Figures/A2_test_positions.pdf}
\caption{For the valid detectors,
  we show the mean (red crosses) and the median (black squares)
  positions of each pixel, as obtained from each beam map.
  Units are arcseconds. {\bf FM : color code ?}}
\label{fig:mean_vs_median}
\end{center}
\end{figure}


% LP: copy from fov.tex + modif according to Samuel's comment
\begin{table}[ht]
\begin{center}  
  \begin{tabular}{|c|c|c|c|}
    \hline
    Array & Designed detectors &  Valid detectors & Fraction\\
    \hline\hline
    A1 & 1140 & 952 &  84\%\\
    A3 & 1140 & 961 &  84\%\\
    A2 & 616  & 553 &  90\%\\
    \hline
  \end{tabular}
  \caption{ CAPTION}
  \label{tab:number_of_kids}
\end{center}    
\end{table}




%   Distortion from the grid
%----------------------------------------------------------------------------------------
\subsection{FOV grid distortion}
\label{se:grid_distortion}

We studied the matching of the KIDs position on the sky to the
\emph{design} position, as decribed in detail in this wiki post\footnote{see
  {\tt http$://$www.iram.fr$/$wiki$/$nika2$/$index.php$/$}
  
  {\tt April$\_$19,$\_$2017,$\_$FXD,$\_$KID$\_$position$\_$mapping$\_$and$\_$Field$\_$distortion$\_$for$\_$Run9}
}
The global result is scaling, rotation, and shift parameters for each
array. They are described in Table~\ref{ta:gridmatch}.

\begin{table}[ht]
\label{ta:gridmatch}
\begin{center}
\begin{tabular}{|c|c|c|c|}
\hline
Array 1  &	Array 3   &	Array 2   &	Comment \\
\hline
1 mm      &       1 mm     &        2 mm  & \\
1140 	 &      1140 	   &        616  &	Total of designed Kids \\
736/673  &	758/734  &	444/437  &	Found/Well-placed Kids \\
91/59 	 &    96/64 	 &      98/71 	 & Fraction [\%] of WPK/FoundKids and WPK/Total \\
0.87 	 &     0.84 	  & 0.66     &	Median deviation (arcsec) for pixels with a deviation smaller than 5 arcsec. \\
0.52 	 &     0.69 	 &        0.68 	 & Mean distortion across the FoV in arcsec \\
2.3 -4.5  &	2.0 -5.8  &	9.3 -7.5  &	Array center in Nasmyth coordinates (arcsec) \\
4.90  &	4.88  &	4.88  &	Plate scaling (arcsec/mm) in the Design x and y (averaged) \\
77.3  &	76.4  &	78.2  &	Plate rotation angle (degree) from the Design to Nasmyth coordinates \\
6.6  &	6.6  &	6.6  &	FOV (Total kids) \\
9.8/2.00  &	9.7/2.00  &	13.3/2.75  &	Distance between near detectors [arcsec, mm] \\
1.24  &	1.22  &	0.97  &	Distance between near detectors [in lambda/D] \\
\end{tabular}
\end{center}
\caption{Linear 2D fit of the observed position of the detectors in the sky
  against their mechanical designed position for N2R9. }
\end{table}

It shows that on average the position of each detector is known to better than
an arcsecond. The 1mm arrays have almost the same center but this center
differs by 7 and 2 arseconds from the 2mm array center. The sampling is above
$\lambda/D$ at 1 mm. Note that the plate rotation angle was designed as
76.2\,degrees, less than 2 degrees from what is observed. We find that array 1
has the most deviant detectors (above 4 arcseconds from their expected
position). These detectors should be excluded from further analysis. We call
distortion (in the table) the $x.y$ term in the polynomial fitting between the
design grid and the observed position (the fitting is done with the $x$ and
$y$ linear terms and $x.y$ term). 



This has been compared to expectations obtained using ZEMAX
simulation. The grid diagram generated using ZEMAX provides us with
the maximum dispersion in the field defined by

\begin{equation}
P = \frac{\sqrt{(x_p - x_r)^2 + (y_p - y_r)^2}}{\sqrt{x_p^2 + y_p^2}},
\end{equation}

where $(x_p, y_p)$ and $(x_r, y_r)$ are respectivelly the predicted
and real coordinates on the image surface relative to the reference
field position image location (see page 170 of the ZEMAX manual, 2007).
The predicted coordinates for the whole field are obtained using a
linear interpolation of a small area in the field central part,
whereas the real coordinates are calculated by ray tracing through the
optical system.

\begin{figure}[ht] 
\begin{center}
\includegraphics[width=0.9\textwidth]{Figures/NIKA2_Final_grid.png}
\caption{NIKA2 grid diagram simulated using ZEMAX. Crosses indicate
  the real coordinates on the Nasmyth image plan. {\bf question a
    Samuel: pourquoi les dimensions indiquees sont environs 4.5 arcmin
  de cote (et pas 6.5)?}}
 \label{fig:fov_grid_distortion_zemax}
\end{center}
\end{figure}

Figure \ref{fig:fov_grid_distortion_zemax} show the ZEMAX grid diagram for
NIKA2 simulated optic system. The maximum grid distortion is expected
to be of $2.7\%$ in NIKA2 $6.5'$ FOV. The distortion is the most
noticeable in the upper right corner of the Nasmyth plan, which is
also the area of the largest defocus w.r.t. to the center. 

An expected distortion of $2.7\%$ is at most a 5 arcsecond shift from the
center to the outside of the array.  The quoted measured distortions are not
too dissimilar once the different fitting methods have been taken into
account.

% FXD: this would need to be more ascertained. Lack of time to go further.


%% [Optimal Focus]
%%________________________________________________________
\subsection{Reconstruction of the focus surfaces}
\label{sec:focus}

Owing to the NIKA2 $6.5~\rm{arcmin}$ FOV, the focus is expected to
slightly changes across the FOV, defining curved focal surfaces at the
location of the three arrays. Therefore, beam patterns are expected to
show some scatter across the FOV accordingly to the focal
surfaces. Although all the detectors cannot be individually focalised,
an optimal axial focus of the telescope can be found to maximize the
number of detectors at the best focus and hence, maximize the
resolution of the NIKA2 maps.
This optimal z-focus setting is obtained
in measuring the focus at the center of the arrays as described
Sect.~\ref{sec:focus-meas} and apply a focus shift, which is primary
predicted using Zemax simulation, and ultimately verified by measuring
the focus surfaces as decribed here.

\begin{figure}
\begin{center}
  \includegraphics[trim={0, 1cm, 0, 1cm}, clip, angle=0, scale=0.5]{Figures/fov_focus_mv_5.png}
\caption{Focus surface of A1, A3 and A2 arrays from left to
  right. From top to bottom, the focus estimates rely on
  FWHM-minimization, amplitude-maximization of an elliptical
  Gaussian of fixed FWHMs and amplitude-maximization of an elliptical
  Gaussian.}
\label{fig:focus-surfaces}
\end{center}
\end{figure}

\emph{Method. } We measure NIKA2 focal surfaces by means of a sequence of five 'beam-map'
scan observations of bright point-like sources, typically Planets or
bright quasars,
for various settings of the telescope axial focus around the
optimal focus $z_{\rm{opt}}$. A beam-map scan consists of a deep-integrated
$13.5' \times 7.8'$ OTF-scan observation comprizing $99$ sub-scans and
with a scanning speed of either $65''/s$ whenever the mean integration
elevation is $< 60$ degree or $39"/s$ at higher elevation. The z-focus is changed in step of
$0.6~\rm{mm}$ to probe a large focus range for measuring even the
extreme variation of the focus surfaces,
namely $z \in \{-1.2, -0.6, 0, 0.6, 1.2 \} + z_{\rm{opt}}$.
Each beam-map scans allow for $4''$-resolution individual maps per kid to
be projected. Before the projection, the correlated noise is mitigated
from each KID timeline in subtrating out a common mode, which is obtained
using, amongst the other detectors, those that correlates the most
with this KID and that are located outside a radius of $90''$
around the source centroid.
Therefore, a series of five cleaned maps at various focus is
available for each detector, from which the best focus is estimated as
described in Sect.~\ref{sec:focus-meas}. The ensemble of the relative
focus estimate per KIDs with respect to the best focus at the center
of the array constitutes the focus surface. An accurate estimate of
the center focus is obtained as the
weighted average focus estimate of the KIDs lying in a $30''$ radius
around the geometrical center of the array. This average does not
induce any sizeable bias thanks to the flatness of the focus surface
in the innermost regions. For robustness test, we consider three focus
estimates: the two first ones are the same as discussed in
Sect.~\ref{sec:focus-meas} -- namely i) $\hat z_{\rm{fwhm}}$ the focus that
minimizes the geometrical FWHM and ii) $\hat z_{\rm{peak}}$ the focus
that maximizes the amplitude of the best-fitting ellitical Gaussian --
whereas the third one is $\hat z_{\rm{flux}}$ the focus that maximizes
the amplitude of the best-fitting elliptical Gaussian of fixed FWHM
(at $12''$ at $260~\rm{GHz}$ and $18''$ at $150~\rm{GHz}$). The 
comparison between the two amplitude-based estimators
($\hat z_{\rm{peak}}$ and $\hat z_{\rm{flux}}$), will test the
stability of the focus results against the exact choice of the beam fitting
function. Since the ellipticity-based estimator $\hat z_{\rm{ellip}}$ is
less sensitive to focus changes and yields larger uncertainties than the
others, we do not use it for the focus surface reconstruction.     


\emph{Data selection. }
During the three commissioning campaigns that occured after the change of A1
lens and the improvement of internal optics alignment (hence in the
final NIKA2 optic configuration),
nine out-of-focus
beam-map scan sequences have been acquired, including incomplete
sequences and sequences hindered by poor atmospheric conditions. We
select sequences that i) comprises at least four scans, ii) have been
observed at zenith opacity at $225~\rm{GHz}$ (as indicated by
the IRAM taumeter) below 0.5 and iii) have a maximal central focus
drift between the starting time and the end of the sequence of
$0.5~\rm{mm}$. These criteria preserve five sequences from which focus
surfaces can be reconstructed. Namely, we consider the sequences
$20170226s415\mbox{--}419$, $20170419s133\mbox{--}137$, $20170420s113\mbox{--}117$,
$20170421s160\mbox{--}164$ and $20170424s123\mbox{--}127$, which consist of observations
of the bright quasar '3C84' and Neptune.

\emph{Results. }
For each detector $k$ and each beam-map sequence $s$, we obtain for
the array $a$, a focus measurement $z_k^{a, s} \pm \sigma_k^{a, s}$,
where $\sigma_k^{a, s}$ is the $1\mbox{--}\sigma$ error of the least-square
polynomial fit. The focus surface measurements per array obtained from the five
beam-map sequences are combined using an inverse-variance weighting
scheme to obtain the focus surface estimates 
\begin{equation}
\label{eq:mv_focus_surf}
z_k^{(a)} = \left( \sigma_k^{(a)} \right)^2 \,  \sum_s \frac{z_k^{a,s}}{\left(\sigma_k^{a,s}\right)^2}\, \,  ,
\end{equation}
with uncertainties 
\begin{equation}
\label{eq:error_mv_focus_surf}
\sigma_k^{(a)} = \left[ \sum_s \frac{1}{\left(\sigma_k^{a,s}\right)^2}\right]^{-1/2}\, .
\end{equation}


We present NIKA2 focus surfaces per arrays obtained as in
Eq.~\ref{eq:mv_focus_surf} 
%from the inverse-variance weighted combination of the five
%reconstructed focus surfaces per arrays
in Fig.~\ref{fig:focus-surfaces}.
The three flavours of focus-estimators provide us with focus surfaces
per arrays that are in good agreement with each others and that have a
non-axisymetrical flatten bowl shape consistent with expectations from
simulation {\bf [TBA, as discussed further below]}.
The median defocus (that is the relative focus w.r.t. the center)
across the detectors is about
$-0.1~\rm{mm}$ for the three arrays. Maximal defocus values of about
$-0.6~\rm{mm}$ are found for detectors located in the outer top and
left regions of the FOV. Finally, a fraction comprised between $20$
and $30\%$ of the KIDs has a relative $z\le -0.2~\rm{mm}$.  

We primarily estimate the uncertainty of the focus
surface measurements using the standard deviation between the three
estimators $z_k^{(a)}|_{\rm{fwhm}}$, $z_k^{(a)}|_{\rm{peak}}$ and
$z_k^{(a)}|_{\rm{flux}}$. We found approximatively homogeneous
standard deviation surfaces per arrays, which have median values across
the FOV of about $0.03~\rm{mm}$.
However, we cross-check this error estimate by forming the quadratic mean of
the three inverse-variance error surfaces per arrays, which are defined in
Eq.~\ref{eq:error_mv_focus_surf} and quoted
$\sigma_k^{(a)}|_{\rm{fwhm}}$, $\sigma_k^{(a)}|_{\rm{peak}}$ and
$\sigma_k^{(a)}|_{\rm{flux}}$. This provides us with more optimistic
error surfaces per array, which do not show any clear pattern across
the FOV and which have a median value across the detectors of about
$0.015~\rm{mm}$.  

%[EXPAND THE DISCUSSION ON COMPARISON WITH SIMULATION]

\emph{Stability across sequences. }
By comparing the focus surface obtained from the five individual focus
sequences, we test the stability of the NIKA2 focus surfaces across
the time and the atmospheric conditions. In
Figs.~\ref{fig:focus-stability-H}-\ref{fig:focus-stability-V}, we compare
the defocus along two perpendicular diameters across the
FOV. Although any direction would have been equivalent for this test, we choose to
position the diameters along-with and perpendicular-to the KID geometrical
grid to avoid the scatter due to KID non-alignement in any other
direction. The scatter is further mitigated by considering
four-detector-wide diameters as shown in upper the left corner of
Figs.~\ref{fig:focus-stability-H}-\ref{fig:focus-stability-V}.

{\bf add a sentence to conclude on the stability}


\begin{figure}
  %\begin{center}
  \includegraphics[trim={-2cm, 2cm, 0, 2cm}, clip, angle=0, scale=0.1]{Figures/fov_focus_stability_check_D1.png}
  \begin{center}
  \includegraphics[trim={0, 2cm, 0, 2cm}, clip, angle=0, scale=0.45]{Figures/fov_focus_1D_Vband_5.png}
  \end{center}
  \caption{Stability of the focus surface across the sequences. This
    series of plot show the relative focus with respect to the center
    (defocus) along the 'vertical diameter', that is a band of
    four-detector width across the FOV, which is vertical with respect to
    the detector geometrical grid, as illustrated by the plot in the
    upper left corner. The datapoints show the defocus along the
    'vertical diameter' estimated from the five focus sequences,
    namely $20170226s415\mbox{--}419$ (sky blue),
    $20170419s133\mbox{--}137$ (dark blue), $20170420s113\mbox{--}117$ (red),
    $20170421s160\mbox{--}164$ (yellow) and $20170424s123\mbox{--}127$
    (green), using the $z^{(a)}|_{\rm{fwhm}}$, $z^{(a)}|_{\rm{flux}}$ and
    $z^{(a)}|_{\rm{peak}}$ estimators from top to bottom, and for A1, A3 and
    A2 arrays from left to right. The black datapoints are the five-sequence combined defocus, as
    presented in Fig.~\ref{fig:focus-surfaces}, taken along the
    'vertical diameter', and the errorbars, the
    five-sequence combined defocus errors along the 'vertical
    diameter'.}
\label{fig:focus-stability-H}
\end{figure}


\begin{figure}  
  \begin{center}
  \includegraphics[trim={0, 2cm, 0, 2cm},clip, angle=0, scale=0.45]{Figures/fov_focus_1D_Hband_5.png}
  \caption{Stability of the focus surface across the sequences. Same
    legend as in Fig.~\ref{fig:focus-stability-H}, but for the
    detectors located in an 'horizontal diameter', i.e. a band of
    four-detector width across the FOV, which is horizontal with respect to
    the detector geometrical grid, as illustrated by the plot in the
    upper left corner. }
\label{fig:focus-stability-V}
\end{center}
\end{figure}


