
NIKA2 photometric capabilities after the calibration presented in
Chapter~\ref{se:calibration}, are assessed in this section. Firstly,
we use observation of secondary calibrators (planetary nebulae NGC7027, CRL2688, and
MWC349A) to test the consistency of the flux density estimates with
expectations. Then, we verify the stability of the photometry with
respect to the atmospheric conditions using a large amount of bright
source observations.


The methodology to evaluate the photometric
capabilities and calibration results are further described in
Sect.~\ref{se:photometry_criteria}.
%, which we recall
%comprizes the opacity correction as described in
%Sect.~\ref{se:opacity_correction}, the KID inter-calibration as discussed in
%Sect.~\ref{se:intercalibration} and the absolute calibration as
%addressed in Sect.~\ref{se:calibration}, are assessed in this section. First, we check that  
%secondary calibrators (planetary nebulae NGC7027, CRL2688, and
%MWC349A) %the two largest asteroids Ceres and Vesta were also
%observed.
In Sect.~\ref{se:ref_flux_secondaries}, the flux density expectations
in NIKA2 bands for the considered secondary calibrators are
determined. Then, we evaluate both the calibrator measured-to-expected
flux density ratio and the calibration statistical errors for the
baseline calibration method in Sect.~\ref{se:photometry_baseline}. 
In Sect.~\ref{se:photometry_others}, we compare these results with other
calibration method results. 
%Finally, in Sect.~\ref{se:aperture_photometry}, we check the robustness of the calibration
%results using an alternative calibration method that relies on
%aperture photometry.

\section{Calibration validation criteria}% {\color{LimeGreen} Laurence}}
\label{se:photometry_criteria}

\subsection{Calibration bias}
The calibration accuracy is primarily assessed by checking
that the flux density measurement of known sources is unbiased.

This performance test also depends on the accuracy with which the
chosen secondary calibrator SED is known. Our main secondary
calibrator is MWC349, for which we have derived precise NIKA2 flux
density espectations as discussed in
Sect.~\ref{se:ref_flux_secondaries}.
These are mainly limited by the absolute calibration of the Plateau de
Bure interferometer and VLA.

We define the calibration bias $b$ for array $i$ as
the ratio between the measured flux density $\hat{S_{i}}$ using the
fixed-width Gaussian beam photometry as discussed in
Sect.~\ref{se:flux_density_equation} and the flux density expectations $S^{0}_{i}$ as
given in Sect.~\ref{se:ref_flux_secondaries}. From a series of
secondary calibrator scans, we evaluate the average calibration bias
per array $b_{\rm A_i}$, which by construction, should be equal to
unity within the precision with which the expected flux densities are
known. Moreover, the calibration bias stability against the observed
opacity provides us with a robustness test of the opacity correction,
and the stability againts the measured beam size, a test of the
photometric susceptibility to optical variations.
%(driven by the main dish distortions)

\subsection{Calibration stability}

We use a large amount of bright source observation scans to test the
stability of the measured flux densities with respect to the observing
conditions.

The selected bright source scans consists of the OTF scans that meet
the baseline selection criteria and for which the flux estimate is
above $800~\rm{mJy}$ at $1~\rm{mm}$ and $400~\rm{mJy}$ at
$2~\rm{mm}$. We consider only the sources for which a minimum of $10$
scans are available after selection.  

We evaluate the standard deviation of the bright source measured-to-median flux
density ratio $\sigma_{\rm A_i}$ for each array or array combination. 
This quantity constitutes an estimate of the statistical calibration
uncertainties that encloses errors of optical, atmospheric, noise and
data processing origins.
Added in quadrature with the model uncertainties reported in
Moreno et al. and with the bandpass uncertainties, it represents a
conservative estimate of the total absolute calibration errors.
