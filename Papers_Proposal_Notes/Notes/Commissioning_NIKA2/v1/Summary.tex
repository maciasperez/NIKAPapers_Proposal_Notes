%----------------------------------------------------------------------------------------
%	SUMMARY
%----------------------------------------------------------------------------------------

\subsection*{Scope $\&$ objectives}
NIKA2 performance in intensity have been assessed, as defined in the
MoU, after the end of the commissioning in intensity (phase 1). This
commissioning phase ended after the April 2017 technical campaign,
which included a science verification phase, and was officialised at
the 'End-of-commissioning' review that took place at the IRAM in September
2017. This document, which comes along the instrument delivery,
presents the performance, as well as the methods we developped
for their assessement and the robustness tests we performed.

\subsection*{Data set}
The performance assessment is based on observations acquired at
the February 2017 technical campaign (a.k.a. N2R9), during which NIKA2
instrumental set-up was in final configuration for the first time, and
on observations taken during the first science pools of October 2017
(a.k.a. N2R12) and January 2018 (a.k.a. N2R14). Data acquired during
April 2017 technical campaign (a.k.a. N2R10) were not considered for
the sensitivity assessement due to extra noise induced by an
end-of-life turbo pump, which disappears once the turbo pump was
replaced. The considered data set allow us to check the performance
statibility for one year and a large range of atmospheric conditions. 

\subsection*{Methods}
The performance assessment utilizes an IDL-based data reduction
pipeline that was internally developed in the NIKA2 collaboration
to go from raw time-ordered data to calibrated maps. We give a
synthetic description of the entire pipeline in
Sect.~\ref{se:pipeline_overview} for the reader convenience. The
calibration and performance assessment methods are briefly presented
in Sect.~\ref{se:pipeline_overview} and further detailed in Chapters 3
to 8, which are outlined below.

In Chapter~\ref{se:opacities}, the atmospheric attenuation is corrected
using two-step method: first, the line-of-sight opacity of each
observing scans is estimated using a series skydip observations
performed with the NIKA2 instrument, then a correction coefficient is
applied to the skydip opacity to ensure the flus density stablity
against the atmospheric condition. Opacity derivation based on the
resident IRAM taumeter measures is also tested for consistency checks.

In Chapter~\ref{se:fp_reconstruction}, \bm\ scans are used to derive
the position of each KIDs in the FoV and perform a first selection to
discard noisy and cross-talking KIDs. The KIDs are further selected on
the comparison between their measured position and the expected
designed position. Finally, series of defocused \bm\ scans are used to
reconstruct the focus surfaces and estimate the telescope focus
setting that optimizes the focus across the arrays.   

In Chapter~\ref{se:beams}, noticeable features of the beam pattern are
identified using \bm\ scans. Radial profiles of the full beam are
measured up to a radius of $180''$ and are fitted using a
three-Gaussian model. The main beam of each array is evaluated using
three complementary methods: the first Gaussian that best fits the full
beam profile, a single Gaussian fit on a side-lobe-masked version of
the profile and a 2D Gaussian fit on a side-lobe-masked version of the
beam map. Then we use the knowledge we have gained on the full beam
and main beam to derive the main beam efficiency up to $180''$. A more
precise description of the beam pattern above $180''$, which is useful for
diffuse emission studies, is let for a future dedicated study that
will be conducted as part of the diffuse emission NIKA2 Large
Programs.

Chapter~\ref{se:calibration} deals with all the aspects related to the
calibration. Sections~\ref{se:cal_HA_reference}
and \ref{se:cal_HA_main} describe the reference photometric system
that has been adopted: flux densities rely on a fit
of a Gaussian with a fixed FWHM at reference values and are given at
reference frequencies. However, we also tested a method based on aperture
photometry and reported the conversion coefficients between the
point-source and aperture photometry calibration in
Sect.~\ref{se:aperture_photo_calibration}.
The main primary calibrator is Uranus, although Neptune
is also considered. In Sect.~\ref{se:ref_flux_primaries}, the flux
expectations for the primary calibrators are derived using the Moreno
model, which is precise at a $5\%$ level. In
Sect.~\ref{se:flatfields}, the KID-to-KID relative calibration is
performed as part of the FOV reconstruction using the fixed-Gaussian fit on
individual maps per KID, which are projected from \bm\ scans. We test
the stability of the KID response across the FoV by considering both
flat fields toward point sources (a.k.a. main beam flat fields) and
flat fields of the atmosphere (a.k.a. forward beam flat
fields). Section~\ref{se:obsdate_variations} deals with the well-known
effect of the telescope heating during the afternoon. This induces a
widening of the beam, which comes along with a drop of the measured
flux densities. We use calibration observations to determine the most
impacted UT hours of the day. The baseline calibration, as presented
in Sect.~\ref{se:baseline_calibration}, is drawn from
scans acquired between 22:00 and 9:00 UT and between 10:00 and 15:00
UT hours, whereas the Sun-rising and afternoon scans are
discarded. However, in Sect.~\ref{se:photocorr_calibration}, we also
tested a calibration method which resorts to a photometric correction
to retrieve robust flux density estimates while the beam experiences
instabilities. This method, which is based on a monitoring of the beam
size throughout an observation campaign, demands dedicated scans to be
made on an hour basis to reach science-grade level of accuracy and
robustness. We present a test case based on a beam monitoring using
the pointing scans, which yields promising results but a mild lake of
robustness. Thus, we choose to base the sensitivity assessment on the
baseline calibration only.

The accuracy and stability of the photometry are treaded in
Chapter~\ref{se:photometry}. We define two calibration performance
criteria: first we measure the ratio between the measured flux density
and the expectations for MWC349, a secondary calibrator which is
monitored at Plateau de Bure, and secondly we estimate the rms errors
with respect to the median flux density using a series of bright
point-like sources. This latter criterion provides us with an estimate
of the statistical uncertainties of the calibration, whiwh includes
errors sourced by the dispersion in the observing conditions
(atmosphere, elevation, source brightness, integration time, etc.) and
in the data analysis.

% first we characterise the features of the noise in the time domain
% by describing both the noise power spectrum and the full KID-to-KID
% correlation matrices, and illustrate the impact of various noise
% decorrelation methods. 
In Chapter~\ref{se:nefd}, we characterise NIKA2 sensitivity by
estimating the Noise Equivalent Flux Density (NEFD). We developped
three complementary methods that
differ in the measurement of the flux density variance but rely on the
same estimate of the on-source integration time. The first method is
based on fitting the variance decrease with the integration time, the
second utilizes statistically equivalent data splits to produce a
noise map estimate and the third method resorts to measuring the noise
far from the source. Regarding the integration time, we ensure the
robustness of the estimate by cross-checking the results of two
approaches: integration times directly derived from the sample counts
are compared to integration time estimates with the array footprint
that best account for the scan strategy.  

\subsection*{Results}

All the designed KIDs detect the signal at least in some observation
scans. We conservatively retain only the most stable KIDs, which are
immune to the cross-talking effect and yield good signal-to-noise
measurement. We defined 'valid' KIDs as the detectors that have been
retained in at least two independent KID selection, as discussed in
Sect.~\ref{se:fov_geometry}. We report valid KID fractions of $84\%$
for the $1\,\rm{mm}$ channel arrays and $90\%$ for Array 2. The other
KIDs, which do not meet the defined validity criterion, ramdomly
distribute across the FoV, so that the whole $6.5\,\rm{arcmin}$ FoV is
covered. Considering the designed KID grid as discussed in
Sect.~\ref{se:grid_distortion}, the distance between
neighbour KIDs is $9.8''$, $9.7''$ and $13.3''$ for Array 1, 3 and 2
respectively, which defined an occupation surface per KID. The surface
covered by all valid KID is in turn associated to an effective FoV of
$5.7'$ for Array 1$\&$3 and of $5.9'$ for Array 2.       
%gs   = [9.8, 9.7, 13.3] ;; gridstep in arcsec
%ndet = [952, 961, 553]
%print, sqrt((ndet*(gs/60.0)^2)/!dpi*4.0)
%% F\lambda = gridstep\times D(30m)/\lambda

While the full beam pattern presents a complexe structure as discussed
in Sect.~\ref{se:fullbeam}, the main beam is well described with a 2D
Gaussian of FWHM of $11$ for the $1\,\rm{mm}$ channel arrays
and $17.6''$ for Array 2, with uncertainties of  $0.2''$ for the
combination of Array 1$\&$3 and for Array 2, as presented in
Sect.~\ref{se:fwhm_results}.
Comparing the main beam fit to the measured full beam, we derive the
main beam efficiency up to a radius of $180''$. We find beam
efficiencies of $55 \pm 3 \%$ at $1\,\rm{mm}$ and $77 \pm 2 \%$ at
$2\,\rm{mm}$. These meet the expectations for an instrument that does not
resort to feed horns. The chosen $180''$ cutting radius ensures an accurate
measure of the beam patterning using a single \bm\ scan. However,
dedicated studies based on the Lunar edge observations showed that
a sizable fraction of the full beam stems from beyond this
radius. This fraction is about $35\%$ at $1\,\rm{mm}$ and about $20\%$
at $2\,\rm{mm}$ (using Kramer et al 2013). Using individual map per
KID, we evaluate the rms dispersion of the main beam FWHM across the
FoV, as discussed in Sect.~\ref{se:fwhm_fov}. We measured rms
dispersions of about $0.6''$ at both wavelengths using N2R9 data. These
constitute a conservative estimate since the rms dispersion has been
further improved by optimizing the telescope focus settings from N2R12
on.  

% calib
The rms calibration uncertainties are well better than $10\%$ at 1mm
and even better than $5\%$ at 2mm, which are state-of-the -art
performance for instrument operated at these wavelengths

% noise&sensitivity
The noise well integrates as the square root of the integration time
for series of scans acquired in similar observing condition. The NEFD
surpasses the specifications and reach the goal value at 2mm, but are
slighlty below the specifications at 1mm due to the absorption of one
of the 1-mm polarisation component by the dichroic element, which
degrades the sensitivity of Array 1.
The mapping speed is defined as the sky area that is covered in
one hour of observations to a noise level of 1 mJy.  NIKA2 mapping
speed is an order of magnitude better than the previous generation of
resident instruments. 


The main measured parameters that define the measured NIKA2 performance
are gathered in Table~\ref{tab:nika2summary}.

\begin{table}[h]
\begin{center}    
  \begin{threeparttable}
    \begin{tabular}{|r|c|c|c|c|l|}
      \hline
      & Array 1 & Array 3  & Array 1\&3 & Array 2 & Reference \\
      \hline
      \hline
      Reference Wavelength  [mm]  &  1.2    &  1.2   & 1.2  & 2.0   &   \\
      Reference Frequency  [GHz]  &  260    &  260   & 260  & 150   &  Sect.~\ref{se:cal_HA_reference}  \\
      Central Frequency [GHz]     &  255.5  &  257.8 &      & 151.6 &  Sect.~\ref{se:bandpasses}  \\
      Bandwidth         [GHz]     &  47.8   &  45.7  &      & 42.1  &   \\
      \hline
      Number of designed detectors                   & 1140      &  1140    &    &    616  & Sect.~\ref{se:array}\\
      Number of valid detectors                      &  952      &   961    &    &    553  & Sect.~\ref{se:fov_geometry}\\
      Fraction of valid detectors [$\%$]             &  84       &   84     &    &     90  & \\
      Effective FOV\tnote{a}\hspace{1mm} [arcmin]    &  5.7      &   5.7    &    &    5.9  & Sect.~\ref{se:grid_distortion}\\
      Pixel size in beam sampling unit\tnote{b}\hspace{1mm} [$\lambda \over D$] & 1.24 & 1.22  &  &  0.97 & \\
      \hline
      FWHM\tnote{c}\hspace{1mm} [arcsec]    &  $11.1 \pm 0.2$   &  $11.0 \pm 0.2$  &   $11.1 \pm 0.1$  &  $17.6 \pm 0.1$  &  Sect.~\ref{se:fwhm_results}\\
      Beam efficiency\tnote{d}\hspace{1mm} [\% ] &  $55 \pm 3$  &  $56 \pm 3$   &  $55 \pm 3$   &  $77 \pm 2$  &  Sect.~\ref{se:beam_efficiency}\\
      rms of the FWHM on the FOV [$\%$]          &    6         &      6        &       6        &      3        & Sect.~\ref{se:fwhm_fov}\\
      \hline
      rms pointing error    [arcsec]             & \multicolumn{4}{|c|}{$<3$} &  Sect.~\ref{se:pointing} \\
      \hline
      Absolute calibration uncertainty [\%]      &  \multicolumn{4}{|c|}{5}   & Sect.~\ref{se:ref_flux_primaries} \\
      \hline
      rms calibration error [\%]                 &    5.5       &     6.0       &      5.7       &     3.0       & Sect.~\ref{se:photometry_baseline} \\
      \hline
      $\alpha$ noise integration in time\tnote{d}\hspace{1mm}  & 0.5  & 0.5  &  0.5 & 0.5 & Sect.~\ref{se:nefd_m1} \\
      \hline
      NEFD\tnote{f}\hspace{1mm} [$\rm{mJy} \cdot \rm{s}^{1/2}/\rm{beam}$]  & $47 \pm 4$ & $38 \pm 3$  & $30 \pm 3$  & $9 \pm 1$ & Sect.~\ref{se:nefd_estimation_methods}\\
      NEFD\tnote{g}\hspace{1mm} [$\rm{mJy} \cdot \rm{s}^{1/2}/\rm{beam}$]  & $57 \pm 5$ & $46 \pm 4$  & $36 \pm 3$  & $10 \pm 1$ & \\
      Mapping speed\tnote{h}\hspace{1mm} [arcmin$^2$/h/mJy$^2$] & 45  & 70  & 111  &  1388 &  \\
      Mapping speed\tnote{i}\hspace{1mm} [arcmin$^2$/h/mJy$^2$] & 31  & 48  &  77  &  1119 &  \\
\hline

\end{tabular}
  \begin{tablenotes}
{\small     
  \item[(a)] Equivalent FOV covered by the valid detectors
  \item[(b)] Calculated from real array pixel size [2.75 mm / 2.0 mm] %and unvignetted pupil diameter [27m]
  \item[(c)] Full-width at half-maximum of the main beam using the combined results of the three methods
  \item[(d)] Ratio between the main beam power and the total beam power up to a radius of 180 arcsec
  \item[(e)] Effective power law of noise reduction with integration time
  \item[(f)] NEFD extrapolated at zero opacity
  \item[(g)] NEFD in typical IRAM good sky opacity condition: 2mm pwv, $60^o$ elevation
  \item[(h)] Mapping speed at zero opacity
  \item[(i)] Mapping speed in typical IRAM good sky opacity condition: 2mm pwv, $60^o$ elevation
}
  \end{tablenotes}
\end{threeparttable}
\caption[Main performance measurements]{Summary of the main characteristics describing the measured 
performances of NIKA2.}
\label{tab:nika2summary}
\end{center}  
\end{table}


The main characteristics, as defined in the MOU, are listed in
Table~\ref{tab:nika2summary_main}, along with a reminder of
the \emph{specifications} that are the requirements to be met by the
instrument, and the \emph{goals} that are the values targeted by the
collaboration.
%The performance parameters given in Table~\ref{tab:nika2summary} are
%splitted in two different lists: first, the main characteristics, as
%defined in the MOU, are listed in Table~\ref{tab:nika2summary_main},
%second, other parameters, which are derived from the instrument
%characteristics described in the MOU, and that need to be
%characterized to complete the commissioning phase are given in
%Table~\ref{tab:nika2summary_second}. Table~\ref{tab:nika2summary_second} is
%constructed from the 'secondary' and 'tertiary' tables of Samuel's
%summary document.


\begin{table}[h]
  \caption[Main performance requirement]{Summary of the main characteristics describing the measured performances of NIKA2, as listed in MoU}
  \label{tab:nika2summary_main}
  \begin{threeparttable}
    \begin{tabular}{|r|r|c|c|c|c|}
      \hline
      \multicolumn{2}{|r|}{}           & Array 1 & Array 3  & Array 1\&3 & Array 2 \\
      \hline
      \hline
      FOV diameter [arcmin] & Goal     &  6.5 & 6.5  & 6.5 & 6.5   \\
                            & Specs    &   5  &   5  &   5 &   5   \\
                            & Measure  &  6.5 & 6.5  & 6.5 &  6.5  \\
      \hline
      Pixel size in beam sampling unit\tnote{b}\hspace{1mm} [F$\lambda$]  & Goal    & 0.6  &  0.6  &    &  0.6 \\
                                                     & Specs   & 0.9  &  0.9  &    &  0.9 \\
                                                     & Measure & 1.09 &  1.09 &    &  0.93 \\
      \hline
      Fraction of valid detectors [$\%$] & Goal     &   90      &    90    &      &     90  \\
                                         & Specs    &   50     &    50    &      &     50  \\
                                         & Measure  &   84     &    84    &      &     90  \\
      \hline
      \multicolumn{2}{|r|}{NEFD\tnote{a}\hspace{1mm} [$\rm{mJy} \cdot \rm{s}^{1/2}/\rm{beam}$] goal on $90\%$ of the KIDs} &  &  &  15  & 10 \\
      \multicolumn{2}{|r|}{NEFD\tnote{a}\hspace{1mm} [$\rm{mJy} \cdot \rm{s}^{1/2}/\rm{beam}$] specification on $50\%$ of the KIDs}  &       &     &  30  & 20 \\
      \multicolumn{2}{|r|}{Measured NEFD\tnote{a}\hspace{1mm} on all valid KIDs [$\rm{mJy} \cdot \rm{s}^{1/2}/\rm{beam}$]}           &       &     &  36  & 10 \\
      \hline\hline      
\end{tabular}
  \begin{tablenotes}
  \item[(a)] NEFD in typical IRAM good sky opacity condition: 2mm pwv, $60^o$ elevation
  \item[(b)] Calculated from real array pixel size [2.75 mm / 2.0 mm] and unvignetted pupil diameter [27m]
    \end{tablenotes}
\end{threeparttable}
\end{table} 


%\begin{table}[h]
%  \caption[Secondary performance measurements]{Summary of other NIKA2 performance characteristics either defined in the MoU or extracted from SL's summary document {\color{magenta} LP to report the up-to-date values of Tab.~\ref{tab:nika2summary} here}}
%  \label{tab:nika2summary_second}
%  \begin{threeparttable}
%    \begin{tabular}{|r|c|c|c|c|}
%      \hline
%      & Array 1 & Array 3  & Array 1\&3 & Array 2 \\
%      \hline
%      \hline
%      Reference Wavelength  [mm]  &  1.2   &  1.2  & 1.2 & 2.0   \\
%      Reference Frequency  [GHz]  &  260   &  260  & 260 & 150  \\
%      Central Frequency [GHz]     &  255.5  &    257.8     &     &   151.6  \\
%      Bandwidth         [GHz]     &   47.8  &     45.7     &     &    42.1  \\
%      \hline
%      Beam efficiency\tnote{a}\hspace{1mm} [\% ]    &        &    &     &    \\
%      rms of the FWHM on the FOV [$\%$]   &   &    &   &  \\
%      \hline 
%      rms calibration error [\%]            & 4.5  & 6.6  &   & 5  \\
%      \hline
%     Absolute calibration uncertainty [\%] &  \multicolumn{4}{|c|}{5} \\
%      \hline
%      $\alpha$ noise integration in time\tnote{d}\hspace{1mm}  &   &   &   &  \\
%      \hline
%      rms pointing error    [arcsec]    & \multicolumn{4}{|c|}{$<3$}  \\
%      \hline
%      Mapping speed\tnote{b}\hspace{1mm} [arcmin$^2$/h/mJy$^2$] & 302  & 454  & 775 (1184)  & 7542 (10861)  \\
%\hline
%\end{tabular}
%  \begin{tablenotes}
%  \item[(a)] Ratio between the main beam power and the total beam power up to a radius of XXX arcsec
%  \item[(b)] Average (best) mapping speed at zero opacity for the February 2017 observation campaign. 
%  \end{tablenotes}
%\end{threeparttable}
%\end{table}

%\subsection*{Conclusion}


