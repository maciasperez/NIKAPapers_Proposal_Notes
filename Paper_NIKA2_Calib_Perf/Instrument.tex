%----------------------------------------------------------------------------------------
%	2./ INSTRUMENT
%----------------------------------------------------------------------------------------
%\section{General view of the instrument}
%\label{se:instru}

NIKA2 simultaneously images a field-of-view of
6.5' in diameter at 150 and 260~GHz. It also has polarimetric
capabilities at 260~GHz, which are not discussed here. The optics of
the telescope receiver cabin have been modified in order
to increase the IRAM 30-m telescope field-of-view compared to earlier
resident instruments{\color{magenta} (Ref)}. To achieve these goals without degrading the
telescope angular resolution, it employs a total of around
2,900\,detectors split over three distinct arrays of Kinetic
Inductance Detectors (KID).

A detailed description of the instrument can be found in
\citet{Adam2018}. We briefly present the main sub-systems in this
section focusing on the elements which are specific to NIKA2 or
which drive the development of dedicated procedures for the data reduction or
calibration.


\subsection{Cryogenics}

The optimal operation of the detectors is achieved at a temperature of around
150\,mK, well below the Aluminium superconducting transition. For this reason,
NIKA2 employs a custom dilution fridge to cool down the focal plane, and the
refractive portion of the optics, for a total mass around 100 kg, deeply in the
sub-Kelvin regime. Despite the complexity and size of the system, the operation
of NIKA2 does not require external cryogenic liquids and is fully remotely
controllable.


\subsection{Optics}

The NIKA2 camera optics include two cold mirrors, and the filtering of unwanted
(off-band) radiation is provided by a suitable stack of multi-mesh filters
placed at all temperature stages between 150 mK and room temperature. An air-gap
dichroic splits the 150\,GHz (reflection) from the 260\,GHz (transmission)
beams. A grid polariser ensures then the separation of the two linear
polarizations on the 260\,GHz channel (V and H). Band-defining filters,
custom-designed to optimally match the atmospheric windows, are installed in
front of each array. A half-wave polarization modulator is added at room
temperature when operating the instrument in polarimetric mode.

\subsection{Bandpasses}
\label{se:instru_bandpass}

\begin{figure}[ht!] % Inline image example
\begin{center}
\includegraphics[clip,trim={0, 1cm, 0, 2cm},width=0.5\textwidth]{Figures/bandpasses_nika2_colorsok.png}
\caption[NIKA2 transmission]{Relative system response of the three NIKA2 arrays as a
  function of frequency. For illustration we also plot in black
  atmospheric transmission obtained with the ATM model \citep{ATM,
    Pardo2002} for two values of precipitable water vapor. The spectra
  of ESA4 model of Uranus (pink) and ESA5 model of
  Neptune\footnote{The ESA4 and ESA5 model are available in the Herschel
    Calibrator web page at \url{https://www.cosmos.esa.int/web/herschel/calibrator-models}} (forest green) in the frequency range are overplotted with arbitrary normalization with respect to NIKA2 transmission.} 
 \label{spectralband1}
\end{center}
\end{figure}

The NIKA2 spectral bands were measured in the laboratory using a
Martin-Puplett interferometer built in-house \cite{durand}.  Both
arrays and filter bands were considered in the measurements. These
were obtained from the difference of two black bodies, hence they
include a $\nu^2$ Rayleigh-Jeans (RJ) spectral term.
Figure~\ref{spectralband1} shows the relative spectral response for
the three arrays (corrected of the RJ term).  Notice that array A2 was
replaced by a new one in N2R5 and that the spectral transmissions are
not the same (red and orange lines in the figure).

The two arrays operating at 260 GHz, mapping different linear polarisations,
exhibit a slightly different spectral behaviour as can be
seen on Fig.~\ref{spectralband1}. This may be explained by a tiny
difference in the silicon wafer and/or Aluminium film thicknesses. For
instance, the observed shift of the peak frequency, 265 GHz for the V
(A1) array versus 258 GHz for the H one (A3), can be explained by
about 5 microns change in the substrate thickness. 

It is clear from Fig.~\ref{spectralband1} that the atmosphere will
modify the overall transmission of the system, especially at the tails
for the A2 array.
For each array, we define reference frequencies that are chosen
as round numbers in the middle of the bands to define NIKA2
photometric system as will be discussed in
Sect.~\ref{se:calibration}. These are 260~GHz for the A1 and A3 and
150~GHz for the A2 arrray.

\subsection{KIDs and electronics}
\label{se:array}

The 150\,GHz channel is equipped with A2 that is an array of
616\,pixels, arranged to cover a 78\,mm diameter circle. Each pixel has a size of
$2.8\times2.8\textrm{\,mm}^2$. The array A2 is connected over four different
readout lines. In the case of the 260\,GHz band detectors, the pixel size is
$2\times 2\mathrm{\,mm}^2$, to ensure a comparable sampling of the focal
plane. In order to fill the two 260\,GHz arrays A1 and A3, a total of 1,140 pixels are
needed in each of them. The focal planes are all based on thin Aluminium films
deposited by e-beam evaporation under ultra-high vacuum conditions over a
Silicon substrate.

The key advantage of the KID technology is the simplicity of the cold
electronics and the multiplexing scheme. In NIKA2, each block of around 150
detectors is connected to single coaxial line providing the excitation and the
readout at the two ends. Each of the readout lines is linked to the input of a
cryogenic (4 K) low-noise amplifier. The warm electronics required to digitize
and process the pixels signals is composed of twenty custom readout cards (one
per feed-line).

%In this document, the 2\,mm array is called A2, while the two 1\,mm arrays are
%called A1 and A3.


\subsection{KID photometry and tuning}
\label{se:tuning}

Kinetic Inductance Detectors are superconducting resonators whose resonance
frequency shifts linearly depend on the incoming optical power. The
measure of such frequency shift $\Delta f$ is what allows us to use
KID as mm-wave detectors.

For the KID readout, an excitation signal is sent into the cryostat on the
feedline coupled to the KID. The transmitted signal can be described by its
amplitude and phase, or, as is common practice for KID, by its components that
are In-phase ($I$) and in Quadrature $Q$ with respect to the excitation
signal. The goal is now to relate the measured variations of the KID response
to excitation signal $(\Delta I, \Delta Q)$, which are induced by incident light, to
$\Delta f$. For this, the electronics modulates the excitation
frequency at about 1\,kHz with a known frequency variation $\delta f$
and the read out gives the induced transmitted signal variations
$(dI, dQ)$. Projecting linearly
$(\Delta I, \Delta Q)$ on $(dI, dQ)$ therefore
provides $\Delta f$. This value, in Hz, constitutes the raw
time-ordered data, which are sampled at a frequency of 23.84
Hz. Ingested into the calibration pipeline, it will be futher calibrated into astronomical units
(Sect.~\ref{se:calibration}). For historical reasons, this way of deriving KID
signals has been nicknamed \emph{RfdIdQ}. More details on this process are given
in \citet{Calvo2013}.\\

Not only incident astronomical light reaches the KIDs and contributes to Cooper
pair breaking. Any change in the background optical load (due, for example, to changes in
the atmospheric transmission or in the elevation) contributes as well to the
shift of the resonances. In order to maximize the sensitivity of a KID, the
excitation signal used to read it out must always be near its resonance
frequency. We therefore have developped a tuning algorithm that takes care of
this optimization. Tunings are performed during the first subscan of each
observation in order to be optimally tuned at the same elevation and sky
conditions as the source. It takes only a few seconds when the $\ftone$ are
close to the current functionning point. In order to always be in these
conditions, continuous tunings are done between two scans when NIKA2 is not observing.


