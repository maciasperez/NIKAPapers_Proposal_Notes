\abstract
% context heading (optional)
{The New IRAM KID Array (\NIKA) instrument is a dual-band imaging camera
operating with Kinetic Inductance Detectors (KID) cooled at 100~mK.  \NIKA\
is designed to observe the sky at wavelengths of 1.25 and 2.14~mm from
the IRAM 30~m telescope at Pico Veleta with an estimated resolution of 13\,arcsec
and 18~arcsec, respectively.  This work presents the performance
of the \NIKA\ camera prior to its opening to the astrophysical community as an
IRAM common-user facility in early 2014. \NIKA\ is a test bench for
the final \NIKAii\ instrument to be installed at the end of 2015. The last
\NIKA\ observation campaigns on November 2012 and June 2013 have been used to
evaluate this performance and to improve the control of systematic effects.
We discuss here the dynamical tuning of the readout electronics to optimize the
KID working point with respect to background changes and the new technique of atmospheric absorption correction. These modifications significantly improve the overall linearity, sensitivity, and absolute calibration
performance of \NIKA. This is proved on observations of point-like sources for
which we obtain a best sensitivity (averaged over all valid detectors) of 40 and
14~mJy.s$^{1/2}$ for optimal weather conditions for the 1.25 and 2.14~mm arrays,
respectively.  \NIKA\ observations of well known extended sources (DR21
complex and the Horsehead nebula) are presented.  This performance
makes the \NIKA\ camera a competitive astrophysical instrument.}

