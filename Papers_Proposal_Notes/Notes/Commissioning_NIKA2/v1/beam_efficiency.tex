
\section{Beam efficiency {\color{blue} Laurence $\&$ Jean-Fran\c cois}}

%% LP
%%%%%%%%%%%%%%%%%%%%%%%%%%%%%%%%%%%%%%%%%%%%%%%%%%%%%%%%%%%%%%%%%%%%%%%%%%
\noindent \emph{method 1:}

Eff = integral of the 2D Gaussian Main Beam / integral of the beam map

\noindent \emph{method 2:}

Eff = integral of the 2D Gaussian Main Beam / integral of the measured profile

\noindent \emph{'Instru' Paper}

Comparing the 2D gaussian main beam fit to the full beam
pattern measurement up to a radius of $250''$, we compute the
beam efficiencies defined as the ratio of power between the main
beam and this full beam. We find beam efficiencies $\approx 55 \%$
and $\approx 75 \%$ for the 260 and 150 GHz channels, respectively.
Heterodyne observations of the lunar edge and of the forward
beam efficiency derived from skydips show that a significant
fraction of the full beam is received from beyond a radius of
$250''$. This fraction is not considered here.


The beam efficiency estimates for the three arrays and the
1mm-array combination are given in Table~\ref{tab:beam_efficiency}.


\begin{table}[h]
  \caption[]{Beam efficiency}
  \centering
  \begin{threeparttable}
  \begin{tabular}{|l|c|c|c|c|}
    \hline
    
       &    \multicolumn{4}{|c|}{Array or array combination} \\
    \cline{2-5}
    Method & A1 &  A3 & A1 $\&$ A3 &  A2  \\
    \hline
    \hline
    2D elliptical-over-deep map beam\tnote{a} &  $0.75 \pm 0.07$  &
    $0.69 \pm 0.04$  &  $0.73 \pm 0.06$  &  $0.85 \pm 0.05$  \\
    2D elliptical-over-measured profile\tnote{a} &  $0.63 \pm 0.10$  &
    $0.56 \pm 0.05$  &  $0.58 \pm 0.06$  &  $0.79 \pm 0.06$  \\
    $250''$ estimates\tnote{b}    &  $0.55$  &   $0.55$ &   $0.55$ &  $0.75$  \\
    \hline
  \end{tabular}
  \begin{tablenotes}
  \item[(a)] Laurence's study
  \item[(b)] values reported in the 'Instrument' paper
  \end{tablenotes}
  \end{threeparttable}
  \label{tab:beam_efficiency}
\end{table}




%%
%%  JFL
%%  --> extracted from results_Ap.tex
%%
%%%%%%%%%%%%%%%%%%%%%%%%%%%%%%%%%%%%%%%%%%%%%%%%%%%%%%%%%%%%%%%%%%%%%%%%%%%%%

Observations of Uranus and Neptune during runs 9, 12 and 14 have been used to measure the true beam, i.e. total beam incluing main and error beams, 
the beam efficiencies, and the ratio between the flux densities determined with the aperture photometry and gaussian fit.

In Table \ref{tab:solid}, the solid angle of the true beam  $\Omega_{true} (\nu,r_{max}) = \int_0^{r_{max}} B(\nu, r) 2 \pi r dr$
is estimated at $r_{max} =180''$ and
the gaussian solid angle $\Omega_{gauss}={{2\pi} \over {(2 \sqrt{2\ln2})^2}} (fwhm)^2$ is derived with the $fwhm$ determined
from a fit of a gaussian main beam within a radius of $0.65 \times fwhm$. 

\begin{table*}[!h]
\caption{Solid angle of true beam based on Uranus and Neptune observations}
\label{tab:solid}
\centering
\begin{tabular}{l| c | c c c | c c c}
\hline\hline
\noalign{\smallskip}
run  & Nber of scans & \multicolumn{3}{c}{$\Omega_{true}$ (arcsec$^{2}$)} & \multicolumn{3}{c}{$\Omega_{true}/\Omega_{gauss}$} \\
\hline
     &               &  A1    &    A2   &  A3  & A1  &  A2  & A3   \\
            \hline
r9    & 27  &  265$\pm$ 23    &  466$\pm$ 17 & 252 $\pm$ 23 &  1.80 $\pm$ 0.12    &  1.35 $\pm$ 0.05   &   1.74 $\pm$ 0.13   \\
r12   & 20  &  229$\pm$ 11    &  437$\pm$  9 & 221 $\pm$ 10 &  1.71 $\pm$ 0.06   &  1.30 $\pm$ 0.02   &   1.68 $\pm$ 0.06   \\
r14   & 28  &  251$\pm$ 16    &  457$\pm$ 15 & 245 $\pm$ 18 &  1.73 $\pm$ 0.08   &  1.32 $\pm$ 0.03   &   1.72 $\pm$ 0.08   \\
mean  &     &  248            &  453         &  239         &  1.74              &   1.32             &   1.71              \\
       \noalign{\smallskip}
            \hline
\end{tabular}
\end{table*}






In Table \ref{tab:MB}, the main beam efficiency is defined as the ratio between the power in the gaussian main beam fitted and 
the power out to $r_{max}=180''$ by summing intensities of all pixels within this radius. The level of the error beam 
is given relative to the main beam peak (we recall that -12dB as found is 6\%). 

\begin{table*}[!h]
\caption{Main beam efficiency and level of error beam}
\label{tab:MB}
\centering
\begin{tabular}{l| c | c c c | c c c}
\hline\hline
\noalign{\smallskip}
run  & Nber of scans & \multicolumn{3}{c}{Main beam efficiency } & \multicolumn{3}{c}{Error beam level} \\
\hline
     &               &  A1    &    A2   &  A3  & A1  &  A2  & A3   \\
            \hline
r9    & 27  &  54.1$\pm$ 3.2\%    &  74.7$\pm$ 2.9\% & 55.9 $\pm$ 3.7\%   &  -11.5 $\pm$ 0.8    &  -14.9 $\pm$ 0.6   &  -12.0 $\pm$ 0.6   \\
r12   & 20  &  55.7$\pm$ 2.0\%    &  77.4$\pm$ 1.0\% & 57.1 $\pm$ 2.0\%   &  -13.4 $\pm$ 0.3    &  -16.1 $\pm$ 0.3   &  -13.8 $\pm$ 0.3   \\
r14   & 28  &  55.0$\pm$ 2.7\%    &  76.0$\pm$ 1.8\% & 56.1 $\pm$ 2.6\%   &  -12.5 $\pm$ 0.6    &  -15.3 $\pm$ 0.6   &  -12.7 $\pm$ 0.8   \\
            \noalign{\smallskip}
            \hline
\end{tabular}
\end{table*}

