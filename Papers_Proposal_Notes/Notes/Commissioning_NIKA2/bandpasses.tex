%\begin{figure}[ht] % Inline image example
%\begin{center}
%\includegraphics[width=0.9\textwidth]{Figures/SpectralBands/atm_transmission.pdf}
%\caption{Spectral transmission of the the three NIKA2 arrays as a function of frequency in GHz. For illustration we also plot the ATM atmospheric model for different values of pwv. \label{spectralband}}
%\end{center}
%\end{figure}


\begin{figure}[ht] % Inline image example
\begin{center}
\includegraphics[width=0.9\textwidth]{Figures/SpectralBands/bandpasses_nika2.png}
\caption{Relative system response of the the three NIKA2 arrays as a
  function of frequency. For illustration we also plot the ATM model
  atmospheric transmission for different values of precipitable water
  vapor. The spectra of ESA4 model of Uranus and ESA5 model of
  Neptune in the frequency range are overplotted with arbitrary
  normalization}
 \label{spectralband1}
\end{center}
\end{figure}



\begin{figure}[ht] % Inline image example
\begin{center}
\includegraphics[width=\textwidth]{Figures/SpectralBands/opacity_ratio_vs_tau1.pdf}
\caption{Expected atmospheric opacity ratio of the 2 and 1 mm channels as function of the opacity at 1 mm. \label{thopacities}}
\end{center}
\end{figure}



The NIKA2 spectral bands were measured in the laboratory using a
Martin-Puplett interferometer built in-house \cite{durand}.
Both arrays and filter bands were considered in the
measurements. These were obtained from the difference of two
black bodies, hence they include a $\nu^2$ Rayleigh-Jeans (RJ) spectral term.
Figure~\ref{spectralband1} shows the
relative spectral response for the three arrays (corrected of the RJ
term). Notice that array A2 was replaced by a new one in N2R5 and that
the spectral transmissions are not the same (green and red lines in the
figure).

{\bf parag. below copied from the Instru paper}

The two arrays operating at 260 GHz, mapping different polarisations, exhibit a slightly different
spectral behaviour probably due to a tiny difference in the
silicon wafer and/or Aluminium film thicknesses. The observed
shift of the peak frequency, 265 GHz for the V (A1) array versus
258 GHz of the H (A3), can be explained by about 5 microns
change in the substrate thickness.

{\bf LP: pourquoi pas les memes chiffes que dans le tableau d'Herv\'e ?
  (A1: 254 GHz, A3: 256 GHz)}


%\begin{table}[h]
%\caption{Spectral transmission characteristics for the NIKA2 arrays.%
%\label{nika2runs}}
%\begin{tabular}{|c|c|c|c|c|}
%\hline 
 % &     A1  &  A3 &  A2 2015 & A2 2016 \\ 
%\hline 
%Central Frequency [GHz] &   255.5  & 257.8   &   147.7  & 151.6 \\  
%Bandwidth [GHz]         &   47.8   & 45,7    &   41.8   & 42.1 \\
%\hline 
%\end{tabular} 
%\end{table} 

\begin{table}[h]
\caption{Effective frequencies (for Uranus) and bandwidth of the NIKA2 bands for
  various atmospheric conditions and elevation.}
\label{tab:bandwidths}
\begin{tabular}{|l|l|r|r|r|r|r|r|}
\hline 
\multirow{3}{*}{Water vapor} & \multirow{3}{*}{Elevation} & \multicolumn{2}{|c|}{1mmH} & \multicolumn{2}{|c|}{1mmV} & \multicolumn{2}{|c|}{2mm} \\
 & & $\nu_eff$ & $\Delta \nu$  & $\nu_eff$ & $\Delta \nu$  & $\nu_eff$ & $\Delta \nu$ \\
 & & (GHz) & (GHz)  & (GHz)  & (GHz)   & (GHz)  & (GHz)  \\
\hline
\multicolumn{2}{|c|}{No atmosphere} & 254.71 & 49.21 & 257.39 & 48.05 & 150.93 & 40.72 \\
\hline
\multirow{4}{*}{1 mm H2O $\rightarrow \tau_{225}=$0.067} & 90 deg &  254.46 & 48.72 & 257.12 & 47.95 & 150.93 & 39.71 \\
 & 60 deg & 254.42 & 48.68 & 257.08 & 47.93 & 150.92 & 39.60 \\
 & 40 deg & 254.33 & 48.57 & 256.98 & 47.89 & 150.88 & 39.32 \\
 & 20 deg & 254.00 & 48.21 & 256.62 & 47.77 & 150.75 & 38.45 \\
\hline
\multirow{4}{*}{2 mm H2O $\rightarrow \tau_{225}=$0.120} & 90 deg &  254.26 & 48.74 & 256.91 & 48.06 & 150.64 & 39.34 \\
 & 60 deg & 254.20 & 48.70 & 256.84 & 48.07 & 150.60 & 39.19 \\
 & 40 deg & 254.02 & 48.60 & 256.65 & 48.08 & 150.48 & 38.80 \\
 & 20 deg & 253.43 & 48.30 & 256.01 & 47.93 & 150.13 & 37.62 \\
\hline
\multirow{4}{*}{3 mm H2O $\rightarrow \tau_{225}=$0.173} & 90 deg &  254.06 & 48.76 & 256.70 & 48.19 & 150.39 & 39.03 \\
 & 60 deg & 253.97 & 48.73 & 256.60 & 48.21 & 150.32 & 38.84 \\
 & 40 deg & 253.71 & 48.65 & 256.33 & 48.28 & 150.14 & 38.35 \\
 & 20 deg & 252.86 & 48.41 & 255.40 & 47.86 & 149.60 & 36.94 \\
\hline
\multirow{4}{*}{5 mm H2O $\rightarrow \tau_{225}=$0.278} & 90 deg &  253.67 & 48.82 & 256.28 & 48.45 & 149.96 & 38.47 \\
 & 60 deg & 253.51 & 48.81 & 256.11 & 48.44 & 149.84 & 38.22 \\
 & 40 deg & 253.10 & 48.77 & 255.68 & 48.26 & 149.54 & 37.58 \\
 & 20 deg & 251.74 & 48.67 & 254.20 & 47.75 & 148.68 & 35.82 \\
\hline
\multirow{4}{*}{8 mm H2O $\rightarrow \tau_{225}=$0.437} & 90 deg &  253.08 & 48.94 & 255.66 & 48.42 & 149.38 & 37.76 \\
 & 60 deg & 252.84 & 48.93 & 255.39 & 48.35 & 149.20 & 37.42 \\
 & 40 deg & 252.21 & 48.94 & 254.71 & 48.16 & 148.77 & 36.64 \\
 & 20 deg & 250.12 & 49.38 & 252.43 & 47.91 & 147.52 & 34.57 \\
\hline
\multirow{4}{*}{10 mm H2O $\rightarrow \tau_{225}=$0.542} & 90 deg &  252.70 & 49.00 & 255.24 & 48.38 & 149.04 & 37.34 \\
 & 60 deg & 252.39 & 49.01 & 254.92 & 48.29 & 148.82 & 36.97 \\
 & 40 deg & 251.62 & 49.11 & 254.08 & 48.13 & 148.31 & 36.12 \\
 & 20 deg & 249.07 & 49.75 & 251.28 & 48.24 & 146.85 & 33.93 \\
\hline
\end{tabular}
\end{table}

%What actually matters more than the ``central frequency'' that depends on many
%assumptions and definitions are the bandpasses. We should make available in a
%.fits file, clearly, our bandpasses to avoid future misunderstanding and propagation of
%false numbers. Official values should be 150 and 260~GHz. We should also clearly
%state that these measured bandpasses were done with the difference of two
%black-bodies, hence they include a $\nu^2$ RJ term.\\

The total system response is the multiplication of the atmospheric
transmission with the relative system response. To derive the
atmospheric transmission, we use GILDAS ATM 2009 model, computed for
the IRAM 30m, with 'midlatwinter' conditions. We select in the model
grid an atmosphere with T=268.3 K and a pressure of 703.5 hPa. The
effective frequency of the passband is defined by:
\begin{equation}
\nu_{eff}( \sec z, mm_{H_{2}O}) = \frac{ \int_{0}^{+\infty} S_{\nu}
  T_{\nu}(\sec z, mm_{H_{2}O}) \nu d\nu } { \int_{0}^{+\infty} S_{\nu} T_{\nu} d\nu}
\label{eq:nueff0}
\end{equation}
where $T_{\nu}$ is the total system response, $S_{\nu}$ is the source
spectrum. Table~\ref{tab:bandwidths} list this effective frequency,
computed for Uranus spectrum (ESA4 model), for different atmospheric
water vapor content and different elevations. 
Table~\ref{tab:bandwidths} also list the bandwidth, defined as:
\begin{equation}
\Delta\nu = \int_{0}^{+\infty} \frac{T_{\nu}}{Max(T_{\nu})}
d\nu
\end{equation}
where the $Max((T_{\nu})$ ensure the RSR span the whole 0.0 to 1.0 range.
From Table~\ref{tab:bandwidths}, we see that the 2mm band is somewhat
sensitive to the atmospheric conditions, especially at low elevation.


{\bf Can we be more specific about what was computed next ? Is it the
  opacity weighted by the band ? Or the resultant opacity ?}

Using the NIKA2 bandpasses for N2R9, we can integrate the ATM
atmospheric model to compute the expected ratio between the
atmospheric opacity of the two NIKA2 channels. 
Figure~\ref{thopacities} shows the atmospheric opacity
ratio of the 2 and 1 mm channels as a function of the opacity for the
1 mm one.


