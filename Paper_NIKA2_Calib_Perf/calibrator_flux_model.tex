%\section{Reference flux density of the calibrators}
%\label{se:flux_theo}
%%\section{Reference flux density of the calibrators}
%\label{se:ref_flux_calibrator}
%%\section{Reference flux density of the calibrators}
%\label{se:ref_flux_calibrator}
%%\section{Reference flux density of the calibrators}
%\label{se:ref_flux_calibrator}
%\input{calibrator_flux_model.tex}


\subsection{Uranus and Neptune}
\label{se:ref_flux_uranus_neptune}

For the flux densities of the giant planets, we use the ESA model from
\cite{ESAmodel}: Version 5 for Neptune and Version 4 for Uranus. 
Both models provide the planet brightness temperature in the
Rayleigh-Jeans approximation as a function of the frequency. The
resulting flux is therefore: 
\begin{equation}
S_{\nu} = \Omega \times \frac{2 \nu^{2} k T_{RJ}}{c^2}
\end{equation}
where $\Omega$ is the solid angle of the planet on the sky. Following
\cite{bendo2013} and correcting their equation 12 we have:
%
\begin{equation}
\Omega = \pi \frac{r_{e} r_{p-a}}{D^{2}} 
\label{eq:omega}
\end{equation}
where $r_{e}$ is the equatorial radius of the planet and $r_{p-a}$ is
its apparent polar radius, and $D$ the distance to the
planet. $r_{p-a}$ can be computed from the sub-observer latitude $\phi$
({\it e.g.} the latitude of the 30-m telescope as seen from the planet in the
planet equatorial reference frame) and $r_{p}$ the polar radius of the
planet as:
\begin{equation}
r_{p-a} = \sqrt{r_{p}^2 \cos^{2}\phi + r_{e}^2 \sin^{2} \phi}
\end{equation}
All quantities to compute the planet flux are obtained from the NASA
Horizons web site \cite{NASAHorizon}, and are
listed in Table~\ref{tab:planetphysparam}. To compute the planet fluxes for a given date, we use the python
photometry package available at \cite{gith-Haussel}.

\begin{table}[ht]
\begin{center}
\begin{tabular}{|c|c|c|}
\hline
     & Uranus & Neptune \\
\hline
$r_{e}$ [km]  & 25559 & 24764 \\ 
\hline
$r_{p}$ [km]  & 24973 & 24341  \\
\hline
$\phi$         & Ob-lat & Ob-lat \\
\hline
$D$   [AU]    & delta   & delta \\
\hline
\end{tabular}
\end{center}
\caption[Primary calibrator flux models]{Physical quantities used for the Uranus and Neptune fluxes
  computation (equation~\ref{eq:omega}. Ob-lat and delta are quantities 
  tabulated by NASA Horizons system \cite{NASAHorizon} as a function of the date}
\label{tab:planetphysparam}
\end{table}


The model spectra are linearly interpolated in log space at the
reference frequencies of the NIKA2 bandpasses. Fluxes for all NIKA2
calibration runs are listed in Table~\ref{tab:fluxPred}, together with
the expected variation between the start and end of a run. 

The Uranus and Neptune models have been compared to 
observations of these planets with the \emph{Planck}
satellite~\cite{PLCK-LII}.
For Uranus, the model used in the comparison
is the ESA V2, and it is found to overpredict by 4 K (about 4\%) the
observed RJ temperature at 143~GHz, to agree at 217 GHz, and
to underpredict at 353 GHz. We use for NIKA2 calibration ESA model V4,
that predict a flux respectively -3.3\%, 0.3\% and 4.7\% higher in the
143, 217 and 353 GHz bands, that would lead to a few percents accuracy
with respect to \emph{Planck} observations.


For Neptune, the same study compared Planck observation with the ESA V5
model, {\it i. e.} the same one used for NIKA2 calibration. For this
planet, temperatures are found to disagree at most by 5 K, i.e 4.1\%,
with the same trend with frequency as observed for Uranus. All thing
considered, this study confirm that Uranus ESA V4 and Neptune ESA V5
models are accurate to 5\% for predicting planet fluxes. Calibration
values tabulated in Table~\ref{tab:fluxPred} show that the variations
of Uranus and Neptune over the duration of a typical NIKA2 run are
negligible compared to the model accuracy. On the other hand, not
taking into account the planet shape and orientation with respect to
the observer in the computations of its solid angle can lead to errors
between 1 and 2\% as illustrated in the Python notebook
\cite{gith-Haussel-Note}
distributed with the software.


\subsection{Secondary calibrators}
\label{se:ref_flux_secondaries}


The secondary calibrator MWC349 consists of a stellar
binary system, including the young Be star MWC349A, surrounded by a
disk. Its radio continuum emission
originates in an ionized bipolar outflow~\cite{Tafoya}.  MWC349A has
been monitored with the Plateau de Bure interferometer and VLA, and
shown to be stable in time and only slightly angularly resolved,
making it a point source for the 30-metre telescope.
%The SED of MWC349A~\cite{krips} is presented in
%Fig.~\ref{fig:Krips2017}.
We have computed the flux densities at the NIKA2 reference frequencies 150 and
260 GHz with $S_\nu = 1.16\pm0.01 \times
(\nu/100 \rm{GHz})^{0.60\pm0.01}$ provided by this
monitoring~\cite{krips}.

%\noindent {\bf FM: MWC349A or MWC349 ?}\\

The secondary calibrator CRL2688 is an Asympotic Giant Branch
star. Its radio continuum emission is mostly from circumstellar dust
and is somewhat extended \cite{Knapp}.  Its flux densities at
$850\ \mu$m and $450 \ \mu$m have been stable at the 5\% level as
monitored by SCUBA2 in 2011-2012~\citep{Dempsey2013_SCUBA2}.
We have extrapolated these flux densities to 150 and 260 GHz
with the power law $S_{\nu} \propto \nu^{\alpha}$ and index
$\alpha=2.44\pm0.18$ derived from the SCUBA2 measurements.

% cahier I p. 128 et 129.

The secondary calibrator NGC7027 is a young, dusty, carbon rich
Planetary Nebula with an ionized core.  It is extended in the
continuum and molecular lines~\cite{Bieging1991}, and is not a point
source for the 30-metre telescope.  Its most recent flux densities are
reported at $1100\mu$m and $2000\mu$m in~\cite{Hoare1992}. It has
been reported to decrease by $\sim$ 0.145 percent/yr in the optically
thin part of its spectrum above $6$~GHz from VLA
observations (\cite{Zijlstra2008}~and~\cite{Hafez2008}) that makes
these flux densities uncertain by 3.6\% currently. Its SED from cm
wavelengths to optical is also presented in~\cite{Hafez2008}.
Its flux densities have been extrapolated to 150 and 260 GHz and the
modeled decrease since 1992 has been included.

All these expected flux densities extrapolated from the literature are in Table~\ref{tab:flux_ref_sec}.

%\begin{figure}[h]
%\begin{center}
%  \includegraphics[clip, angle=0, scale=0.4]{Figures/MWC_349_pap-flux.eps}
%  \caption[MWC349 spectral energy density]{SED of MWC349A from its flux density monitored at PdBI and VLA \cite{krips}.
%  Symbols are for primary calibrators used (Uranus, Neptune and Mars).}
%\label{fig:Krips2017}
%\end{center}
%\end{figure}

\begin{table*}[!thbp]
  \caption[Reference flux densities of secondary calibrators]{Reference flux densities of secondary calibrators at the NIKA2 reference frequencies 150 and 260 GHz. Uncertainties of flux densities extrapolated
    at 150 and 260 GHz include contribution of the uncertainty on
    $\alpha$.}
  \label{tab:flux_ref_sec}
  \centering    
  \begin{tabular}{|l|c|c|c|l|}
    \hline\hline
    \multicolumn{1}{|c}{}  & \multicolumn{3}{|c}{flux  densities (Jy)} & \multicolumn{1}{|c|}{}  \\
    \hline
    &    A1 \& A3       &  A2             &            &   Ref. \\
    &  260 GHz          &  150 GHz        & $\alpha$\tablefootmark{i} &      \\
    \hline
    MWC349A   &   $2.06\pm0.04$  &  $1.48\pm0.02$ &  $+0.60\pm0.01$      &  PdB \cite{krips}    \\
    NGC7027  &   $3.46\pm0.11$   &  $4.26\pm0.24$  &  $-0.34\pm0.10$     &  Hoare et al 1992 \cite{Hoare1992}      \\
    CRL2688  &   $2.91\pm0.23$   &  $0.76\pm0.14$  &  $+2.44\pm0.18$     &  \citet{Dempsey2013_SCUBA2} \\
    \hline
  \end{tabular}
  \tablefoot{ \\
    \tablefoottext{i}{Spectral index defined as $S_{\nu} \propto \nu^{\alpha}$}
  }
\end{table*}

%\addparag{Color correction}

Measured flux densities however are determined over the broad
bandwidth of each array and so must be color-corrected to be compared
to the expected flux densities of Table~\ref{tab:flux_ref_sec}.  For
this purpose, we have derived color-corrections for sources with spectral
indices $\alpha$ comprised between -2 and +4 in Table~\ref{tab:mod} of
Sect.~\ref{se:color_correction}. %\ref{ap:color_correction_JFL}.
As it can be seen, this effect can be a few \% for MWC349, NGC7027, and CDL2688.




\subsection{Uranus and Neptune}
\label{se:ref_flux_uranus_neptune}

For the flux densities of the giant planets, we use the ESA model from
\cite{ESAmodel}: Version 5 for Neptune and Version 4 for Uranus. 
Both models provide the planet brightness temperature in the
Rayleigh-Jeans approximation as a function of the frequency. The
resulting flux is therefore: 
\begin{equation}
S_{\nu} = \Omega \times \frac{2 \nu^{2} k T_{RJ}}{c^2}
\end{equation}
where $\Omega$ is the solid angle of the planet on the sky. Following
\cite{bendo2013} and correcting their equation 12 we have:
%
\begin{equation}
\Omega = \pi \frac{r_{e} r_{p-a}}{D^{2}} 
\label{eq:omega}
\end{equation}
where $r_{e}$ is the equatorial radius of the planet and $r_{p-a}$ is
its apparent polar radius, and $D$ the distance to the
planet. $r_{p-a}$ can be computed from the sub-observer latitude $\phi$
({\it e.g.} the latitude of the 30-m telescope as seen from the planet in the
planet equatorial reference frame) and $r_{p}$ the polar radius of the
planet as:
\begin{equation}
r_{p-a} = \sqrt{r_{p}^2 \cos^{2}\phi + r_{e}^2 \sin^{2} \phi}
\end{equation}
All quantities to compute the planet flux are obtained from the NASA
Horizons web site \cite{NASAHorizon}, and are
listed in Table~\ref{tab:planetphysparam}. To compute the planet fluxes for a given date, we use the python
photometry package available at \cite{gith-Haussel}.

\begin{table}[ht]
\begin{center}
\begin{tabular}{|c|c|c|}
\hline
     & Uranus & Neptune \\
\hline
$r_{e}$ [km]  & 25559 & 24764 \\ 
\hline
$r_{p}$ [km]  & 24973 & 24341  \\
\hline
$\phi$         & Ob-lat & Ob-lat \\
\hline
$D$   [AU]    & delta   & delta \\
\hline
\end{tabular}
\end{center}
\caption[Primary calibrator flux models]{Physical quantities used for the Uranus and Neptune fluxes
  computation (equation~\ref{eq:omega}. Ob-lat and delta are quantities 
  tabulated by NASA Horizons system \cite{NASAHorizon} as a function of the date}
\label{tab:planetphysparam}
\end{table}


The model spectra are linearly interpolated in log space at the
reference frequencies of the NIKA2 bandpasses. Fluxes for all NIKA2
calibration runs are listed in Table~\ref{tab:fluxPred}, together with
the expected variation between the start and end of a run. 

The Uranus and Neptune models have been compared to 
observations of these planets with the \emph{Planck}
satellite~\cite{PLCK-LII}.
For Uranus, the model used in the comparison
is the ESA V2, and it is found to overpredict by 4 K (about 4\%) the
observed RJ temperature at 143~GHz, to agree at 217 GHz, and
to underpredict at 353 GHz. We use for NIKA2 calibration ESA model V4,
that predict a flux respectively -3.3\%, 0.3\% and 4.7\% higher in the
143, 217 and 353 GHz bands, that would lead to a few percents accuracy
with respect to \emph{Planck} observations.


For Neptune, the same study compared Planck observation with the ESA V5
model, {\it i. e.} the same one used for NIKA2 calibration. For this
planet, temperatures are found to disagree at most by 5 K, i.e 4.1\%,
with the same trend with frequency as observed for Uranus. All thing
considered, this study confirm that Uranus ESA V4 and Neptune ESA V5
models are accurate to 5\% for predicting planet fluxes. Calibration
values tabulated in Table~\ref{tab:fluxPred} show that the variations
of Uranus and Neptune over the duration of a typical NIKA2 run are
negligible compared to the model accuracy. On the other hand, not
taking into account the planet shape and orientation with respect to
the observer in the computations of its solid angle can lead to errors
between 1 and 2\% as illustrated in the Python notebook
\cite{gith-Haussel-Note}
distributed with the software.


\subsection{Secondary calibrators}
\label{se:ref_flux_secondaries}


The secondary calibrator MWC349 consists of a stellar
binary system, including the young Be star MWC349A, surrounded by a
disk. Its radio continuum emission
originates in an ionized bipolar outflow~\cite{Tafoya}.  MWC349A has
been monitored with the Plateau de Bure interferometer and VLA, and
shown to be stable in time and only slightly angularly resolved,
making it a point source for the 30-metre telescope.
%The SED of MWC349A~\cite{krips} is presented in
%Fig.~\ref{fig:Krips2017}.
We have computed the flux densities at the NIKA2 reference frequencies 150 and
260 GHz with $S_\nu = 1.16\pm0.01 \times
(\nu/100 \rm{GHz})^{0.60\pm0.01}$ provided by this
monitoring~\cite{krips}.

%\noindent {\bf FM: MWC349A or MWC349 ?}\\

The secondary calibrator CRL2688 is an Asympotic Giant Branch
star. Its radio continuum emission is mostly from circumstellar dust
and is somewhat extended \cite{Knapp}.  Its flux densities at
$850\ \mu$m and $450 \ \mu$m have been stable at the 5\% level as
monitored by SCUBA2 in 2011-2012~\citep{Dempsey2013_SCUBA2}.
We have extrapolated these flux densities to 150 and 260 GHz
with the power law $S_{\nu} \propto \nu^{\alpha}$ and index
$\alpha=2.44\pm0.18$ derived from the SCUBA2 measurements.

% cahier I p. 128 et 129.

The secondary calibrator NGC7027 is a young, dusty, carbon rich
Planetary Nebula with an ionized core.  It is extended in the
continuum and molecular lines~\cite{Bieging1991}, and is not a point
source for the 30-metre telescope.  Its most recent flux densities are
reported at $1100\mu$m and $2000\mu$m in~\cite{Hoare1992}. It has
been reported to decrease by $\sim$ 0.145 percent/yr in the optically
thin part of its spectrum above $6$~GHz from VLA
observations (\cite{Zijlstra2008}~and~\cite{Hafez2008}) that makes
these flux densities uncertain by 3.6\% currently. Its SED from cm
wavelengths to optical is also presented in~\cite{Hafez2008}.
Its flux densities have been extrapolated to 150 and 260 GHz and the
modeled decrease since 1992 has been included.

All these expected flux densities extrapolated from the literature are in Table~\ref{tab:flux_ref_sec}.

%\begin{figure}[h]
%\begin{center}
%  \includegraphics[clip, angle=0, scale=0.4]{Figures/MWC_349_pap-flux.eps}
%  \caption[MWC349 spectral energy density]{SED of MWC349A from its flux density monitored at PdBI and VLA \cite{krips}.
%  Symbols are for primary calibrators used (Uranus, Neptune and Mars).}
%\label{fig:Krips2017}
%\end{center}
%\end{figure}

\begin{table*}[!thbp]
  \caption[Reference flux densities of secondary calibrators]{Reference flux densities of secondary calibrators at the NIKA2 reference frequencies 150 and 260 GHz. Uncertainties of flux densities extrapolated
    at 150 and 260 GHz include contribution of the uncertainty on
    $\alpha$.}
  \label{tab:flux_ref_sec}
  \centering    
  \begin{tabular}{|l|c|c|c|l|}
    \hline\hline
    \multicolumn{1}{|c}{}  & \multicolumn{3}{|c}{flux  densities (Jy)} & \multicolumn{1}{|c|}{}  \\
    \hline
    &    A1 \& A3       &  A2             &            &   Ref. \\
    &  260 GHz          &  150 GHz        & $\alpha$\tablefootmark{i} &      \\
    \hline
    MWC349A   &   $2.06\pm0.04$  &  $1.48\pm0.02$ &  $+0.60\pm0.01$      &  PdB \cite{krips}    \\
    NGC7027  &   $3.46\pm0.11$   &  $4.26\pm0.24$  &  $-0.34\pm0.10$     &  Hoare et al 1992 \cite{Hoare1992}      \\
    CRL2688  &   $2.91\pm0.23$   &  $0.76\pm0.14$  &  $+2.44\pm0.18$     &  \citet{Dempsey2013_SCUBA2} \\
    \hline
  \end{tabular}
  \tablefoot{ \\
    \tablefoottext{i}{Spectral index defined as $S_{\nu} \propto \nu^{\alpha}$}
  }
\end{table*}

%\addparag{Color correction}

Measured flux densities however are determined over the broad
bandwidth of each array and so must be color-corrected to be compared
to the expected flux densities of Table~\ref{tab:flux_ref_sec}.  For
this purpose, we have derived color-corrections for sources with spectral
indices $\alpha$ comprised between -2 and +4 in Table~\ref{tab:mod} of
Sect.~\ref{se:color_correction}. %\ref{ap:color_correction_JFL}.
As it can be seen, this effect can be a few \% for MWC349, NGC7027, and CDL2688.




\subsection{Uranus and Neptune}
\label{se:ref_flux_uranus_neptune}

For the flux densities of the giant planets, we use the ESA model from
\cite{ESAmodel}: Version 5 for Neptune and Version 4 for Uranus. 
Both models provide the planet brightness temperature in the
Rayleigh-Jeans approximation as a function of the frequency. The
resulting flux is therefore: 
\begin{equation}
S_{\nu} = \Omega \times \frac{2 \nu^{2} k T_{RJ}}{c^2}
\end{equation}
where $\Omega$ is the solid angle of the planet on the sky. Following
\cite{bendo2013} and correcting their equation 12 we have:
%
\begin{equation}
\Omega = \pi \frac{r_{e} r_{p-a}}{D^{2}} 
\label{eq:omega}
\end{equation}
where $r_{e}$ is the equatorial radius of the planet and $r_{p-a}$ is
its apparent polar radius, and $D$ the distance to the
planet. $r_{p-a}$ can be computed from the sub-observer latitude $\phi$
({\it e.g.} the latitude of the 30-m telescope as seen from the planet in the
planet equatorial reference frame) and $r_{p}$ the polar radius of the
planet as:
\begin{equation}
r_{p-a} = \sqrt{r_{p}^2 \cos^{2}\phi + r_{e}^2 \sin^{2} \phi}
\end{equation}
All quantities to compute the planet flux are obtained from the NASA
Horizons web site \cite{NASAHorizon}, and are
listed in Table~\ref{tab:planetphysparam}. To compute the planet fluxes for a given date, we use the python
photometry package available at \cite{gith-Haussel}.

\begin{table}[ht]
\begin{center}
\begin{tabular}{|c|c|c|}
\hline
     & Uranus & Neptune \\
\hline
$r_{e}$ [km]  & 25559 & 24764 \\ 
\hline
$r_{p}$ [km]  & 24973 & 24341  \\
\hline
$\phi$         & Ob-lat & Ob-lat \\
\hline
$D$   [AU]    & delta   & delta \\
\hline
\end{tabular}
\end{center}
\caption[Primary calibrator flux models]{Physical quantities used for the Uranus and Neptune fluxes
  computation (equation~\ref{eq:omega}. Ob-lat and delta are quantities 
  tabulated by NASA Horizons system \cite{NASAHorizon} as a function of the date}
\label{tab:planetphysparam}
\end{table}


The model spectra are linearly interpolated in log space at the
reference frequencies of the NIKA2 bandpasses. Fluxes for all NIKA2
calibration runs are listed in Table~\ref{tab:fluxPred}, together with
the expected variation between the start and end of a run. 

The Uranus and Neptune models have been compared to 
observations of these planets with the \emph{Planck}
satellite~\cite{PLCK-LII}.
For Uranus, the model used in the comparison
is the ESA V2, and it is found to overpredict by 4 K (about 4\%) the
observed RJ temperature at 143~GHz, to agree at 217 GHz, and
to underpredict at 353 GHz. We use for NIKA2 calibration ESA model V4,
that predict a flux respectively -3.3\%, 0.3\% and 4.7\% higher in the
143, 217 and 353 GHz bands, that would lead to a few percents accuracy
with respect to \emph{Planck} observations.


For Neptune, the same study compared Planck observation with the ESA V5
model, {\it i. e.} the same one used for NIKA2 calibration. For this
planet, temperatures are found to disagree at most by 5 K, i.e 4.1\%,
with the same trend with frequency as observed for Uranus. All thing
considered, this study confirm that Uranus ESA V4 and Neptune ESA V5
models are accurate to 5\% for predicting planet fluxes. Calibration
values tabulated in Table~\ref{tab:fluxPred} show that the variations
of Uranus and Neptune over the duration of a typical NIKA2 run are
negligible compared to the model accuracy. On the other hand, not
taking into account the planet shape and orientation with respect to
the observer in the computations of its solid angle can lead to errors
between 1 and 2\% as illustrated in the Python notebook
\cite{gith-Haussel-Note}
distributed with the software.


\subsection{Secondary calibrators}
\label{se:ref_flux_secondaries}


The secondary calibrator MWC349 consists of a stellar
binary system, including the young Be star MWC349A, surrounded by a
disk. Its radio continuum emission
originates in an ionized bipolar outflow~\cite{Tafoya}.  MWC349A has
been monitored with the Plateau de Bure interferometer and VLA, and
shown to be stable in time and only slightly angularly resolved,
making it a point source for the 30-metre telescope.
%The SED of MWC349A~\cite{krips} is presented in
%Fig.~\ref{fig:Krips2017}.
We have computed the flux densities at the NIKA2 reference frequencies 150 and
260 GHz with $S_\nu = 1.16\pm0.01 \times
(\nu/100 \rm{GHz})^{0.60\pm0.01}$ provided by this
monitoring~\cite{krips}.

%\noindent {\bf FM: MWC349A or MWC349 ?}\\

The secondary calibrator CRL2688 is an Asympotic Giant Branch
star. Its radio continuum emission is mostly from circumstellar dust
and is somewhat extended \cite{Knapp}.  Its flux densities at
$850\ \mu$m and $450 \ \mu$m have been stable at the 5\% level as
monitored by SCUBA2 in 2011-2012~\citep{Dempsey2013_SCUBA2}.
We have extrapolated these flux densities to 150 and 260 GHz
with the power law $S_{\nu} \propto \nu^{\alpha}$ and index
$\alpha=2.44\pm0.18$ derived from the SCUBA2 measurements.

% cahier I p. 128 et 129.

The secondary calibrator NGC7027 is a young, dusty, carbon rich
Planetary Nebula with an ionized core.  It is extended in the
continuum and molecular lines~\cite{Bieging1991}, and is not a point
source for the 30-metre telescope.  Its most recent flux densities are
reported at $1100\mu$m and $2000\mu$m in~\cite{Hoare1992}. It has
been reported to decrease by $\sim$ 0.145 percent/yr in the optically
thin part of its spectrum above $6$~GHz from VLA
observations (\cite{Zijlstra2008}~and~\cite{Hafez2008}) that makes
these flux densities uncertain by 3.6\% currently. Its SED from cm
wavelengths to optical is also presented in~\cite{Hafez2008}.
Its flux densities have been extrapolated to 150 and 260 GHz and the
modeled decrease since 1992 has been included.

All these expected flux densities extrapolated from the literature are in Table~\ref{tab:flux_ref_sec}.

%\begin{figure}[h]
%\begin{center}
%  \includegraphics[clip, angle=0, scale=0.4]{Figures/MWC_349_pap-flux.eps}
%  \caption[MWC349 spectral energy density]{SED of MWC349A from its flux density monitored at PdBI and VLA \cite{krips}.
%  Symbols are for primary calibrators used (Uranus, Neptune and Mars).}
%\label{fig:Krips2017}
%\end{center}
%\end{figure}

\begin{table*}[!thbp]
  \caption[Reference flux densities of secondary calibrators]{Reference flux densities of secondary calibrators at the NIKA2 reference frequencies 150 and 260 GHz. Uncertainties of flux densities extrapolated
    at 150 and 260 GHz include contribution of the uncertainty on
    $\alpha$.}
  \label{tab:flux_ref_sec}
  \centering    
  \begin{tabular}{|l|c|c|c|l|}
    \hline\hline
    \multicolumn{1}{|c}{}  & \multicolumn{3}{|c}{flux  densities (Jy)} & \multicolumn{1}{|c|}{}  \\
    \hline
    &    A1 \& A3       &  A2             &            &   Ref. \\
    &  260 GHz          &  150 GHz        & $\alpha$\tablefootmark{i} &      \\
    \hline
    MWC349A   &   $2.06\pm0.04$  &  $1.48\pm0.02$ &  $+0.60\pm0.01$      &  PdB \cite{krips}    \\
    NGC7027  &   $3.46\pm0.11$   &  $4.26\pm0.24$  &  $-0.34\pm0.10$     &  Hoare et al 1992 \cite{Hoare1992}      \\
    CRL2688  &   $2.91\pm0.23$   &  $0.76\pm0.14$  &  $+2.44\pm0.18$     &  \citet{Dempsey2013_SCUBA2} \\
    \hline
  \end{tabular}
  \tablefoot{ \\
    \tablefoottext{i}{Spectral index defined as $S_{\nu} \propto \nu^{\alpha}$}
  }
\end{table*}

%\addparag{Color correction}

Measured flux densities however are determined over the broad
bandwidth of each array and so must be color-corrected to be compared
to the expected flux densities of Table~\ref{tab:flux_ref_sec}.  For
this purpose, we have derived color-corrections for sources with spectral
indices $\alpha$ comprised between -2 and +4 in Table~\ref{tab:mod} of
Sect.~\ref{se:color_correction}. %\ref{ap:color_correction_JFL}.
As it can be seen, this effect can be a few \% for MWC349, NGC7027, and CDL2688.




\subsection{Uranus and Neptune}
\label{se:ref_flux_uranus_neptune}

For the flux densities of the giant planets, we use the ESA model from
\cite{ESAmodel}: Version 5 for Neptune and Version 4 for Uranus. 
Both models provide the planet brightness temperature in the
Rayleigh-Jeans approximation as a function of the frequency. The
resulting flux is therefore: 
\begin{equation}
S_{\nu} = \Omega \times \frac{2 \nu^{2} k T_{RJ}}{c^2}
\end{equation}
where $\Omega$ is the solid angle of the planet on the sky. Following
\cite{bendo2013} and correcting their equation 12 we have:
%
\begin{equation}
\Omega = \pi \frac{r_{e} r_{p-a}}{D^{2}} 
\label{eq:omega}
\end{equation}
where $r_{e}$ is the equatorial radius of the planet and $r_{p-a}$ is
its apparent polar radius, and $D$ the distance to the
planet. $r_{p-a}$ can be computed from the sub-observer latitude $\phi$
({\it e.g.} the latitude of the 30-m telescope as seen from the planet in the
planet equatorial reference frame) and $r_{p}$ the polar radius of the
planet as:
\begin{equation}
r_{p-a} = \sqrt{r_{p}^2 \cos^{2}\phi + r_{e}^2 \sin^{2} \phi}
\end{equation}
All quantities to compute the planet flux are obtained from the NASA
Horizons web site \cite{NASAHorizon}, and are
listed in Table~\ref{tab:planetphysparam}. To compute the planet fluxes for a given date, we use the python
photometry package available at \cite{gith-Haussel}.

\begin{table}[ht]
\begin{center}
\begin{tabular}{|c|c|c|}
\hline
     & Uranus & Neptune \\
\hline
$r_{e}$ [km]  & 25559 & 24764 \\ 
\hline
$r_{p}$ [km]  & 24973 & 24341  \\
\hline
$\phi$         & Ob-lat & Ob-lat \\
\hline
$D$   [AU]    & delta   & delta \\
\hline
\end{tabular}
\end{center}
\caption[Primary calibrator flux models]{Physical quantities used for the Uranus and Neptune fluxes
  computation (equation~\ref{eq:omega}. Ob-lat and delta are quantities 
  tabulated by NASA Horizons system \cite{NASAHorizon} as a function of the date}
\label{tab:planetphysparam}
\end{table}


The model spectra are linearly interpolated in log space at the
reference frequencies of the NIKA2 bandpasses. Fluxes for all NIKA2
calibration runs are listed in Table~\ref{tab:fluxPred}, together with
the expected variation between the start and end of a run. 

The Uranus and Neptune models have been compared to 
observations of these planets with the \emph{Planck}
satellite~\cite{PLCK-LII}.
For Uranus, the model used in the comparison
is the ESA V2, and it is found to overpredict by 4 K (about 4\%) the
observed RJ temperature at 143~GHz, to agree at 217 GHz, and
to underpredict at 353 GHz. We use for NIKA2 calibration ESA model V4,
that predict a flux respectively -3.3\%, 0.3\% and 4.7\% higher in the
143, 217 and 353 GHz bands, that would lead to a few percents accuracy
with respect to \emph{Planck} observations.


For Neptune, the same study compared Planck observation with the ESA V5
model, {\it i. e.} the same one used for NIKA2 calibration. For this
planet, temperatures are found to disagree at most by 5 K, i.e 4.1\%,
with the same trend with frequency as observed for Uranus. All thing
considered, this study confirm that Uranus ESA V4 and Neptune ESA V5
models are accurate to 5\% for predicting planet fluxes. Calibration
values tabulated in Table~\ref{tab:fluxPred} show that the variations
of Uranus and Neptune over the duration of a typical NIKA2 run are
negligible compared to the model accuracy. On the other hand, not
taking into account the planet shape and orientation with respect to
the observer in the computations of its solid angle can lead to errors
between 1 and 2\% as illustrated in the Python notebook
\cite{gith-Haussel-Note}
distributed with the software.


\subsection{Secondary calibrators}
\label{se:ref_flux_secondaries}


The secondary calibrator MWC349 consists of a stellar
binary system, including the young Be star MWC349A, surrounded by a
disk. Its radio continuum emission
originates in an ionized bipolar outflow~\cite{Tafoya}.  MWC349A has
been monitored with the Plateau de Bure interferometer and VLA, and
shown to be stable in time and only slightly angularly resolved,
making it a point source for the 30-metre telescope.
%The SED of MWC349A~\cite{krips} is presented in
%Fig.~\ref{fig:Krips2017}.
We have computed the flux densities at the NIKA2 reference frequencies 150 and
260 GHz with $S_\nu = 1.16\pm0.01 \times
(\nu/100 \rm{GHz})^{0.60\pm0.01}$ provided by this
monitoring~\cite{krips}.

%\noindent {\bf FM: MWC349A or MWC349 ?}\\

The secondary calibrator CRL2688 is an Asympotic Giant Branch
star. Its radio continuum emission is mostly from circumstellar dust
and is somewhat extended \cite{Knapp}.  Its flux densities at
$850\ \mu$m and $450 \ \mu$m have been stable at the 5\% level as
monitored by SCUBA2 in 2011-2012~\cite{Dempsey}.
We have extrapolated these flux densities to 150 and 260 GHz
with the power law $S_{\nu} \propto \nu^{\alpha}$ and index
$\alpha=2.44\pm0.18$ derived from the SCUBA2 measurements.

% cahier I p. 128 et 129.

The secondary calibrator NGC7027 is a young, dusty, carbon rich
Planetary Nebula with an ionized core.  It is extended in the
continuum and molecular lines~\cite{Bieging1991}, and is not a point
source for the 30-metre telescope.  Its most recent flux densities are
reported at $1100\mu$m and $2000\mu$m in~\cite{Hoare1992}. It has
been reported to decrease by $\sim$ 0.145 percent/yr in the optically
thin part of its spectrum above $6$~GHz from VLA
observations (\cite{Zijlstra2008}~and~\cite{Hafez2008}) that makes
these flux densities uncertain by 3.6\% currently. Its SED from cm
wavelengths to optical is also presented in~\cite{Hafez2008}.
Its flux densities have been extrapolated to 150 and 260 GHz and the
modeled decrease since 1992 has been included.

All these expected flux densities extrapolated from the literature are in Table~\ref{tab:flux_ref_sec}.

%\begin{figure}[h]
%\begin{center}
%  \includegraphics[clip, angle=0, scale=0.4]{Figures/MWC_349_pap-flux.eps}
%  \caption[MWC349 spectral energy density]{SED of MWC349A from its flux density monitored at PdBI and VLA \cite{krips}.
%  Symbols are for primary calibrators used (Uranus, Neptune and Mars).}
%\label{fig:Krips2017}
%\end{center}
%\end{figure}

\begin{table*}[!thbp]
  \caption[Reference flux densities of secondary calibrators]{Reference flux densities of secondary calibrators at the NIKA2 reference frequencies 150 and 260 GHz. Uncertainties of flux densities extrapolated
    at 150 and 260 GHz include contribution of the uncertainty on
    $\alpha$.}
  \label{tab:flux_ref_sec}
  \centering    
  \begin{tabular}{|l|c|c|c|l|}
    \hline\hline
    \multicolumn{1}{|c}{}  & \multicolumn{3}{|c}{flux  densities (Jy)} & \multicolumn{1}{|c|}{}  \\
    \hline
    &    A1 \& A3       &  A2             &            &   Ref. \\
    &  260 GHz          &  150 GHz        & $\alpha$\tablefootmark{i} &      \\
    \hline
    MWC349A   &   $2.06\pm0.04$  &  $1.48\pm0.02$ &  $+0.60\pm0.01$      &  PdB \cite{krips}    \\
    NGC7027  &   $3.46\pm0.11$   &  $4.26\pm0.24$  &  $-0.34\pm0.10$     &  Hoare et al 1992 \cite{Hoare1992}      \\
    CRL2688  &   $2.91\pm0.23$   &  $0.76\pm0.14$  &  $+2.44\pm0.18$     &  Dempsey et al 2013  \cite{Dempsey} \\
    \hline
  \end{tabular}
  \tablefoot{ \\
    \tablefoottext{i}{Spectral index defined as $S_{\nu} \propto \nu^{\alpha}$}
  }
\end{table*}

%\addparag{Color correction}

Measured flux densities however are determined over the broad
bandwidth of each array and so must be color-corrected to be compared
to the expected flux densities of Table~\ref{tab:flux_ref_sec}.  For
this purpose, we have derived color-corrections for sources with spectral
indices $\alpha$ comprised between -2 and +4 in Table~\ref{tab:mod} of
Sect.~\ref{se:color_correction}. %\ref{ap:color_correction_JFL}.
As it can be seen, this effect can be a few \% for MWC349, NGC7027, and CDL2688.

