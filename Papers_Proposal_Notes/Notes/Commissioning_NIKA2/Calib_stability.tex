\subsection{Calibration stability across the run}

\begin{figure}[ht]
\begin{center}
\includegraphics[clip, angle=0, scale = 0.7]{Figures/FluxIndScans/flux_ratio_run22.pdf}
\includegraphics[clip, angle=0, scale = 0.7]{Figures/FluxIndScans/flux_ap_ratio_run22.pdf}
\caption{Ratio between the measured flux per scan and the averaged flux for all sources observed in N2R9. We considered both fixed FWHM (top) and aperture photometry fluxes for the 1 (red) and 2 (cyan) mm channels. }
\label{fig:fluxvsscan}
\end{center}
\end{figure}

\begin{figure}[ht]
\begin{center}
\includegraphics[clip, angle=0, scale = 0.7]{Figures/FluxIndScans/flux_ratio_rz_run22.pdf}
\includegraphics[clip, angle=0, scale = 0.7]{Figures/FluxIndScans/flux_ap_ratio_rz_run22.pdf}
\caption{Ratio between the measured flux per scan and the averaged flux as a function atmospheric backgroud for all sources observed in N2R9. We considered both fixed FWHM (top) and aperture photometry fluxes for the 1 (red) and 2 (cyan) mm channels. }
\label{fig:fluxvsbackground}
\end{center}
\end{figure}

\begin{figure}[ht]
\begin{center}
\includegraphics[clip, angle=0, scale = 0.7]{Figures/flux_1mm_ratio_run22_23.pdf}
\includegraphics[clip, angle=0, scale = 0.7]{Figures/flux_1mm_ap_ratio_run22_23.pdf}
\caption{Ratio between the measured flux per scan and the averaged flux for all sources observed in N2R9 and N2R10. We considered both fixed FWHM (top) and aperture photometry fluxes for the 1 mm channel }
\label{fig:fluxvsscan}
\end{center}
\end{figure}

\begin{figure}[ht]
\begin{center}
\includegraphics[clip, angle=0, scale = 0.7]{Figures/flux_2mm_ratio_run22_23.pdf}
\includegraphics[clip, angle=0, scale = 0.7]{Figures/flux_2mm_ap_ratio_run22_23.pdf}
\caption{Ratio between the measured flux per scan and the averaged flux for all sources observed in N2R9 and N2R10. We considered both fixed FWHM (top) and aperture photometry fluxes for the 2 mm channel }
\label{fig:fluxvsscan}
\end{center}
\end{figure}
