%While neither of their individual deviations is greater than $3\sigma$, their combined
%significance is greater than $3\sigma$, and given their spatial coincidence
%We find it plausible that this deviation (at $\sims$12\asec, radially) is due to a weak point source.

%For NIKA, this deviation occurs in the innermost bin, $r < 17$\asec, while MUSTANG is able to localize the
%deviation in a central bin: $9 < r < 23$\asec. To test the potential for a point source to account for these
%deviations, we add a point source to the virtual fits mentioned previously (Section~\ref{sec:robustness}).
%We run a suite of fits, where we place a point source over the radial range implied by MUSTANG and NIKA,
%and allow for different flux densities at 90 and 150 GHz. We find that a point source at a radius of 12\asecs
%with flux densities of 0.5 mJy and 1.4 mJy at 90 and 150 GHz roughly reproduces the observed deviations from
%a gNFW curve.

%Given the area enclosed for the MUSTANG and NIKA bins which
%show this low pressure, such a source would necessarily be $\gtrsim 0.5$ mJy at 90 GHz and 150 GHz.
%The sensitivity in the MUSTANG and NIKA maps are 0.1 and 0.2 mJy respectively. Such a point source,
%with its central location, could be masked by the SZ signal in the NIKA map. Indeed, in our simulated
%images (simulated cluster + point source), the surface brightness at the location of the point source
%is (1) still negative, and (2) not visually evident to host a point source (pertubation to the smooth
%surface brightness from the tSZ of the cluster).

%The location at 12\asecs is well matched to the point source found in \citet{korngut2011}. The potential
%existence of a point source at this location has already been investigated in the 260 GHz NIKA data
%\citep{adam2015}, as well as at lower frequencies and higher frequencies (Section~\ref{sec:preprocessing}).
