\part{L'effet de lentille gravitationnelle dans \emph{Planck}}

%----------------------------------------------------------------------------------------
%
%   
%
%----------------------------------------------------------------------------------------
\chapter{L'effet de lentille gravitationnelle sur le CMB}
{\color{vert}\lipsum[1-2]}
%----------------------------------------------------------------------------------------
%
%   
%
%----------------------------------------------------------------------------------------
\chapter{La reconstruction de l'effet de lentille gravitationnelle}

\section{Méthodologie}
{\color{vert}\lipsum[1-2]}
        
\section{La carte du potentiel gravitationnel intégré}
{\color{vert}\lipsum[1-2]}
        
%  VOIR PROCEEDIND MORIOND 2014 :
%/Users/perotto/MesTrucs/Talks/proceedings/Moriond-2014
%--------------------------------------------------------
\begin{figure}
  \centering
  \includegraphics[width=0.8\linewidth]{Figures/Lensing/estimated_phimap_smica_s_mollweide_blue.pdf}
  \caption[]{Planck's Wiener-filtered gravitational potential map $\phi^{\mathrm{WF}}_{LM} \equiv C_L^{\phi}/N_L^{(0)} \hat{\phi}_{LM}$ using the {\sc smica} foreground cleaned CMB map (left) and the minimum-variance combination of the 143 and 217~GHz maps (right). Whereas noise dominated at all scales, these maps encode the integrated Dark Matter distribution back to $z\sim1100$~\cite{planck2013:lensing}}
  \label{fig:maps}
\end{figure}

\begin{figure}
\hfill
\begin{minipage}{0.45\linewidth}\centerline{
\begin{overpic}[width=1\linewidth, height=5.7cm]{Figures/Lensing/analysis_lens_clpp_resid_pub_top2.pdf}
\put(30, -3){\small{Lensing Multipole $L$}}
\put(-28, 5){\includegraphics[width=0.5\linewidth]{Figures/Lensing/comp_filt_ordo.pdf}}
\end{overpic}}
%\includegraphics[width=1\linewidth]{figs/analysis_lens_clpp_resid_pub_top.pdf}}
\end{minipage}
\hfill
\begin{minipage}{0.45\linewidth}\centerline{
\begin{overpic}[width=1\linewidth, height=5.5cm]{Figures/Lensing/comp_filt_top3.pdf}
\put(30, -4.){\small{Lensing Multipole $L$}}
\put(-28, 5){\includegraphics[width=0.5\linewidth]{Figures/Lensing/comp_filt_ordo.pdf}}
\end{overpic}}
%\centerline{\includegraphics[width=1\linewidth, height=5.5cm]{figs/P12_clpp_allmethods.pdf}}
\end{minipage}
\caption[]{Planck's lensing potential power spectrum for the minimum variance (MV) combination of 143 and 217~GHz (empty black boxes). Left panel shows a comparison to the reconstructions from the ACT~\cite{Das2013} and the SPT~\cite{vanEngelen2012} as well as the Planck's estimates for the 143 and 217~GHz maps. Right panel shows the $C_L^{\phi}$ estimates for the 143 and 217~GHz maps using the four analysis pipelines, based on different methodologies of sky cut treatment, that were developed by the Planck Collab.~\cite{planck2013:lensing} to ensure robustness. }
\label{fig:spectre}
\end{figure}

\section{Implication pour la cosmologie}
{\color{vert}\lipsum[1-2]}
        
%  VOIR PROCEEDIND VIETNAM 2015 :
%/Users/perotto/MesTrucs/Talks/proceedings/Vietnam-2015
%--------------------------------------------------------
\begin{figure}
\begin{minipage}{0.42\linewidth}
\centerline{\includegraphics[width=1\linewidth]{Figures/Lensing/omegak_rmsdeflect.pdf}}
\end{minipage}
\hfill
\begin{minipage}{0.48\linewidth}\centerline{
\begin{overpic}[width=1\linewidth]{Figures/Lensing/DX9_AA_neutrinomasses_delta_88mm_crop.pdf}
\put(-5, -5){\includegraphics[scale=1.2]{Figures/Lensing/DX9_AA_neutrinomasses_delta_88mm_ordo.pdf}}
\end{overpic}}
\end{minipage}
\caption[]{Examples of the lensing reconstruction cosmological implications. \emph{Left}: MCMC samples in the plan of the root mean squared lensing deflection, $<d^2>^{1/2}$, and curvature density, $\Omega_K$, using Planck's $C_\ell^{T}$ and low-$\ell$ polarization (Planck $TT+\rm{low}P$) only. The points are color coded by the value of the Hubble parameter, $H_0$. The confidence intervals from the combination of Planck $TT+\rm{low}P$ and lensing (in black), show that closed-universe models, which are associated to low values of $H_0$, are strongly suppressed since they predict too much lensing power. Including the lensing (colored contours) also provides additional information to the combination of Planck $TT+\rm{low}P$ and BAO measurements (red contours). See the Planck Collab.~\cite{planck2015:lensing} for further discussions. \emph{Right}: $C_L^{\phi}$ calculated for various values of the total (active) neutrino mass $M_\nu$ ranging from 0 to 1~eV divided by the $C_L^{\phi}$ in the mass-less neutrino case. Planck 2013 $C_L^{\phi}$ measure divided by the $M_\nu=0$ model is plotted to give a hint of the sensitivity to $M_\nu$.}
\label{fig:cosmo}
\end{figure}

%----------------------------------------------------------------------------------------
%
%   
%
%----------------------------------------------------------------------------------------
\chapter{Les modes B de polarisation induits par l'effet de lentille}

{\color{vert}\lipsum[2-5]}

\begin{figure}
\begin{overpic}[width=0.46\linewidth]{Figures/Lensing/Fig5_smica_B_lowres_beamcorr_gauss60_256_masklens_0p75_88mm.pdf}
\end{overpic}
%\begin{minipage}{0.45\linewidth}
%\centerline{\includegraphics[width=1\linewidth]{figs/Fig5_smica_B_highres_beamcorr_gauss10_1024_masklens_88mm.pdf}}
%\end{minipage}
\hfill
%\begin{minipage}{0.45\linewidth}
\begin{overpic}[width=0.46\linewidth]{Figures/Lensing/Fig9_10_consistency_to_external_bkonly_88mm.pdf}
      %\put(50.5,54){\Large{$\}$}}
      %\put(50.5,43){\Huge{$\}$}}
      \put(48,54.5){$\left.
            \begin{array}{@{}ll@{}}
              \\
               
            \end{array}\right\}$} 
      \put(48,45){\small{$\left.
            \begin{array}{@{}ll@{}}
              \\
              \\
 
            \end{array}\right\}$}}
    \end{overpic}
%\centerline{\includegraphics[width=1\linewidth]{figs/Fig9_10_consistency_to_external_bkonly_88mm.pdf}}
%\end{minipage}
\caption[]{Lensing-induced B-mode results. \emph{Left}: Planck's map
  of the lensing $B$-modes, which has been smoothed to highlight
  angular scales greater or similar to one degree. No spurious features are evidence at the angular scales relevant for primordial $B$-modes. The grey area shows the aggressive mask that preserves $80\%$ of the sky. \emph{Right}: The Planck-derived lensing $B$-modes band-powers (red points) are consistent with the indirect (cross-correlation) measurements from SPTpol (green points), as well as with direct (total CMB $B$-modes) measurements from the BICEP2/Keck Array/Planck joint analysis(dark blue boxes), POLARBEAR (light blue boxes) and SPTpol (yellow boxes).}
\label{fig:Bmap}
\end{figure}

{\color{vert}\lipsum[6-9]}

