

%\section{Goals of the total power commissioning}

\begin{table}[ht]
\begin{center}
\begin{threeparttable}
\begin{tabular}{|r|c|c|}
  \hline
  \hline
Reference Wavelength  [mm]  &  1.2 & 2.0  \\
Reference Frequency  [GHz]  &  260 & 150  \\
\hline  
\hline
FOV diameter [arcmin]       &  5 (6.5)    &  5 (6.5)   \\
Pixel size in beam sampling unit [F$\lambda$]  &  0.9 (0.6)   &   0.9 (0.6)  \\
FWHM  [arcsec]              &  12 (10)   &  18 (16) \\
Fraction of valid detectors [$\%$] &  50 (90)   &  50 (90) \\
NEFD\tnote{a}\hspace{1mm}   [$\rm{mJy} \cdot \rm{s}^{1/2}/\rm{beam}$]  &  30 (15)   &  20 (10) \\
\hline
NEFD [$\rm{mJy} \cdot \rm{s}^{1/2}/\rm{beam}$] goal on $90\%$ of the pixels  &  15  & 10 \\
NEFD [$\rm{mJy} \cdot \rm{s}^{1/2}/\rm{beam}$] specification on $50\%$ of the pixels  &  30  &  20  \\
\hline
\end{tabular}
\begin{tablenotes}
{\small
  \item[(a)] NEFD in typical IRAM good sky opacity condition: 2 mm pwv, $60^\circ$ elevation
%  \tablefoottext{a}{NEFD in typical IRAM good sky opacity condition: 2mm pwv, $60^o$ elevation}
}
\end{tablenotes}
\end{threeparttable}
\caption[Main characteristics defining the expected performances of NIKA2.]{Each parameter is associated with two values: the first one indicates the \emph{specifications}, i.e. the requirements to be met by the instrument, while the second bracketed one gives the \emph{goals}, i. e. the values targeted by the collaboration.}
\label{nika2specs}
\end{center}
\end{table} 

\begin{table}[h]
\small
\begin{center} 
\begin{tabular}{|c|c|c|c|c|}
\hline 
RUN  & NIKA Run & Starting date    & End date         &  General comments \\
\hline
N2R1     & 13       & 29-October-2015   & 10-November-2015 & Not full instrumentation        \\
N2R2     & 14       & 24-November-2015  & 02-December-2015 & 13 NIKEL boards working         \\
N2R3     & 15       & 12-January-2016   & 01-February-2016 & 20 NIKEL boards                 \\
N2R4     & 16       & 1-March-2016      & 15-March-2016    & 	                               \\
Dark run & 17       & 4-May-2016        & 4-May-2016       & Dark tests with N2R4 conditions  \\
\hline
N2R5     & 18       & 16-September-2016 & 11-October-2016  & New dichroic, corrugated lenses, \\
         &          &                   &                  &  new array 2 mm, new electronics \\
N2R6     & 19       & 25-October-2016   & 1-November-2016  &                                  \\
N2R7     & 20       & 6-December-2016   & 13-December-2016 & Test external calibrator         \\
N2R8     & 21       & 9-January-2017    & 13-January-2017  & Replace array 1 lens by smooth one, \\
         &          &                   &                  &  adjust the alignement of internal optics \\ 
         &          & 24-January-2017   & 25-January-2017  & Tests on the sky   \\
N2R9     & 22       & 21-February-2017  & 28-February-2017 &                                   \\
N2R10    & 23       & 18-April-2017     & 25-April-2017    & End of commisioning phase 1,     \\
         &          &                   &                  & Science verification  \\  
% HA: 24
N2R11  & 24       &  8-June-2017   & 13-June-2017  &  polarization commissioning  \\
\hline
\end{tabular}
\caption[Commissioning campaigns, dates and general comments.]{
\label{nika2runs}}
\end{center} 
\end{table} 

This document describes the work accomplished by the NIKA2 consortium in order
to validate the performances of NIKA2. More specifically, the MoU defined
specifications and goals on some key parameters. They are summarized in
Tab~\ref{nika2specs}. Additionnal information on the instrument and ancillary
parameters were also requested by IRAM, together with a document demonstrating
how the performances have been assessed, both in terms of observations and data
reduction. This document is the answer.\\

Table~\ref{nika2runs} summarizes the timeline of NIKA2 integration and tests
since the replacement of NIKA1 by NIKA2 in Oct.~2015. Based on this overview and
the observing conditions, we have focused our analysis on data from
the N2R9 calibration campaign and the N2R12 and N2R14 sciance pools.
The final results of this commissioning are given in the last section
essentially via a summary table~\ref{tab:nika2summary} with links to the
relevant sections for detailed demonstration of the results.

In order to be self-consistent, the document is also meant to be read in chronological
order. Chapter~\ref{se:chap_instru_obs_methods} summarizes the main features of
the instrument and of the data reduction that will be used in the remaining of
the document for all performance derivations. It also lists and introduces the necessary inputs
that are required for the full data reduction and that are discussed in details
in the following sections.

%We start with atmospheric opacity in
%chapter~\ref{se:opacities}, then proceed to detector positional offsets in
%chapter~\ref{se:fp_reconstruction} and beam characterization in chapter~\ref{se:beams}.
%Chapter~\ref{se:calibration} addresses the overall calibration of
%observations. At last, chapters~\ref{se:photometry}~and~\ref{se:nefd} address
%the final performances of the instrument in terms of photometry and sensitivity,
%before the final summary section~\ref{se:summary}.
In Chapter~\ref{se:opacities}, the atmospheric attenuation is corrected
using two-step method: first, the line-of-sight opacity of each
observing scans is estimated using a series skydip observations
performed with the NIKA2 instrument, then a correction coefficient is
applied to the skydip opacity to ensure the flus density stablity
against the atmospheric condition. Opacity derivation based on the
resident IRAM taumeter measures is also tested for consistency checks.

In Chapter~\ref{se:fp_reconstruction}, \bm\ scans are used to derive
the position of each KIDs in the FoV and perform a first selection to
discard noisy and cross-talking KIDs. The KIDs are further selected on
the comparison between their measured position and the expected
designed position. Finally, series of defocused \bm\ scans are used to
reconstruct the focus surfaces and estimate the telescope focus
setting that optimizes the focus across the arrays.   

In Chapter~\ref{se:beams}, noticeable features of the beam pattern are
identified using \bm\ scans. Radial profiles of the full beam are
measured up to a radius of $180''$ and are fitted using a
three-Gaussian model. The main beam of each array is evaluated using
three complementary methods: the first Gaussian that best fits the full
beam profile, a single Gaussian fit on a side-lobe-masked version of
the profile and a 2D Gaussian fit on a side-lobe-masked version of the
beam map. Then we use the knowledge we have gained on the full beam
and main beam to derive the main beam efficiency up to $180''$. A more
precise description of the beam pattern above $180''$, which is useful for
diffuse emission studies, is let for a future dedicated study that
will be conducted as part of the diffuse emission NIKA2 Large
Programs.

Chapter~\ref{se:calibration} deals with all the aspects related to the
calibration. Sections~\ref{se:cal_HA_reference}
and \ref{se:cal_HA_main} describe the reference photometric system
that has been adopted: flux densities rely on a fit
of a Gaussian with a fixed FWHM at reference values and are given at
reference frequencies. However, we also tested a method based on aperture
photometry and reported the conversion coefficients between the
point-source and aperture photometry calibration in
Sect.~\ref{se:aperture_photo_calibration}.
The main primary calibrator is Uranus, although Neptune
is also considered. In Sect.~\ref{se:ref_flux_primaries}, the flux
expectations for the primary calibrators are derived using the Moreno
model, which is precise at a $5\%$ level. In
Sect.~\ref{se:flatfields}, the KID-to-KID relative calibration is
performed as part of the FOV reconstruction using the fixed-Gaussian fit on
individual maps per KID, which are projected from \bm\ scans. We test
the stability of the KID response across the FoV by considering both
flat fields toward point sources (a.k.a. main beam flat fields) and
flat fields of the atmosphere (a.k.a. forward beam flat
fields). Section~\ref{se:obsdate_variations} deals with the well-known
effect of the telescope heating during the afternoon. This induces a
widening of the beam, which comes along with a drop of the measured
flux densities. We use calibration observations to determine the most
impacted UT hours of the day. The baseline calibration, as presented
in Sect.~\ref{se:baseline_calibration}, is drawn from
scans acquired between 22:00 and 9:00 UT and between 10:00 and 15:00
UT hours, whereas the Sun-rising and afternoon scans are
discarded. However, in Sect.~\ref{se:photocorr_calibration}, we also
tested a calibration method which resorts to a photometric correction
to retrieve robust flux density estimates while the beam experiences
instabilities. This method, which is based on a monitoring of the beam
size throughout an observation campaign, demands dedicated scans to be
made on an hour basis to reach science-grade level of accuracy and
robustness. We present a test case based on a beam monitoring using
the pointing scans, which yields promising results but a mild lake of
robustness. Thus, we choose to base the sensitivity assessment on the
baseline calibration only.

The accuracy and stability of the photometry are treaded in
Chapter~\ref{se:photometry}. We define two calibration performance
criteria: first we measure the ratio between the measured flux density
and the expectations for MWC349, a secondary calibrator which is
monitored at Plateau de Bure, and secondly we estimate the rms errors
with respect to the median flux density using a series of bright
point-like sources. This latter criterion provides us with an estimate
of the statistical uncertainties of the calibration, whiwh includes
errors sourced by the dispersion in the observing conditions
(atmosphere, elevation, source brightness, integration time, etc.) and
in the data analysis.

% first we characterise the features of the noise in the time domain
% by describing both the noise power spectrum and the full KID-to-KID
% correlation matrices, and illustrate the impact of various noise
% decorrelation methods. 
In Chapter~\ref{se:nefd}, we characterise NIKA2 sensitivity by
estimating the Noise Equivalent Flux Density (NEFD). We developped
three complementary methods that differ in the measurement of the flux
density variance but rely on the
same estimate of the on-source integration time. The first method is
based on fitting the variance decrease with the integration time, the
second utilizes statistically equivalent data splits to produce a
noise map estimate and the third method resorts to measuring the noise
far from the source. Regarding the integration time, we ensure the
robustness of the estimate by cross-checking the results of two
approaches: integration times directly derived from the sample counts
are compared to integration time estimates with the array footprint
that best account for the scan strategy.  
