

A few tasks do not need to be done at each scan reduction. They involve
instrument characterization, like the focal plane reconstruction and the
calibration, telescope focus and pointing monitoring. They are presented in this
section.

\subsection{Focal plane reconstruction}

Because of \nika's frequency multiplexing, we do not know {\it a priori} where
kids point on the sky. To acquire such information, we raster scan a strong
point source, typically a planet\footnote{{\bf update with values of run8 ?} the
  angular diameter of Uranus was 3.54~arcsec; if we compare it to the size of
  the main beam (13.5 and 18.4~arcsec), the convolution of the corresponding
  disk with a gaussian with 13.5 and 18.4~arcsec FWHM broaden our beam by only
  0.17 and 0.12~arcsec at 1.25~mm and 2.14~mm respectively.}, to derive the
pointing offsets of each kid w.r.t to the nominal pointing of the telescope. We
proceed in the following way:

\begin{enumerate}
\item We apply a median filter of approximately 5~FWHM of width to the detector
  timelines to subtract most of the atmospheric signal and the low frequency
  correlated electronic noise while preserving most of the planet signal (less
  than 1~\% lost at scales smaller than 2.5 x FWHM). These timelines are then
  projected onto individual maps per detector.
\item We discard kids that show ghost images at this stage to keep only well
  behaved kids (see Sect.~\ref{se:doubles} for further details).
\item We fit a Gaussian ellipse of each selected kid, look at the distribution
  of all kids FWHMs, ellipticities, amplitudes and discard outlyers.
\item With this selection of well behaved kids (the majority), we can make a
  first combined map of the source. Indeed, the Gaussian amplitude of each
  gaussian can be used as a proxy to relative calibration and the Gaussian peak
  position is a good first estimate of the kid's pointing offsets w.r.t nominal
  pointing coordinates.
\item With the source combined map in hand, we can now improve the subtraction
  of atmosphere and correlated noise by applying the point source common mode
  subtraction described in Sect.~\ref{se:point_source_decorr}.
\item We repeat the process of projecting these clean timelines onto indivicual
  kid maps and fitting gaussian ellipses to derive the final pointing offsets
  and beam parameters. The amplitude of the gaussian is our primary relative and
  absolute calibration.
\end{enumerate}

{\bf show plots : focal plane beam centroid positions, beam shapes and profiles
  to show the absence of ringing with this common mode subtraction ?}

This focal plane reconstruction and calibration needs to be done only once per
instrument configuration (frequency tuning), and not before each scientific
scan. In practice, we monitor these parameters about once a day during an
obseration to acquire statistics, check their robustness and improve their
relative precision. Once we have all kid pointing offsets, we choose one
reference pixel near the center of the 2mm array (because quazars are usually
brighter at 2mm than at 1mm) for the determination of the pointing model and for
further pointing monitoring (see Sect.~\ref{se:pointing_cross}). We also choose
a 1mm kid close to it for focus monitoring (see Sect.~\ref{se:focus}).

%% \subsubsection{Ghost beams}
%% 
%% Some kids ({\bf give fraction number}) also show ghost images of other kids and
%% we introduce the mixing matrix $M$ to account for this effect. In practice, the
%% raw timelines are therefore better described by
%% 
%% \begin{equation}
%% d_k'(t) = M_{kk_1}d_{k_1}(t)
%% \end{equation}

\subsection{Pointing reconstruction}

{\bf Do we want to talk about holes in the data and pointing inpainting ? I'd
  say no because it's not a feature of the pipeline, but if we think that yes we
  should, here is the place.}

\subsection{Pointing correction}
\label{se:pointing_cross}

Due to ambient temperature variations essentially, the telescope absolute
pointing may vary by a few arcseconds compared to the pointing model. To correct
for such effects, we periodically observe a bright quazar (or a planet) close to
the science field that is going to be observed to determine the current pointing
offsets that must be subtracted from our observation coordinates. We have two
ways to perform such observation. For convenience and further reference, we name
them after their IRAM operation command.

\paragraph{Method 1: {\tt pointing\_track.}} The telescope scans the
pointing source back and forth in azimuth, then back and forth in elevation. The
pointing offsets can then be derived in two ways: first by fitting a Gaussian on
the one-dimensional profiles of the source as seen in the timeline of the
reference detector, second by fitting a Gaussian on the combined map of all
kids. In each case, the atmosphere and correlated electronic noise are
subtracted with the method described in Sect.~\ref{se:point_source_decorr}. In
most cases, both determinations agree to sub-arcsecond precision, but when the
opacity is large, the combined map has higher signal to noise. This makes it our
reference method.

\paragraph{Method 2: {\tt pointing\_liss.}} The TOI processing is the same as in the
previous method but the this time, the telescope scans the source with a
Lissajous pattern and fit the pointing parameters on the combined map.

{\bf comment on the relative precision of each method.}

\subsection{Focus monitoring}
\label{se:focus}

Ambient temperature variations also impact the focusing of the telescope. About
once every three hours, we therefore optimize the position of the secondary
mirror. Here again we have two methods to do this

\paragraph{Method 1: {\tt focus\_track.}} The telescope points to a planet or a
bright point source and the secondary mirror is moved by $\pm 1$mm in steps. At
each step, we determine the total flux that falls on the reference pixel and fit
a parabola as a function of the mirror position. The position of the peak gives
the optimal focus position. The subtraction of atmosphere and
electronic correlated noise is done using kids far from the source {\bf detail
  further...}

\paragraph{Method 2: {\tt focus\_liss.}} The telescope performs a Lissajou scan of
the source for 5 different positions of the secondary mirror. Each scan is
reduced separately and an elliptical Gaussian is fit for each one. We fit the amplitude
of the Gaussian as a function of the mirror position with a parabola. The peak
position provides the optimal focus position. The same can be done this time
fitting the minimum of the FWHM as a function of the mirror position. Both
results agree to a fraction of arcsec.

{\bf comment on the comparison of the two methods, pros and cons of each one}
