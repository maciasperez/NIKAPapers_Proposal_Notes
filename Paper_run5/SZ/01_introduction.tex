Galaxy clusters are the largest gravitationally bound objects in the Universe. Their formation strongly depends on the content and the history of the Universe within the framework of a bottom-up scenario \citep[{\it e.g.,} ][]{kravtsov_2012}, where there is merging of small clusters to form larger ones. They are classically probed using \mbox{X-ray} produced via bremsstrahlung emission of the electrons in the intracluster medium (ICM) but are also measured in the optical and infrared wavelengths, which trace the stellar populations in the member galaxies. Their radio emission is related to the acceleration of charged particles, and the lensing of background objects provides surface mass density measurements from multi-band optical and infrared data. See \cite{bohringer_2010, gal_2006, oliver_2012, feretti_2012, kneib_2011} for reviews on the different cluster observables.

The thermal Sunyaev-Zel'dovich (tSZ) effect \citep{SunyaevZeldovich1, SunyaevZeldovich2}, which consists of the inverse Compton scatter of Cosmic Microwave Background (CMB) photons on hot electrons in the ICM, can be used as a complementary method to probe galaxy clusters \citep[see][for a detailed review on the tSZ effect]{Birkinshaw, Carlstrom_et_al_2002}. Three-dimensional information on the cluster may be inferred using the characteristic dependences of \mbox{X-ray} (sensitive to the line-of-sight integral of the density squared and the square root of the temperature) and tSZ (sensitive to the integrated pressure along the line-of-sight) with the properties of the ICM. This gives a more accurate picture than \mbox{X-ray} or tSZ alone, especially in the case of merging systems \citep{Basu}. In addition, unlike other observational approaches, the tSZ signal is not affected by cosmological dimming. Only the angular size of the observed cluster depends on the distance to the source. High angular resolution tSZ observations are therefore of particular interest to probe structure formation at high redshift.

The resolutions of the main current instruments measuring the tSZ effect are of the order of the~arcmin. It is larger than 5~arcmin for the {\it Planck} satellite \citep{PLANCK_mission} and about 1~arcmin for the South Pole Telescope \citep[{\it SPT};][]{SPT} and the Atacama Cosmology Telescope \citep[{\it ACT};][]{ACT}. Higher resolution instruments, such as {\it MUSTANG} \citep[$\sim 8$~arcsec resolution at 90~GHz;][]{mason_2010, korngut_2011}, may suffer from filtering of large-scale structures due to the atmospheric noise removal when observing at a single frequency band. High redshift tSZ observations, therefore, need a new generation of instruments. The New IRAM KID Arrays ({\it NIKA}) is a prototype of a high-resolution camera based on Kinetic Inductance Detectors (KIDs) \citep{day_2003,KID_for_CMB} in development for millimeter wave astronomy \citep{NIKA_2011}. It consists of two arrays of 132 and 224 detectors, which observe at 140 and 240~GHz with resolutions of 18.5 and 12.5~arcsec, respectively. Due to the characteristic spectral distortion of the CMB photons induced by the tSZ effect, {\it NIKA} is an ideal instrument for high resolution tSZ observations. Indeed, the tSZ signal is strongly negative at 140~GHz and positive but close to zero at 240~GHz. The {\it NIKA} prototype has already been successfully tested during four observation campaigns \citep{NIKA_2010,NIKA_2011} at the Institut de Radio Astronomie Millim\'etrique (IRAM) 30-meter telescope at Pico Veleta, Granada, Spain. These observations have demonstrated performances comparable to state-of-the-art bolometer arrays operating at these wavelengths, such as {\it GISMO} \citep{gismo}. The final camera, {\it NIKA2}, will contain 1000 and 4000 detectors at 140 and 240~GHz, respectively, and should be operational in 2015.

We report the first observation of a galaxy cluster via the tSZ effect here using the {\it NIKA} prototype. It has been imaged during the fifth observation campaign of {\it NIKA} in November 2012. The targeted source is the massive intermediate redshift galaxy cluster \mbox{RX~J1347.5-1145} at $z = 0.4516$. It has been selected for both its tSZ intensity and angular size with the latter being comparable to the field of view of the {\it NIKA }prototype. Moreover, \mbox{RX~J1347.5-1145} is known to be a complex merging system that we aim at characterizing further with respect to previous works at scales in the range of 20 to 200~arcsec.
   
This paper is organized as follows. In Sect.~\ref{sec:previous_obs}, we give the status of the previous observations of \mbox{RX~J1347.5-1145}. In Sect.~\ref{sec:observations}, we provide a brief description of the {\it NIKA} camera and give an overview of the observations that is carried out during the November 2012 campaign at the IRAM 30-meter telescope. Sect.~\ref{sec:SZ_analysis} describes the tSZ dedicated data analysis and its validation on simulations is reported in Sect.~\ref{sec:simu_and_idcs}. We present the map of \mbox{RX~J1347.5-1145} in Sect.~\ref{sec:results} and the results on the pressure profile for this cluster of galaxies. These results are then compared to other experiments in Sect.~\ref{sec:comparison}. Throughout this paper, we assume a flat $\Lambda$CDM cosmology according to the lastest {\it Planck} results \citep{PLANCK_param_cosmo} with $H_0 = 67.11$ km.s$^{-1}$.Mpc$^{-1}$, $\Omega_M = 0.3175$, and $\Omega_{\Lambda} = 0.6825$.
