Contemporary cosmology at millimetre wavelengths requires multi-wavelengths study for foregrounds subtraction (e.g. thermal dust, synchrotron, dusty and radio galaxies) and components separation of the various effects for the Cosmic Microwave Background.

In this framework, the necessity to observe at different frequencies makes suitable the exploitation of spectrometers. Among this category a candidate is the FTS. The Fourier Transform  technique applied to spectrometry is a method that exploits the interference of light rather than separate the wavelengths and, compared to competitors (on-chip (\cite{deshima}) and grating (\cite{grating}) spectroscopy), FTS allows larger instantaneous FoV.

Firstly, as in the past for FIRAS (\cite{FIRAS}), the selection of the FTS for the will of developing an instrument fells on a Martin Puplett: an interferometer that measures the difference between the powers of their two input beams (\cite{mpi}). The Martin Puplett Interferometer transposes the observing frequencies paradigm from time to spacial domain and allows to use a calibration source: an ideal choice for precise measurements.

Secondly, the necessity to go fast on data acquisition makes suitable to choose fast detectors like KIDs (for an overview see \cite{kids}). Taking in consideration these premises, using a ``brute-force'' acquisition technique, where you just stock all the information available, would result on a huge data production. In this scenario, it is desirable to adopt a smart solution that minimises the amount of data.

Aiming this optimisation, we want to describe the method developed for KISS.

In sec. \ref{sec:acq} we introduce the method used in past for NIKA2 and it explained the new created for KISS. In the same section it is deeply described the new method and showed a preliminary simulation that allowed the utilisation of the same for the data.

In the sec. \ref{sec:results} we present the first results characterisation of on-sky data.


