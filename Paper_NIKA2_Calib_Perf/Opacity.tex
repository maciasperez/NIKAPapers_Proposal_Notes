%----------------------------------------------------------------------------------------
%	7./ Opacity derivation
%----------------------------------------------------------------------------------------
%\section{Opacity derivation}
%\label{se:opacity}

Only a fraction of the signal is transmitted by the atmosphere and
reaches NIKA2 detectors. 
The relation between uncorrected observed flux densities
$\tilde{S}_{\nu}$ and top-of-the-atmosphere flux densities $S_{\nu}$
is parametrized by the zenith opacity $\tau_{\nu}$
and the line-of-sight airmass $x = \left(\sin\delta\right)^{-1}$ ($\delta$ is the elevation), such as
\begin{equation}
\tilde{S}_{\nu} = S_{\nu} \, e^{-\tau_{\nu}  x}.
\label{eq:uncorr_flux}
\end{equation}

An accurate derivation of the opacity condition for each scan is
of outmost importance to retrieve the source signal at the top of the
atmosphere and to ensure low calibration uncertainties.
%Opacity correction uncertainties even prevail in the
%final calibration error budget.
We developed three opacity derivation methods, which are described in
Sect.~\ref{se:opacity_methods}, and extensively tested their
robustness against observing conditions, as presented in
Sect.~\ref{opacity_tests}.


\subsection{Opacity derivation}
\label{se:opacity_methods}

To assess the robustness of our opacity measurements, we have devised
three procedures for the opacity derivation: {\tt taumeter} is
described in Sect~\ref{se:taumeter-method} and relies on measurements
provided by the resident IRAM tau-meter operated at $225~\rm{GHz}$ and
a fit of the opacity estimates in NIKA2 frequency bands by imposing
the flux density stability against atmospheric conditions;
{\tt skydip}, described in Sect~\ref{se:skydip-method}, consists in
using NIKA2 as a taumeter (assuming the resonance frequencies are
linearly related to the total power) by resorting to a series of
skydip scans, the selection of which is addressed in
Sect.~\ref{se:skydip-selection}; {\tt corrected skydip}, presented in
Sect.~\ref{se:corrected-skydip}, is a modifed
version of the {\tt skydip} method that minimizes the dependence of the
estimated top-of-the-atmosphere flux density on the opacity.

All these methods i) do not rely on an atmospheric model nor on any
hypothesis on the atmospheric contents for the sake of robustness, and
ii) do not use the laboratory measurements of the bandpass (see 
Sect.~\ref{se:instru_bandpass}) for more accuracy.  

\subsubsection{{\tt taumeter}}
\label{se:taumeter-method}

The IRAM 30-m telescope facility is equipped with a resident taumeter,
which performs elevation scans at a fixed azimuth and is operated at
225~GHz, to monitor the atmospheric opacity. The IRAM science support
team provides us with time-stamped zenith opacities at $225$~GHz
$\tau_{225}$, as derived from the taumeter measures. The
$\tau_{225}$ estimates come in two different flavours: one relying
on a linear model and the other on an exponential fitting model. They
are then filtered by removing outliers and by thresholding on
goodness-of-fit criteria.
%$0< \tau_{225} <1.2$ and $R^2 > 0.99$, where $R^2$ is the coefficient
%of multiple determination of the $\tau_{225}$
%estimates, which quantifies the goodness of fit of each model.
%Redundant samples and outliers are removed.
Based on IRAM experience, we use the linear filtered $\tau_{225}$
data for the NIKA2 analysis. The time-stamped $\tau_{225}$ estimates,
which are sampled about every 4 minutes, are interpolated to the time
of the NIKA2 scans (we consider the time of the middle of the
scan). For cross-check we also produce a smooth version of time-stamped
$\tau_{225}$ data by
filtering with a running median of 7 samples, which is then
interpolated to the NIKA2 scan times.

We fit the relations between the IRAM
$225\,\rm{GHz}$ taumeter opacities and NIKA2 band pass opacities using
observation of calibration sources which spans a large range of air
masses. This method has the advantages of not relying on
any atmosphere model nor on the bandpass measurements in laboratory.
We use a series of 64 scans of MWC349, which consists of the
baseline selected subset of scans from the 68 available scans for
this source during N2R9.
It constitutes an homogeneous data set in flux density but
heterogeneous in atmospheric conditions: zenith opacities at
$225\,\rm{GHz}$ 
range from 0.08 to 0.32 and elevations from $23$ to $73$
degrees, spanning a large range of air mass as required. NIKA2 opacities
$\tau_\nu$, for $\nu$ corresponding
to Array 1, 2, 3 and the combination of Arrays 1 and 3, are estimated
from the $225\,\rm{GHz}$ taumeter median-filtered exponential-based opacity
estimates $\tau_{225}$ as
\begin{equation}  
  \tau_\nu =  a_\nu^{225}\tau_{225} + b_\nu^{225},          
\end{equation}
where the parameters $a_\nu^{225}$ and $b_\nu^{225}$ are fitted to ensure
that the non-corrected flux densities $\tilde{S}_\nu$ are stable against
$\tau_{225}$ after correction of the atmospheric attenuation by
inversion of Eq.~\ref{eq:uncorr_flux} using 
\begin{equation}  
  S_\nu = \tilde{S}_\nu e^{(a_\nu^{225}\tau_{225} + b_\nu^{225}) \cdot x}.
  \label{eq:opacorr_taumeter}
\end{equation}

\begin{table}[!htbp]
  \begin{center}
    \caption[IRAM taumeter to NIKA2 opacity model]{IRAM taumeter to NIKA2 opacity model.}
    \label{tab:tau225-to-taunika}  
    \begin{tabular}{lrrrr}
      \hline
      \hline
      \noalign{\smallskip}
      Parameters & Array 1 & Array 3  & Array 1$\&$3 & Array 2  \\
      \noalign{\smallskip}
      \hline
      \noalign{\smallskip}
      $a_\nu^{225}$         & $1.94$   &  $1.90$ &  $1.92$ & $0.94$ \\
      $b_\nu^{225}$         & $-0.04$  & $-0.07$ & $-0.06$ & $0.00$ \\
      $\Delta a_\nu^{225}$  & $0.15$  & $0.08$  &  $0.09$ & $0.10$ \\
      $\Delta b_\nu^{225}$  & $0.05$  & $0.03$  & $0.04$ & $0.03$ \\
      \noalign{\smallskip}
      \hline
    \end{tabular}
  \end{center}    
\end{table}

We tested two estimators of the flux stability. The first one relies
on minimising the standard deviation of the measured-to-median flux
densities ratio after correction of the opacity using
Eq.~\ref{eq:opacorr_taumeter}. The second one consists in rewriting
the rms minimisation as an unweighted $\chi^2$ minimisation using:
\begin{equation}
\chi^2 = \sum_{i=1}^{N} \frac{1}{\sigma^2} \, \left( \frac{S_\nu}{Med(S_\nu)} -1 \right)^2,  
\end{equation}
where $\sigma$ is the rms error of the flux density estimates. Note
that these estimators do not depend on
the absolute scale of the flux density of the source. Both estimators
yield consistent results that are combined and gathered in
Table.~\ref{tab:tau225-to-taunika}. The quoted errors
$\Delta a_\nu^{225}$ and $\Delta b_\nu^{225}$ are 1-$\sigma$ errors of
the fit.

Because the IRAM tau-meter observes at a fixed azimuth, the
taumeter-based opacities are not exactly the line-of-sight opacities for
the observation scans. As this will be verified in
Sect.~\ref{se:photometry}, this induces larger rms errors of
the top-of-the-atmosphere flux density estimates with respect to
opacity correction methods that relies on skydip-based measurements using
NIKA2 instrument itself. The taumeter-based method will thus be used
as an alternative method in case of i) failure of the NIKA2 skydip-based
methods and ii) to perform consistency checks.



\subsubsection{NIKA2 skydip-based method}
\label{se:skydip-method}


\subsubsection{{\tt corrected skydip}}
\label{se:corrected-skydip}
