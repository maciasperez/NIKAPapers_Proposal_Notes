
%\documentclass{article}
\documentclass[12pt]{article}
\usepackage[utf8]{inputenc}
\usepackage[english]{babel}

\title{Referee report: NIKA polarization observations of the Crab nebula}
%\author{Alessia Ritacco}
%\date{March 2018}

\begin{document}
\maketitle

\noindent Dear Referee,\\
Thank you again for your attentive reading. 
In the following we answer (in boldface) to your questions and comments. 
We have also included in the text your comments to ease the reading.
In the last version of the paper we have boldfaced the changes. \\  \\
Best regards,\\
A. Ritacco, J.F. Macías-Pérez, N. Ponthieu on behalf of the authors.


\begin{enumerate}
    \item 
{
As described in my previous report, the measurements and analysis 
presented in this paper are of important value to the community. I am 
pleased to see that the latest version of the paper addresses all the 
main issues I raised. It is exciting to see the new Planck 
measurements. There are some minor issues that I believe 
should be resolved. 
}

Equation (2), for the uncertainty on p appears to be incorrect, based 
on dimensional analysis. I believe equation (3) is also incorrect; 
the terms under the square root should be cross-terms, such as 
$Q^2\sigma_U^2$. Whatever the final correct equations are, I ask the 
authors to guide me in how these results were obtained from the 3 
references given (i.e. what equations under what assumptions).

{\bf The assumptions we have made here are: (i) correlations between the total and polarized intensities are negligible; (ii) correlated noise between Q and U is also negligible; and (iii) equal noise is observed on Q and U measurements.
  In this case the full covariance matrix (Eq. 3) represented in Montier et al. 2015 I reduces to the diagonal terms. These assumptions have been verified as described in Ritacco PhD thesis 2016 (Paragraph 6.3.4) and in addition a white noise spectrum is observed in Q and U as described by Ritacco et al. 2017 A$\&$A, 599, A34. This leads our computation of the uncertainty on p which is without dimension, as expected.}

In section 3.1, a "zero level" subtraction is described, with 
particular reference to the annulus between 4 and 4.5 arcminute 
radius. Later, "aperture photometry" is used to estimate fluxes with 
"radius of 5', 7' and 10' FWHM from the center of the map". What is 
meant by "FWHM" here? If this is a circular aperture then the FWHM 
would mean the diameter, not the radius. Are the radii 5', 7', and 
10', or is that the diameter? If it's the diameter, doesn't that go 
outside the annulus at which the "zero level" was computed and 
removed? The authors need to clarify whether these are radii or 
diameters, and I would encourage them to not use "FWHM" in this 
context. 

\textbf{Indeed this section is a bit confusing. When we finally decided to use aperture photometry to compute the fluxies we forgot to specify that we are now talking of diameters. We consider circular aperture with radii of: 2.5', 3.5' and 5'. }

At various points, the color correction of 1.05 is mentioned. A few 
more words are warranted here... what assumptions (spectral index, 
pass band) go into the color correction? How big is the uncertainty? 
After color correction, are the quoted flux densities valid at the 
nominal frequency of 150 GHz, or some other frequency? 

\textbf{}

In section 4.1, it is claimed that the flux at radio and millimeter 
wavelengths is due to synchrotron from a single population of 
particles in the "same" magnetic field. This leads to the statement 
that "The direction of the polarization is therefore expected to be 
constant across the frequency range 30-300 GHz while the polarization 
degree may vary." I do not see how the polarization degree, and 
angle, could be decoupled like this. Either different parts of the 
nebula could scale differently with frequency (producing varying 
degree and angle of polarization), or they all scale together, locking 
all polarization properties to the total intensity.
\textbf{}

\end{enumerate}

\end{document}
