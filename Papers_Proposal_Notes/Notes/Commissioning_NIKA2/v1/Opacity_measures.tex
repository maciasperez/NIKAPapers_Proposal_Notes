%----------------------------------------------------------------------------------------
%	OPACITY MEASURES
%----------------------------------------------------------------------------------------

\section{Opacity monitoring {\color{YellowGreen} Juan $\&$ Laurence}}
\label{se:opacity_measures}

%{\bf copy from the 'Instru' paper}

%\begin{figure}[ht]
%\begin{center}
%\includegraphics[scale=0.8]{Figures/test_allskd_N2R10v3commiss2.pdf}
%\caption{Atmospheric opacity as measured from the NIKA2 data 
%at 260 (top) and 150\,GHz (bottom) during N2R10
%commissioning campaign. Each block of 40 KIDs gives an independent estimate of
%the opacity value for each skydip scan (the integer abscissae). The block
%number is the decimal value of the abscissae.
%\label{fig:taumeas_paper}}
%\end{center}
%\end{figure}

%\begin{figure}[ht]
%\begin{center}
%\includegraphics[scale=0.8]{Figures/test_allskd_N2R10v2commiss1.pdf}
%\caption{Atmospheric opacity as measured from the NIKA2 data 
%at 260 and 150\,GHz during N2R10
%commissioning campaign. The error bars are in fact dispersion of the deduced
%opacities between blocks of 40 KIDs.
%\label{fig:taumeas_paper}}
%\end{center}
%\end{figure}


\begin{figure}[ht]
\begin{center}
\includegraphics[scale=1.0]{Figures/opacity_evol_run_9_12_14.pdf}
\caption[Zenith opacity monitoring during N2R9, N2R12 and
  N2R14]{Atmospheric opacity as measured from the IRAM 225\,GHz
  taumeter (yellow green), and from the NIKA2 data at 150 (red) and
  260\,GHz (blue) during N2R9 commissioning campaign (Feb. 2017) and
  N2R12 (Oct. 2017) and N2R14 (Jan. 2018) scientific pools. We
  stress the fact that the IRAM 225\,GHz taumeter data is not used for
  the atmospheric correction and is plotted here for comparison only.
  \label{fig:taumeas}}
\end{center}
\end{figure}


%\begin{figure}[ht]
%\begin{center}
%\includegraphics[width=\linewidth]{Figures/opacity_vs_index_N2R9_N2R10.png}
%\caption[Zenith opacity monitoring during N2R9 and N2R10]{Atmospheric opacity as measured from the IRAM 225\,GHz
%  taumeter (black crosses), and from the NIKA2 data at 150 (red) and 260\,GHz (blue) during 
%  N2R9 and N2R10 commissioning campaigns.  We stress the fact that the IRAM 225\,GHz taumeter data is not used for the %atmospheric correction and is plotted here just for comparison.
%  \label{fig:taumeas}}
%\end{center}
%\end{figure}


In Fig.~\ref{fig:taumeas}  we present the evolution of the NIKA2 in-band
opacities for all the 'OTF' scans (about 1300 scans per runs) of the
N2R9 commissioning campaign (Feb. 2017) and N2R12 (Oct. 2017) and
N2R14 (Jan. 2018) scientific pools. These are compared to the IRAM
tau-meter values. For both opacity estimates, plateaux correspond to
the repeated propagation of the last valid value in case of failure of
the opacity measurement. We observe an agreement on the
global trend between the IRAM tau-meter opacity (225 GHz) and the
NIKA2 values. These latter show, however, a smaller dispersion (less
than one percent).



