%----------------------------------------------------------------------------------------
%	8./ Calibration
%----------------------------------------------------------------------------------------
%\section{Calibration}
%\label{se:calibration}

We present the calibration of the NIKA2 instrument in this section in using Uranus as the
main primary calibrator. Neptune and Mars are also considered as
valuable alternatives to calibrate when Uranus is not visible.


\subsection{Photometric System}

We detail here the procedure for the calibrating the absolute scale of
the flux density and the chosen photometric system.

We parametrize the primary calibrator flux density
$S_{\rm{c}}(\nu) = S_{\rm{c}}(\nu_0)\, f(\nu/\nu_{0})$, where $f(\nu/\nu_{0})$
encloses the spectral dependence, 
as a function of a reference frequency $\nu_{0}$ that we choose
arbitrarily to be: $\nu_{0} = 150$~GHz for the 2mm array and
$\nu_{0}= 260$~GHz for both 1mm arrays. Projecting the raw data (in
units of the KID resonance frequency shift or $\rm Hz$) of a
calibrator, we model the raw map with a fixed-width Gaussian
\begin{equation}
  R_{\rm{c}}(\theta, \phi)  = \frac{A_{\rm{c}}}{2 \pi \sigma_{0}^{2}}
e^{-\frac{\theta^{2}}{2\sigma_{0}^{2}}},
\end{equation}
where $\sigma_{0}$ is derived from the
reference FWHM, labelled FWHM$_{0}$, which are $12.5''$ for the 1mm
arrays and $18.5''$ for the 2mm array. These values have
been chosen sizably larger than the main beam values, as reported in
Sect.~\ref{se:beam}, to account for a fraction of the signal smeared 
in the first error beam and first side lobes.
Both the reference frequency and FWHM, $\nu_0$ and FWHM$_{0}$, define our reference photometric system, as
summarized in Table~\ref{tab:definitions}.

\begin{table}[!htbp]
  \begin{center}
    \caption{NIKA2 reference frequencies and FWHM}
    \begin{tabular}{lcc}
      \hline\hline
      \noalign{\smallskip}
      & 1 mm & 2 mm \\
      \noalign{\smallskip}
      \hline
      \noalign{\smallskip}
      Reference frequency $\nu_{0}$ & 260 GHz & 150 GHz \\
      Reference FWHM  FWHM$_{0}$    & 12.5'' & 18.5'' \\
      \noalign{\smallskip}
      \hline
    \end{tabular}
  \end{center}
  \label{tab:definitions}
\end{table}

The absolute calibration coefficients are estimated as the ratio of
the flux density expectations at the reference frequency
$S_{\rm{c}}(\nu_0)$ and the amplitude estimate of the fixed-width reference
FWHM Gaussian $A_{\rm{c}}$. For any point-like source $\rm{s}$, the map
\begin{equation}
  M_{\rm{s}}(\theta, \phi) = \frac{S_{\rm{c}} (\nu_{0})}{A_{\rm{c}}}
  R_{\rm{s}}(\theta,\phi),
\end{equation}
where $R_{\rm{s}}(\theta,\phi)$ is the raw data projection, is calibrated in Jy/FWHM$_{0}$
beam. The flux density estimate for the source is then:
\begin{equation}
S(\nu_{0})  = \frac{S_{\rm{c}}(\nu_{0})}{A_{\rm{c}}} \, A_{\rm{s}},
\label{eq:pointsourcephot}
\end{equation}
where $A_{\rm{s}}$ is the amplitude estimate of a FWHM$_0$ Gaussian.


The flux density estimate $S(\nu_{0})$ gives the
flux of the source at the reference frequency only if the source has
the same spectral index as the calibrator. In general, to retrieve the
flux of the source at the reference frequency, a color correction
$C_{\rm{s}}$ has to be applied
\begin{equation}
S_{\rm{s}}(\nu_{0}) = S(\nu_{0})  C_{\rm{s}}(\nu_{0}, \alpha_{\rm{s}}),
\end{equation}
which depends on the reference frequency $\nu_{0}$ and the source
spectral index $\alpha_{\rm{s}}$.
Neglecting the effect of the atmosphere on NIKA2 transmission, we compute the color correction
factor for target sources of spectral indices $\alpha_{\rm{s}}$ that are
different from Uranus using
\begin{equation}
  C_{\rm{s}}(\nu_{0}, \alpha_{\rm{s}}) = \frac{\int_{0}^{+\infty} (\nu/\nu_0)^{1.6} ~T({\nu}) d\nu}{ \int_{0}^{+\infty} (\nu
    /\nu_0)^{\alpha_S} ~ T({\nu}) d\nu}.
\end{equation}

Color correction factors for eight values of $\alpha_{\rm{s}}$, and in particular
for $\alpha_{\rm{s}}= 0.6$ which is the spectral index of MWC349, are
given in Table~\ref{tab:mod}. 

\begin{table*}[!h]
\caption{Color correction factor for a target source  $S \propto \nu ^{\alpha_{\rm{s}}}$}
\label{tab:mod}
\centering 
\begin{tabular}{l| c c c c c c c c}
\hline\hline
\noalign{\smallskip}
Array  & \multicolumn{8}{c}{$\alpha_{\rm{s}}$} \\
\noalign{\smallskip}
\hline
          &  -2 &  -1    &    0  & + 0.6 & +1  &  +2  & +3 & +4  \\
%            \noalign{\smallskip}
            \hline
%            \noalign{\smallskip}
          A1   & 0.876  &  0.916   &   0.951  & 0.969 &  0.981   &  1.005  &    1.024  &  1.037   \\
          A2   & 0.945  &  0.972   &   0.990  & 0.996 &  0.998   &  0.997  &    0.986  &  0.966      \\ 
          A3   & 0.907  &  0.940   &   0.967  & 0.980 &  0.987   &  1.001  &    1.009  &  1.011     \\
            \noalign{\smallskip}
            \hline
\multicolumn{8}{c}{Note : Uranus/Moreno model used for Uranus in this
  Table.}
\end{tabular}
\end{table*}

