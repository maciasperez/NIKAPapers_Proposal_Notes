
%hi

\subsection{Overview of the calibration pipeline}

The steps to go from raw timeline data in Hertz to calibrated data in Jansky per beam comprize:
\begin{itemize}
\item[] Opacity correction
\item[] Field-of-view geometry and KIDs selection
\item[] KID-to-KID intercalibration (flat fielding)
\item[] Absolute calibration  
\end{itemize}


\subsection{Data reduction summary {\color{blue} Nico}}

The performance assessment relies on a data reduction pipeline that consists of the following steps:
\begin{itemize}
\item[] reading of the raw timeline 
\item[] implementation of the calibration
\item[] substraction of the correlated part of the noise 
\item[] projection of the timeline onto maps
\end{itemize}


\subsection{Data selection {\color{blue} Laurence}}
\label{se:data_selection}

For calibration and performance assessment, we select scans in average
observing conditions by performing mild selection cuts. These scan
cuts rely on zenith opacity estimates in NIKA2 bands $\tau$, as
described in Sect.~\ref{se:opacities}, and on the observation date:
%
\begin{itemize}
\item[i)] $\tau_{3} < 0.5$, where $\tau_{3}$ is $\tau$ estimate for
  Array 3, corresponding to a decrease of the signal by a factor of
  two at $45^{o}$ of elevation;
\item[ii)] $x\, \tau_{3} < 0.7$ and $\elev > 20^{o}$, where $\elev$ is the
  elevation of the telescope and $x$ the
  air mass, which depends on the elevation as $x=\sec \elev$. This
  threshold corresponds to a decrease of the signal by a factor of two;
\item[iii)] observation date from 22:00 to 9:00 UT and from 10:00 to
  15:00, that is excluding the sunrise period and the late afternoon.
\end{itemize}
%
As discussed in Sect.~\ref{se:obsdate_variations}, the late afternoon
observation are impacted by telescope-driven beam broadening. Around
sunrise, the focus shifts continuously due to the ambiant temperature
change until the temperature stabilizes, so that the scans taken from
9:00 to 10:00 UT are likely not to be optimally focused.
After the focus stabilisation, morning period 
from 10:00 to 15:00 UT offers stable observing conditions
provided the telescope is not heated due to observations in a
direction close to the Sun.  Otherwise, further scan selection based on the
observation exact historic might be needed before using these
observations for performance assessment.

   
In addition to the above scan selection cuts, we use a Gaussian beam
size criterion for the absolute calibration on Planets
(e.g. Uranus). Namely, the FWHM estimated from the Planet observation
map is asked to be lower than $13''$ at 1mm and lower than $18.3''$ at
2mm, which correspond to a beam about $15\%$ larger than the average
beam (see Sect.~\ref{se:beams}). The rational of this extra cut is
mitigating the flux scatter due to beam broadening, and thus
preserving the absolute calibration accuracy, as discussed in
Sect.~\ref{se:calibration}.




