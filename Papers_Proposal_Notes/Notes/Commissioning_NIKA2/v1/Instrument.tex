
\section{Overview of the NIKA2 instrument {\color{blue} Nico} }

NIKA2 is a millimeter camera able to simultaneously image a field-of-view of
6.5\,arcmin in diameter at 150 and 260~GHz, with polarimetric capabilities at
260~GHz. The optics of the telescope receiver cabin have been modified in order
to increase the telescope field-of-view compared to earlier IRAM instruments. To
achieve these goals without degrading the telescope angular resolution, it
employs a total of around 2,900\,detectors split over three distinct arrays of
Kinetic Inductance Detectors (KID). A KID is a planar superconducting resonator
properly designed to absorb the incoming radiation and at the same time
incarnate the multiplexing scheme. The quality factors of the KID used in NIKA2
are of the order of 15,000. The measurable quantity, proportional to the
incoming power per pixel, is the shift in frequency of each resonance
(pixel). This shift is a consequence of variations of the resonator (kinetic)
inductance with incoming radiation.

\subsection{Optics}

The NIKA2 camera optics includes two cold mirrors, and the filtering of unwanted
(off-band) radiation is provided by a suitable stack of multi-mesh filters
placed at all temperature stages between 150 mK and room temperature. An air-gap
dichroic splits the 150\,GHz (reflection) from the 260\,GHz (transmission)
beams. A grid polariser ensures then the separation of the two linear
polarizations on the 260\,GHz channel (V and H). Band-defining filters,
custom-designed to optimally match the atmospheric windows, are installed in
front of each array. A half-wave polarization modulator is added at room
temperature when operating the instrument in polarimetric mode.

%-------------------------
% HERVE
\input{bandpass}
%-------------------------

\subsection{Cryogenics}

The optimal operation of the detectors is achieved at a temperature of around
150\,mK, well below the Aluminium superconducting transition. For this reason,
NIKA2 employs a custom dilution fridge to cool down the focal plane, and the
refractive portion of the optics, for a total mass around 100 kg, deeply in the
sub-Kelvin regime. Despite the complexity and size of the system, the operation
of NIKA2 does not require external cryogenic liquids and is fully remotely
controllable.

\subsection{KIDs and electronics}

The 150\,GHz channel is equipped with an array of 616\,pixels, arranged to cover
a 78\,mm diameter circle. Each pixel has a size of
$2.8\times2.8\textrm{\,mm}^2$. The array is connected over four different
readout lines. In the case of the 260\,GHz band detectors, the pixel size is
$2\times 2\mathrm{\,mm}^2$, to ensure a comparable sampling of the focal
plane. In order to fill the two 260\,GHz arrays, a total of 1,140 pixels are
needed in each of them. The focal planes are all based on thin Aluminium films
deposited by e-beam evaporation under ultra-high vacuum conditions over a
Silicon substrate.

The key advantage of the KID technology is the simplicity of the cold
electronics and the multiplexing scheme. In NIKA2, each block of around 150
detectors is connected to single coaxial line providing the excitation and the
readout at the two ends. Each of the readout lines is linked to the input of a
cryogenic (4 K) low-noise amplifier. The warm electronics required to digitize
and process the pixels signals is composed of twenty custom readout cards (one
per feed-line).

In this document, the 2\,mm array is called A2, while the two 1\,mm arrays are
called A1 and A3.

%\addparag{NIKEL}

\subsection{KID tuning}

\addparag{$\ftone$ definition}

\subsection{KID photometry}

\todo{With the above definitions, explain RfdIdQ}.



