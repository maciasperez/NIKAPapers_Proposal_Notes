

\section{Calibration statistical errors}
\label{se:calibration_rms}

We use a large amount of bright source observation scans to test the
stability of the measured flux densities with respect to the observing
conditions.

The selected bright source scans consists of the OTF scans that meet
the baseline selection criteria and for which the flux estimate is
above $800~\rm{mJy}$ at $1~\rm{mm}$ and $400~\rm{mJy}$ at
$2~\rm{mm}$. We consider only the sources for which a minimum of $10$
scans are available after selection.  

We evaluate the standard deviation of the bright source measured-to-median flux
density ratio $\sigma_{\rm A_i}$ for each array or array combination. 
This quantity constitutes an estimate of the statistical calibration
uncertainties that encloses errors of optical, atmospheric, noise and
data processing origins.
Added in quadrature with the model uncertainties reported in
Moreno et al. and with the bandpass uncertainties, it represents a
conservative estimate of the total absolute calibration errors.


\subsection{Baseline calibration accuracy}



\subsection{Comparison with other calibration methods}
