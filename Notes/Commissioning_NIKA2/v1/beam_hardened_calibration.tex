\subsection{Beam-variation hardened calibration {\color{blue} Laurence} }
%\label{se:allscan_calibration}

As discussed in Sect.~\ref{se:beam_variation}, observations during the
afternoon session or the morning session after observations close to the
direction of the Sun are deeply affected by the telescope-driven beam
size variation. For the baseline calibration presented in
Sect.~\ref{se:baseline_calibration}, the made choice is to discard the
scans acquired during these periods.

However, in this section, we address the issue of calibrating in
telescope-driven unstable observing conditions. We discuss a method,
which relies on a beam-size-dependent photometric correction, to
calibrate while the beam size broaden.

We perform two case studies: Sect.~\ref{se:cal_democase} presents a demonstration
calibration assuming the beam is precisely monitored, whereas
Sect.~\ref{se:cal_pointings} addresses a practical calibration relying
on a beam monitoring using pointing scans. 


\subsubsection{Demonstration case}
\label{se:cal_democase}

\addparag{FIG: Flux vs obs. tau, color code
  obsdate, all runs}

\subsubsection{Practical case using pointing scans}
\label{se:cal_pointings}

\addparag{FIG: Flux vs obs. tau, color code
  obsdate, all runs}
