\section{Introduction}
\label{sec:introduction}

%Pour citer les papiers: \citep{planck2013mission} ou alors
%\citep{2010A&A...518L.100M,arzoumianian}.


%\begin{itemize}
%\item Why we want to measure CMB polarization B modes
%\item The need for matrices and the KID solution, quickly mention other
%  solutions and multiplexing
%\item Importance to master systematic effects (ref. to previous papers)
%\item KIDs are a new technology that has not yet reached a ``space proof''
%  maturity. In this paper, we address a few specific systematics to KIDs.
%\item Outline
%\end{itemize}


Precise measurement of the cosmic microwave background (CMB) predicted by the $\Lambda$CDM model is one of the major challenges in cosmology. In fact, the CMB anisotropies represent an essential source of information on the physics of the early Universe as it will help us determine the origin of large scales structure and constrain the cosmological parameters of the $\Lambda$CDM model. Since the discovery of the CMB (PENZIAS), several experiments, COBE \citep{1992ApJ...396L...1S}, WMAP \citep{2013ApJS..208...20B}, Planck \citep{2016A&A...594A...1P}, have made precise measurement of CMB temperature anisotropies. These detections have permitted to support the predictions of the $\Lambda$CDM model concerning the primordial nucleosynthesis, formation of structure and on inflation, however these measurements alone can not determine all of the cosmological parameters. The next objectif for the CMB community is the measurement of the CMB polarization and in particular the detection of $B$ modes polarization. The key motivation for CMB polarization measurements is that  inflation models predict the generation of a gravitational wave background. It has been shown that a primordial gravitational wave background would leave an imprint on the polarization pattern of the CMB in the form of B mode polarization \citep{1997PhRvD..55.1830Z, 1997PhRvD..55.7368K}. The measurement of B mode polarization induced by primordial gravitational wave would then provide a powerful probe of inflation. The level of B modes is characterized by the tensor-to-scalar ratio, $r$. The current upper limit on $r$ is < 0.07 ($95 \%$ confidence) \citep{2016PhRvL.116c1302B}. Several missions focused on the detection of B modes, such as LiteBIRD \citep{2014JLTP..176..733M} or COrE \citep{2011arXiv1102.2181T}, are proposed with the objectif of having a sensitivity to be able to detect B modes at the level of $r \sim 10^{-3}$. The level of B mode polarization is several order of magnitude below the level of temperature fluctuations and polarized foregrounds, consequently, B modes can easily be degraded by various systematic effect due for exemple to temperature to polarization leakage. Together with high sensitivity, an instrument aiming at measuring the B modes requires a strict control of systematic effects.  
Some of these effects can be reduced in the design of the instruments while others have to be simulated and corrected a posteriori. Numerous systematic effects arise from optical and instrumental imperfections, temperature drifts of the optics and detectors, coupling between instrument imperfections and astrophysical foregrounds and the detector non-linearity. These effects can be modelled and characterized by the spurious signal $C_{l}$ they produce \citep{2008PhRvD..77h3003S,quickpol}. \\
The faintness of B modes signal requires the need to study the Universe at high sensitivity. To do so, the development of large detector arrays is required. Most of the detectors currently used are bolometers such as Transition Edge Sensors (TES), they are installed on experiments such as POLARBEAR \citep{2014ApJ...794..171P} or EPIC \citep{2008arXiv0805.4207B}. Performances of current bolometers are reaching the CMB photon noise limit. The need to build large detector arrays necessitate the development of detectors adapted to strong frequency domain multiplexing. In this context, TES are limited and a new kind of detectors was developed : the Kinetic Inductance Detectors (KIDs), proposed by \citet{2003Natur.425..817D}. Since 2007, these detectors have been developed for the construction of NIKA2 and its prototype NIKA (New IRAM Kids Arrays), which is the first operational instrument using KIDs \citep{2010A&A...521A..29M,2016JLTP..184..816C}. The main characteristic that makes KIDs one of the best candidates to large size detector array is their natural multiplexing capability. Indeed, the resonant frequencies of a resonator can be easily controlled during the array design, and the sharpness of the resonances allows many resonators to be placed into a limited bandwidth. Thus, a large number of resonators, can be coupled to a single transmission line, each one resonating at a different frequency $f_{0}^{i}$ \citep{2010A&A...521A..29M,Calvo2008}.\\
In this paper, we adress specifically the issue of non-linearity. This issue has so far not been addressed in the literature and is of particular relevance to the way KIDs are read out. Many polarization experiments such as POLARBEAR \citep{2017JCAP...05..008T} and \nika2 \citep{2017A&A...599A..34R} use a half wave plate (HWP) to improve the sensitivity in polarizatioon measurements, however the modulation of the HWP induces a strong parasitic signal that is several order of magnitude higher than the noise level. In this paper we will also adress the question of non-linearity that can be brought by this parasitic signal. These issues complement the list of specifications that must be put in instrumental designs in view of a space mission.\\
The paper is organized as follows:
Sect. \ref{sec2} presents a brief description of KIDs...
\todo{Say also clearly that this paper both aims at showing the linearity of
  RFdidq and improving on this with the new method}
