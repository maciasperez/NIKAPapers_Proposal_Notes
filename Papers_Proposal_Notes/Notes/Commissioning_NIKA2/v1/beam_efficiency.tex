
\section{Beam efficiency}% {\color{YellowGreen} Laurence $\&$ Jean-Fran\c cois}}
\label{se:beam_efficiency}

Building upon the description of the full-beam pattern in
Sect.~\ref{se:fullbeam} and the main beam in Sect.~\ref{se:MB},
we derive the beam efficiency for each array, which is defined as the
ratio of the solid angle sustained by the main beam to the total beam
solid angle.

We derive an estimate of the total solid angle
\begin{equation}
  \Omega_{\rm{tot}} (A_i, r_{max}) = \int_0^{r_{max}} B_{A_i}(r)/B_{A_i}(0) \times 2 \pi r dr
  \label{eq:omega_tot}
\end{equation}
from the normalised beam profile $B_{A_i}(r)/B_{A_i}(0)$ of the array
$A_i$  at $r_{max} = 180''$ and the main beam solid angle is
evaluated as the volume of the Gaussian main beam as
$\Omega_{\rm{mb}} = 2 \pi\,  \sigma_{\rm{mb}}^2$.

The choice of the maximum radius is entailed both by the depth of
the \bm\ scans and the filtering due to the data processing.   
However, heterodyne observations of the lunar edge and of the forward
beam efficiency derived from skydips show that a non-neglectible
fraction of the full beam is received from beyond a radius of
$180''$~\cite{Greve2010}, \cite{Kramer2013}. This fraction is not
considered here.
\new{Beam efficiency estimators that are based on
the total solid angle estimates using Eq.~\ref{eq:omega_tot} thus
overestimates the beam efficiency, whose the accurate evaluation would
requires the measure of the $4\pi$ integrale of the full beam pattern.}

A large set of \bm\ scans of Uranus and Neptune acquired during the
N2R9, N2R12 and N2R14 campaigns have been used to evaluate
$\Omega_{\rm{tot}}$ from the measured beam profiles up to
$r_{max} =180''$ and $\Omega_{\rm{mb}}$  derived with the main beam FWHM
estimates using the sidelobe-masked 1D method. The result is given in
Table~\ref{tab:solid}.

\begin{table*}[!h]
\caption{Solid angle of true beam based on Uranus and Neptune observations}
\label{tab:solid}
\centering
\begin{tabular}{l| c | c c c | c c c}
\hline\hline
\noalign{\smallskip}
run  & Nber of scans & \multicolumn{3}{c}{$\Omega_{\rm{tot}}$ (arcsec$^{2}$)} & \multicolumn{3}{c}{$\Omega_{\rm{tot}}/\Omega_{gauss}$} \\
\hline
     &               &  A1    &    A2   &  A3  & A1  &  A2  & A3   \\
            \hline
N2R9    & 27  &  265$\pm$ 23    &  466$\pm$ 17 & 252 $\pm$ 23 &  1.80 $\pm$ 0.12    &  1.35 $\pm$ 0.05   &   1.74 $\pm$ 0.13   \\
N2R12   & 20  &  229$\pm$ 11    &  437$\pm$  9 & 221 $\pm$ 10 &  1.71 $\pm$ 0.06   &  1.30 $\pm$ 0.02   &   1.68 $\pm$ 0.06   \\
N2R14   & 28  &  251$\pm$ 16    &  457$\pm$ 15 & 245 $\pm$ 18 &  1.73 $\pm$ 0.08   &  1.32 $\pm$ 0.03   &   1.72 $\pm$ 0.08   \\
mean    &     &  248            &  453         &  239         &  1.74              &   1.32             &   1.71              \\
\noalign{\smallskip}
\hline
\end{tabular}
\end{table*}



This 75 \bm\ scan set, as well as the 12 \bm\ scan sub-set, which is
defined in Sect.~\ref{se:mb_with_beammap}, are further used to measure the beam
efficiencies. We compare the results based on various estimates of
$\Omega_{\rm{tot}}$ and $\Omega_{\rm{mb}}$:
\begin{itemize}
  \item{\emph{Method 1} relies on the best-fitting parameters of the
    three-Gaussian model of the full beam to derive the two solid
    angles. The main beam solid angle thus corresponds to the volume
    enclosed by the first Gaussian;}
  \item{\emph{Method 2} consists in using the measured beam profile to
    estimate $\Omega_{\rm{tot}}$, while $\Omega_{\rm{mb}}$ is derived
    with the main beam Gaussian FWHM fit using the sidelobe-masked 1D
    method, as discussed in Sect.~\ref{se:beam_mb_1D};}
  \item{\emph{Method 3} is similar to method 2 but the main beam FWHM is
    fitted using the sidelobe-masked 2D method, as in Sect.~\ref{se:beam_mb_2D}.}  
\end{itemize}

The beam efficiency estimates using the three methods are gathered
in Table~\ref{tab:beam_efficiency}: central values and error
bars are evaluated as the median and the rms error of the
estimates on individual \bms\ respectively. The rms error estimates
for \emph{method 2}, which are based on 75 \bm\ scans, provide us with
a robust evaluation of the beam efficiency uncertainties. By contrast, error
estimates for \emph{methods 1$\&$3} rely on 12 scans and are thus less
robust. We combined the results of three methods using an error-weighted
average. \emph{Method 2} rms errors are conservatively used as a lower
limit of the error estimates for all methods. The combined beam
efficiency are given in Table~\ref{tab:beam_efficiency}.  
%res1 = [0.54, 0.54, 0.53, 0.74]
%res2 = [0.55, 0.56, 0.55, 0.76]
%res3 = [0.59, 0.58, 0.59, 0.80]
%res = [[res1], [res2], [res3]]
%s1 = [0.03, 0.04, 0.03, 0.04]
%s2 = [0.03, 0.03, 0.03, 0.02]
%s3 = [0.07, 0.04, 0.04, 0.02]
%sig = [[s1], [s2], [s3]]
%for i = 0, 3 do print, total(res[i, *]/sig[i, *]^2)/ total(1d0/sig[i, *]^2)
%for i = 0, 3 do print, sqrt(1d0/ total(1d0/sig[i, *]^2))

\begin{table}[!h]
  \caption[]{Main beam efficiency in $\%$}
  \begin{center}
  \begin{threeparttable}
    \renewcommand{\TPTminimum}{0.6\textwidth}
    {\centering
    \begin{tabular}{l|c|c|c|c}
      \hline\hline
      
      &    \multicolumn{4}{|c}{Array or array combination} \\
      \cline{2-5}
      Method & A1 &  A3 & A1 $\&$ A3 &  A2  \\
      \hline
      method 1\tnote{a} &  $54 \pm 3$  & $54 \pm 4$  &  $53 \pm 3$  &  $74 \pm 4$  \\
      method 2\tnote{b} &  $55 \pm 3$  & $56 \pm 3$  &  $55 \pm 3$  &  $76 \pm 2$  \\
        method 3\tnote{c} &  $59 \pm 7$  & $58 \pm 4$  &  $59 \pm 4$  &  $80 \pm 1$  \\
        combined          &  $55 \pm 3$  & $56 \pm 3$  &  $55 \pm 3$  &  $77 \pm 2$  \\
        \hline\hline
    \end{tabular}
    }
    \begin{tablenotes}
      \small
    \item[(a)] based on the three-Gaussian model best-fitting parameters
    \item[(b)] based on the sidelobe-masked 1D main beam FWHM 
    \item[(c)] based on the sidelobe-masked 2D main beam FWHM 
    \end{tablenotes}
  \end{threeparttable}
  \end{center}
  \label{tab:beam_efficiency}
\end{table}


As a stability check of the beam efficiency, the detailed beam
efficiency estimates using \emph{method 2}, as well as the level of
the error beam, for each observation campaigns are given in Table
\ref{tab:MB}. The level of the error beam is given relative to the
main beam peak (we recall that -12dB as found is $6\%$). We find
stable beam efficiencies using observations acquired one year apart.

%We note a
%mild improvement of the beam efficiency between the N2R9 campaign and
%the following campaigns. This could be due to the change of the method
% of setting the telescope focus:  while the focus was set at the
%best value for the center the arrays during N2R9, from N2R12
%on, the focus is set to the optimal value across the array (see the
%discussion in Sect.~\ref{}).    


\begin{table*}[!h]
\caption{Main beam efficiency and level of error beam}
\label{tab:MB}
\centering
\begin{tabular}{l| c | c c c | c c c}
\hline\hline
%\noalign{\smallskip}
run  & Nber of scans & \multicolumn{3}{|c|}{Main beam efficiency ($\%$)} & \multicolumn{3}{c}{Error beam level (dB)} \\
\hline
     &               &  A1    &    A2   &  A3    & A1  &  A2  & A3   \\
            \hline
N2R9    & 27  &  54.1$\pm$ 3.2   &  74.7$\pm$ 2.9  & 55.9 $\pm$ 3.7   &  -11.5 $\pm$ 0.8    &  -14.9 $\pm$ 0.6   &  -12.0 $\pm$ 0.6   \\
N2R12   & 20  &  55.7$\pm$ 2.0   &  77.4$\pm$ 1.0  & 57.1 $\pm$ 2.0   &  -13.4 $\pm$ 0.3    &  -16.1 $\pm$ 0.3   &  -13.8 $\pm$ 0.3   \\
N2R14   & 28  &  55.0$\pm$ 2.7   &  76.0$\pm$ 1.8  & 56.1 $\pm$ 2.6   &  -12.5 $\pm$ 0.6    &  -15.3 $\pm$ 0.6   &  -12.7 $\pm$ 0.8   \\
            %\noalign{\smallskip}
            \hline\hline
\end{tabular}
\end{table*}

